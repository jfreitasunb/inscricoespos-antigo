\documentclass[11pt]{article}
\usepackage{graphicx,color}
\usepackage{pdfpages}
\usepackage[brazil]{babel}
\usepackage[utf8]{inputenc}
\addtolength{\hoffset}{-3cm} \addtolength{\textwidth}{6cm}
\addtolength{\voffset}{-.5cm} \addtolength{\textheight}{1cm}
%%%%%%%%%%%%%%%%%%%%%%%%%%%%%%%%%%%  To use Colors 
\title{\vspace*{-4cm} Ficha de Inscrição: \\Cod: 1212\ \ Clayton Cristiano Silva\ \ - \ \ Mestrado 
 }
\date{}

\begin{document}
\maketitle
\vspace*{-1.5cm}
\noindent Data de Nascimento:26/1/1991
\ \ \ Idade: 23   \ \ \ Sexo: Masculino
\\
Naturalidade: manhuaçuense  
\ \ \  Estado: MG
\ \ \  Nacionalidade: brasileira
\ \ \ País: Brasil
\\        
Nome do pai : Sebastião Oliveira da Silva
\ \ \ Nome da mãe: Neli Francisca da Silva          
\\[0.2cm]                     
\textbf{Endereço Pessoal} 
\\ 
\noindent Endereço residencial: Rua Fabiano Janote, n 45, Bairro Maria Eugênia
\\
        CEP: 36570-000 
\ \ \ Cidade: Viçosa 
\ \ \ Estado: MG 
\ \ \ País: Brasil
\\		
		Telefone comercial : +55(33)84448978
\ \ \ Telefone residencial: +55(33)84448978
\ \ \ Telefone celular : +55(33)84448978
\\
E-mail principal: clayton.silva@ufv.br
\ \ \ E-mail alternativo: ccris22@gmail.com 
\\[0.2cm] 
\textbf{Documentos Pessoais}
\\
\noindent Número de CPF : 09999717608
\ \ \ Número de Identidade (ou Passaporte para estrangeiros): MG 16714042
\\
Orgão emissor: SSP
\ \ \ Estado: MG
\ \ \ Data de emissão :17/2/2009
\\[0.3cm]
\textbf{Grau acadêmico mais alto obtido}
\\	
Curso:Matemática
\ \ \ Grau : bacharel
\ \ \ Instituição : Universidade Federal de Viçosa
\\			
Ano de Conclusão ou Previsão: 2014
\\ 
Experiência Profissional mais recente. \ \  
Tem experiência: 0 Discente  
\ \ \ Instituição: Universidade Federal de Viçosa
\\  
Período - início: 1-2009
\ \ \ fim: 1-2014
\\[0.2cm] 
\textbf{Programa Pretendido:} Mestrado\\
Interesse em bolsa: Sim
\\[0.3cm]		
\textbf{Dados dos Recomendantes} 
\\
1- Nome: Marinês Guerreiro
\ \ \ \  e-mail: marinesguerreiro0208@gmail.com 
\\
2- Nome: Mércio Botelho Faria
\ \ \ \ e-mail: mercio@gmail.com
\\
3- Nome: Rogério Carvalho Picanço
\ \ \ \ e-mail: rogerio@ufv.br
\\[0.2cm]
Motivação e expectativa do candidato em relação ao programa pretendido:
\\O meu principal interesse com relação ao Programa de Mestrado em Matemática da UNB é ampliar a minha formação e dar continuidade aos meus estudos, principalmente na área de Álgebra. Durante o meu curso de Graduação em Matemática na UFV, desenvolvi um forte interesse por essa área e realizei alguns projetos de Iniciação Científica com ênfase no estudo de Álgebras de Lie e Teoria de Representações de Grupos, de Álgebras Associativas e de Álgebras de Lie. Desse modo, pretendo utilizar os conhecimentos adquiridos nesses projetos para elaborar uma dissertação de mestrado e, futuramente, realizar pesquisa em Álgebra. \newpage\vspace*{-4cm}\subsection*{Carta de Recomendação - Marinês Guerreiro}Código Identificador: 1329\\Conhece-o candidato há quanto tempo (For how long have you known the applicant)? 
\ Desde 2011.
\\ Conhece-o sob as seguintes circunstâncias: aulas\ \ orientacao
	\ \ seminarios\ \  
\\ Conheçe o candidato sob outras circunstâncias: 
\\	Avaliações:\\
\begin{tabular}{|l|c|c|c|c|c|}
\hline
 & Excelente & Bom & Regular & Insuficiente & Não sabe \\
\hline
Desempenho acadêmico &  & X &  &  & \\
\hline
Capacidade de aprender novos conceitos & X &  &  &  & \\
\hline
Capacidade de trabalhar sozinho & X &  &  &  & \\
\hline
Criatividade & X &  &  &  & \\
\hline
Curiosidade & X &  &  &  & \\
\hline
Esforço, persistência & X &  &  &  & \\
\hline
Expressão escrita &  & X &  &  & \\
\hline
Expressão oral &  & X &  &  & \\
\hline
Relacionamento com colegas & X &  &  &  & \\
\hline
\end{tabular}\\
\\
\textbf{Opinião sobre os antecedentes acadêmicos, profissionais e/ou técnicos do candidato:}
\\Clayton foi medalhista da OBMEP e participou do PIC. Como estudante de graduação, foi bolsista de I.C. em vários projetos. Sob minha orientação, 
estudou álgebras de Lie de dimensão finita e infinita. Tem cursado disciplinas avançadas do bacharelado com bom entendimento dos conceitos e teorias.\\
\\
\textbf{Opinião sobre seu possível aproveitamento, se aceito no Programa:}
\\Clayton tem potencial para desenvolver pesquisa original, pois resolve problemas com criatividade e tem boa iniciativa na busca de novos conhecimentos.\\ 
\\
\textbf{Outras informações relevantes:} \\Ele tem direito a uma bolsa do PICME de mestrado que deve ser 
encaminhada junto a coordenação local do PICME até o dia 10 de julho de 2014.
\\[0.3cm]
\textbf{Entre os estudantes que já conheceu, você diria que o candidato está entre os:}
\\
\begin{tabular}{|l|c|c|c|c|c|}
\hline
 & 5\% melhores & 10\% melhores & 25\% melhores & 50\% melhores & Não sabe \\
\hline
Como aluno, em aulas &  & X &  &  & \\
\hline
Como orientando & X &  &  &  & \\
\hline
\end{tabular}
\subsection*{Dados Recomendante} 
	Instituição (Institution): UNIVERSIDADE FEDERAL DE VIÇOSA
\\ 
	Grau acadêmico mais alto obtido: doutor
	\ \ Área: MATEMÁTICA
	\\
	Ano de obtenção deste grau: 1997
	\ \ 
	Instituição de obtenção deste grau : UNIVERSITY OF MANCHESTER
	\\ 
	Endereço institucional do recomendante: \\ UFV  Departamento de Matemática
36570900  Viçosa  MG\newpage\vspace*{-4cm}\subsection*{Carta de Recomendação - Mércio Botelho Faria}Código Identificador: 1330\\Conhece-o candidato há quanto tempo (For how long have you known the applicant)? 
\ Um ano
\\ Conhece-o sob as seguintes circunstâncias: aulas\ \ 
	\ \ \ \  
\\ Conheçe o candidato sob outras circunstâncias: 
\\Avaliações: \\
\begin{tabular}{|l|c|c|c|c|c|}
\hline
 & Excelente & Bom & Regular & Insuficiente & Não sabe \\
\hline
Desempenho acadêmico &  & X &  &  & \\
\hline
Capacidade de aprender novos conceitos &  & X &  &  & \\
\hline
Capacidade de trabalhar sozinho &  & X &  &  & \\
\hline
Criatividade &  & X &  &  & \\
\hline
Curiosidade & X &  &  &  & \\
\hline
Esforço, persistência & X &  &  &  & \\
\hline
Expressão escrita & X &  &  &  & \\
\hline
Expressão oral & X &  &  &  & \\
\hline
Relacionamento com colegas & X &  &  &  & \\
\hline
\end{tabular}\\
\\
\textbf{Opinião sobre os antecedentes acadêmicos, profissionais e/ou técnicos do candidato:}
\\Trata se de um estudante dedicado e curioso. Inicialmente começou outro curso na Universidade vindo posteriormente para a Matemática e desde então tem crescido semestre após semestre.\\
\\
\textbf{Opinião sobre seu possível aproveitamento, se aceito no Programa:}
\\O desempenho vem crescendo ao longo dos semestre o que demonstra seu interesse e esforço em amadurecer matematicamente.\\ 
\\
\textbf{Outras informações relevantes:} \\
\\[0.3cm]
\textbf{Entre os estudantes que já conheceu, você diria que o candidato está entre os:}
\\
\begin{tabular}{|l|c|c|c|c|c|}
\hline
 & 5\% melhores & 10\% melhores & 25\% melhores & 50\% melhores & Não sabe \\
\hline
Como aluno, em aulas &  & X &  &  & \\
\hline
Como orientando &  &  &  &  & X\\
\hline
\end{tabular}
\subsection*{Dados Recomendante} 
	Instituição (Institution): UFV
\\ 
	Grau acadêmico mais alto obtido: doutor
	\ \ Área: Geometria
	\\
	Ano de obtenção deste grau: 2005
	\ \ 
	Instituição de obtenção deste grau : UNICAMP
	\\ 
	Endereço institucional do recomendante: \\ Departamento de Matemática  UFV
av. P. H. Rolfs, sem número
Campus UFV
VIÇOSA  MG
36570000\newpage\vspace*{-4cm}\subsection*{Carta de Recomendação - Rogério Carvalho Picanço}Código Identificador: 915\\Conhece-o candidato há quanto tempo (For how long have you known the applicant)? 
\ 02 anos
\\ Conhece-o sob as seguintes circunstâncias: aulas\ \ 
	\ \ \ \  
\\ Conheçe o candidato sob outras circunstâncias: 
\\Avaliações: \\
\begin{tabular}{|l|c|c|c|c|c|}
\hline
 & Excelente & Bom & Regular & Insuficiente & Não sabe \\
\hline
Desempenho acadêmico &  & X &  &  & \\
\hline
Capacidade de aprender novos conceitos &  & X &  &  & \\
\hline
Capacidade de trabalhar sozinho & X &  &  &  & \\
\hline
Criatividade &  &  &  &  & X\\
\hline
Curiosidade &  & X &  &  & \\
\hline
Esforço, persistência &  & X &  &  & \\
\hline
Expressão escrita &  &  &  &  & X\\
\hline
Expressão oral &  & X &  &  & \\
\hline
Relacionamento com colegas &  & X &  &  & \\
\hline
\end{tabular}\\
\\
\textbf{Opinião sobre os antecedentes acadêmicos, profissionais e/ou técnicos do candidato:}
\\Clayton é meu aluno neste semestre no curso de Tópicos em Álgebra, onde está sendo estudado Teoria de Categorias. Clayton está apresentando um rendimento bom no curso. É um estudante muito aplicado e bastante responsável. Também tenho conhecimento do seu trabalho de IC sobre Álgebra de Lie. É um estudante com bastante motivação e perfil para trabalhar em pesquisa na área de matemática pura.\\
\\
\textbf{Opinião sobre seu possível aproveitamento, se aceito no Programa:}
\\Clayton está concluindo o curso de Bacharelado em Matemática, o que lhe garante uma formação sólida, em várias áreas. Acredito que seu potencial, junto com a formação que obteve fornece a ele todas as possibilidades para um bom desempenho num programa de Mestrado.\\ 
\\
\textbf{Outras informações relevantes:} \\
\\[0.3cm]
\textbf{Entre os estudantes que já conheceu, você diria que o candidato está entre os:}
\\
\begin{tabular}{|l|c|c|c|c|c|}
\hline
 & 5\% melhores & 10\% melhores & 25\% melhores & 50\% melhores & Não sabe \\
\hline
Como aluno, em aulas &  & X &  &  & \\
\hline
Como orientando &  &  &  &  & X\\
\hline
\end{tabular}
\subsection*{Dados Recomendante} 
	Instituição (Institution): UFV - Universidade Federal de Viçosa
\\ 
	Grau acadêmico mais alto obtido: doutor
	\ \ Área: Álgebra
	\\
	Ano de obtenção deste grau: 2010
	\ \ 
	Instituição de obtenção deste grau : UFMG - Universidade Federal de Minas Gerais
	\\ 
	Endereço institucional do recomendante: \\ UFV, Departamento de Matematica, Vicosa, MG\includepdf[pages={-},offset=35mm 0mm]{../../../upload/1212_2014-05-23_documentos.pdf}\includepdf[pages={-},offset=35mm 0mm]{../../../upload/1212_2014-05-23_historico.pdf} 
\begin{center}
Anexos.
\end{center}
\end{document}