\documentclass[11pt]{article}
\usepackage{graphicx,color}
\usepackage{pdfpages}
\usepackage[brazil]{babel}
\usepackage[utf8]{inputenc}
\addtolength{\hoffset}{-3cm} \addtolength{\textwidth}{6cm}
\addtolength{\voffset}{-.5cm} \addtolength{\textheight}{1cm}
%%%%%%%%%%%%%%%%%%%%%%%%%%%%%%%%%%%  To use Colors 
\title{\vspace*{-4cm} Ficha de Inscrição: \\Cod: 1384\ \ Celimar Reijane Alves Damasceno Paiva Alves Damasceno Paiva\ \ - \ \ Mestrado 
 }
\date{}

\begin{document}
\maketitle
\vspace*{-1.5cm}
\noindent Data de Nascimento:6/10/1976
\ \ \ Idade: 37   \ \ \ Sexo: Feminino
\\
Naturalidade: Janaúba  
\ \ \  Estado: MG
\ \ \  Nacionalidade: Brasileira
\ \ \ País: Brasil
\\        
Nome do pai : Pedro Damasceno 
\ \ \ Nome da mãe: Arcena Alves Ferreira Damasceno          
\\[0.2cm]                     
\textbf{Endereço Pessoal} 
\\ 
\noindent Endereço residencial: Avenida Leão XIII, 900
\\
        CEP: 39480-000 
\ \ \ Cidade: Januária 
\ \ \ Estado: MG 
\ \ \ País: Brasil
\\		
		Telefone comercial : +0(38)36294600
\ \ \ Telefone residencial: +0(38)36212048
\ \ \ Telefone celular : +0(38)91715960
\\
E-mail principal: ceelimarreijane@yahoo.com.br
\ \ \ E-mail alternativo: ceelimarreijane@gmail.com 
\\[0.2cm] 
\textbf{Documentos Pessoais}
\\
\noindent Número de CPF : 03075070688
\ \ \ Número de Identidade (ou Passaporte para estrangeiros): m-9232265
\\
Orgão emissor: SSPMG
\ \ \ Estado: MG
\ \ \ Data de emissão :12/6/2006
\\[0.3cm]
\textbf{Grau acadêmico mais alto obtido}
\\	
Curso:Matemática
\ \ \ Grau : licenciado
\ \ \ Instituição : Universidade Federal de Viçosa
\\			
Ano de Conclusão ou Previsão: 2006
\\ 
Experiência Profissional mais recente. \ \  
Tem experiência: Docente Discente  
\ \ \ Instituição: Instituto Federal do Norte de Minas Gerais
\\  
Período - início: 1-2009
\ \ \ fim: 1-0
\\[0.2cm] 
\textbf{Programa Pretendido:} Mestrado\\
Interesse em bolsa: Nao
\\[0.3cm]		
\textbf{Dados dos Recomendantes} 
\\
1- Nome: José Ermelino Alves Damasceno
\ \ \ \  e-mail: ermelinod@bol.com.br 
\\
2- Nome: Edinei Canuto Paiva
\ \ \ \ e-mail: edineifis98@yahoo.com.br
\\
3- Nome: Lilian Isabel Amorim
\ \ \ \ e-mail: lilian.boc@gmail.com
\\[0.2cm]
Motivação e expectativa do candidato em relação ao programa pretendido:
\\ Atualmente sou docente do Instituto Federal de Educação e  Tecnologia do Norte de Minas gerais campus Januária, trabalho nos cursos superiores de matemática, administração, agronomia e engenharia agrícola.A minha área de atuação no departamento de ensino superior abrange as disciplinas de álgebra moderna, álgebra linear, estatística, cálculos I, II, III e IV .Desde que fui aprovada no concurso público, tenho esperado o momento de sair para me qualificar, para melhorar a qualidade do meu trabalho.Infelizmente a minha instituição tem oferecido poucas vagas para afastamento, priorizando mestrados semipresenciais. Eu particularmente tenho interesse em mestrado acadêmico e presencial,pesquisando sobre o programa de mestrado em matemática oferecido pela UNB o que me motivou foram as áreas de pesquisa oferecida pelo programa, em particular Álgebra.Assim como descrevi anteriormente,álgebra é uma área que tenho trabalhado, inclusive com projetos de pesquisa junto a minha instituição.Espero que se for selecionada, eu tenha a oportunidade de aprender mais para melhorar a minha atuação no ensino e na pesquisa. \newpage\vspace*{-4cm}\subsection*{Carta de Recomendação - José Ermelino Alves Damasceno}Código Identificador: 1408\\Conhece-o candidato há quanto tempo (For how long have you known the applicant)? 
\ 37 anos
\\ Conhece-o sob as seguintes circunstâncias: aulas\ \ 
	\ \ seminarios\ \ outra 
\\ Conheçe o candidato sob outras circunstâncias: Projeto de Pesquisa
\\	Avaliações:\\
\begin{tabular}{|l|c|c|c|c|c|}
\hline
 & Excelente & Bom & Regular & Insuficiente & Não sabe \\
\hline
Desempenho acadêmico &  & X &  &  & \\
\hline
Capacidade de aprender novos conceitos & X &  &  &  & \\
\hline
Capacidade de trabalhar sozinho & X &  &  &  & \\
\hline
Criatividade &  & X &  &  & \\
\hline
Curiosidade & X &  &  &  & \\
\hline
Esforço, persistência & X &  &  &  & \\
\hline
Expressão escrita &  & X &  &  & \\
\hline
Expressão oral & X &  &  &  & \\
\hline
Relacionamento com colegas &  & X &  &  & \\
\hline
\end{tabular}\\
\\
\textbf{Opinião sobre os antecedentes acadêmicos, profissionais e/ou técnicos do candidato:}
\\A candidata em questão sempre foi muito esforçada, conheço sua trajetória acadêmica desde a educação básica, que foi cursada em escola pública, ela sempre conseguiu perceber oportunidades diante de todas as dificuldades enfrentadas.\\
\\
\textbf{Opinião sobre seu possível aproveitamento, se aceito no Programa:}
\\Pela sua trajetória acadêmica na Universidade Federal de Viçosa e pela sua atuação como professora no Instituto Federal, tenho plena confiança que conseguirá desenvolver as propostas apresentadas pelo programa com maestria, pois além de autonomia e responsabilidade ela é muito persistente e obstinada, qualidades essas que com certeza lhe dará destaque entre os outros participantes.Portanto, me sinto a vontade para recomendar a sua participação ao Programa de Mestrado em Matemática dessa conceituada Instituição.\\ 
\\
\textbf{Outras informações relevantes:} \\
\\[0.3cm]
\textbf{Entre os estudantes que já conheceu, você diria que o candidato está entre os:}
\\
\begin{tabular}{|l|c|c|c|c|c|}
\hline
 & 5\% melhores & 10\% melhores & 25\% melhores & 50\% melhores & Não sabe \\
\hline
Como aluno, em aulas &  &  &  & X & \\
\hline
Como orientando &  &  &  &  & X\\
\hline
\end{tabular}
\subsection*{Dados Recomendante} 
	Instituição (Institution): Unimontes
\\ 
	Grau acadêmico mais alto obtido: mestre
	\ \ Área: Matemática e Estatística
	\\
	Ano de obtenção deste grau: 2003
	\ \ 
	Instituição de obtenção deste grau : Universidade Federal de Lavras
	\\ 
	Endereço institucional do recomendante: \\ Av. Reinaldo Viana sn Bico da Pedra 
Campus Avançado Janaúba MG.\newpage\vspace*{-4cm}\subsection*{Carta de Recomendação - Edinei Canuto Paiva}Código Identificador: 1422\\Conhece-o candidato há quanto tempo (For how long have you known the applicant)? 
\ 13 anos
\\ Conhece-o sob as seguintes circunstâncias: \ \ 
	\ \ seminarios\ \ outra 
\\ Conheçe o candidato sob outras circunstâncias: Colega de Trabalho
\\Avaliações: \\
\begin{tabular}{|l|c|c|c|c|c|}
\hline
 & Excelente & Bom & Regular & Insuficiente & Não sabe \\
\hline
Desempenho acadêmico &  & X &  &  & \\
\hline
Capacidade de aprender novos conceitos & X &  &  &  & \\
\hline
Capacidade de trabalhar sozinho & X &  &  &  & \\
\hline
Criatividade & X &  &  &  & \\
\hline
Curiosidade & X &  &  &  & \\
\hline
Esforço, persistência & X &  &  &  & \\
\hline
Expressão escrita & X &  &  &  & \\
\hline
Expressão oral & X &  &  &  & \\
\hline
Relacionamento com colegas & X &  &  &  & \\
\hline
\end{tabular}\\
\\
\textbf{Opinião sobre os antecedentes acadêmicos, profissionais e/ou técnicos do candidato:}
\\A candidata iniciou o curso em licenciatura em física na UFV onde fizermos algumas disciplina juntos, sempre demonstrou muita habilidade  na área de matemática que foi o que a  levou períodos depois fazer prova de transferência interna e se formar em matemática.\\
\\
\textbf{Opinião sobre seu possível aproveitamento, se aceito no Programa:}
\\A candidata hoje atua como professora dos cursos superiores de licenciatura em matemática e física, engenharias e administração.Trabalha com as disciplinas de Cálculo, Álgebra e Estatística desde 2009.O que percebemos é que ela procura desenvolver o seu trabalho com excelência e tem contribuído  muito para a formação dos nossos acadêmicos.Acredito que este programa de mestrado tem o seu perfil, pois é bastante dedicada na área do ensino e também na pesquisa, mesmo sem titulação desenvolve projeto de pesquisa a nível institucional.\\ 
\\
\textbf{Outras informações relevantes:} \\Mesmo sem titulação a candidata orienta trabalho de conclusão de curso e projeto de pesquisa.
\\[0.3cm]
\textbf{Entre os estudantes que já conheceu, você diria que o candidato está entre os:}
\\
\begin{tabular}{|l|c|c|c|c|c|}
\hline
 & 5\% melhores & 10\% melhores & 25\% melhores & 50\% melhores & Não sabe \\
\hline
Como aluno, em aulas &  & X &  &  & \\
\hline
Como orientando & X &  &  &  & \\
\hline
\end{tabular}
\subsection*{Dados Recomendante} 
	Instituição (Institution): Instituto Federal de Educação Ciência e Tecnologia do Norte de Minas Gerais
\\ 
	Grau acadêmico mais alto obtido: doutor
	\ \ Área: Engenharia Agrícola
	\\
	Ano de obtenção deste grau: 2009
	\ \ 
	Instituição de obtenção deste grau : Universidade Federal de Viçosa
	\\ 
	Endereço institucional do recomendante: \\ Fazenda São Geraldo, sn, Estrada Januária km 06, Bom Jardim
Januária, Minas Gerais CEP 39480 000\newpage\vspace*{-4cm}\subsection*{Carta de Recomendação - Lilian Isabel Amorim}Código Identificador: 1410\\Conhece-o candidato há quanto tempo (For how long have you known the applicant)? 
\ Há 3 anos
\\ Conhece-o sob as seguintes circunstâncias: \ \ 
	\ \ \ \ outra 
\\ Conheçe o candidato sob outras circunstâncias: Vínculo profissional
\\Avaliações: \\
\begin{tabular}{|l|c|c|c|c|c|}
\hline
 & Excelente & Bom & Regular & Insuficiente & Não sabe \\
\hline
Desempenho acadêmico & X &  &  &  & \\
\hline
Capacidade de aprender novos conceitos & X &  &  &  & \\
\hline
Capacidade de trabalhar sozinho & X &  &  &  & \\
\hline
Criatividade & X &  &  &  & \\
\hline
Curiosidade & X &  &  &  & \\
\hline
Esforço, persistência & X &  &  &  & \\
\hline
Expressão escrita & X &  &  &  & \\
\hline
Expressão oral & X &  &  &  & \\
\hline
Relacionamento com colegas & X &  &  &  & \\
\hline
\end{tabular}\\
\\
\textbf{Opinião sobre os antecedentes acadêmicos, profissionais e/ou técnicos do candidato:}
\\Meu vínculo com a candidata é profissional, uma vez que trabalhamos juntas há três anos. Posso atestar que a mesma é uma excelente profissional, busca trabalhar no ensino, pesquisa e extensão. Atualmente orientamos dois trabalhos de iniciação científica, sendo que um deles, o tema é Criptografia, com ênfase em Teoria dos Números. Gostaria de deixar claro que não posso avaliar a candidata como aluna, mas comparando com a forma que ela cobra dos seus alunos e a responsabilidade com que lida com as suas disciplinas, penso que a postura como estudante não será diferente.\\
\\
\textbf{Opinião sobre seu possível aproveitamento, se aceito no Programa:}
\\Penso que se aceita, a candidata terá um excelente desenvolvimento, pois demonstra elevada capacidade intelectual, grande motivação quando se trata de estudos avançados. Possui ótima expressão oral e escrita. Possui também habilidades humanas, portanto, capacidade de trabalho em equipe. \\ 
\\
\textbf{Outras informações relevantes:} \\Gostaria de enfatizar que a conclusão desse mestrado trará benefícios pessoais para a candidata, mas sobretudo, profissionais, não somente para ela, mas contribuirá para o desenvolvimento do norte de Minas Gerais, região onde atua, especificamente, para a formação de melhores profissionais, pois atualmente, a candidata é professora do IFNMG campus Januária, ministrando aulas nos cursos de Licenciatura em Matemática, Licenciatura em Física e outros.
\\[0.3cm]
\textbf{Entre os estudantes que já conheceu, você diria que o candidato está entre os:}
\\
\begin{tabular}{|l|c|c|c|c|c|}
\hline
 & 5\% melhores & 10\% melhores & 25\% melhores & 50\% melhores & Não sabe \\
\hline
Como aluno, em aulas &  &  &  &  & X\\
\hline
Como orientando &  &  &  &  & X\\
\hline
\end{tabular}
\subsection*{Dados Recomendante} 
	Instituição (Institution): Instituto Federal de Educação, Ciência e Tecnologia do Norte de Minas Gerais campus Januária
\\ 
	Grau acadêmico mais alto obtido: mestre
	\ \ Área: Educação Matemática
	\\
	Ano de obtenção deste grau: 2011
	\ \ 
	Instituição de obtenção deste grau : Universidade Federal de Ouro Preto
	\\ 
	Endereço institucional do recomendante: \\ Fazenda São Geraldo, sn km 6
Januária MG
Cep 39480000\includepdf[pages={-},offset=35mm 0mm]{../../../upload/1384_2014-05-30_documentos.pdf}\includepdf[pages={-},offset=35mm 0mm]{../../../upload/1384_2014-05-30_historico.pdf} 
\begin{center}
Anexos.
\end{center}
\end{document}