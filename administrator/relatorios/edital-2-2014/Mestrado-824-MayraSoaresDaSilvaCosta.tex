\documentclass[11pt]{article}
\usepackage{graphicx,color}
\usepackage{pdfpages}
\usepackage[brazil]{babel}
\usepackage[utf8]{inputenc}
\addtolength{\hoffset}{-3cm} \addtolength{\textwidth}{6cm}
\addtolength{\voffset}{-.5cm} \addtolength{\textheight}{1cm}
%%%%%%%%%%%%%%%%%%%%%%%%%%%%%%%%%%%  To use Colors 
\title{\vspace*{-4cm} Ficha de Inscrição: \\Cod: 824\ \ Mayra  Soares da Silva Costa\ \ - \ \ Mestrado 
 }
\date{}

\begin{document}
\maketitle
\vspace*{-1.5cm}
\noindent Data de Nascimento:29/12/1993
\ \ \ Idade: 20   \ \ \ Sexo: Feminino
\\
Naturalidade: Gama  
\ \ \  Estado: DF
\ \ \  Nacionalidade: Brasileira
\ \ \ País: Brasil
\\        
Nome do pai : Pedro Costa Filho
\ \ \ Nome da mãe: Rosangela Soares da Silva Costa          
\\[0.2cm]                     
\textbf{Endereço Pessoal} 
\\ 
\noindent Endereço residencial: Quadra 02 A casa 31 Pedregal
\\
        CEP: 72860-415 
\ \ \ Cidade: Novo Gama 
\ \ \ Estado: GO 
\ \ \ País: Brasil
\\		
		Telefone comercial : +55(61)36083152
\ \ \ Telefone residencial: +55(61)36083152
\ \ \ Telefone celular : +55(61)92505220
\\
E-mail principal: ssc\_mayra@hotmail.com
\ \ \ E-mail alternativo: 0 
\\[0.2cm] 
\textbf{Documentos Pessoais}
\\
\noindent Número de CPF : 03271864101
\ \ \ Número de Identidade (ou Passaporte para estrangeiros): 2998225
\\
Orgão emissor: SSP
\ \ \ Estado: GO
\ \ \ Data de emissão :28/7/2010
\\[0.3cm]
\textbf{Grau acadêmico mais alto obtido}
\\	
Curso:Matemática
\ \ \ Grau : bacharel
\ \ \ Instituição : Universidade de Brasília
\\			
Ano de Conclusão ou Previsão: 2014
\\ 
Experiência Profissional mais recente. \ \  
Tem experiência: Docente Discente  
\ \ \ Instituição: 0
\\  
Período - início: 0-0
\ \ \ fim: 0-0
\\[0.2cm] 
\textbf{Programa Pretendido:} Mestrado\\
Interesse em bolsa: Sim
\\[0.3cm]		
\textbf{Dados dos Recomendantes} 
\\
1- Nome: Liliane de Almeida Maia
\ \ \ \  e-mail: lilimaia.unb@gmail.com 
\\
2- Nome: Noraí Rocco
\ \ \ \ e-mail: norai.rocco@gmail.com
\\
3- Nome: Leandro Cioletti
\ \ \ \ e-mail: leandro.mat@gmail.com
\\[0.2cm]
Motivação e expectativa do candidato em relação ao programa pretendido:
\\Sou aluna de bacharelado do departamento de matemática da Unb, estou concluindo meu curso esse semestre, e quero ingressar no programa de mestrado da Unb,porque pretendo atuar como docente em matemática na área de Análise, e sei que a Unb é uma das melhores no país nessa área. Me sinto motivada pela afeição que tenho à matemática, e por ter certeza que é com a matemática que eu me identifico. Sei que meu futuro profissional está voltado para isso.


Atenciosamente,


Mayra Soares.\newpage\vspace*{-4cm}\subsection*{Carta de Recomendação - Liliane de Almeida Maia}Código Identificador: 655\\Conhece-o candidato há quanto tempo (For how long have you known the applicant)? 
\ 2 anos
\\ Conhece-o sob as seguintes circunstâncias: aulas\ \ 
	\ \ \ \  
\\ Conheçe o candidato sob outras circunstâncias: Análise II
\\	Avaliações:\\
\begin{tabular}{|l|c|c|c|c|c|}
\hline
 & Excelente & Bom & Regular & Insuficiente & Não sabe \\
\hline
Desempenho acadêmico & X &  &  &  & \\
\hline
Capacidade de aprender novos conceitos &  & X &  &  & \\
\hline
Capacidade de trabalhar sozinho & X &  &  &  & \\
\hline
Criatividade &  & X &  &  & \\
\hline
Curiosidade & X &  &  &  & \\
\hline
Esforço, persistência & X &  &  &  & \\
\hline
Expressão escrita & X &  &  &  & \\
\hline
Expressão oral & X &  &  &  & \\
\hline
Relacionamento com colegas & X &  &  &  & \\
\hline
\end{tabular}\\
\\
\textbf{Opinião sobre os antecedentes acadêmicos, profissionais e/ou técnicos do candidato:}
\\Mayra foi minha aluna em Análise II em 2.2012.
Foi excelente aluna na disciplina.\\
\\
\textbf{Opinião sobre seu possível aproveitamento, se aceito no Programa:}
\\Será uma ótima aluna no Mestrado se repetir a sua atuação em Análise II.\\ 
\\
\textbf{Outras informações relevantes:} \\Recomendo fortemente a aceitação da aluna no mestrado.
\\[0.3cm]
\textbf{Entre os estudantes que já conheceu, você diria que o candidato está entre os:}
\\
\begin{tabular}{|l|c|c|c|c|c|}
\hline
 & 5\% melhores & 10\% melhores & 25\% melhores & 50\% melhores & Não sabe \\
\hline
Como aluno, em aulas & X &  &  &  & \\
\hline
Como orientando &  &  &  &  & X\\
\hline
\end{tabular}
\subsection*{Dados Recomendante} 
	Instituição (Institution): UnB
\\ 
	Grau acadêmico mais alto obtido: doutor
	\ \ Área: Análise
	\\
	Ano de obtenção deste grau: 1992
	\ \ 
	Instituição de obtenção deste grau : University of Wisconsin - Madison EUA
	\\ 
	Endereço institucional do recomendante: \\ UnB\newpage\vspace*{-4cm}\subsection*{Carta de Recomendação - Noraí Rocco}Código Identificador: 1201\\Conhece-o candidato há quanto tempo (For how long have you known the applicant)? 
\ há uns dois anos ou mais.
\\ Conhece-o sob as seguintes circunstâncias: aulas\ \ 
	\ \ \ \  
\\ Conheçe o candidato sob outras circunstâncias: 
\\Avaliações: \\
\begin{tabular}{|l|c|c|c|c|c|}
\hline
 & Excelente & Bom & Regular & Insuficiente & Não sabe \\
\hline
Desempenho acadêmico & X &  &  &  & \\
\hline
Capacidade de aprender novos conceitos & X &  &  &  & \\
\hline
Capacidade de trabalhar sozinho & X &  &  &  & \\
\hline
Criatividade &  & X &  &  & \\
\hline
Curiosidade &  & X &  &  & \\
\hline
Esforço, persistência & X &  &  &  & \\
\hline
Expressão escrita &  & X &  &  & \\
\hline
Expressão oral & X &  &  &  & \\
\hline
Relacionamento com colegas & X &  &  &  & \\
\hline
\end{tabular}\\
\\
\textbf{Opinião sobre os antecedentes acadêmicos, profissionais e/ou técnicos do candidato:}
\\A candidata é oriunda da nossa graduação e seu histórico escolar exibe um excelente desempenho. De fato, é uma candidata fortíssima para o nosso Mestrado. Ela cursou comigo disciplinas de Álgebra 1 e 2 e obteve a menção SS em ambas. Acho que o seu desempenho nas outras disciplinas também é semelhante. Tem todas as condições para prosseguir com sucesso nos estudos. \\
\\
\textbf{Opinião sobre seu possível aproveitamento, se aceito no Programa:}
\\Altamente promissora. \\ 
\\
\textbf{Outras informações relevantes:} \\Mayra é uma excelente pessoa também. Muito bem educada, persistente, tem muita garra e  boa vontade para superar as dificuldades. Além de muito talentosa. 
\\[0.3cm]
\textbf{Entre os estudantes que já conheceu, você diria que o candidato está entre os:}
\\
\begin{tabular}{|l|c|c|c|c|c|}
\hline
 & 5\% melhores & 10\% melhores & 25\% melhores & 50\% melhores & Não sabe \\
\hline
Como aluno, em aulas &  & X &  &  & \\
\hline
Como orientando &  &  &  &  & X\\
\hline
\end{tabular}
\subsection*{Dados Recomendante} 
	Instituição (Institution): UnB
\\ 
	Grau acadêmico mais alto obtido: doutor
	\ \ Área: Álgebra
	\\
	Ano de obtenção deste grau: 1980
	\ \ 
	Instituição de obtenção deste grau : UnB
	\\ 
	Endereço institucional do recomendante: \\ UnB ICC Centro BT 403 
Tel 31077347\newpage\vspace*{-4cm}\subsection*{Carta de Recomendação - Leandro Cioletti}Código Identificador: 110\\Conhece-o candidato há quanto tempo (For how long have you known the applicant)? 
\ 6 meses.
\\ Conhece-o sob as seguintes circunstâncias: aulas\ \ 
	\ \ \ \  
\\ Conheçe o candidato sob outras circunstâncias: 
\\Avaliações: \\
\begin{tabular}{|l|c|c|c|c|c|}
\hline
 & Excelente & Bom & Regular & Insuficiente & Não sabe \\
\hline
Desempenho acadêmico & X &  &  &  & \\
\hline
Capacidade de aprender novos conceitos & X &  &  &  & \\
\hline
Capacidade de trabalhar sozinho & X &  &  &  & \\
\hline
Criatividade & X &  &  &  & \\
\hline
Curiosidade & X &  &  &  & \\
\hline
Esforço, persistência & X &  &  &  & \\
\hline
Expressão escrita & X &  &  &  & \\
\hline
Expressão oral & X &  &  &  & \\
\hline
Relacionamento com colegas & X &  &  &  & \\
\hline
\end{tabular}\\
\\
\textbf{Opinião sobre os antecedentes acadêmicos, profissionais e/ou técnicos do candidato:}
\\Mayra foi minha aluna no curso de verão de topologia e foi a estudante de melhor desempenho no curso com notas 9.9, 9.9 e 10 nas provas 1,2 e 3, respectivamente. Ela é ótima estudante tem boas ideias escreve de maneira clara e é muito independente.\\
\\
\textbf{Opinião sobre seu possível aproveitamento, se aceito no Programa:}
\\Acredito que a Mayra concluíra o Mestrado sem problemas e provavelmente com ótimo desempenho.\\ 
\\
\textbf{Outras informações relevantes:} \\Mayra é uma pessoa muito educada, agradável, respeitosa e prestativa.
Recomendo fortemente sua aceitação no programa de Pós Graduação com bolsa.
\\[0.3cm]
\textbf{Entre os estudantes que já conheceu, você diria que o candidato está entre os:}
\\
\begin{tabular}{|l|c|c|c|c|c|}
\hline
 & 5\% melhores & 10\% melhores & 25\% melhores & 50\% melhores & Não sabe \\
\hline
Como aluno, em aulas & X &  &  &  & \\
\hline
Como orientando &  &  &  &  & X\\
\hline
\end{tabular}
\subsection*{Dados Recomendante} 
	Instituição (Institution): Universidade de Brasília
\\ 
	Grau acadêmico mais alto obtido: doutor
	\ \ Área: Física-Matemática
	\\
	Ano de obtenção deste grau: 2008
	\ \ 
	Instituição de obtenção deste grau : Universidade Federal de Minas Gerais
	\\ 
	Endereço institucional do recomendante: \\ Universidade de Brasília, Campus Darci Ribeiro.\includepdf[pages={-},offset=35mm 0mm]{../../../upload/824_2014-05-04_documentos.pdf}\includepdf[pages={-},offset=35mm 0mm]{../../../upload/824_2014-05-04_historico.pdf}\includepdf[pages={-},offset=35mm 0mm]{../../../upload/824_2013-11-13_documentos.pdf}\includepdf[pages={-},offset=35mm 0mm]{../../../upload/824_2013-11-13_historico.pdf} 
\begin{center}
Anexos.
\end{center}
\end{document}