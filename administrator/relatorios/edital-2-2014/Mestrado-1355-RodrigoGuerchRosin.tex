\documentclass[11pt]{article}
\usepackage{graphicx,color}
\usepackage{pdfpages}
\usepackage[brazil]{babel}
\usepackage[utf8]{inputenc}
\addtolength{\hoffset}{-3cm} \addtolength{\textwidth}{6cm}
\addtolength{\voffset}{-.5cm} \addtolength{\textheight}{1cm}
%%%%%%%%%%%%%%%%%%%%%%%%%%%%%%%%%%%  To use Colors 
\title{\vspace*{-4cm} Ficha de Inscrição: \\Cod: 1355\ \ Rodrigo Guerch Rosin\ \ - \ \ Mestrado 
 }
\date{}

\begin{document}
\maketitle
\vspace*{-1.5cm}
\noindent Data de Nascimento:8/10/1992
\ \ \ Idade: 21   \ \ \ Sexo: Masculino
\\
Naturalidade: São Vicente do Sul  
\ \ \  Estado: RS
\ \ \  Nacionalidade: Brasileiro
\ \ \ País: Brasil
\\        
Nome do pai : Osmar Vanderlei Rosin
\ \ \ Nome da mãe: Iracema Guerch Rosin          
\\[0.2cm]                     
\textbf{Endereço Pessoal} 
\\ 
\noindent Endereço residencial: Rua 20 de Setembro
\\
        CEP: 97420-000 
\ \ \ Cidade: São Vicente do Sul 
\ \ \ Estado: RS 
\ \ \ País: Brasil
\\		
		Telefone comercial : +55(55)96652854
\ \ \ Telefone residencial: +55(55)96652854
\ \ \ Telefone celular : +55(55)96652854
\\
E-mail principal: rodrigo.rosin@yahoo.com.br
\ \ \ E-mail alternativo: rodrigoguerch@hotmail.com 
\\[0.2cm] 
\textbf{Documentos Pessoais}
\\
\noindent Número de CPF : 02299936050
\ \ \ Número de Identidade (ou Passaporte para estrangeiros): 5082984955
\\
Orgão emissor: SSP
\ \ \ Estado: RS
\ \ \ Data de emissão :2/12/2011
\\[0.3cm]
\textbf{Grau acadêmico mais alto obtido}
\\	
Curso:Matemática
\ \ \ Grau : licenciado
\ \ \ Instituição : Universidade Federal de Santa Maria
\\			
Ano de Conclusão ou Previsão: 2014
\\ 
Experiência Profissional mais recente. \ \  
Tem experiência: Docente Discente  
\ \ \ Instituição: Universidade Federal de Santa Maria
\\  
Período - início: 2-2010
\ \ \ fim: 0-2014
\\[0.2cm] 
\textbf{Programa Pretendido:} Mestrado\\
Interesse em bolsa: Sim
\\[0.3cm]		
\textbf{Dados dos Recomendantes} 
\\
1- Nome: Antonio Carlos Lyrio Bidel
\ \ \ \  e-mail: bidelac@gmail.com 
\\
2- Nome: Ricardo Fajardo
\ \ \ \ e-mail: rfaj@ufsm.br
\\
3- Nome: Celene Buriol
\ \ \ \ e-mail: celene.buriol@gmail.com
\\[0.2cm]
Motivação e expectativa do candidato em relação ao programa pretendido:
\\Realizo minha inscrição neste, no intuito de continuar minha formação e ampliar meus conhecimentos na área que quero trabalhar, desenvolvendo minhas capacidades e sanando minhas dificuldades. Além claro de buscar uma melhor condição financeira sendo oriundo da classe baixa. 
Pretendo encontrar pessoas capasses e compreensíveis, dispostas a fazer um bom trabalhar e a auxiliar nas dificuldades que vou encontrar durante esse processo.   
Ressalto que seria de grade satisfação ingressar nesta instituição e no mestrado pretendido.

Sem mais delongas, Rodrigo Guerch Rosin.  
\newpage\vspace*{-4cm}\subsection*{Carta de Recomendação - Antonio Carlos Lyrio Bidel}Código Identificador: 221\\Conhece-o candidato há quanto tempo (For how long have you known the applicant)? 
\ 4 anos
\\ Conhece-o sob as seguintes circunstâncias: \ \ orientacao
	\ \ \ \ outra 
\\ Conheçe o candidato sob outras circunstâncias: Orinetação de atividades de ensino e extensão vinculadas ao PET Matemática da UFSM do qual sou tutor desde junho de 2005
\\	Avaliações:\\
\begin{tabular}{|l|c|c|c|c|c|}
\hline
 & Excelente & Bom & Regular & Insuficiente & Não sabe \\
\hline
Desempenho acadêmico &  & X &  &  & \\
\hline
Capacidade de aprender novos conceitos & X &  &  &  & \\
\hline
Capacidade de trabalhar sozinho & X &  &  &  & \\
\hline
Criatividade & X &  &  &  & \\
\hline
Curiosidade & X &  &  &  & \\
\hline
Esforço, persistência & X &  &  &  & \\
\hline
Expressão escrita & X &  &  &  & \\
\hline
Expressão oral &  & X &  &  & \\
\hline
Relacionamento com colegas & X &  &  &  & \\
\hline
\end{tabular}\\
\\
\textbf{Opinião sobre os antecedentes acadêmicos, profissionais e/ou técnicos do candidato:}
\\O candidato desenvolve atividades, de Iniciação Científica, com professor orientador do Depto de Matemática da UFSM,extensão e ensino vinculadas ao PET Matemática da UFSM. Tem demonstrado capacidade, motivação e persistência no planejamento, execução e avaliação das atividades. Se relaciona muito bem com colegas e professores.\\
\\
\textbf{Opinião sobre seu possível aproveitamento, se aceito no Programa:}
\\Concluirá seus estudos em tempo hábil e com bom aproveitamento.\\ 
\\
\textbf{Outras informações relevantes:} \\Apesar de dedicar 20h semanais as atividades do PET, o candidato tem mantido um bom aproveitamento.
\\[0.3cm]
\textbf{Entre os estudantes que já conheceu, você diria que o candidato está entre os:}
\\
\begin{tabular}{|l|c|c|c|c|c|}
\hline
 & 5\% melhores & 10\% melhores & 25\% melhores & 50\% melhores & Não sabe \\
\hline
Como aluno, em aulas &  & X &  &  & \\
\hline
Como orientando & X &  &  &  & \\
\hline
\end{tabular}
\subsection*{Dados Recomendante} 
	Instituição (Institution): Universidade Federal de Santa Maria
\\ 
	Grau acadêmico mais alto obtido: doutor
	\ \ Área: Engenharia Mecânica - Mecânica dos Sólidos
	\\
	Ano de obtenção deste grau: 2003
	\ \ 
	Instituição de obtenção deste grau : UFRGS
	\\ 
	Endereço institucional do recomendante: \\ Avenida Roraíma número 1000 - Prédio 13 CCNE
Campus Universitário - Camobi - Santa Maria RS 
97105 - 900\newpage\vspace*{-4cm}\subsection*{Carta de Recomendação - Ricardo Fajardo}Código Identificador: 1356\\Conhece-o candidato há quanto tempo (For how long have you known the applicant)? 
\ Quatro anos
\\ Conhece-o sob as seguintes circunstâncias: aulas\ \ 
	\ \ \ \  
\\ Conheçe o candidato sob outras circunstâncias: 
\\Avaliações: \\
\begin{tabular}{|l|c|c|c|c|c|}
\hline
 & Excelente & Bom & Regular & Insuficiente & Não sabe \\
\hline
Desempenho acadêmico & X &  &  &  & \\
\hline
Capacidade de aprender novos conceitos & X &  &  &  & \\
\hline
Capacidade de trabalhar sozinho & X &  &  &  & \\
\hline
Criatividade &  & X &  &  & \\
\hline
Curiosidade & X &  &  &  & \\
\hline
Esforço, persistência & X &  &  &  & \\
\hline
Expressão escrita &  & X &  &  & \\
\hline
Expressão oral &  & X &  &  & \\
\hline
Relacionamento com colegas & X &  &  &  & \\
\hline
\end{tabular}\\
\\
\textbf{Opinião sobre os antecedentes acadêmicos, profissionais e/ou técnicos do candidato:}
\\O candidato é sério, responsável e interessado com relação ao conteúdo matemático. Possui um raciocínio lógico muito bom, que continua desenvolvendo. Trabalha muito bem individualmente, mas também se relaciona bem em trabalho de grupo.\\
\\
\textbf{Opinião sobre seu possível aproveitamento, se aceito no Programa:}
\\O candidato sempre demonstrou responsabilidade, seriedade e interesse na carreira. Tenho certeza que o candidato se dedicará ao programa e irá terminar o mesmo.\\ 
\\
\textbf{Outras informações relevantes:} \\O candidato será uma grande aquisição para o programa.
\\[0.3cm]
\textbf{Entre os estudantes que já conheceu, você diria que o candidato está entre os:}
\\
\begin{tabular}{|l|c|c|c|c|c|}
\hline
 & 5\% melhores & 10\% melhores & 25\% melhores & 50\% melhores & Não sabe \\
\hline
Como aluno, em aulas & X &  &  &  & \\
\hline
Como orientando & X &  &  &  & \\
\hline
\end{tabular}
\subsection*{Dados Recomendante} 
	Instituição (Institution): Universidade Federal de Santa Maria
\\ 
	Grau acadêmico mais alto obtido: doutor
	\ \ Área: Probability Theory
	\\
	Ano de obtenção deste grau: 1993
	\ \ 
	Instituição de obtenção deste grau : University of Rochester  NY EUA
	\\ 
	Endereço institucional do recomendante: \\ Departamento de Matemática
Centro de Ciências Naturais e Exatas
Universidade Federal de Santa Maria
Avenida Roraima, 1000
Bairro Camobi
97105900 Santa Maria RS\newpage\vspace*{-4cm}\subsection*{Carta de Recomendação - Celene Buriol}Código Identificador: 1358\\Conhece-o candidato há quanto tempo (For how long have you known the applicant)? 
\ 2 anos
\\ Conhece-o sob as seguintes circunstâncias: aulas\ \ 
	\ \ \ \  
\\ Conheçe o candidato sob outras circunstâncias: 
\\Avaliações: \\
\begin{tabular}{|l|c|c|c|c|c|}
\hline
 & Excelente & Bom & Regular & Insuficiente & Não sabe \\
\hline
Desempenho acadêmico &  & X &  &  & \\
\hline
Capacidade de aprender novos conceitos &  & X &  &  & \\
\hline
Capacidade de trabalhar sozinho &  & X &  &  & \\
\hline
Criatividade &  & X &  &  & \\
\hline
Curiosidade &  & X &  &  & \\
\hline
Esforço, persistência & X &  &  &  & \\
\hline
Expressão escrita &  &  & X &  & \\
\hline
Expressão oral &  &  & X &  & \\
\hline
Relacionamento com colegas & X &  &  &  & \\
\hline
\end{tabular}\\
\\
\textbf{Opinião sobre os antecedentes acadêmicos, profissionais e/ou técnicos do candidato:}
\\Foi meu aluno em análise na reta. Tem dificuldade mas é muito esforçado.Aluno muito estudioso e dedicado.\\
\\
\textbf{Opinião sobre seu possível aproveitamento, se aceito no Programa:}
\\Creio que pela sua dedicação e esforço será um bom aluno de mestrado.\\ 
\\
\textbf{Outras informações relevantes:} \\
\\[0.3cm]
\textbf{Entre os estudantes que já conheceu, você diria que o candidato está entre os:}
\\
\begin{tabular}{|l|c|c|c|c|c|}
\hline
 & 5\% melhores & 10\% melhores & 25\% melhores & 50\% melhores & Não sabe \\
\hline
Como aluno, em aulas &  & X &  &  & \\
\hline
Como orientando &  &  &  &  & X\\
\hline
\end{tabular}
\subsection*{Dados Recomendante} 
	Instituição (Institution): Universidade Federal de Santa Maria
\\ 
	Grau acadêmico mais alto obtido: doutor
	\ \ Área: Análise,  Equações Diferenciais Parciais 
	\\
	Ano de obtenção deste grau: 2004
	\ \ 
	Instituição de obtenção deste grau : Universidade Federal do Rio de Janeiro
	\\ 
	Endereço institucional do recomendante: \\ Rua Diamante número 475,Camobi. Cep 97110764. Santa Maria, RS.\includepdf[pages={-},offset=35mm 0mm]{../../../upload/1355_2014-05-26_documentos.pdf}\includepdf[pages={-},offset=35mm 0mm]{../../../upload/1355_2014-05-26_historico.pdf} 
\begin{center}
Anexos.
\end{center}
\end{document}