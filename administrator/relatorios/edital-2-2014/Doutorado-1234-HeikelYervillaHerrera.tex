\documentclass[11pt]{article}
\usepackage{graphicx,color}
\usepackage{pdfpages}
\usepackage[brazil]{babel}
\usepackage[utf8]{inputenc}
\addtolength{\hoffset}{-3cm} \addtolength{\textwidth}{6cm}
\addtolength{\voffset}{-.5cm} \addtolength{\textheight}{1cm}
%%%%%%%%%%%%%%%%%%%%%%%%%%%%%%%%%%%  To use Colors 
\title{\vspace*{-4cm} Ficha de Inscrição: \\Cod: 1234\ \ Heikel Yervilla Herrera\ \ - \ \ Doutorado 
 }
\date{}

\begin{document}
\maketitle
\vspace*{-1.5cm}
\noindent Data de Nascimento:25/4/1983
\ \ \ Idade: 31   \ \ \ Sexo: Masculino
\\
Naturalidade: Cuba  
\ \ \  Estado: Outro
\ \ \  Nacionalidade: cubana
\ \ \ País: Cuba
\\        
Nome do pai : Pantaleon Félix Yervilla Diaz
\ \ \ Nome da mãe: Marta María Herrera Zamora          
\\[0.2cm]                     
\textbf{Endereço Pessoal} 
\\ 
\noindent Endereço residencial: 4ta 369A, e G y Doble Vía, Vigía
\\
        CEP: 50200-000 
\ \ \ Cidade: Santa Clara 
\ \ \ Estado: Outro 
\ \ \ País: Cuba
\\		
		Telefone comercial : +53(42)224746
\ \ \ Telefone residencial: +53(42)204695
\ \ \ Telefone celular : +53(53)228898
\\
E-mail principal: yervilla@gmail.com
\ \ \ E-mail alternativo: yervilla@uclv.edu.cu 
\\[0.2cm] 
\textbf{Documentos Pessoais}
\\
\noindent Número de CPF : 0
\ \ \ Número de Identidade (ou Passaporte para estrangeiros): B972032
\\
Orgão emissor: MININT
\ \ \ Estado: Outro
\ \ \ Data de emissão :24/1/2011
\\[0.3cm]
\textbf{Grau acadêmico mais alto obtido}
\\	
Curso:Ciencia de la Computación
\ \ \ Grau : mestre
\ \ \ Instituição : Universidad Central de Las Villas
\\			
Ano de Conclusão ou Previsão: 2012
\\ 
Experiência Profissional mais recente. \ \  
Tem experiência: Docente 0  
\ \ \ Instituição: Universidad Central de Las Villas,
\\  
Período - início: 1-2014
\ \ \ fim: 1-2014
\\[0.2cm] 
\textbf{Programa Pretendido:} Doutorado\ \ \ \textbf{Área:} TeoriaDaComputacao\\
Interesse em bolsa: Sim
\\[0.3cm]		
\textbf{Dados dos Recomendantes} 
\\
1- Nome: Carlos Alexander Recarey Morfa
\ \ \ \  e-mail: carecarey@gmail.com 
\\
2- Nome: Daniel Galvez Lio
\ \ \ \ e-mail: dgalvez@uclv.edu.cu
\\
3- Nome: Maria Matilde Garcia Lorenzo
\ \ \ \ e-mail: mmgarcia@uclv.edu.cu
\\[0.2cm]
Motivação e expectativa do candidato em relação ao programa pretendido:
\\Mi nombre es Heikel Yervilla Herrera soy graduado en Ciencias de la Computación en la Universidad Central de las Villas UCLV. Estoy interesado en el programa de estudios de postgrado en Ciencias de la Computación que la UnB ofrece debido a la relevancia y actualidad de las investigaciones llevadas a cabo en el centro y específicamente su vinculación con la Inteligencia Artificial, Robótica, Computación Gráfica y Bioinformática.

Trabajo actualmente como investigador en la UCLV, mis temas de investigación  están relacionados con la mecánica computacional, representaciones espaciales, generación de mallas y visualización científica. Específicamente he desarrollado e implementado técnicas de visualización científica para métodos  numéricos  en la ingeniería, métodos  para la generación de mallas y nubes de puntos, algoritmos para el manejo de grandes volúmenes de información, he trabajado con algoritmos de inteligencia artificial aplicados a problemas de ingeniería y poseo, además, un fuerte grado de familiaridad con estructuras de datos de búsqueda y particionamiento  espacial octree, covertree, kdtree, spatial grids, etc. Tengo, además,  experiencia en varios lenguajes de programación tales como C, C, TCL, entre otros y APIs tales como OpenGL, OpenMP, GTK, VTK, etcétera.

Trabajo, además, en colaboración con la Universidad Politécnica de Catalua  UPC, específicamente con el Centro Internacional de Métodos Numéricos en la Ingeniería CIMNE en el cual realicé una pasantía de trabajo desarrollando e  implementando algoritmos computacionales para los grupos de GID y Bioinformática. He presentado y publicado varios artículos en eventos nacionales e internacionales y revistas de alto factor de impacto.
Considerando mis calificaciones e investigaciones fui aceptado en la Maestría en Ciencia de la Computación ofrecida por el Centro de Estudios Informáticos de la UCLV la cual complete recién en el 2012. Todo ello en conjunto me permiten tener una fuerte base y perspectiva para enfrentar diversos y disímiles problemas de investigación. 

Mi licenciatura y en especial la maestría en Ciencia de la Computación tienen una fuerte base en Inteligencia Artificial y en  Métodos de Solución de Problemas y siempre he tenido como objetivo combinar estas  temáticas a mi trabajo y la robótica es un tema de actualidad y mucha aplicación hoy en día. Es por ello que he escogido su universidad para comenzar a investigar en la  materia y combinar mis  áreas  de investigación con temas de inteligencia  artificial, optimización y robótica.

La posible aplicación de las investigaciones con la culminación de los  estudios si procede es bien diversa, desde fines académicos con la divulgación de conocimientos hasta la aplicación real a problemas sociales. Demás esta mencionar el impacto a nivel personal y la satisfacción de trabajar en conjunto con otros estudiantes y profesionales en un centro de relevancia a nivel internacional en la solución de problemas de investigación básica e innovación tecnológica.\newpage\vspace*{-4cm}\subsection*{Carta de Recomendação - Carlos Alexander Recarey Morfa}Código Identificador: 1416\\Conhece-o candidato há quanto tempo (For how long have you known the applicant)? 
\ 12 years
\\ Conhece-o sob as seguintes circunstâncias: \ \ orientacao
	\ \ \ \  
\\ Conheçe o candidato sob outras circunstâncias: 
\\	Avaliações:\\
\begin{tabular}{|l|c|c|c|c|c|}
\hline
 & Excelente & Bom & Regular & Insuficiente & Não sabe \\
\hline
Desempenho acadêmico & X &  &  &  & \\
\hline
Capacidade de aprender novos conceitos & X &  &  &  & \\
\hline
Capacidade de trabalhar sozinho & X &  &  &  & \\
\hline
Criatividade & X &  &  &  & \\
\hline
Curiosidade & X &  &  &  & \\
\hline
Esforço, persistência & X &  &  &  & \\
\hline
Expressão escrita & X &  &  &  & \\
\hline
Expressão oral & X &  &  &  & \\
\hline
Relacionamento com colegas & X &  &  &  & \\
\hline
\end{tabular}\\
\\
\textbf{Opinião sobre os antecedentes acadêmicos, profissionais e/ou técnicos do candidato:}
\\Si es aceptado en el programa de doctorado el candidato podrá beneficiarse de una formación en la investigación altamente especializada, además contribuirá de alguna manera al desarrollo de la ciencia en nuestro país, se espera que todos los conocimientos adquiridos jueguen un papel clave en la economía del conocimiento y de la nación porque estará en una mejor posición de impulsar avances en la ciencia, la tecnología y en el conocimiento en la sociedad.\\
\\
\textbf{Opinião sobre seu possível aproveitamento, se aceito no Programa:}
\\El candidato Heikel Yervilla Herrera posee muy buena actitud para desenvolverse en el ambiente académico, sus notas son sobresalientes, su capacidad de aprendizaje es buena, como profesionalinvestigador ha publicado disimiles artículos en la web, ha participado en congresos internacionales y nacionales. Técnicamente, como profesional de la computación, posee un conocimiento amplio de varios lenguajes de programación y muy buena capacidad de adaptación a cualquier ambiente de desarrollo.\\ 
\\
\textbf{Outras informações relevantes:} \\En el ambito laboral el candidato posee buenas ralaciones con los demás, como estudiante y profesional muestra un gran nivel de responsabilidad, comprensión, iniciativa y entusiasmo para desarrollar actividades académicas y profesionales.
\\[0.3cm]
\textbf{Entre os estudantes que já conheceu, você diria que o candidato está entre os:}
\\
\begin{tabular}{|l|c|c|c|c|c|}
\hline
 & 5\% melhores & 10\% melhores & 25\% melhores & 50\% melhores & Não sabe \\
\hline
Como aluno, em aulas & X &  &  &  & \\
\hline
Como orientando & X &  &  &  & \\
\hline
\end{tabular}
\subsection*{Dados Recomendante} 
	Instituição (Institution): Universidad Central Marta Abreu de las Villas 
\\ 
	Grau acadêmico mais alto obtido: doutor
	\ \ Área: Ingeniría Civil
	\\
	Ano de obtenção deste grau: 1996
	\ \ 
	Instituição de obtenção deste grau : Universidad Central Marta Abreu de las Villas 
	\\ 
	Endereço institucional do recomendante: \\ Universidad Central Marta Abreu de Las Villas, Carretera a Camajuaní, km 5 , Santa Clara, Villa Clara, Cuba. CP. 54830\newpage\vspace*{-4cm}\subsection*{Carta de Recomendação - Daniel Galvez Lio}Código Identificador: 1417\\Conhece-o candidato há quanto tempo (For how long have you known the applicant)? 
\ 8 years
\\ Conhece-o sob as seguintes circunstâncias: aulas\ \ orientacao
	\ \ \ \ outra 
\\ Conheçe o candidato sob outras circunstâncias: investigaciones
\\Avaliações: \\
\begin{tabular}{|l|c|c|c|c|c|}
\hline
 & Excelente & Bom & Regular & Insuficiente & Não sabe \\
\hline
Desempenho acadêmico &  & X &  &  & \\
\hline
Capacidade de aprender novos conceitos & X &  &  &  & \\
\hline
Capacidade de trabalhar sozinho & X &  &  &  & \\
\hline
Criatividade & X &  &  &  & \\
\hline
Curiosidade & X &  &  &  & \\
\hline
Esforço, persistência & X &  &  &  & \\
\hline
Expressão escrita &  & X &  &  & \\
\hline
Expressão oral &  & X &  &  & \\
\hline
Relacionamento com colegas & X &  &  &  & \\
\hline
\end{tabular}\\
\\
\textbf{Opinião sobre os antecedentes acadêmicos, profissionais e/ou técnicos do candidato:}
\\Heikel Yervilla Herrera realizó una buena labor como estudiante del programa de Licenciatura en Ciencias de la Computación 20022007 en la Universidad Central Marta Abreu de Las Villas UCLV, se gradúa con un índice académico de 4,67 en base a 5. En esta etapa de pregrado se le aplicó un plan de estudio especial para estudiantes de alto aprovechamiento mediante el cual cursó asignaturas de aos superiores obteniendo también calificaciones excelentes y permitiéndole adquirir otros contenidos fuera del plan de estudio de la carrera y profundizar en temas como la visualización científica de datos provenientes de métodos numéricos en la ingeniería para el estudio de propiedades en simulaciones numéricas.
Ha formado parte del Centro de Investigación de Métodos Computacionales y Numéricos en la Ingeniería Aula CIMNE en Cuba desde su creación en el 2003. A este grupo se integró como estudiante y labora como investigador, allí participa en proyectos de impacto nacional e internacional tales como Desarrollo e implementación de tecnologías de avanzada para la modernización y automatización del sistema de inventario, diagnóstico, evaluación, mantenimiento y conservación de puentes de ferrocarriles., Desarrollo de herramientas computacionales avanzadas para su aplicación en la Ingeniería Estructural en zonas sísmicas de Cuba, entre otros. Durante esta etapa ha recibido evaluaciones excelentes por su desempeo en su trabajo como investigador y profesor de la UCLV.
Cursó la maestría en Ciencia de la Computación de la Universidad Central Marta Abreu de Las Villas, programa certificado de excelencia a nivel nacional y por la AUIP, defendió su tesis de master con el trabajo Algoritmos de pre y post proceso para métodos numéricos de puntos, métodos de partículas y libres de mallas mostrando una buena capacidad intelectual, motivación para estudios avanzados, capacidad de trabajo individual y colectivo y facilidad de expresión.
\\
\\
\textbf{Opinião sobre seu possível aproveitamento, se aceito no Programa:}
\\Considero que de ser aceptado en el programa se desarrollará satisfactoriamente todos los objetivos propuestos. El canddidato es una persona qcon capacidades de trabajo y de asimilación.\\ 
\\
\textbf{Outras informações relevantes:} \\
\\[0.3cm]
\textbf{Entre os estudantes que já conheceu, você diria que o candidato está entre os:}
\\
\begin{tabular}{|l|c|c|c|c|c|}
\hline
 & 5\% melhores & 10\% melhores & 25\% melhores & 50\% melhores & Não sabe \\
\hline
Como aluno, em aulas &  & X &  &  & \\
\hline
Como orientando & X &  &  &  & \\
\hline
\end{tabular}
\subsection*{Dados Recomendante} 
	Instituição (Institution): Universidad Central Marta Abreu de Las Villas
\\ 
	Grau acadêmico mais alto obtido: doutor
	\ \ Área: Programacion e Inteligencia Artificial
	\\
	Ano de obtenção deste grau: 1995
	\ \ 
	Instituição de obtenção deste grau : Universidad Central Marta Abreu de Las Villas
	\\ 
	Endereço institucional do recomendante: \\ Carretera a Camajuani, km 5, Santa Clara, Villa Clara, Cuba.\newpage\vspace*{-4cm}\subsection*{Carta de Recomendação - Maria Matilde Garcia Lorenzo}Código Identificador: 1418\\Conhece-o candidato há quanto tempo (For how long have you known the applicant)? 
\ 8 years
\\ Conhece-o sob as seguintes circunstâncias: aulas\ \ 
	\ \ seminarios\ \  
\\ Conheçe o candidato sob outras circunstâncias: 
\\Avaliações: \\
\begin{tabular}{|l|c|c|c|c|c|}
\hline
 & Excelente & Bom & Regular & Insuficiente & Não sabe \\
\hline
Desempenho acadêmico & X &  &  &  & \\
\hline
Capacidade de aprender novos conceitos & X &  &  &  & \\
\hline
Capacidade de trabalhar sozinho & X &  &  &  & \\
\hline
Criatividade & X &  &  &  & \\
\hline
Curiosidade & X &  &  &  & \\
\hline
Esforço, persistência &  & X &  &  & \\
\hline
Expressão escrita &  & X &  &  & \\
\hline
Expressão oral &  & X &  &  & \\
\hline
Relacionamento com colegas &  & X &  &  & \\
\hline
\end{tabular}\\
\\
\textbf{Opinião sobre os antecedentes acadêmicos, profissionais e/ou técnicos do candidato:}
\\MSc. Heikel Yervilla Herrera es graduado de Licenciado en Ciencias de la Computación en la Universidad Central Marta Abreu de Las Villas, es un excelente profesional con un nivel satisfactorio de cumplimiento de las actividades que se le asignan.

Durante su etapa como estudiante tanto en la licenciatura donde le impartí la asignatura de Inteligencia Artificial como en la maestría, de la cual soy coordinadora, mostró cualidades admirables entre las que se destacan la disciplina, madurez y responsabilidad, excelente capacidad de adaptación, iniciativa y motivación. Se gradúa, en pregrado, con un índice académico de 4.67 en base 5, un excelente rendimiento, destacándose entre los primeros 5 de su promoción. Desde el pregrado se integra activamente a grupos de investigación y se le aplicó un plan de estudios especial de alto aprovechamiento donde cursa asignaturas de grados superiores con excelentes resultados lo que le permite profundizar en otros estudios afines a su interés como estudiante investigador.

Como investigador, dentro de la universidad, forma parte de un grupo dedicado a las investigaciones en métodos numéricos aplicado a la ingeniería, donde se dedica al desarrollo de técnicas de visualización científicas, manejo de grandes volúmenes de datos y generación de mallas. Desarrolla tareas como parte de proyectos nacionales e internacionales y defiende el Master en Ciencia de la Computación de la Universidad Central Marta Abreu de Las Villas, programa certificado de excelencia a nivel nacional y por la AUIP, con la tesis Algoritmos de pre y post proceso para métodos numéricos de puntos, métodos de partículas y libres de mallas.

Asimismo, destacamos su capacidad académica e intelectual, sana competitividad, liderazgo y persistencia para cursar estudios superiores en una institución académica altamente reconocida. \\
\\
\textbf{Opinião sobre seu possível aproveitamento, se aceito no Programa:}
\\El candidato tiene  las actitudes y conocimientos requeridos para el programa de doctorado\\ 
\\
\textbf{Outras informações relevantes:} \\
\\[0.3cm]
\textbf{Entre os estudantes que já conheceu, você diria que o candidato está entre os:}
\\
\begin{tabular}{|l|c|c|c|c|c|}
\hline
 & 5\% melhores & 10\% melhores & 25\% melhores & 50\% melhores & Não sabe \\
\hline
Como aluno, em aulas & X &  &  &  & \\
\hline
Como orientando & X &  &  &  & \\
\hline
\end{tabular}
\subsection*{Dados Recomendante} 
	Instituição (Institution): Universidad Central Marta Abreu de Las Villas
\\ 
	Grau acadêmico mais alto obtido: doutor
	\ \ Área: Inteligencia Artificial
	\\
	Ano de obtenção deste grau: 1997
	\ \ 
	Instituição de obtenção deste grau : Universidad Central arta Abreu de Las Villas
	\\ 
	Endereço institucional do recomendante: \\ Carretera a Camajuani Km 5, Santa Clara Villa Clara, Cuba\includepdf[pages={-},offset=35mm 0mm]{../../../upload/1234_2014-05-30_documentos.pdf}\includepdf[pages={-},offset=35mm 0mm]{../../../upload/1234_2014-05-30_historico.pdf} 
\begin{center}
Anexos.
\end{center}
\end{document}