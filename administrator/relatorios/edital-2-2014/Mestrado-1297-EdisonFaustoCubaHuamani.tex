\documentclass[11pt]{article}
\usepackage{graphicx,color}
\usepackage{pdfpages}
\usepackage[brazil]{babel}
\usepackage[utf8]{inputenc}
\addtolength{\hoffset}{-3cm} \addtolength{\textwidth}{6cm}
\addtolength{\voffset}{-.5cm} \addtolength{\textheight}{1cm}
%%%%%%%%%%%%%%%%%%%%%%%%%%%%%%%%%%%  To use Colors 
\title{\vspace*{-4cm} Ficha de Inscrição: \\Cod: 1297\ \ edison fausto  Cuba Huamani\ \ - \ \ Mestrado 
 }
\date{}

\begin{document}
\maketitle
\vspace*{-1.5cm}
\noindent Data de Nascimento:3/11/1991
\ \ \ Idade: 22   \ \ \ Sexo: Masculino
\\
Naturalidade: Arequipa  
\ \ \  Estado: PE
\ \ \  Nacionalidade: Peruana
\ \ \ País: Peru
\\        
Nome do pai : Fausto Cuba Ccoa
\ \ \ Nome da mãe: Sulma Huamani Ugarte          
\\[0.2cm]                     
\textbf{Endereço Pessoal} 
\\ 
\noindent Endereço residencial: Av. Independencia Americana Mz.47 Lt.7
\\
        CEP: 04012-040 
\ \ \ Cidade: Arequipa 
\ \ \ Estado: PE 
\ \ \ País: Peru
\\		
		Telefone comercial : +0(51)496626
\ \ \ Telefone residencial: +0(51)496626
\ \ \ Telefone celular : +0(51)940239803
\\
E-mail principal: edisonfausto@gmail.com
\ \ \ E-mail alternativo: navegante\_14\_93@hotmail.com 
\\[0.2cm] 
\textbf{Documentos Pessoais}
\\
\noindent Número de CPF : 0
\ \ \ Número de Identidade (ou Passaporte para estrangeiros): 6347100
\\
Orgão emissor: Peru
\ \ \ Estado: PE
\ \ \ Data de emissão :16/4/2012
\\[0.3cm]
\textbf{Grau acadêmico mais alto obtido}
\\	
Curso:Matemáticas
\ \ \ Grau : bacharel
\ \ \ Instituição : Universidad Nacional de San Agustin
\\			
Ano de Conclusão ou Previsão: 2013
\\ 
Experiência Profissional mais recente. \ \  
Tem experiência: 0 0  
\ \ \ Instituição: 0
\\  
Período - início: 0-0
\ \ \ fim: 0-0
\\[0.2cm] 
\textbf{Programa Pretendido:} Mestrado\\
Interesse em bolsa: Sim
\\[0.3cm]		
\textbf{Dados dos Recomendantes} 
\\
1- Nome: Walter Torres Montes
\ \ \ \  e-mail: wtorresm@unsa.edu.pe 
\\
2- Nome: Vladimir Alfonso Rosas Meneses
\ \ \ \ e-mail: vrosas@unsa.edu.pe
\\
3- Nome: Angel Carlos Wilfredo Sangiacomo Carazas
\ \ \ \ e-mail: asangiacomo@unsa.edu.pe
\\[0.2cm]
Motivação e expectativa do candidato em relação ao programa pretendido:
\\Primeramente, escogí inscribirme en la Universidade de Brasília por el nivel que se encuentra en sus cursos de postgraduación de acuerdo a la clasificación del programa CAPES ya que se encuentra en la más alta calificación la cual es el puntaje 7, donde pocas universidades obtuvieron tal puntuación.

Segundo, porque varios profesores de la universidad donde estudie la mencionaron con gran afecto y admiración.

Tercero, la mayoría de los profesores con los que lleve cursos en pregrado me recomendaron la UNB porque tiene un muy bien nivel en lo que concierne a la carrera que estudié, la cual es la Escuela Profesional de Matemáticas.

Otra de las razones por la que decidí postular a esta casa de estudios es porque varios chicos que hicieron y continúan haciendo su Maestría en Brasil me recomendaron esta Universidad porque es una de las mejores hoy por hoy en Sudamérica.

Por último, elegí la UNB para hacer mi Maestría, porque una de mis metas que me trace fue hacer mi postgrado en el extranjero y por todas las sugerencias que recibí anteriormente, tome la decisión de postular a la Universidade de Brasília.\newpage\vspace*{-4cm}\subsection*{Carta de Recomendação - Walter Torres Montes}Código Identificador: 1381\\Conhece-o candidato há quanto tempo (For how long have you known the applicant)? 
\ 6 years
\\ Conhece-o sob as seguintes circunstâncias: aulas\ \ 
	\ \ \ \  
\\ Conheçe o candidato sob outras circunstâncias: 
\\	Avaliações:\\
\begin{tabular}{|l|c|c|c|c|c|}
\hline
 & Excelente & Bom & Regular & Insuficiente & Não sabe \\
\hline
Desempenho acadêmico & X &  &  &  & \\
\hline
Capacidade de aprender novos conceitos & X &  &  &  & \\
\hline
Capacidade de trabalhar sozinho & X &  &  &  & \\
\hline
Criatividade & X &  &  &  & \\
\hline
Curiosidade & X &  &  &  & \\
\hline
Esforço, persistência & X &  &  &  & \\
\hline
Expressão escrita & X &  &  &  & \\
\hline
Expressão oral & X &  &  &  & \\
\hline
Relacionamento com colegas & X &  &  &  & \\
\hline
\end{tabular}\\
\\
\textbf{Opinião sobre os antecedentes acadêmicos, profissionais e/ou técnicos do candidato:}
\\Ha sido mi alumno en las asignaturas de Álgebra y Análisis Real.
En ambas asignaturas ha obtenido la nota de 18, lo que significa que es un excelente alumno.\\
\\
\textbf{Opinião sobre seu possível aproveitamento, se aceito no Programa:}
\\Habiendo obtenido buenas calificaciones en el pregrado de Matematicas de la Universidad Nacional de San Agustín, estoy seguro que abordara con exito los estudios en Maestría en Matemáticas.\\ 
\\
\textbf{Outras informações relevantes:} \\Durante el desarrollo de mis clases acostumbro hacer preguntas y Edison era el que mejor respondia lo que para mi significaba que tiene talento para las Matematicas.
\\[0.3cm]
\textbf{Entre os estudantes que já conheceu, você diria que o candidato está entre os:}
\\
\begin{tabular}{|l|c|c|c|c|c|}
\hline
 & 5\% melhores & 10\% melhores & 25\% melhores & 50\% melhores & Não sabe \\
\hline
Como aluno, em aulas & X &  &  &  & \\
\hline
Como orientando & X &  &  &  & \\
\hline
\end{tabular}
\subsection*{Dados Recomendante} 
	Instituição (Institution): Universidad Nacional San Agustin de Arequipa
\\ 
	Grau acadêmico mais alto obtido: licenciado
	\ \ Área: Análisis Funcional
	\\
	Ano de obtenção deste grau: 1962
	\ \ 
	Instituição de obtenção deste grau : Universidad Nacional de Educación, La Cantuta,Lima
	\\ 
	Endereço institucional do recomendante: \\ Casilla 1304, Serpost, Arequipa,Peru\newpage\vspace*{-4cm}\subsection*{Carta de Recomendação - Vladimir Alfonso Rosas Meneses}Código Identificador: 1382\\Conhece-o candidato há quanto tempo (For how long have you known the applicant)? 
\ four years
\\ Conhece-o sob as seguintes circunstâncias: aulas\ \ orientacao
	\ \ \ \  
\\ Conheçe o candidato sob outras circunstâncias: 
\\Avaliações: \\
\begin{tabular}{|l|c|c|c|c|c|}
\hline
 & Excelente & Bom & Regular & Insuficiente & Não sabe \\
\hline
Desempenho acadêmico &  & X &  &  & \\
\hline
Capacidade de aprender novos conceitos & X &  &  &  & \\
\hline
Capacidade de trabalhar sozinho & X &  &  &  & \\
\hline
Criatividade & X &  &  &  & \\
\hline
Curiosidade &  & X &  &  & \\
\hline
Esforço, persistência &  & X &  &  & \\
\hline
Expressão escrita &  & X &  &  & \\
\hline
Expressão oral &  & X &  &  & \\
\hline
Relacionamento com colegas & X &  &  &  & \\
\hline
\end{tabular}\\
\\
\textbf{Opinião sobre os antecedentes acadêmicos, profissionais e/ou técnicos do candidato:}
\\Se que el candidato ha tenido un desenpenho aceptable durante su estadía en el pregrado en matemáticas\\
\\
\textbf{Opinião sobre seu possível aproveitamento, se aceito no Programa:}
\\El candidato no debe tener problemas en caso de ser aceptado una vez que demostro ser buen estudiante en el pregado. \\ 
\\
\textbf{Outras informações relevantes:} \\Actualmente soy orientador del candidato y él está trabajando sobre el espacio hiperbólico donde tiene buen desempenho.
Dedo indicar que Edison tiene excelentes aptitudes para la asimilacion de nuevos conocimientos.
\\[0.3cm]
\textbf{Entre os estudantes que já conheceu, você diria que o candidato está entre os:}
\\
\begin{tabular}{|l|c|c|c|c|c|}
\hline
 & 5\% melhores & 10\% melhores & 25\% melhores & 50\% melhores & Não sabe \\
\hline
Como aluno, em aulas &  & X &  &  & \\
\hline
Como orientando &  & X &  &  & \\
\hline
\end{tabular}
\subsection*{Dados Recomendante} 
	Instituição (Institution): Universidad Nacional de San Agustín, Perú
\\ 
	Grau acadêmico mais alto obtido: doutor
	\ \ Área: Geometria
	\\
	Ano de obtenção deste grau: 2000
	\ \ 
	Instituição de obtenção deste grau : PUC, Rio
	\\ 
	Endereço institucional do recomendante: \\ Universidad Nacional de San Agustin.
Calle Santa Catalina Nro 117
Arequipa Perú\newpage\vspace*{-4cm}\subsection*{Carta de Recomendação - Angel Carlos Wilfredo Sangiacomo Carazas}Código Identificador: 1383\\Conhece-o candidato há quanto tempo (For how long have you known the applicant)? 
\ 5 years
\\ Conhece-o sob as seguintes circunstâncias: aulas\ \ 
	\ \ \ \  
\\ Conheçe o candidato sob outras circunstâncias: 
\\Avaliações: \\
\begin{tabular}{|l|c|c|c|c|c|}
\hline
 & Excelente & Bom & Regular & Insuficiente & Não sabe \\
\hline
Desempenho acadêmico &  & X &  &  & \\
\hline
Capacidade de aprender novos conceitos & X &  &  &  & \\
\hline
Capacidade de trabalhar sozinho & X &  &  &  & \\
\hline
Criatividade & X &  &  &  & \\
\hline
Curiosidade &  & X &  &  & \\
\hline
Esforço, persistência & X &  &  &  & \\
\hline
Expressão escrita &  & X &  &  & \\
\hline
Expressão oral &  & X &  &  & \\
\hline
Relacionamento com colegas & X &  &  &  & \\
\hline
\end{tabular}\\
\\
\textbf{Opinião sobre os antecedentes acadêmicos, profissionais e/ou técnicos do candidato:}
\\A Edison lo conozco como estudiante en el curso de ecuaciones diferenciales en la Escuela Profesional de matemática, de 30 alumnos esta entre los 4 primeros.\\
\\
\textbf{Opinião sobre seu possível aproveitamento, se aceito no Programa:}
\\En cuanto a estudio sera bueno, pues, es muy persistente y es interesado en terminar lo que empieza.
Se que pondrá de su parte pues, se interesa en terminar lo que empieza y 
si es aceptado tendrán un nuevo maestro en matemática.\\ 
\\
\textbf{Outras informações relevantes:} \\Edison es un estudiante inquieto, conversador, expresivo, y le gusta los temas nuevos.
\\[0.3cm]
\textbf{Entre os estudantes que já conheceu, você diria que o candidato está entre os:}
\\
\begin{tabular}{|l|c|c|c|c|c|}
\hline
 & 5\% melhores & 10\% melhores & 25\% melhores & 50\% melhores & Não sabe \\
\hline
Como aluno, em aulas & X &  &  &  & \\
\hline
Como orientando & X &  &  &  & \\
\hline
\end{tabular}
\subsection*{Dados Recomendante} 
	Instituição (Institution): Universidad Nacional de San Agustín, Arequipa, PERU
\\ 
	Grau acadêmico mais alto obtido: doutor
	\ \ Área: Matemática y Educación
	\\
	Ano de obtenção deste grau: 2007
	\ \ 
	Instituição de obtenção deste grau : Universidad Nacional de San Agustín, Arequipa, PERU
	\\ 
	Endereço institucional do recomendante: \\ Santa  Catalina 117   cercado
Arequipa PERU\includepdf[pages={-},offset=35mm 0mm]{../../../upload/1297_2014-05-28_documentos.pdf}\includepdf[pages={-},offset=35mm 0mm]{../../../upload/1297_2014-05-28_historico.pdf} 
\begin{center}
Anexos.
\end{center}
\end{document}