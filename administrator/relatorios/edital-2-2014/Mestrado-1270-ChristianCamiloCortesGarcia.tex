\documentclass[11pt]{article}
\usepackage{graphicx,color}
\usepackage{pdfpages}
\usepackage[brazil]{babel}
\usepackage[utf8]{inputenc}
\addtolength{\hoffset}{-3cm} \addtolength{\textwidth}{6cm}
\addtolength{\voffset}{-.5cm} \addtolength{\textheight}{1cm}
%%%%%%%%%%%%%%%%%%%%%%%%%%%%%%%%%%%  To use Colors 
\title{\vspace*{-4cm} Ficha de Inscrição: \\Cod: 1270\ \ Christian Camilo Cortes Garcia\ \ - \ \ Mestrado 
 }
\date{}

\begin{document}
\maketitle
\vspace*{-1.5cm}
\noindent Data de Nascimento:5/5/1992
\ \ \ Idade: 22   \ \ \ Sexo: Masculino
\\
Naturalidade: Neiva  
\ \ \  Estado: Outro
\ \ \  Nacionalidade: Colombiana
\ \ \ País: Colombia
\\        
Nome do pai : Dagoberto Cortes Bahamon
\ \ \ Nome da mãe: Amparo Garcia Fierro          
\\[0.2cm]                     
\textbf{Endereço Pessoal} 
\\ 
\noindent Endereço residencial: Calle 1 C BIS 27 59
\\
        CEP: 41000-100 
\ \ \ Cidade: Neiva 
\ \ \ Estado: Outro 
\ \ \ País: Colombia
\\		
		Telefone comercial : +57(1)8600104
\ \ \ Telefone residencial: +57(1)8600104
\ \ \ Telefone celular : +57(1)321347473
\\
E-mail principal: chriscort05@gmail.com
\ \ \ E-mail alternativo: chriscort05@hotmail.com 
\\[0.2cm] 
\textbf{Documentos Pessoais}
\\
\noindent Número de CPF : 0
\ \ \ Número de Identidade (ou Passaporte para estrangeiros): 1075261911
\\
Orgão emissor: Cedula
\ \ \ Estado: Outro
\ \ \ Data de emissão :18/5/2010
\\[0.3cm]
\textbf{Grau acadêmico mais alto obtido}
\\	
Curso:Matematica Aplicada
\ \ \ Grau : licenciado
\ \ \ Instituição : Universidad Surcolombiana
\\			
Ano de Conclusão ou Previsão: 2014
\\ 
Experiência Profissional mais recente. \ \  
Tem experiência: Docente 0  
\ \ \ Instituição: Universidad Surcolombiana
\\  
Período - início: 2-2011
\ \ \ fim: 2-2012
\\[0.2cm] 
\textbf{Programa Pretendido:} Mestrado\\
Interesse em bolsa: Sim
\\[0.3cm]		
\textbf{Dados dos Recomendantes} 
\\
1- Nome: Mauro Montealegre Cardenas
\ \ \ \  e-mail: mmonteal@usco.edu.co 
\\
2- Nome: Luis Arturo Polania Quiza
\ \ \ \ e-mail: lapola@usco.edu.co
\\
3- Nome: Ronaldo Garcia
\ \ \ \ e-mail: ragarcia@mat.ufg.br
\\[0.2cm]
Motivação e expectativa do candidato em relação ao programa pretendido:
\\La Matemática Aplicada se utiliza en casi todas las áreas del conocimiento científico y tecnológico, el cual dicho desarrollo se basa, en gran parte, en la aplicación del conocimiento matemático en la solución de problemas y toma de decisiones asociados a fenómenos ocurridos en la vida actual y que se ilustran a través de modelos matemáticos. Mi intención en realizar la Maestría en Matemática en la Universidad de Brasilia está encaminada a profundizar en los conocimientos de las matemáticas, en particular en el Análisis Matemático, y aplicar dichos conocimientos a la Teoría de los Sistemas Dinámicos porque es interesante que a través de ellos se pueda estudiar la evolución de fenómenos de las ciencias naturales, sociales y así poder controlar yo predecir el comportamiento a futuro de dichos fenómenos.\newpage\vspace*{-4cm}\subsection*{Carta de Recomendação - Mauro Montealegre Cardenas}Código Identificador: 1371\\Conhece-o candidato há quanto tempo (For how long have you known the applicant)? 
\ 5 anos
\\ Conhece-o sob as seguintes circunstâncias: aulas\ \ orientacao
	\ \ seminarios\ \  
\\ Conheçe o candidato sob outras circunstâncias: 
\\	Avaliações:\\
\begin{tabular}{|l|c|c|c|c|c|}
\hline
 & Excelente & Bom & Regular & Insuficiente & Não sabe \\
\hline
Desempenho acadêmico & X &  &  &  & \\
\hline
Capacidade de aprender novos conceitos & X &  &  &  & \\
\hline
Capacidade de trabalhar sozinho & X &  &  &  & \\
\hline
Criatividade & X &  &  &  & \\
\hline
Curiosidade & X &  &  &  & \\
\hline
Esforço, persistência & X &  &  &  & \\
\hline
Expressão escrita & X &  &  &  & \\
\hline
Expressão oral & X &  &  &  & \\
\hline
Relacionamento com colegas & X &  &  &  & \\
\hline
\end{tabular}\\
\\
\textbf{Opinião sobre os antecedentes acadêmicos, profissionais e/ou técnicos do candidato:}
\\Christian  Camilo  Cortes  García  obtuvo  las  más  altas  calificaciones
en  el  programa  de  Matemáticas de  la  Universidad  Surcolombiana, dirigí  su  trabajo  de  grado  que  obtuvo  el mejor  reconocimiento de la banca  de  evaluadora.\\
\\
\textbf{Opinião sobre seu possível aproveitamento, se aceito no Programa:}
\\Chistian  tiene  todas  las  condiciones  para  obtener  buenos  resultados en  el  programa  de  maestría al  que  aspira  en  la  Universidad de Brasilia.\\ 
\\
\textbf{Outras informações relevantes:} \\Chistian es  joven y  esta  muy  estusiasmado para cotinuar  sus  estudios avanzados.
\\[0.3cm]
\textbf{Entre os estudantes que já conheceu, você diria que o candidato está entre os:}
\\
\begin{tabular}{|l|c|c|c|c|c|}
\hline
 & 5\% melhores & 10\% melhores & 25\% melhores & 50\% melhores & Não sabe \\
\hline
Como aluno, em aulas & X &  &  &  & \\
\hline
Como orientando & X &  &  &  & \\
\hline
\end{tabular}
\subsection*{Dados Recomendante} 
	Instituição (Institution): Universidad  Surcolombiana
\\ 
	Grau acadêmico mais alto obtido: doutor
	\ \ Área: Sistemas  Dinámicos
	\\
	Ano de obtenção deste grau: 1996
	\ \ 
	Instituição de obtenção deste grau : IMEUSP
	\\ 
	Endereço institucional do recomendante: \\ Rua  Pastrana 1  Neiva\newpage\vspace*{-4cm}\subsection*{Carta de Recomendação - Luis Arturo Polania Quiza}Código Identificador: 1372\\Conhece-o candidato há quanto tempo (For how long have you known the applicant)? 
\ cuatro
\\ Conhece-o sob as seguintes circunstâncias: aulas\ \ 
	\ \ \ \  
\\ Conheçe o candidato sob outras circunstâncias: 
\\Avaliações: \\
\begin{tabular}{|l|c|c|c|c|c|}
\hline
 & Excelente & Bom & Regular & Insuficiente & Não sabe \\
\hline
Desempenho acadêmico &  & X &  &  & \\
\hline
Capacidade de aprender novos conceitos &  & X &  &  & \\
\hline
Capacidade de trabalhar sozinho &  & X &  &  & \\
\hline
Criatividade &  & X &  &  & \\
\hline
Curiosidade &  & X &  &  & \\
\hline
Esforço, persistência &  & X &  &  & \\
\hline
Expressão escrita &  & X &  &  & \\
\hline
Expressão oral &  & X &  &  & \\
\hline
Relacionamento com colegas & X &  &  &  & \\
\hline
\end{tabular}\\
\\
\textbf{Opinião sobre os antecedentes acadêmicos, profissionais e/ou técnicos do candidato:}
\\En los cursos que le oriente al joven Christian Camilo Cortes mostró interés por comprender el porque de un conceptos, el porque de un resultado y el como aplicarlos dentro de la misma matemática, en otras áreas del conocimiento así como también usarlos en la solución de problemas de vida real.\\
\\
\textbf{Opinião sobre seu possível aproveitamento, se aceito no Programa:}
\\El joven Christian por su dedicación y perseverancia lo mas probable es que el tenga un buen desempeno académico e investigación en los cursos necesarios que deba tomar para la realización de una buena tesis de maestría.\\ 
\\
\textbf{Outras informações relevantes:} \\El candidato elaboro un trabajo de grado en la linea de teoría de bifurcaciones teniendo una evaluación muy buena en el momento de su sustentación, el cual fue mostrado en un evento nacional.
\\[0.3cm]
\textbf{Entre os estudantes que já conheceu, você diria que o candidato está entre os:}
\\
\begin{tabular}{|l|c|c|c|c|c|}
\hline
 & 5\% melhores & 10\% melhores & 25\% melhores & 50\% melhores & Não sabe \\
\hline
Como aluno, em aulas & X &  &  &  & \\
\hline
Como orientando & X &  &  &  & \\
\hline
\end{tabular}
\subsection*{Dados Recomendante} 
	Instituição (Institution): Universidad Surcolombiana
\\ 
	Grau acadêmico mais alto obtido: mestre
	\ \ Área: Matematica
	\\
	Ano de obtenção deste grau: 1995
	\ \ 
	Instituição de obtenção deste grau : Universidad Surcolombiana
	\\ 
	Endereço institucional do recomendante: \\ Avenida Pastrana Calle 1\newpage\vspace*{-4cm}\subsection*{Carta de Recomendação - Ronaldo Garcia}Código Identificador: 1373\\Conhece-o candidato há quanto tempo (For how long have you known the applicant)? 
\ 
\\ Conhece-o sob as seguintes circunstâncias: \ \ 
	\ \ \ \  
\\ Conheçe o candidato sob outras circunstâncias: 
\\Avaliações: \\
\begin{tabular}{|l|c|c|c|c|c|}
\hline
 & Excelente & Bom & Regular & Insuficiente & Não sabe \\
\hline
Desempenho acadêmico &  &  &  &  & \\
\hline
Capacidade de aprender novos conceitos &  &  &  &  & \\
\hline
Capacidade de trabalhar sozinho &  &  &  &  & \\
\hline
Criatividade &  &  &  &  & \\
\hline
Curiosidade &  &  &  &  & \\
\hline
Esforço, persistência &  &  &  &  & \\
\hline
Expressão escrita &  &  &  &  & \\
\hline
Expressão oral &  &  &  &  & \\
\hline
Relacionamento com colegas &  &  &  &  & \\
\hline
\end{tabular}\\
\\
\textbf{Opinião sobre os antecedentes acadêmicos, profissionais e/ou técnicos do candidato:}
\\\\
\\
\textbf{Opinião sobre seu possível aproveitamento, se aceito no Programa:}
\\\\ 
\\
\textbf{Outras informações relevantes:} \\
\\[0.3cm]
\textbf{Entre os estudantes que já conheceu, você diria que o candidato está entre os:}
\\
\begin{tabular}{|l|c|c|c|c|c|}
\hline
 & 5\% melhores & 10\% melhores & 25\% melhores & 50\% melhores & Não sabe \\
\hline
Como aluno, em aulas &  &  &  &  & \\
\hline
Como orientando &  &  &  &  & \\
\hline
\end{tabular}
\subsection*{Dados Recomendante} 
	Instituição (Institution): 
\\ 
	Grau acadêmico mais alto obtido: 
	\ \ Área: 
	\\
	Ano de obtenção deste grau: 
	\ \ 
	Instituição de obtenção deste grau : 
	\\ 
	Endereço institucional do recomendante: \\ \includepdf[pages={-},offset=35mm 0mm]{../../../upload/1270_2014-05-28_documentos.pdf}\includepdf[pages={-},offset=35mm 0mm]{../../../upload/1270_2014-05-28_historico.pdf} 
\begin{center}
Anexos.
\end{center}
\end{document}