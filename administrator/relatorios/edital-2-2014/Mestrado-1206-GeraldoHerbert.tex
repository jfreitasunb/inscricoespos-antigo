\documentclass[11pt]{article}
\usepackage{graphicx,color}
\usepackage{pdfpages}
\usepackage[brazil]{babel}
\usepackage[utf8]{inputenc}
\addtolength{\hoffset}{-3cm} \addtolength{\textwidth}{6cm}
\addtolength{\voffset}{-.5cm} \addtolength{\textheight}{1cm}
%%%%%%%%%%%%%%%%%%%%%%%%%%%%%%%%%%%  To use Colors 
\title{\vspace*{-4cm} Ficha de Inscrição: \\Cod: 1206\ \ Geraldo Herbert\ \ - \ \ Mestrado 
 }
\date{}

\begin{document}
\maketitle
\vspace*{-1.5cm}
\noindent Data de Nascimento:2/7/1991
\ \ \ Idade: 23   \ \ \ Sexo: Masculino
\\
Naturalidade: Amazonense  
\ \ \  Estado: AM
\ \ \  Nacionalidade: Brasileiro
\ \ \ País: Brasil
\\        
Nome do pai : Geraldo Linhares de Souza
\ \ \ Nome da mãe: Ivany Beltrão de Souza          
\\[0.2cm]                     
\textbf{Endereço Pessoal} 
\\ 
\noindent Endereço residencial: Clarindo de Queiroz
\\
        CEP: 69079-080 
\ \ \ Cidade: Manaus 
\ \ \ Estado: AM 
\ \ \ País: Brasil
\\		
		Telefone comercial : +0(92)93158592
\ \ \ Telefone residencial: +0(92)92231052
\ \ \ Telefone celular : +0(92)84327291
\\
E-mail principal: geraldo.herbert@yahoo.com.br
\ \ \ E-mail alternativo: herbertmatematica@gmail.com 
\\[0.2cm] 
\textbf{Documentos Pessoais}
\\
\noindent Número de CPF : 01556883242
\ \ \ Número de Identidade (ou Passaporte para estrangeiros): 23940964
\\
Orgão emissor: PAC
\ \ \ Estado: AM
\ \ \ Data de emissão :8/11/2006
\\[0.3cm]
\textbf{Grau acadêmico mais alto obtido}
\\	
Curso:Matemática
\ \ \ Grau : licenciado
\ \ \ Instituição : Universidade do Estado do Amazonas
\\			
Ano de Conclusão ou Previsão: 2014
\\ 
Experiência Profissional mais recente. \ \  
Tem experiência: Docente Discente  
\ \ \ Instituição: Universidade do Estado do Amazonas
\\  
Período - início: 1-2010
\ \ \ fim: 1-2014
\\[0.2cm] 
\textbf{Programa Pretendido:} Mestrado\\
Interesse em bolsa: Sim
\\[0.3cm]		
\textbf{Dados dos Recomendantes} 
\\
1- Nome: Marcus Antônio Mendonça Marrocos
\ \ \ \  e-mail: marcusmarrocos@gmail.com 
\\
2- Nome: Flávia Morgana de Oliveira Jacinto
\ \ \ \ e-mail: flavia.jacinto@gmail.com
\\
3- Nome: Sandro Dimy Barbosa Bitar
\ \ \ \ e-mail: sandrobitar@gmail.com
\\[0.2cm]
Motivação e expectativa do candidato em relação ao programa pretendido:
\\Minha maior motivação é fato de a UnB ser uma das poucas instituições cujo programa de pósgraduação em matemática tem conceito 7 pela capes e também pela concentração de análise, pois conta com um bom número de pesquisadores nessa área.\newpage\vspace*{-4cm}\subsection*{Carta de Recomendação - Marcus Antônio Mendonça Marrocos}Código Identificador: 1301\\Conhece-o candidato há quanto tempo (For how long have you known the applicant)? 
\ 2 anos
\\ Conhece-o sob as seguintes circunstâncias: aulas\ \ orientacao
	\ \ \ \  
\\ Conheçe o candidato sob outras circunstâncias: 
\\	Avaliações:\\
\begin{tabular}{|l|c|c|c|c|c|}
\hline
 & Excelente & Bom & Regular & Insuficiente & Não sabe \\
\hline
Desempenho acadêmico &  & X &  &  & \\
\hline
Capacidade de aprender novos conceitos & X &  &  &  & \\
\hline
Capacidade de trabalhar sozinho & X &  &  &  & \\
\hline
Criatividade &  & X &  &  & \\
\hline
Curiosidade & X &  &  &  & \\
\hline
Esforço, persistência & X &  &  &  & \\
\hline
Expressão escrita & X &  &  &  & \\
\hline
Expressão oral &  & X &  &  & \\
\hline
Relacionamento com colegas &  & X &  &  & \\
\hline
\end{tabular}\\
\\
\textbf{Opinião sobre os antecedentes acadêmicos, profissionais e/ou técnicos do candidato:}
\\Sempre muito responsável dedicado e no curso em que fui seu professor esteve sempre entre os melhores alunos. A disciplina e determinação são certamente algumas de suas qualidades. Mesmo cursando licenciatura sempre apresentou talento natural no entendimento de conceitos abstratos.\\
\\
\textbf{Opinião sobre seu possível aproveitamento, se aceito no Programa:}
\\Apesar de apresentar algumas lacunas na formação durante a graduação, por conta da universidade em que cursou licencuiatura em matemática ao ter disciplinas do bacahrelado, acredito que essas lacunas não representarão obstáculos para o seu desempenhos nas disciplinas.\\ 
\\
\textbf{Outras informações relevantes:} \\Se expressa muito bem tanto escrita quanto oralmente e trabalha muito bem em grupo.
\\[0.3cm]
\textbf{Entre os estudantes que já conheceu, você diria que o candidato está entre os:}
\\
\begin{tabular}{|l|c|c|c|c|c|}
\hline
 & 5\% melhores & 10\% melhores & 25\% melhores & 50\% melhores & Não sabe \\
\hline
Como aluno, em aulas & X &  &  &  & \\
\hline
Como orientando & X &  &  &  & \\
\hline
\end{tabular}
\subsection*{Dados Recomendante} 
	Instituição (Institution): Universidade Federal do Amazonas
\\ 
	Grau acadêmico mais alto obtido: doutor
	\ \ Área: Equações diferenciais parciais
	\\
	Ano de obtenção deste grau: 2011
	\ \ 
	Instituição de obtenção deste grau : USP
	\\ 
	Endereço institucional do recomendante: \\ Endereço Av. General Rodrigo  
 Octávio, 6200, Coroado I 
 Cep 69077000
Manaus, AM\newpage\vspace*{-4cm}\subsection*{Carta de Recomendação - Flávia Morgana de Oliveira Jacinto}Código Identificador: 1302\\Conhece-o candidato há quanto tempo (For how long have you known the applicant)? 
\ 2 anos
\\ Conhece-o sob as seguintes circunstâncias: \ \ orientacao
	\ \ seminarios\ \  
\\ Conheçe o candidato sob outras circunstâncias: 
\\Avaliações: \\
\begin{tabular}{|l|c|c|c|c|c|}
\hline
 & Excelente & Bom & Regular & Insuficiente & Não sabe \\
\hline
Desempenho acadêmico &  & X &  &  & \\
\hline
Capacidade de aprender novos conceitos &  & X &  &  & \\
\hline
Capacidade de trabalhar sozinho & X &  &  &  & \\
\hline
Criatividade & X &  &  &  & \\
\hline
Curiosidade & X &  &  &  & \\
\hline
Esforço, persistência & X &  &  &  & \\
\hline
Expressão escrita &  & X &  &  & \\
\hline
Expressão oral &  & X &  &  & \\
\hline
Relacionamento com colegas &  & X &  &  & \\
\hline
\end{tabular}\\
\\
\textbf{Opinião sobre os antecedentes acadêmicos, profissionais e/ou técnicos do candidato:}
\\O candidato foi meu orientando de Iniciação Científica no Projeto Integrando a Amazônia. Ele foi aluno de graduação da UEA e nos procurou na UFAM para participar dos nossos cursos de nivelamento para o Mestrado do nosso programa PPGM. Mediante se ue grande interesse e bom desempenho no curso, atuando como ouvinte, o convidamos a participar do projeto. Acrescento que sempre desenvolveu todas as atividades previstas com muita dedicação, pontualidade e esmero.\\
\\
\textbf{Opinião sobre seu possível aproveitamento, se aceito no Programa:}
\\O candidato é muito interessado e persistente. Procura atender a tudo no que é solicitado com dedicaçcão e presteza. É perseverante e tem uma boa capacidade de aprender e manusear novos conceitos. Procura sempre dar novas soluções aos exercícios e prepara os seminários com muita clareza e empenho, apesar de que a falta de maturidade o faça pecar no rigor por vezes. Acredito qu,e caso seja aceito, desenvolverá seu curso de maneira muito satisfatória. \\ 
\\
\textbf{Outras informações relevantes:} \\Ele como aluno do projeto, participou de dois cursos de Verão, um na UFC e o outro no IMPA, e em ambos obteve bom desempenho, apesar de não ter sido selecionado para continuar no Mestrado.
\\[0.3cm]
\textbf{Entre os estudantes que já conheceu, você diria que o candidato está entre os:}
\\
\begin{tabular}{|l|c|c|c|c|c|}
\hline
 & 5\% melhores & 10\% melhores & 25\% melhores & 50\% melhores & Não sabe \\
\hline
Como aluno, em aulas &  & X &  &  & \\
\hline
Como orientando & X &  &  &  & \\
\hline
\end{tabular}
\subsection*{Dados Recomendante} 
	Instituição (Institution): Universidade Federal do Amazonas
\\ 
	Grau acadêmico mais alto obtido: doutor
	\ \ Área: Matemáica Aplicada em Otimização Contínua 
	\\
	Ano de obtenção deste grau: 2007
	\ \ 
	Instituição de obtenção deste grau : PESC, COPPE, UFRJ
	\\ 
	Endereço institucional do recomendante: \\ Avenida Gal Rodrigo Otávio, 6200, Campus Universitário da UFAM, Departamento de Matemática, CEP  69077000, Manaus, Amazonsa.\newpage\vspace*{-4cm}\subsection*{Carta de Recomendação - Sandro Dimy Barbosa Bitar}Código Identificador: 1303\\Conhece-o candidato há quanto tempo (For how long have you known the applicant)? 
\ 3 anos
\\ Conhece-o sob as seguintes circunstâncias: aulas\ \ 
	\ \ \ \ outra 
\\ Conheçe o candidato sob outras circunstâncias: Discussões sobre temas avançados
\\Avaliações: \\
\begin{tabular}{|l|c|c|c|c|c|}
\hline
 & Excelente & Bom & Regular & Insuficiente & Não sabe \\
\hline
Desempenho acadêmico &  & X &  &  & \\
\hline
Capacidade de aprender novos conceitos &  & X &  &  & \\
\hline
Capacidade de trabalhar sozinho &  & X &  &  & \\
\hline
Criatividade &  & X &  &  & \\
\hline
Curiosidade &  & X &  &  & \\
\hline
Esforço, persistência & X &  &  &  & \\
\hline
Expressão escrita &  & X &  &  & \\
\hline
Expressão oral &  &  &  &  & X\\
\hline
Relacionamento com colegas &  &  &  &  & X\\
\hline
\end{tabular}\\
\\
\textbf{Opinião sobre os antecedentes acadêmicos, profissionais e/ou técnicos do candidato:}
\\O candidato realizou sua graduação em outra instituição de ensino. Interessado em avançar nos seus estudos, resolveu assistir algumas aulas como aluno ouvinte. Com essa estratégia conseguiu melhorar sua formação, demonstrando muito interesse pela matemática, ao mesmo tempo que deixou claro a capacidade para tal.\\
\\
\textbf{Opinião sobre seu possível aproveitamento, se aceito no Programa:}
\\Acredito que o mestrado lhe renderá ótima experiência. Será um divisor na sua formação. Acredito no sucesso desse mestrado com base no conhecimento que disponho desse candidato. É responsável e determinado. São características essenciais que carecem em nossos alunos.\\ 
\\
\textbf{Outras informações relevantes:} \\
\\[0.3cm]
\textbf{Entre os estudantes que já conheceu, você diria que o candidato está entre os:}
\\
\begin{tabular}{|l|c|c|c|c|c|}
\hline
 & 5\% melhores & 10\% melhores & 25\% melhores & 50\% melhores & Não sabe \\
\hline
Como aluno, em aulas &  & X &  &  & \\
\hline
Como orientando &  & X &  &  & \\
\hline
\end{tabular}
\subsection*{Dados Recomendante} 
	Instituição (Institution): UNIVERSIDADE FEDERAL DO AMAZONAS
\\ 
	Grau acadêmico mais alto obtido: doutor
	\ \ Área: Matemática Aplicada
	\\
	Ano de obtenção deste grau: 2009
	\ \ 
	Instituição de obtenção deste grau : UNIVERSIDADE FEDERAL DO PARÁ
	\\ 
	Endereço institucional do recomendante: \\ Avenida General Rodrigo Octávio Jordão Ramos, 3000, Campus Universitário, Setor Norte, Instituto de Ciências Exatas, Departamento de Matemática, CEP 69077000.\includepdf[pages={-},offset=35mm 0mm]{../../../upload/1206_2014-05-19_documentos.pdf}\includepdf[pages={-},offset=35mm 0mm]{../../../upload/1206_2014-05-19_historico.pdf} 
\begin{center}
Anexos.
\end{center}
\end{document}