\documentclass[11pt]{article}
\usepackage{graphicx,color}
\usepackage{pdfpages}
\usepackage[brazil]{babel}
\usepackage[utf8]{inputenc}
\addtolength{\hoffset}{-3cm} \addtolength{\textwidth}{6cm}
\addtolength{\voffset}{-.5cm} \addtolength{\textheight}{1cm}
%%%%%%%%%%%%%%%%%%%%%%%%%%%%%%%%%%%  To use Colors 
\title{\vspace*{-4cm} Ficha de Inscrição: \\Cod: 1362\ \ Kamila Costa Santos\ \ - \ \ Mestrado 
 }
\date{}

\begin{document}
\maketitle
\vspace*{-1.5cm}
\noindent Data de Nascimento:2/8/1991
\ \ \ Idade: 23   \ \ \ Sexo: Feminino
\\
Naturalidade: Guanhães  
\ \ \  Estado: MG
\ \ \  Nacionalidade: Brasileira
\ \ \ País: Brasil
\\        
Nome do pai : Joel Matias dos Santos
\ \ \ Nome da mãe: Joviana de Padua Costa Santos          
\\[0.2cm]                     
\textbf{Endereço Pessoal} 
\\ 
\noindent Endereço residencial: Rua Menina Daniela, 68
\\
        CEP: 39705-000 
\ \ \ Cidade: São João Evangelista 
\ \ \ Estado: MG 
\ \ \ País: Brasil
\\		
		Telefone comercial : +55(33)34122506
\ \ \ Telefone residencial: +55(33)88350444
\ \ \ Telefone celular : +55(33)88480776
\\
E-mail principal: kamilacostasantos@hotmail.com
\ \ \ E-mail alternativo: kamila.santos@ifmg.edu.br 
\\[0.2cm] 
\textbf{Documentos Pessoais}
\\
\noindent Número de CPF : 11035848635
\ \ \ Número de Identidade (ou Passaporte para estrangeiros): MG18107689
\\
Orgão emissor: ssp
\ \ \ Estado: MG
\ \ \ Data de emissão :23/12/2009
\\[0.3cm]
\textbf{Grau acadêmico mais alto obtido}
\\	
Curso:Matemática
\ \ \ Grau : licenciado
\ \ \ Instituição : Instituto Federal de Minas Gerais Campus São João Evangelista
\\			
Ano de Conclusão ou Previsão: 2013
\\ 
Experiência Profissional mais recente. \ \  
Tem experiência: Docente Discente  
\ \ \ Instituição: Instituto Federal de Minas Gerais Campus São João Evangelista
\\  
Período - início: 1-2014
\ \ \ fim: 1-0
\\[0.2cm] 
\textbf{Programa Pretendido:} Mestrado\\
Interesse em bolsa: Sim
\\[0.3cm]		
\textbf{Dados dos Recomendantes} 
\\
1- Nome: Jossara Bazílio de Souza Bicalho
\ \ \ \  e-mail: jossara.bicalho@ifmg.edu.br 
\\
2- Nome: Karina Dutra de Carvalho Lemos
\ \ \ \ e-mail: karina.dutra@ifmg.edu.br
\\
3- Nome: Amilton Ferreira da Silva Júnior
\ \ \ \ e-mail: amilton.junior@ifmg.edu.br
\\[0.2cm]
Motivação e expectativa do candidato em relação ao programa pretendido:
\\Venho por meio deste documento, demonstrar o meu interesse em participar do PROGRAMA DE PÓS GRADUAÇÃO EM MATEMÁTICA, nível mestrado,oferecido pela Universidades de Brasília. Inicialmente, farei um breve histórico da minha formação acadêmica e experiência profissional. A minha formação acadêmica consiste na graduação em Licenciatura em Matemática pelo Instituto Federal de Minas Gerais Campus São João Evangelista. Durante a minha trajetória acadêmica fui bolsista da CAPES por dois anos, trabalhei com o PIBID Programa Institucional de Bolsa de Iniciação à Docência e com o PRODOCÊNCIA Programa de Consolidação das Licenciaturas. Concernente à experiência profissional,trabalhei em uma Escola Estadual, onde tive a oportunidade de aprender e trabalhar efetivamente dando aula de Matemática para o ensino Fundamental.  Durante essa experiência, enfrentei desafios estimulantes com desdobramentos positivos na minha formação, além de aprender a desenvolver novas metodologias de ensino. Atualmente, trabalho no Instituto Federal de Minas Gerais Campus São João Evangelista, exercendo o cargo de Professora Substituta dando aulas de Matemática para o Ensino Médio.  Em 2013 participei do VII Congresso Iberoamericano de Educação Matemática Montevideo, Uruguay, onde tive a oportunidade de apresentar trabalhos de pesquisa realizado por mim. Com essa experiência e outras tive um grande interesse em me aprofundar cientificamente, hoje em dia a sociedade exige muito do conhecimento do ser humano e não dá para ficar parado, temos que correr atrás de crescer profissionalmente não só no currículo mas pessoalmente também. Acredito que esse curso é fundamental para o plano de carreira que almejo alcançar.\newpage\vspace*{-4cm}\subsection*{Carta de Recomendação - Jossara Bazílio de Souza Bicalho}Código Identificador: 1428\\Conhece-o candidato há quanto tempo (For how long have you known the applicant)? 
\ 4 anos e meio
\\ Conhece-o sob as seguintes circunstâncias: aulas\ \ orientacao
	\ \ \ \ outra 
\\ Conheçe o candidato sob outras circunstâncias: Bolsista de iniciação à docência
\\	Avaliações:\\
\begin{tabular}{|l|c|c|c|c|c|}
\hline
 & Excelente & Bom & Regular & Insuficiente & Não sabe \\
\hline
Desempenho acadêmico &  & X &  &  & \\
\hline
Capacidade de aprender novos conceitos &  & X &  &  & \\
\hline
Capacidade de trabalhar sozinho &  & X &  &  & \\
\hline
Criatividade &  & X &  &  & \\
\hline
Curiosidade &  & X &  &  & \\
\hline
Esforço, persistência &  & X &  &  & \\
\hline
Expressão escrita &  & X &  &  & \\
\hline
Expressão oral &  & X &  &  & \\
\hline
Relacionamento com colegas &  & X &  &  & \\
\hline
\end{tabular}\\
\\
\textbf{Opinião sobre os antecedentes acadêmicos, profissionais e/ou técnicos do candidato:}
\\A candidata apresentou rendimento acadêmico de regular para bom, tendo se destacado em atividades de iniciação à docência.\\
\\
\textbf{Opinião sobre seu possível aproveitamento, se aceito no Programa:}
\\O empenho e a dedicação apresentados ao longo da graduação, se estendido ao Programa de Mestrado, poderão levar a candidata ao sucesso.\\ 
\\
\textbf{Outras informações relevantes:} \\
\\[0.3cm]
\textbf{Entre os estudantes que já conheceu, você diria que o candidato está entre os:}
\\
\begin{tabular}{|l|c|c|c|c|c|}
\hline
 & 5\% melhores & 10\% melhores & 25\% melhores & 50\% melhores & Não sabe \\
\hline
Como aluno, em aulas &  &  & X &  & \\
\hline
Como orientando &  &  & X &  & \\
\hline
\end{tabular}
\subsection*{Dados Recomendante} 
	Instituição (Institution): Instituto Federal de Minas Gerais
\\ 
	Grau acadêmico mais alto obtido: mestre
	\ \ Área: Matemática
	\\
	Ano de obtenção deste grau: 2013
	\ \ 
	Instituição de obtenção deste grau : Universidade Federal de Viçosa
	\\ 
	Endereço institucional do recomendante: \\ Avenida Primeiro de Junho, 1043. Centro. São João Evangelista. Minas Gerais. 39740000\newpage\vspace*{-4cm}\subsection*{Carta de Recomendação - Karina Dutra de Carvalho Lemos}Código Identificador: 1429\\Conhece-o candidato há quanto tempo (For how long have you known the applicant)? 
\ 4 anos
\\ Conhece-o sob as seguintes circunstâncias: aulas\ \ 
	\ \ \ \  
\\ Conheçe o candidato sob outras circunstâncias: 
\\Avaliações: \\
\begin{tabular}{|l|c|c|c|c|c|}
\hline
 & Excelente & Bom & Regular & Insuficiente & Não sabe \\
\hline
Desempenho acadêmico & X &  &  &  & \\
\hline
Capacidade de aprender novos conceitos & X &  &  &  & \\
\hline
Capacidade de trabalhar sozinho &  & X &  &  & \\
\hline
Criatividade &  & X &  &  & \\
\hline
Curiosidade &  & X &  &  & \\
\hline
Esforço, persistência & X &  &  &  & \\
\hline
Expressão escrita &  & X &  &  & \\
\hline
Expressão oral &  & X &  &  & \\
\hline
Relacionamento com colegas & X &  &  &  & \\
\hline
\end{tabular}\\
\\
\textbf{Opinião sobre os antecedentes acadêmicos, profissionais e/ou técnicos do candidato:}
\\Candidata desempenhou durante estes 4 anos suas
atividades de maneira eficiente, demonstrando competência,
liderança, bem como facilidade no aprendizado e trabalhos em
equipe. 

\\
\\
\textbf{Opinião sobre seu possível aproveitamento, se aceito no Programa:}
\\Ela é uma aluna muito aplicada e vai se destacar em
qualidade dos seus trabalhos e provas feitas. Portanto, venho 
através desta afirmar que ela é uma aluna correta e de fácil relacionamento.\\ 
\\
\textbf{Outras informações relevantes:} \\
\\[0.3cm]
\textbf{Entre os estudantes que já conheceu, você diria que o candidato está entre os:}
\\
\begin{tabular}{|l|c|c|c|c|c|}
\hline
 & 5\% melhores & 10\% melhores & 25\% melhores & 50\% melhores & Não sabe \\
\hline
Como aluno, em aulas &  &  & X &  & \\
\hline
Como orientando &  &  &  &  & X\\
\hline
\end{tabular}
\subsection*{Dados Recomendante} 
	Instituição (Institution): Instituto Federal de Minas Gerais
\\ 
	Grau acadêmico mais alto obtido: mestre
	\ \ Área: Modelagem Matemática
	\\
	Ano de obtenção deste grau: 2008
	\ \ 
	Instituição de obtenção deste grau : Universidade Vale do Rio Verde
	\\ 
	Endereço institucional do recomendante: \\ www.agronet.gov.br
karina.dutraarrobaifmg.edu.br\newpage\vspace*{-4cm}\subsection*{Carta de Recomendação - Amilton Ferreira da Silva Júnior}Código Identificador: 1430\\Conhece-o candidato há quanto tempo (For how long have you known the applicant)? 
\ 01 ano
\\ Conhece-o sob as seguintes circunstâncias: aulas\ \ 
	\ \ \ \  
\\ Conheçe o candidato sob outras circunstâncias: 
\\Avaliações: \\
\begin{tabular}{|l|c|c|c|c|c|}
\hline
 & Excelente & Bom & Regular & Insuficiente & Não sabe \\
\hline
Desempenho acadêmico &  & X &  &  & \\
\hline
Capacidade de aprender novos conceitos &  & X &  &  & \\
\hline
Capacidade de trabalhar sozinho &  & X &  &  & \\
\hline
Criatividade &  & X &  &  & \\
\hline
Curiosidade & X &  &  &  & \\
\hline
Esforço, persistência &  & X &  &  & \\
\hline
Expressão escrita &  & X &  &  & \\
\hline
Expressão oral &  & X &  &  & \\
\hline
Relacionamento com colegas &  & X &  &  & \\
\hline
\end{tabular}\\
\\
\textbf{Opinião sobre os antecedentes acadêmicos, profissionais e/ou técnicos do candidato:}
\\Nada declarar\\
\\
\textbf{Opinião sobre seu possível aproveitamento, se aceito no Programa:}
\\Nada a declarar\\ 
\\
\textbf{Outras informações relevantes:} \\
\\[0.3cm]
\textbf{Entre os estudantes que já conheceu, você diria que o candidato está entre os:}
\\
\begin{tabular}{|l|c|c|c|c|c|}
\hline
 & 5\% melhores & 10\% melhores & 25\% melhores & 50\% melhores & Não sabe \\
\hline
Como aluno, em aulas &  &  & X &  & \\
\hline
Como orientando &  &  &  &  & X\\
\hline
\end{tabular}
\subsection*{Dados Recomendante} 
	Instituição (Institution): INSTITUTO FEDERAL DE MINAS GERAIS
\\ 
	Grau acadêmico mais alto obtido: mestre
	\ \ Área: Modelagem Computacional  
	\\
	Ano de obtenção deste grau: 2012
	\ \ 
	Instituição de obtenção deste grau : UFF
	\\ 
	Endereço institucional do recomendante: \\ Av 1 junho, centro, são joão evangelista, MG	
\begin{figure}[!htb]
\includegraphics{../upload/1362_2014-05-30_documentos.jpg}
\end{figure}	
\begin{figure}[!htb]
\includegraphics{../upload/1362_2014-05-30_historico.jpg}
\end{figure}\includepdf[pages={-},offset=35mm 0mm]{../../../upload/1362_2014-05-30_historico.pdf} 
\begin{center}
Anexos.
\end{center}
\end{document}