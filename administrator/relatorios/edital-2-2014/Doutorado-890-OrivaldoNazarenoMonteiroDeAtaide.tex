\documentclass[11pt]{article}
\usepackage{graphicx,color}
\usepackage{pdfpages}
\usepackage[brazil]{babel}
\usepackage[utf8]{inputenc}
\addtolength{\hoffset}{-3cm} \addtolength{\textwidth}{6cm}
\addtolength{\voffset}{-.5cm} \addtolength{\textheight}{1cm}
%%%%%%%%%%%%%%%%%%%%%%%%%%%%%%%%%%%  To use Colors 
\title{\vspace*{-4cm} Ficha de Inscrição: \\Cod: 890\ \ ORIVALDO NAZARENO MONTEIRO DE ATAIDE\ \ - \ \ Doutorado 
 }
\date{}

\begin{document}
\maketitle
\vspace*{-1.5cm}
\noindent Data de Nascimento:16/1/1979
\ \ \ Idade: 35   \ \ \ Sexo: Masculino
\\
Naturalidade: VIGIENSE  
\ \ \  Estado: PA
\ \ \  Nacionalidade: BRASILEIRO
\ \ \ País: BRASIL
\\        
Nome do pai : ORIVALDO NAZARENO DE ATAIDE
\ \ \ Nome da mãe: IRENE MORAES MONTEIRO          
\\[0.2cm]                     
\textbf{Endereço Pessoal} 
\\ 
\noindent Endereço residencial: RUA JOAQUIM MAGALHÃES DOS SANTOS
\\
        CEP: 68909-793 
\ \ \ Cidade: MACAPÁ 
\ \ \ Estado: AP 
\ \ \ País: BRASIL
\\		
		Telefone comercial : +55(96)21011439
\ \ \ Telefone residencial: +55(96)32511604
\ \ \ Telefone celular : +55(96)81123930
\\
E-mail principal: orivaldoataide@yahoo.com.br
\ \ \ E-mail alternativo: orivaldoataide@bol.com.br 
\\[0.2cm] 
\textbf{Documentos Pessoais}
\\
\noindent Número de CPF : 64956881215
\ \ \ Número de Identidade (ou Passaporte para estrangeiros): 306580
\\
Orgão emissor: DPTC
\ \ \ Estado: AP
\ \ \ Data de emissão :10/8/2001
\\[0.3cm]
\textbf{Grau acadêmico mais alto obtido}
\\	
Curso:Metodologia do Ensino de Matemática
\ \ \ Grau : especialista
\ \ \ Instituição : AUPEX FACEL
\\			
Ano de Conclusão ou Previsão: 2014
\\ 
Experiência Profissional mais recente. \ \  
Tem experiência: 0 0  
\ \ \ Instituição: 0
\\  
Período - início: 0-0
\ \ \ fim: 0-0
\\[0.2cm] 
\textbf{Programa Pretendido:} Doutorado\ \ \ \textbf{Área:} GeometriaDiferencial\\
Interesse em bolsa: Sim
\\[0.3cm]		
\textbf{Dados dos Recomendantes} 
\\
1- Nome: Steve Wanderson Calheiros de Araújo
\ \ \ \  e-mail: steve@unifap.br 
\\
2- Nome: Guzmán Eulálio Isla Chamilco
\ \ \ \ e-mail: isla@unifap.br
\\
3- Nome: José Walter Cárdenas Sotil
\ \ \ \ e-mail: cardenas@unifap.br
\\[0.2cm]
Motivação e expectativa do candidato em relação ao programa pretendido:
\\No curso de doutorado, pretendo dar continuidade aos meus estudos de pesquisa em campos da Matemática, tais como na Geometria, na qual tento, desde a graduação, fazer com que a Comunidade Científica reconheça as minhas descobertas, ou seja, a teoria dos Vinculantes de pontos do Plano Cartesiano.Além disso, moro na Região Norte do Brasil, no Estado do Amapá, onde não temos o programa de Doutorado em Matemática, o que temos, desde 2011, é o Mestrado Profissional PROFMAT, da Sociedade Brasileira de Matemática, ministrado na Universidade Federal do Amapá, UNIFAP. Mas, por sua natureza, não atende às minhas expectativas, por se tratar de um curso voltado para professores de Matemática das escolas públicas, trabalhando essencialmente com assuntos do Ensino Básico.Como tenho plena convicção de que posso contribuir mais ainda com o meu país, estudando em uma instituição  reconhecida em todo o mundo e com alta avaliação pela CAPES para o doutorado em Matemática, venho, respeitosamente, solicitar minha inscrição nesse programa.\newpage\vspace*{-4cm}\subsection*{Carta de Recomendação - Steve Wanderson Calheiros de Araújo}Código Identificador: 1221\\Conhece-o candidato há quanto tempo (For how long have you known the applicant)? 
\ 8 anos
\\ Conhece-o sob as seguintes circunstâncias: aulas\ \ 
	\ \ seminarios\ \  
\\ Conheçe o candidato sob outras circunstâncias: 
\\	Avaliações:\\
\begin{tabular}{|l|c|c|c|c|c|}
\hline
 & Excelente & Bom & Regular & Insuficiente & Não sabe \\
\hline
Desempenho acadêmico & X &  &  &  & \\
\hline
Capacidade de aprender novos conceitos &  & X &  &  & \\
\hline
Capacidade de trabalhar sozinho & X &  &  &  & \\
\hline
Criatividade & X &  &  &  & \\
\hline
Curiosidade & X &  &  &  & \\
\hline
Esforço, persistência & X &  &  &  & \\
\hline
Expressão escrita &  & X &  &  & \\
\hline
Expressão oral &  & X &  &  & \\
\hline
Relacionamento com colegas &  & X &  &  & \\
\hline
\end{tabular}\\
\\
\textbf{Opinião sobre os antecedentes acadêmicos, profissionais e/ou técnicos do candidato:}
\\Foi um excelente aluno de graduação e posso relatar isso com propriedade, pois acompanhei seu progresso ao longo de toda sua graduação. \\
\\
\textbf{Opinião sobre seu possível aproveitamento, se aceito no Programa:}
\\Tenho certeza que pela sua determinação e persistência se sairá muito bem.\\ 
\\
\textbf{Outras informações relevantes:} \\Ressalto apenas que o acadêmico tem trabalhos, digamos, inéditos. E é muito criativo.
\\[0.3cm]
\textbf{Entre os estudantes que já conheceu, você diria que o candidato está entre os:}
\\
\begin{tabular}{|l|c|c|c|c|c|}
\hline
 & 5\% melhores & 10\% melhores & 25\% melhores & 50\% melhores & Não sabe \\
\hline
Como aluno, em aulas & X &  &  &  & \\
\hline
Como orientando & X &  &  &  & \\
\hline
\end{tabular}
\subsection*{Dados Recomendante} 
	Instituição (Institution): Universidade Federal do Amapá
\\ 
	Grau acadêmico mais alto obtido: especialista
	\ \ Área: Matemática EDP
	\\
	Ano de obtenção deste grau: 2008
	\ \ 
	Instituição de obtenção deste grau : Faculdade Meta
	\\ 
	Endereço institucional do recomendante: \\ Avenida JK, SN, Jardim Marco Zero, MacapáAP, www.unifap.br\newpage\vspace*{-4cm}\subsection*{Carta de Recomendação - Guzmán Eulálio Isla Chamilco}Código Identificador: 908\\Conhece-o candidato há quanto tempo (For how long have you known the applicant)? 
\ 15 anos
\\ Conhece-o sob as seguintes circunstâncias: aulas\ \ orientacao
	\ \ \ \  
\\ Conheçe o candidato sob outras circunstâncias: 
\\Avaliações: \\
\begin{tabular}{|l|c|c|c|c|c|}
\hline
 & Excelente & Bom & Regular & Insuficiente & Não sabe \\
\hline
Desempenho acadêmico & X &  &  &  & \\
\hline
Capacidade de aprender novos conceitos & X &  &  &  & \\
\hline
Capacidade de trabalhar sozinho & X &  &  &  & \\
\hline
Criatividade & X &  &  &  & \\
\hline
Curiosidade & X &  &  &  & \\
\hline
Esforço, persistência & X &  &  &  & \\
\hline
Expressão escrita &  & X &  &  & \\
\hline
Expressão oral &  & X &  &  & \\
\hline
Relacionamento com colegas &  & X &  &  & \\
\hline
\end{tabular}\\
\\
\textbf{Opinião sobre os antecedentes acadêmicos, profissionais e/ou técnicos do candidato:}
\\É um candidato muito competente e responsavel.\\
\\
\textbf{Opinião sobre seu possível aproveitamento, se aceito no Programa:}
\\É um candidato que possui grande potencial para matemática. Foi meu aluno em licenciatura em matemática. No período que o candidato cursou a licenciatura, o curso não tinha na grade a disciplina de ANALISE REAL. Já falei para éle que estude primeiro o Mestrado.
Em tal sentido, solicito a comissão ver a possibilidade de aceitar ao curso de verão para seleção ao Mestrado Acadêmico.  \\ 
\\
\textbf{Outras informações relevantes:} \\
\\[0.3cm]
\textbf{Entre os estudantes que já conheceu, você diria que o candidato está entre os:}
\\
\begin{tabular}{|l|c|c|c|c|c|}
\hline
 & 5\% melhores & 10\% melhores & 25\% melhores & 50\% melhores & Não sabe \\
\hline
Como aluno, em aulas & X &  &  &  & \\
\hline
Como orientando & X &  &  &  & \\
\hline
\end{tabular}
\subsection*{Dados Recomendante} 
	Instituição (Institution): Universidade Federal do Amapá
\\ 
	Grau acadêmico mais alto obtido: doutor
	\ \ Área: Modelagem Matemático Computacional
	\\
	Ano de obtenção deste grau: 2006
	\ \ 
	Instituição de obtenção deste grau : Laboratório Nacional de Computação Científica.
	\\ 
	Endereço institucional do recomendante: \\ Rodovia JK, Km. 02. Macapá. Campos Marco Zero\newpage\vspace*{-4cm}\subsection*{Carta de Recomendação - José Walter Cárdenas Sotil}Código Identificador: 232\\Conhece-o candidato há quanto tempo (For how long have you known the applicant)? 
\ 8 anos
\\ Conhece-o sob as seguintes circunstâncias: aulas\ \ 
	\ \ seminarios\ \ outra 
\\ Conheçe o candidato sob outras circunstâncias: Membro de Banca de TCC
\\Avaliações: \\
\begin{tabular}{|l|c|c|c|c|c|}
\hline
 & Excelente & Bom & Regular & Insuficiente & Não sabe \\
\hline
Desempenho acadêmico & X &  &  &  & \\
\hline
Capacidade de aprender novos conceitos & X &  &  &  & \\
\hline
Capacidade de trabalhar sozinho & X &  &  &  & \\
\hline
Criatividade & X &  &  &  & \\
\hline
Curiosidade & X &  &  &  & \\
\hline
Esforço, persistência & X &  &  &  & \\
\hline
Expressão escrita & X &  &  &  & \\
\hline
Expressão oral & X &  &  &  & \\
\hline
Relacionamento com colegas & X &  &  &  & \\
\hline
\end{tabular}\\
\\
\textbf{Opinião sobre os antecedentes acadêmicos, profissionais e/ou técnicos do candidato:}
\\O candidato foi um excelente aluno de graduação, sendo considerado o aluno com mais potencial de sua turma. Por diversas circunstancias ele não se formou no tempo previsto. Em 2012 ele ingressou ao Mestrado no PROFMAT, mas não continuo pelo conteúdo do programa, direcionado ao ensino básico. 
Ele atualmente é funcionário público estadual concursado, mas esta decidido a continuar sua vida profissional e acadêmica em uma universidade, fazendo pesquisa e contribuindo com o ensino da matemática. 
Por isto, ele esta tentando uma pós graduação na sua prestigiada instituição.\\
\\
\textbf{Opinião sobre seu possível aproveitamento, se aceito no Programa:}
\\O candidato já é graduado em matemática e mesmo não trabalhando atualmente em uma instituição acadêmica, continua lendo literatura da área. Entre estas leituras estão as coleções do IMPA nas diversas áreas de pesquisa. 
Ele continua sua preparação para caso ser aceito no programa de vocês, ter um ótimo desempenho. Acredito que seu desempenho vai ser ótimo, pelo potencial dele e pelo ganho de estar em sua instituição com a experiência em ensino e pesquisa em pós graduação.\\ 
\\
\textbf{Outras informações relevantes:} \\
\\[0.3cm]
\textbf{Entre os estudantes que já conheceu, você diria que o candidato está entre os:}
\\
\begin{tabular}{|l|c|c|c|c|c|}
\hline
 & 5\% melhores & 10\% melhores & 25\% melhores & 50\% melhores & Não sabe \\
\hline
Como aluno, em aulas & X &  &  &  & \\
\hline
Como orientando &  & X &  &  & \\
\hline
\end{tabular}
\subsection*{Dados Recomendante} 
	Instituição (Institution): UNIFAP
\\ 
	Grau acadêmico mais alto obtido: doutor
	\ \ Área: Matemática Aplicada
	\\
	Ano de obtenção deste grau: 1999
	\ \ 
	Instituição de obtenção deste grau : IME-USP
	\\ 
	Endereço institucional do recomendante: \\ Campus Universitário Marco Zero do Equador
Rod. Juscelino Kubitschek, KM-02 - Jardim Marco Zero - CEP 68.903-419 - Macapá - AP - Brasil\includepdf[pages={-},offset=35mm 0mm]{../../../upload/890_2014-05-07_historico.pdf}\includepdf[pages={-},offset=35mm 0mm]{../../../upload/890_2014-05-07_documentos.pdf}\includepdf[pages={-},offset=35mm 0mm]{../../../upload/890_2013-10-28_documentos.pdf}\includepdf[pages={-},offset=35mm 0mm]{../../../upload/890_2013-10-28_historico.pdf} 
\begin{center}
Anexos.
\end{center}
\end{document}