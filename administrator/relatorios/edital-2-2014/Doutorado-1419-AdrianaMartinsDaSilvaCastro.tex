\documentclass[11pt]{article}
\usepackage{graphicx,color}
\usepackage{pdfpages}
\usepackage[brazil]{babel}
\usepackage[utf8]{inputenc}
\addtolength{\hoffset}{-3cm} \addtolength{\textwidth}{6cm}
\addtolength{\voffset}{-.5cm} \addtolength{\textheight}{1cm}
%%%%%%%%%%%%%%%%%%%%%%%%%%%%%%%%%%%  To use Colors 
\title{\vspace*{-4cm} Ficha de Inscrição: \\Cod: 1419\ \ Adriana Martins da Silva Castro\ \ - \ \ Doutorado 
 }
\date{}

\begin{document}
\maketitle
\vspace*{-1.5cm}
\noindent Data de Nascimento:5/1/1981
\ \ \ Idade: 33   \ \ \ Sexo: Feminino
\\
Naturalidade: Guanambi  
\ \ \  Estado: BA
\ \ \  Nacionalidade: Brasileira
\ \ \ País: Brasil
\\        
Nome do pai : Valdemir Rocha da Silva
\ \ \ Nome da mãe: Isa Maria Pereira MArtins          
\\[0.2cm]                     
\textbf{Endereço Pessoal} 
\\ 
\noindent Endereço residencial: Rua Acácia de Paula
\\
        CEP: 39401-037 
\ \ \ Cidade: Montes Claros 
\ \ \ Estado: MG 
\ \ \ País: Brasil
\\		
		Telefone comercial : +0(38)36294616
\ \ \ Telefone residencial: +0(38)91173339
\ \ \ Telefone celular : +0(38)84268548
\\
E-mail principal: adriana.mcastro@yahoo.com
\ \ \ E-mail alternativo: dicamartins@hotmail.com 
\\[0.2cm] 
\textbf{Documentos Pessoais}
\\
\noindent Número de CPF : 78626323549
\ \ \ Número de Identidade (ou Passaporte para estrangeiros): 0755801792
\\
Orgão emissor: SSP
\ \ \ Estado: MG
\ \ \ Data de emissão :5/8/2010
\\[0.3cm]
\textbf{Grau acadêmico mais alto obtido}
\\	
Curso:Matemática
\ \ \ Grau : mestre
\ \ \ Instituição : UNIFEI
\\			
Ano de Conclusão ou Previsão: 2012
\\ 
Experiência Profissional mais recente. \ \  
Tem experiência: Docente 0  
\ \ \ Instituição: Instituto Federal de Educação, Ciência e Tecnologia do Norte de MInas Gerais
\\  
Período - início: 1-2013
\ \ \ fim: nselecionado-nselecionado
\\[0.2cm] 
\textbf{Programa Pretendido:} Doutorado\ \ \ \textbf{Área:} SistemasDinamicos\\
Interesse em bolsa: Sim
\\[0.3cm]		
\textbf{Dados dos Recomendantes} 
\\
1- Nome: Luis Fernando de Osório Mello
\ \ \ \  e-mail: lfmelo@unifei.edu.br 
\\
2- Nome: Mariza Stefanello Simsen
\ \ \ \ e-mail: mariza@unifei.edu.br
\\
3- Nome: Ednei Canuto Paiva
\ \ \ \ e-mail: edineifis98@yahoo.com.br
\\[0.2cm]
Motivação e expectativa do candidato em relação ao programa pretendido:
\\A importância de se cursar uma Pós graduação Stricto Sensu,doutorado, prendese a dois aspectos aprender mais e melhor, aperfeiçoar naquilo que sei e adentrarme cada vez mais na pesquisa matemática e evoluir dentro da comunidade acadêmica.
\newpage\vspace*{-4cm}\subsection*{Carta de Recomendação - Luis Fernando de Osório Mello}Código Identificador: 1420\\Conhece-o candidato há quanto tempo (For how long have you known the applicant)? 
\ 
\\ Conhece-o sob as seguintes circunstâncias: \ \ 
	\ \ \ \  
\\ Conheçe o candidato sob outras circunstâncias: 
\\	Avaliações:\\
\begin{tabular}{|l|c|c|c|c|c|}
\hline
 & Excelente & Bom & Regular & Insuficiente & Não sabe \\
\hline
Desempenho acadêmico &  &  &  &  & \\
\hline
Capacidade de aprender novos conceitos &  &  &  &  & \\
\hline
Capacidade de trabalhar sozinho &  &  &  &  & \\
\hline
Criatividade &  &  &  &  & \\
\hline
Curiosidade &  &  &  &  & \\
\hline
Esforço, persistência &  &  &  &  & \\
\hline
Expressão escrita &  &  &  &  & \\
\hline
Expressão oral &  &  &  &  & \\
\hline
Relacionamento com colegas &  &  &  &  & \\
\hline
\end{tabular}\\
\\
\textbf{Opinião sobre os antecedentes acadêmicos, profissionais e/ou técnicos do candidato:}
\\\\
\\
\textbf{Opinião sobre seu possível aproveitamento, se aceito no Programa:}
\\\\ 
\\
\textbf{Outras informações relevantes:} \\
\\[0.3cm]
\textbf{Entre os estudantes que já conheceu, você diria que o candidato está entre os:}
\\
\begin{tabular}{|l|c|c|c|c|c|}
\hline
 & 5\% melhores & 10\% melhores & 25\% melhores & 50\% melhores & Não sabe \\
\hline
Como aluno, em aulas &  &  &  &  & \\
\hline
Como orientando &  &  &  &  & \\
\hline
\end{tabular}
\subsection*{Dados Recomendante} 
	Instituição (Institution): 
\\ 
	Grau acadêmico mais alto obtido: 
	\ \ Área: 
	\\
	Ano de obtenção deste grau: 
	\ \ 
	Instituição de obtenção deste grau : 
	\\ 
	Endereço institucional do recomendante: \\ \newpage\vspace*{-4cm}\subsection*{Carta de Recomendação - Mariza Stefanello Simsen}Código Identificador: 1421\\Conhece-o candidato há quanto tempo (For how long have you known the applicant)? 
\ 
\\ Conhece-o sob as seguintes circunstâncias: \ \ 
	\ \ \ \  
\\ Conheçe o candidato sob outras circunstâncias: 
\\Avaliações: \\
\begin{tabular}{|l|c|c|c|c|c|}
\hline
 & Excelente & Bom & Regular & Insuficiente & Não sabe \\
\hline
Desempenho acadêmico &  &  &  &  & \\
\hline
Capacidade de aprender novos conceitos &  &  &  &  & \\
\hline
Capacidade de trabalhar sozinho &  &  &  &  & \\
\hline
Criatividade &  &  &  &  & \\
\hline
Curiosidade &  &  &  &  & \\
\hline
Esforço, persistência &  &  &  &  & \\
\hline
Expressão escrita &  &  &  &  & \\
\hline
Expressão oral &  &  &  &  & \\
\hline
Relacionamento com colegas &  &  &  &  & \\
\hline
\end{tabular}\\
\\
\textbf{Opinião sobre os antecedentes acadêmicos, profissionais e/ou técnicos do candidato:}
\\\\
\\
\textbf{Opinião sobre seu possível aproveitamento, se aceito no Programa:}
\\\\ 
\\
\textbf{Outras informações relevantes:} \\
\\[0.3cm]
\textbf{Entre os estudantes que já conheceu, você diria que o candidato está entre os:}
\\
\begin{tabular}{|l|c|c|c|c|c|}
\hline
 & 5\% melhores & 10\% melhores & 25\% melhores & 50\% melhores & Não sabe \\
\hline
Como aluno, em aulas &  &  &  &  & \\
\hline
Como orientando &  &  &  &  & \\
\hline
\end{tabular}
\subsection*{Dados Recomendante} 
	Instituição (Institution): 
\\ 
	Grau acadêmico mais alto obtido: 
	\ \ Área: 
	\\
	Ano de obtenção deste grau: 
	\ \ 
	Instituição de obtenção deste grau : 
	\\ 
	Endereço institucional do recomendante: \\ \newpage\vspace*{-4cm}\subsection*{Carta de Recomendação - Ednei Canuto Paiva}Código Identificador: 1422\\Conhece-o candidato há quanto tempo (For how long have you known the applicant)? 
\ 2 anos
\\ Conhece-o sob as seguintes circunstâncias: \ \ 
	\ \ \ \ outra 
\\ Conheçe o candidato sob outras circunstâncias: Colega de Trabalho
\\Avaliações: \\
\begin{tabular}{|l|c|c|c|c|c|}
\hline
 & Excelente & Bom & Regular & Insuficiente & Não sabe \\
\hline
Desempenho acadêmico &  & X &  &  & \\
\hline
Capacidade de aprender novos conceitos & X &  &  &  & \\
\hline
Capacidade de trabalhar sozinho & X &  &  &  & \\
\hline
Criatividade & X &  &  &  & \\
\hline
Curiosidade & X &  &  &  & \\
\hline
Esforço, persistência & X &  &  &  & \\
\hline
Expressão escrita & X &  &  &  & \\
\hline
Expressão oral &  & X &  &  & \\
\hline
Relacionamento com colegas & X &  &  &  & \\
\hline
\end{tabular}\\
\\
\textbf{Opinião sobre os antecedentes acadêmicos, profissionais e/ou técnicos do candidato:}
\\A candidata trabalha com a disciplina de cálculo no curso de física, no qual também sou professor, trabalhamos em parceria identificando na ementa dos cálculos pontos que podem ser trabalhados de forma interdisciplinar  para auxiliar as disciplinas de física.\\
\\
\textbf{Opinião sobre seu possível aproveitamento, se aceito no Programa:}
\\A candidata tem muita habilidade na área de abrangência deste programa de doutorado,veio de um mestrado na área e demonstra muita competência sempre que é solicitada, tem muita habilidade em trabalhar em grupo, inclusive faz  parte do Núcleo Docente Estruturante NDE do curso de física onde tem dado muita contribuição. Acredito que se aceita será uma aluna muito comprometida com o programa, pois é assim a sua atuação como professora do IFNMG.\\ 
\\
\textbf{Outras informações relevantes:} \\A candidata tem grande interesse por equações diferenciais, atua como coorientadora em um projeto de pesquisa que trata de soluções de equações diferenciais de problemas de interesse em física. E também atua como professora da disciplinas de equações diferenciais.
\\[0.3cm]
\textbf{Entre os estudantes que já conheceu, você diria que o candidato está entre os:}
\\
\begin{tabular}{|l|c|c|c|c|c|}
\hline
 & 5\% melhores & 10\% melhores & 25\% melhores & 50\% melhores & Não sabe \\
\hline
Como aluno, em aulas &  & X &  &  & \\
\hline
Como orientando &  & X &  &  & \\
\hline
\end{tabular}
\subsection*{Dados Recomendante} 
	Instituição (Institution): Instituto Federal de Educação Ciência e Tecnologia do Norte de Minas Gerais
\\ 
	Grau acadêmico mais alto obtido: doutor
	\ \ Área: Engenharia Agrícola
	\\
	Ano de obtenção deste grau: 2009
	\ \ 
	Instituição de obtenção deste grau : Universidade Federal de Viçosa
	\\ 
	Endereço institucional do recomendante: \\ Fazenda São Geraldo, sn, Estrada Januária km 06, Bom Jardim
Januária, Minas Gerais CEP 39480 000\includepdf[pages={-},offset=35mm 0mm]{../../../upload/1419_2014-05-30_documentos.pdf}\includepdf[pages={-},offset=35mm 0mm]{../../../upload/1419_2014-05-30_historico.pdf} 
\begin{center}
Anexos.
\end{center}
\end{document}