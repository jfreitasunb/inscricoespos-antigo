\documentclass[11pt]{article}
\usepackage{graphicx,color}
\usepackage{pdfpages}
\usepackage[brazil]{babel}
\usepackage[utf8]{inputenc}
\addtolength{\hoffset}{-3cm} \addtolength{\textwidth}{6cm}
\addtolength{\voffset}{-.5cm} \addtolength{\textheight}{1cm}
%%%%%%%%%%%%%%%%%%%%%%%%%%%%%%%%%%%  To use Colors 
\title{\vspace*{-4cm} Ficha de Inscrição: \\Cod: 1243\ \ Tauan Barbosa\ \ - \ \ Mestrado 
 }
\date{}

\begin{document}
\maketitle
\vspace*{-1.5cm}
\noindent Data de Nascimento:1/4/1993
\ \ \ Idade: 21   \ \ \ Sexo: Masculino
\\
Naturalidade: Picos   
\ \ \  Estado: PI
\ \ \  Nacionalidade: Brasileiro
\ \ \ País: Brasil
\\        
Nome do pai : Cleones Mendes Barbosa
\ \ \ Nome da mãe: Elenice Alves de Sousa          
\\[0.2cm]                     
\textbf{Endereço Pessoal} 
\\ 
\noindent Endereço residencial: Rua Trento Quadra 02 Lote 09 Jardim Abaporú
\\
        CEP: 74786-007 
\ \ \ Cidade: Goiânia 
\ \ \ Estado: GO 
\ \ \ País: Brasil
\\		
		Telefone comercial : +0(62)35676807
\ \ \ Telefone residencial: +0(62)35676807
\ \ \ Telefone celular : +0(62)83320410
\\
E-mail principal: tauansousa@hotmail.com
\ \ \ E-mail alternativo: tauansousab@gmail.com 
\\[0.2cm] 
\textbf{Documentos Pessoais}
\\
\noindent Número de CPF : 03601776111
\ \ \ Número de Identidade (ou Passaporte para estrangeiros): 5237916
\\
Orgão emissor: SPTC
\ \ \ Estado: GO
\ \ \ Data de emissão :29/8/2005
\\[0.3cm]
\textbf{Grau acadêmico mais alto obtido}
\\	
Curso:Matemática
\ \ \ Grau : licenciado
\ \ \ Instituição : IFG
\\			
Ano de Conclusão ou Previsão: 2014
\\ 
Experiência Profissional mais recente. \ \  
Tem experiência: Docente 0  
\ \ \ Instituição: Colégio Boas Novas
\\  
Período - início: 2-2013
\ \ \ fim: 0-2014
\\[0.2cm] 
\textbf{Programa Pretendido:} Mestrado\\
Interesse em bolsa: Nao
\\[0.3cm]		
\textbf{Dados dos Recomendantes} 
\\
1- Nome: Iran Martins do Carmo
\ \ \ \  e-mail: iran.carmo@ifg.edu.br 
\\
2- Nome: José Eder Salvador de Vasconcelos
\ \ \ \ e-mail: jose.vasconcelos@ifg.edu.br
\\
3- Nome: Aline Mota de Mesquita Assis
\ \ \ \ e-mail: aline.mesquita@ifg.edu.br
\\[0.2cm]
Motivação e expectativa do candidato em relação ao programa pretendido:
\\
A razão pela qual me motivou a esta candidatura é a ânsia de expandir os conhecimentos matemáticos adquiridos durante a graduação, das quais me tornaram um profissional que reconhece o rigor matemático e a utilidade que essa ciência possui na atualidade. Assim, acredito que não somente a formação superior é suficiente para consolidar um matemático completo sendo necessário, por meio do mestrado, o aprofundamento dos estudos no âmbito científico.

A matemática durante a minha formação superior me proporcionou a aquisição e o desenvolvimento de um pensamento lógico matemático mediante as disciplinas cursadas. Além disso, com a execução de projetos de iniciação científica na área de Álgebra, pude perceber a importância da pesquisa em matemática para o crescimento pessoal, profissional e intelectual, confirmando assim a importância de ingresso neste mestrado para o preenchimento de lacunas que ficaram durante a minha graduação. 

Com o ingresso neste mestrado, tenho como expectativa prosseguir com os estudos na área de matemática, visando uma melhor qualificação profissional. Além disso, iniciar e desenvolver pesquisas, que é totalmente possível dentro desta instituição, uma vez que os professores são totalmente qualificados. Enfim, se aceito, estarei à disposição para estudar e me dedicar a cada momento do mestrado objetivando sua conclusão com bastante êxito. 
\newpage\vspace*{-4cm}\subsection*{Carta de Recomendação - Iran Martins do Carmo}Código Identificador: 1337\\Conhece-o candidato há quanto tempo (For how long have you known the applicant)? 
\ 2 anos
\\ Conhece-o sob as seguintes circunstâncias: aulas\ \ 
	\ \ seminarios\ \  
\\ Conheçe o candidato sob outras circunstâncias: 
\\	Avaliações:\\
\begin{tabular}{|l|c|c|c|c|c|}
\hline
 & Excelente & Bom & Regular & Insuficiente & Não sabe \\
\hline
Desempenho acadêmico & X &  &  &  & \\
\hline
Capacidade de aprender novos conceitos & X &  &  &  & \\
\hline
Capacidade de trabalhar sozinho & X &  &  &  & \\
\hline
Criatividade & X &  &  &  & \\
\hline
Curiosidade & X &  &  &  & \\
\hline
Esforço, persistência & X &  &  &  & \\
\hline
Expressão escrita & X &  &  &  & \\
\hline
Expressão oral &  & X &  &  & \\
\hline
Relacionamento com colegas & X &  &  &  & \\
\hline
\end{tabular}\\
\\
\textbf{Opinião sobre os antecedentes acadêmicos, profissionais e/ou técnicos do candidato:}
\\O estudante Tauan foi meu aluno na disciplina Estatística, onde trabalhamos conteúdos de Probabilidade e Estatística. Demonstrou ser bastante interessado e disciplinado, obtendo nota 10,0. 
Participou da III Semana da Licenciatura em Matemática do IFG,em 2012, compondo a equipe organizadora, com atuação de destaque.
As notícias sobre seu desempenho em outras disciplinas, dadas pelos professores, são muito boas.\\
\\
\textbf{Opinião sobre seu possível aproveitamento, se aceito no Programa:}
\\Por ser um estudante muito interessado, disciplinado e com uma boa base matemática, acredito que o Tauan terá muito exito na sua pós graduação.\\ 
\\
\textbf{Outras informações relevantes:} \\Devo destacar ainda a seriedade com que o Tauan coloca em suas atividades.
\\[0.3cm]
\textbf{Entre os estudantes que já conheceu, você diria que o candidato está entre os:}
\\
\begin{tabular}{|l|c|c|c|c|c|}
\hline
 & 5\% melhores & 10\% melhores & 25\% melhores & 50\% melhores & Não sabe \\
\hline
Como aluno, em aulas & X &  &  &  & \\
\hline
Como orientando & X &  &  &  & \\
\hline
\end{tabular}
\subsection*{Dados Recomendante} 
	Instituição (Institution): IFG
\\ 
	Grau acadêmico mais alto obtido: doutor
	\ \ Área: Probabilidade e Estatística
	\\
	Ano de obtenção deste grau: 2006
	\ \ 
	Instituição de obtenção deste grau : IME  USP
	\\ 
	Endereço institucional do recomendante: \\ IFG Câmpus Goiânia, Departamento de Áreas Acadêmicas II, 
Avenida Contorno, sn, Centro, Goiânia, fone 62 32272700\newpage\vspace*{-4cm}\subsection*{Carta de Recomendação - José Eder Salvador de Vasconcelos}Código Identificador: 1338\\Conhece-o candidato há quanto tempo (For how long have you known the applicant)? 
\ 4 anos
\\ Conhece-o sob as seguintes circunstâncias: aulas\ \ 
	\ \ \ \  
\\ Conheçe o candidato sob outras circunstâncias: 
\\Avaliações: \\
\begin{tabular}{|l|c|c|c|c|c|}
\hline
 & Excelente & Bom & Regular & Insuficiente & Não sabe \\
\hline
Desempenho acadêmico & X &  &  &  & \\
\hline
Capacidade de aprender novos conceitos & X &  &  &  & \\
\hline
Capacidade de trabalhar sozinho & X &  &  &  & \\
\hline
Criatividade &  & X &  &  & \\
\hline
Curiosidade & X &  &  &  & \\
\hline
Esforço, persistência & X &  &  &  & \\
\hline
Expressão escrita &  & X &  &  & \\
\hline
Expressão oral &  & X &  &  & \\
\hline
Relacionamento com colegas & X &  &  &  & \\
\hline
\end{tabular}\\
\\
\textbf{Opinião sobre os antecedentes acadêmicos, profissionais e/ou técnicos do candidato:}
\\Cursou comigo três disciplinas, em todas elas obteve o conceito máximo, se destacando dos demais e sendo referência para a turma.\\
\\
\textbf{Opinião sobre seu possível aproveitamento, se aceito no Programa:}
\\Não tenho dúvida que o Tauan se sairá bem em qualquer programa de pós graduação. Sempre demonstrou muito empenho e dedicação e conta com talento natural para as ciências exatas.\\ 
\\
\textbf{Outras informações relevantes:} \\O aluno tem todas as condições de desenvolver um bom trabalho no mestrado e já teve oportunidade de desenvolver junto com sua orientadora de PIBIC projeto de pesquisa na área de teoria dos números.
\\[0.3cm]
\textbf{Entre os estudantes que já conheceu, você diria que o candidato está entre os:}
\\
\begin{tabular}{|l|c|c|c|c|c|}
\hline
 & 5\% melhores & 10\% melhores & 25\% melhores & 50\% melhores & Não sabe \\
\hline
Como aluno, em aulas & X &  &  &  & \\
\hline
Como orientando &  &  &  &  & X\\
\hline
\end{tabular}
\subsection*{Dados Recomendante} 
	Instituição (Institution): IFG
\\ 
	Grau acadêmico mais alto obtido: mestre
	\ \ Área: Álgebra
	\\
	Ano de obtenção deste grau: 2009
	\ \ 
	Instituição de obtenção deste grau : UFCG
	\\ 
	Endereço institucional do recomendante: \\ Rua 75, n46. Centro. CEP 74055110. GoiâniaGO.
Fone 62 32272700
\newpage\vspace*{-4cm}\subsection*{Carta de Recomendação - Aline Mota de Mesquita Assis}Código Identificador: 1339\\Conhece-o candidato há quanto tempo (For how long have you known the applicant)? 
\ 2 anos e meio
\\ Conhece-o sob as seguintes circunstâncias: aulas\ \ orientacao
	\ \ \ \  
\\ Conheçe o candidato sob outras circunstâncias: 
\\Avaliações: \\
\begin{tabular}{|l|c|c|c|c|c|}
\hline
 & Excelente & Bom & Regular & Insuficiente & Não sabe \\
\hline
Desempenho acadêmico & X &  &  &  & \\
\hline
Capacidade de aprender novos conceitos &  & X &  &  & \\
\hline
Capacidade de trabalhar sozinho & X &  &  &  & \\
\hline
Criatividade &  & X &  &  & \\
\hline
Curiosidade & X &  &  &  & \\
\hline
Esforço, persistência & X &  &  &  & \\
\hline
Expressão escrita &  & X &  &  & \\
\hline
Expressão oral &  & X &  &  & \\
\hline
Relacionamento com colegas & X &  &  &  & \\
\hline
\end{tabular}\\
\\
\textbf{Opinião sobre os antecedentes acadêmicos, profissionais e/ou técnicos do candidato:}
\\É um excelente aluno, que se dispõe a estudar, o que o faz ter uma percepção diferente dos demais colegas quanto aos conteúdos matemáticos, conseguindo fazer ligações entre as diversas áreas da matemática. Tem muita facilidade nos estudos, e quando não entende não desiste do problema, não se dando por satisfeito até a total compreensão e assimilação do conteúdo. Não tem preguiça para pesquisar em outros materiais didáticos nem de procurar o professor para solicitar ajuda. É um aluno de grande destaque.\\
\\
\textbf{Opinião sobre seu possível aproveitamento, se aceito no Programa:}
\\Com certeza ele se empenhará em ser um excelente aluno, como foi na graduação, se dedicando em entender a fundo a matemática em suas diversas áreas. Sei que ele terá dificuldades no início, pois está concluindo uma licenciatura em matemática, o que lhe causa uma pequena desvantagem em relação a quem faz bacharelado, mas isso não é um impedimento para que ele busque os ensinamentos que lhe falta.\\ 
\\
\textbf{Outras informações relevantes:} \\Ele tem um raciocínio bem rápido e lógico, o que lhe proporciona a capacidade para estudos individuais bem sucedidos. Tem enorme satisfação em compartilhar com os colegas o que já sabe, ajudando no aprendizado dos outros e, consequentemente, aumentando o dele.
\\[0.3cm]
\textbf{Entre os estudantes que já conheceu, você diria que o candidato está entre os:}
\\
\begin{tabular}{|l|c|c|c|c|c|}
\hline
 & 5\% melhores & 10\% melhores & 25\% melhores & 50\% melhores & Não sabe \\
\hline
Como aluno, em aulas & X &  &  &  & \\
\hline
Como orientando & X &  &  &  & \\
\hline
\end{tabular}
\subsection*{Dados Recomendante} 
	Instituição (Institution): Instituto Federal de Educação, Ciência e Tecnologia de Goiás
\\ 
	Grau acadêmico mais alto obtido: mestre
	\ \ Área: Matemática
	\\
	Ano de obtenção deste grau: 2007
	\ \ 
	Instituição de obtenção deste grau : Universidade Federal de Goiás
	\\ 
	Endereço institucional do recomendante: \\ aline.mesquita ifg.edu.br \includepdf[pages={-},offset=35mm 0mm]{../../../upload/1243_2014-05-25_documentos.pdf}\includepdf[pages={-},offset=35mm 0mm]{../../../upload/1243_2014-05-25_historico.pdf} 
\begin{center}
Anexos.
\end{center}
\end{document}