\documentclass[11pt]{article}
\usepackage{graphicx,color}
\usepackage{pdfpages}
\usepackage[brazil]{babel}
\usepackage[utf8]{inputenc}
\addtolength{\hoffset}{-3cm} \addtolength{\textwidth}{6cm}
\addtolength{\voffset}{-.5cm} \addtolength{\textheight}{1cm}
%%%%%%%%%%%%%%%%%%%%%%%%%%%%%%%%%%%  To use Colors 
\title{\vspace*{-4cm} Ficha de Inscrição: \\Cod: 1354\ \ Luana  Kuister Xavier\ \ - \ \ Mestrado 
 }
\date{}

\begin{document}
\maketitle
\vspace*{-1.5cm}
\noindent Data de Nascimento:25/5/1993
\ \ \ Idade: 21   \ \ \ Sexo: Feminino
\\
Naturalidade: São Sepé  
\ \ \  Estado: RS
\ \ \  Nacionalidade: Brasileira
\ \ \ País: Brasil
\\        
Nome do pai : Jaures Marion da Silva Xavier
\ \ \ Nome da mãe: Eda Kuister Xavier          
\\[0.2cm]                     
\textbf{Endereço Pessoal} 
\\ 
\noindent Endereço residencial: Cerro do Formigueiro
\\
        CEP: 97210-000 
\ \ \ Cidade: Formigueiro 
\ \ \ Estado: RS 
\ \ \ País: Brasil
\\		
		Telefone comercial : +55(55)99222316
\ \ \ Telefone residencial: +55(55)99222316
\ \ \ Telefone celular : +55(55)99222316
\\
E-mail principal: luana.k.xavier@hotmail.com
\ \ \ E-mail alternativo: luanna.k.xavier@gmail.com 
\\[0.2cm] 
\textbf{Documentos Pessoais}
\\
\noindent Número de CPF : 03051630004
\ \ \ Número de Identidade (ou Passaporte para estrangeiros): 3093511412
\\
Orgão emissor: SSP
\ \ \ Estado: RS
\ \ \ Data de emissão :22/2/2011
\\[0.3cm]
\textbf{Grau acadêmico mais alto obtido}
\\	
Curso:Matemática
\ \ \ Grau : licenciado
\ \ \ Instituição : Universidade Federal de Santa Maria
\\			
Ano de Conclusão ou Previsão: 2014
\\ 
Experiência Profissional mais recente. \ \  
Tem experiência: 0 Discente  
\ \ \ Instituição: Universidade Federal de Santa Maria
\\  
Período - início: 2-2010
\ \ \ fim: 1-2014
\\[0.2cm] 
\textbf{Programa Pretendido:} Mestrado\\
Interesse em bolsa: Sim
\\[0.3cm]		
\textbf{Dados dos Recomendantes} 
\\
1- Nome: Antonio Carlos Lyrio Bidel
\ \ \ \  e-mail: bidelac@gmail.com 
\\
2- Nome: Maurício Fronza da Silva
\ \ \ \ e-mail: mauriciofronzadasilva@hotmail.com
\\
3- Nome: Ricardo Fajardo
\ \ \ \ e-mail: rfaj@ufsm.br
\\[0.2cm]
Motivação e expectativa do candidato em relação ao programa pretendido:
\\Gostaria de fazer o mestrado em matemática na instituição pelo reconhecimento que a mesma possui. Além disso, gosto muito de matemática, de estudar e me aperfeiçoar nesta área. Tenho certeza que seria uma ótima experiência fazer o mestrado em outra instituição, para ampliar meus conhecimentos na profissão. Além disso, muitos professores e antigos colegas do curso que estão aí me incentivaram a fazer a inscrição. Gostaria muito de ser aprovada, pois almejo fazer mestrado, doutorado e seguir carreira em matemática. Quero poder ter essa oportunidade para melhorar minha vida, já que sou de família humilde e também porque tenho certeza que é isso que eu quero para minha carreira profissional.\newpage\vspace*{-4cm}\subsection*{Carta de Recomendação - Antonio Carlos Lyrio Bidel}Código Identificador: 221\\Conhece-o candidato há quanto tempo (For how long have you known the applicant)? 
\ 4 anos
\\ Conhece-o sob as seguintes circunstâncias: \ \ orientacao
	\ \ \ \ outra 
\\ Conheçe o candidato sob outras circunstâncias: Orinetação de atividades de ensino e extensão vinculadas ao PET Matemática da UFSM do qual sou tutor desde junho de 2005
\\	Avaliações:\\
\begin{tabular}{|l|c|c|c|c|c|}
\hline
 & Excelente & Bom & Regular & Insuficiente & Não sabe \\
\hline
Desempenho acadêmico &  & X &  &  & \\
\hline
Capacidade de aprender novos conceitos & X &  &  &  & \\
\hline
Capacidade de trabalhar sozinho & X &  &  &  & \\
\hline
Criatividade & X &  &  &  & \\
\hline
Curiosidade & X &  &  &  & \\
\hline
Esforço, persistência & X &  &  &  & \\
\hline
Expressão escrita & X &  &  &  & \\
\hline
Expressão oral & X &  &  &  & \\
\hline
Relacionamento com colegas & X &  &  &  & \\
\hline
\end{tabular}\\
\\
\textbf{Opinião sobre os antecedentes acadêmicos, profissionais e/ou técnicos do candidato:}
\\A candidata tem boa base matemática. Desenvolve atividades de Iniciação Científica, com professor orientador do Depto de Matemática da UFSM os quais relatam ótimo desempenho nas referidas atividades.\\
\\
\textbf{Opinião sobre seu possível aproveitamento, se aceito no Programa:}
\\Não tenho dúvidas que concluirá seus estudos em tempo hábil e com ótimo aproveitamento.\\ 
\\
\textbf{Outras informações relevantes:} \\Reforço que, a participação da candidata nas atividades vinculadas ao PET, é bastante qualificada. É persistente, motivada, se expressa muito bem nas formas escritas e oral e se relaciona muito bem com colegas e professores.
\\[0.3cm]
\textbf{Entre os estudantes que já conheceu, você diria que o candidato está entre os:}
\\
\begin{tabular}{|l|c|c|c|c|c|}
\hline
 & 5\% melhores & 10\% melhores & 25\% melhores & 50\% melhores & Não sabe \\
\hline
Como aluno, em aulas & X &  &  &  & \\
\hline
Como orientando & X &  &  &  & \\
\hline
\end{tabular}
\subsection*{Dados Recomendante} 
	Instituição (Institution): Universidade Federal de Santa Maria
\\ 
	Grau acadêmico mais alto obtido: doutor
	\ \ Área: Engenharia Mecânica - Mecânica dos Sólidos
	\\
	Ano de obtenção deste grau: 2003
	\ \ 
	Instituição de obtenção deste grau : UFRGS
	\\ 
	Endereço institucional do recomendante: \\ Avenida Roraíma número 1000 - Prédio 13 CCNE
Campus Universitário - Camobi - Santa Maria RS 
97105 - 900\newpage\vspace*{-4cm}\subsection*{Carta de Recomendação - Maurício Fronza da Silva}Código Identificador: 222\\Conhece-o candidato há quanto tempo (For how long have you known the applicant)? 
\ Desde o primeiro semestre de 2013
\\ Conhece-o sob as seguintes circunstâncias: aulas\ \ 
	\ \ \ \  
\\ Conheçe o candidato sob outras circunstâncias: 
\\Avaliações: \\
\begin{tabular}{|l|c|c|c|c|c|}
\hline
 & Excelente & Bom & Regular & Insuficiente & Não sabe \\
\hline
Desempenho acadêmico &  & X &  &  & \\
\hline
Capacidade de aprender novos conceitos &  & X &  &  & \\
\hline
Capacidade de trabalhar sozinho &  & X &  &  & \\
\hline
Criatividade &  & X &  &  & \\
\hline
Curiosidade &  & X &  &  & \\
\hline
Esforço, persistência &  & X &  &  & \\
\hline
Expressão escrita &  & X &  &  & \\
\hline
Expressão oral &  & X &  &  & \\
\hline
Relacionamento com colegas &  &  &  &  & X\\
\hline
\end{tabular}\\
\\
\textbf{Opinião sobre os antecedentes acadêmicos, profissionais e/ou técnicos do candidato:}
\\Trabalhei com a Luana nas disciplinas de Análise na Reta e Análise no Rn, nas quais tem um desempenho muito bom. A mas marcante característica sua, que pude observar, é que tem uma capacidade considerável de memorizar cada uma das hipóteses dos resultados estudados. Além disso, apresenta gosto pela matemática e é, sem dúvida, uma aluna responsável e dedicada. Em Acredito que apresenta uma boa capacidade de resolver exercícios. Creio que sua formação foi prejudicada por uma reforma curricular mal administrada no nosso curso de graduação.
Assim, embora não seja uma aluna brilhante e cuja formação poderia ter sido mais completa, está em uma fase de amadurecimento e de crescimento considerável. Assim, acredito que reúne condições para realizar um curso de mestrado.  \\
\\
\textbf{Opinião sobre seu possível aproveitamento, se aceito no Programa:}
\\Acredito que a candidata, caso seja aceita, terá dificuldades no primeiro semestre do curso. Mas pelas qualidades apresentadas acima, terá um desempenho satisfatório a partir daí.\\ 
\\
\textbf{Outras informações relevantes:} \\
\\[0.3cm]
\textbf{Entre os estudantes que já conheceu, você diria que o candidato está entre os:}
\\
\begin{tabular}{|l|c|c|c|c|c|}
\hline
 & 5\% melhores & 10\% melhores & 25\% melhores & 50\% melhores & Não sabe \\
\hline
Como aluno, em aulas &  & X &  &  & \\
\hline
Como orientando &  & X &  &  & \\
\hline
\end{tabular}
\subsection*{Dados Recomendante} 
	Instituição (Institution): Universidade Federal de Santa Maria
\\ 
	Grau acadêmico mais alto obtido: doutor
	\ \ Área: EDP
	\\
	Ano de obtenção deste grau: 2006
	\ \ 
	Instituição de obtenção deste grau : UFSCar
	\\ 
	Endereço institucional do recomendante: \\ mauriciofronzadasilva hotmail.com\newpage\vspace*{-4cm}\subsection*{Carta de Recomendação - Ricardo Fajardo}Código Identificador: 1356\\Conhece-o candidato há quanto tempo (For how long have you known the applicant)? 
\ Quatro anos
\\ Conhece-o sob as seguintes circunstâncias: aulas\ \ 
	\ \ \ \  
\\ Conheçe o candidato sob outras circunstâncias: 
\\Avaliações: \\
\begin{tabular}{|l|c|c|c|c|c|}
\hline
 & Excelente & Bom & Regular & Insuficiente & Não sabe \\
\hline
Desempenho acadêmico & X &  &  &  & \\
\hline
Capacidade de aprender novos conceitos & X &  &  &  & \\
\hline
Capacidade de trabalhar sozinho & X &  &  &  & \\
\hline
Criatividade &  & X &  &  & \\
\hline
Curiosidade & X &  &  &  & \\
\hline
Esforço, persistência & X &  &  &  & \\
\hline
Expressão escrita &  & X &  &  & \\
\hline
Expressão oral &  & X &  &  & \\
\hline
Relacionamento com colegas & X &  &  &  & \\
\hline
\end{tabular}\\
\\
\textbf{Opinião sobre os antecedentes acadêmicos, profissionais e/ou técnicos do candidato:}
\\O candidato é sério, responsável e interessado com relação ao conteúdo matemático. Possui um raciocínio lógico muito bom, que continua desenvolvendo. Trabalha muito bem individualmente, mas também se relaciona bem em trabalho de grupo.\\
\\
\textbf{Opinião sobre seu possível aproveitamento, se aceito no Programa:}
\\O candidato sempre demonstrou responsabilidade, seriedade e interesse na carreira. Tenho certeza que o candidato se dedicará ao programa e irá terminar o mesmo.\\ 
\\
\textbf{Outras informações relevantes:} \\O candidato será uma grande aquisição para o programa.
\\[0.3cm]
\textbf{Entre os estudantes que já conheceu, você diria que o candidato está entre os:}
\\
\begin{tabular}{|l|c|c|c|c|c|}
\hline
 & 5\% melhores & 10\% melhores & 25\% melhores & 50\% melhores & Não sabe \\
\hline
Como aluno, em aulas & X &  &  &  & \\
\hline
Como orientando & X &  &  &  & \\
\hline
\end{tabular}
\subsection*{Dados Recomendante} 
	Instituição (Institution): Universidade Federal de Santa Maria
\\ 
	Grau acadêmico mais alto obtido: doutor
	\ \ Área: Probability Theory
	\\
	Ano de obtenção deste grau: 1993
	\ \ 
	Instituição de obtenção deste grau : University of Rochester  NY EUA
	\\ 
	Endereço institucional do recomendante: \\ Departamento de Matemática
Centro de Ciências Naturais e Exatas
Universidade Federal de Santa Maria
Avenida Roraima, 1000
Bairro Camobi
97105900 Santa Maria RS	
\begin{figure}[!htb]
\includegraphics{../upload/1354_2014-05-26_documentos.jpg}
\end{figure}\includepdf[pages={-},offset=35mm 0mm]{../../../upload/1354_2014-05-26_historico.pdf} 
\begin{center}
Anexos.
\end{center}
\end{document}