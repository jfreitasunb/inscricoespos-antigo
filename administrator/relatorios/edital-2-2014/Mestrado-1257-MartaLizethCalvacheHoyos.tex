\documentclass[11pt]{article}
\usepackage{graphicx,color}
\usepackage{pdfpages}
\usepackage[brazil]{babel}
\usepackage[utf8]{inputenc}
\addtolength{\hoffset}{-3cm} \addtolength{\textwidth}{6cm}
\addtolength{\voffset}{-.5cm} \addtolength{\textheight}{1cm}
%%%%%%%%%%%%%%%%%%%%%%%%%%%%%%%%%%%  To use Colors 
\title{\vspace*{-4cm} Ficha de Inscrição: \\Cod: 1257\ \ Marta Lizeth Calvache Hoyos\ \ - \ \ Mestrado 
 }
\date{}

\begin{document}
\maketitle
\vspace*{-1.5cm}
\noindent Data de Nascimento:3/1/1990
\ \ \ Idade: 24   \ \ \ Sexo: Feminino
\\
Naturalidade: Bolivar Cauca  
\ \ \  Estado: Outro
\ \ \  Nacionalidade: Colombiana
\ \ \ País: Colombia
\\        
Nome do pai : Fausto Emilio Calvache 
\ \ \ Nome da mãe: Libia Hoyos Bolaos          
\\[0.2cm]                     
\textbf{Endereço Pessoal} 
\\ 
\noindent Endereço residencial: Calle 26 EN N. 4A37
\\
        CEP: 19002-472 
\ \ \ Cidade: Popayán 
\ \ \ Estado: Outro 
\ \ \ País: Colombia
\\		
		Telefone comercial : +57(57)32062415
\ \ \ Telefone residencial: +57(57)232392
\ \ \ Telefone celular : +57(57)320624159
\\
E-mail principal: lizethcalvache235@gmail.com
\ \ \ E-mail alternativo: lizeth632@hotmail.com 
\\[0.2cm] 
\textbf{Documentos Pessoais}
\\
\noindent Número de CPF : 0
\ \ \ Número de Identidade (ou Passaporte para estrangeiros): 1058968227
\\
Orgão emissor: re
\ \ \ Estado: Outro
\ \ \ Data de emissão :18/3/2008
\\[0.3cm]
\textbf{Grau acadêmico mais alto obtido}
\\	
Curso:Matemática
\ \ \ Grau : outro
\ \ \ Instituição : Universidad del Cauca  Colombia
\\			
Ano de Conclusão ou Previsão: 2014
\\ 
Experiência Profissional mais recente. \ \  
Tem experiência: Docente Discente  
\ \ \ Instituição: Centro de Soluciones Educativas
\\  
Período - início: 1-2014
\ \ \ fim: 1-2014
\\[0.2cm] 
\textbf{Programa Pretendido:} Mestrado\\
Interesse em bolsa: Sim
\\[0.3cm]		
\textbf{Dados dos Recomendantes} 
\\
1- Nome: Francisco Eduardo Belalcazar
\ \ \ \  e-mail: enriquezfrank@gmail.com 
\\
2- Nome: Willy Will Sierra Arroyo
\ \ \ \ e-mail: wsierra19@gmail.com
\\
3- Nome: Carlos Alberto Trujillo
\ \ \ \ e-mail: trujillo@unicauca.edu.co
\\[0.2cm]
Motivação e expectativa do candidato em relação ao programa pretendido:
\\Cuando empecé a entender mi carrera y sus objetivos me planteé como meta continuar con mis estudios en matemáticas y enfocarme en tener un mejor perfil académico.Encontrar nuevas herramientas que me permitan aplicar los conocimientos adquiridos en el tiempo de estudio consolidarme profesionalmente y perseguir oportunidades de desarrollo personal y profesional son propósitos primordiales en este momento en mi vida. Desenvolverme ofrecer y desarrollar todas mis capacidades innatas consiguiendo un excelente resultado en mis estudios además adquirir una experiencia inigualable con la cual conseguiré un paso importante en mi superación personal junto con el orgullo de hacer parte de la Universidad de Brasilia catalogada una de las mejores universidades del país. Asimismo sería un privilegio para mí poder seguir con mis estudios de maestría en Brasil debido al alto nivel académico de sus universidades. Además me gustaría aprovechar las oportunidades de becas que ofrece la universidad de Brasilia puesto que en mi país Colombia es difícil continuar con estudios de maestría debido al alto costo. Mis expectativas con respecto a la maestría en matemáticas son profundizar mis conocimientos en análisis y topología puesto que he asistido a varios seminarios del grupo de Análisis funcional de la universidad del Cauca los cuales han despertado mi interés por dichas áreas motivo por le cual decidí enfocar mi trabajo de grado en análisis funcional. Considero estoy totalmente capacitada para adquirir conocimientos a nivel de maestría y con la plena disposición para dedicar todo mi esfuerzo y así tener un excelente rendimiento. Por las razones anteriormente mencionadas deseo realizar  la maestría en matemáticas que ofrece la universidad de Brasilia. Atentamente
Marta Lizeth Calvache Hoyos \newpage\vspace*{-4cm}\subsection*{Carta de Recomendação - Francisco Eduardo Belalcazar}Código Identificador: 1294\\Conhece-o candidato há quanto tempo (For how long have you known the applicant)? 
\ 3 anos 8 meses
\\ Conhece-o sob as seguintes circunstâncias: aulas\ \ orientacao
	\ \ seminarios\ \ outra 
\\ Conheçe o candidato sob outras circunstâncias: Como monitora de los cursos basicos de matematicas en la Universidad del Cauca
\\	Avaliações:\\
\begin{tabular}{|l|c|c|c|c|c|}
\hline
 & Excelente & Bom & Regular & Insuficiente & Não sabe \\
\hline
Desempenho acadêmico & X &  &  &  & \\
\hline
Capacidade de aprender novos conceitos & X &  &  &  & \\
\hline
Capacidade de trabalhar sozinho & X &  &  &  & \\
\hline
Criatividade &  & X &  &  & \\
\hline
Curiosidade & X &  &  &  & \\
\hline
Esforço, persistência & X &  &  &  & \\
\hline
Expressão escrita & X &  &  &  & \\
\hline
Expressão oral &  & X &  &  & \\
\hline
Relacionamento com colegas & X &  &  &  & \\
\hline
\end{tabular}\\
\\
\textbf{Opinião sobre os antecedentes acadêmicos, profissionais e/ou técnicos do candidato:}
\\La candidata posee muy buena formación matematica. En los cursos de análisis que le oriente, se destaco por su capacidad para captar nuevos conceptos y mostro curiosidad e interés en temas que para ella significaron un considerable reto intelectual. Como director de su tesis de Grado Introducción a los espacios de Orlicz, destaco su alto grado de responsabilidad y compromiso con las metas academicas propuestas\\
\\
\textbf{Opinião sobre seu possível aproveitamento, se aceito no Programa:}
\\Las razones expuestas anteriormente me permiten prever que Martha Lizeth podrá culminar exitosamente estudios de maestria, en especial en el área de análisis.\\ 
\\
\textbf{Outras informações relevantes:} \\Martha Lizeth también participo activamente en Seminarios especiales en análisis, que ofrece el Departamento de Matematicas de la Universidad del Cauca. También asistió al cursillo Funciones Generalizadas, orientado en el mes de noviembre de 2013 por el Doctor Burenkov Victor IvanovichCardiff University, U.K. 
\\[0.3cm]
\textbf{Entre os estudantes que já conheceu, você diria que o candidato está entre os:}
\\
\begin{tabular}{|l|c|c|c|c|c|}
\hline
 & 5\% melhores & 10\% melhores & 25\% melhores & 50\% melhores & Não sabe \\
\hline
Como aluno, em aulas &  & X &  &  & \\
\hline
Como orientando & X &  &  &  & \\
\hline
\end{tabular}
\subsection*{Dados Recomendante} 
	Instituição (Institution): Universidad del Cauca   Colombia
\\ 
	Grau acadêmico mais alto obtido: doutor
	\ \ Área: Analisis Matematico   01.01.01 
	\\
	Ano de obtenção deste grau: 2004
	\ \ 
	Instituição de obtenção deste grau : Peooples Friendship University of Russia   Moscow
	\\ 
	Endereço institucional do recomendante: \\ Calle 5 numero 4 70 Universidad del Cauca, Popayán Colombia\newpage\vspace*{-4cm}\subsection*{Carta de Recomendação - Willy Will Sierra Arroyo}Código Identificador: 1295\\Conhece-o candidato há quanto tempo (For how long have you known the applicant)? 
\ 30 meses
\\ Conhece-o sob as seguintes circunstâncias: aulas\ \ 
	\ \ seminarios\ \ outra 
\\ Conheçe o candidato sob outras circunstâncias: Evaluador de su trabajo de grado
\\Avaliações: \\
\begin{tabular}{|l|c|c|c|c|c|}
\hline
 & Excelente & Bom & Regular & Insuficiente & Não sabe \\
\hline
Desempenho acadêmico & X &  &  &  & \\
\hline
Capacidade de aprender novos conceitos & X &  &  &  & \\
\hline
Capacidade de trabalhar sozinho & X &  &  &  & \\
\hline
Criatividade &  & X &  &  & \\
\hline
Curiosidade &  & X &  &  & \\
\hline
Esforço, persistência &  & X &  &  & \\
\hline
Expressão escrita & X &  &  &  & \\
\hline
Expressão oral & X &  &  &  & \\
\hline
Relacionamento com colegas &  &  &  &  & X\\
\hline
\end{tabular}\\
\\
\textbf{Opinião sobre os antecedentes acadêmicos, profissionais e/ou técnicos do candidato:}
\\Los antecedentes académicos de Marta son lo bastante sólidos como para tener un buen rendimiento en el programa al cual aspira. Marta en nuestro pregrado ha cursado y aprobado con buenas notas cursos esenciales para su formación, entre los cuales yo le he orientado Variable Compleja, Análisis III, el cual es un curso de introducción a la teoría de la medida, y Fundamentos de Análisis. Este último es un curso básico de nuestro programa de maestría. Además de lo anterior, fui parte del comité de seguimiento de su trabajo de grado, Introducción a los espacios de Órlicz, donde Marta demostró independencia para estudiar y entender temas que no se tratan en ninguno de los cursos de nuestro programa de Matemáticas.

De otro lado, Marta recientemente inició su actividad profesional y fue monitora de algunos cursos, pero no tengo conocimiento para evaluar su trabajo profesional.\\
\\
\textbf{Opinião sobre seu possível aproveitamento, se aceito no Programa:}
\\Considero que Marta, dada su formación y la independencia que ha demostrado tener, tiene el potencial suficiente para afrontar con éxito sus estudios a nivel de postgrado.\\ 
\\
\textbf{Outras informações relevantes:} \\
\\[0.3cm]
\textbf{Entre os estudantes que já conheceu, você diria que o candidato está entre os:}
\\
\begin{tabular}{|l|c|c|c|c|c|}
\hline
 & 5\% melhores & 10\% melhores & 25\% melhores & 50\% melhores & Não sabe \\
\hline
Como aluno, em aulas & X &  &  &  & \\
\hline
Como orientando &  &  &  &  & X\\
\hline
\end{tabular}
\subsection*{Dados Recomendante} 
	Instituição (Institution): Universidad del Cauca
\\ 
	Grau acadêmico mais alto obtido: doutor
	\ \ Área: Matemáticas, Teoría Geométrica de Funciones.
	\\
	Ano de obtenção deste grau: 2010
	\ \ 
	Instituição de obtenção deste grau : Pontificia Universidad Católica de Chile
	\\ 
	Endereço institucional do recomendante: \\ Calle 5, carrera 4 número 70. Departamento de Matemáticas, Universidad del Cauca, Popayán, Colombia\newpage\vspace*{-4cm}\subsection*{Carta de Recomendação - Carlos Alberto Trujillo}Código Identificador: 1296\\Conhece-o candidato há quanto tempo (For how long have you known the applicant)? 
\ two years
\\ Conhece-o sob as seguintes circunstâncias: aulas\ \ 
	\ \ seminarios\ \  
\\ Conheçe o candidato sob outras circunstâncias: 
\\Avaliações: \\
\begin{tabular}{|l|c|c|c|c|c|}
\hline
 & Excelente & Bom & Regular & Insuficiente & Não sabe \\
\hline
Desempenho acadêmico &  & X &  &  & \\
\hline
Capacidade de aprender novos conceitos & X &  &  &  & \\
\hline
Capacidade de trabalhar sozinho & X &  &  &  & \\
\hline
Criatividade &  & X &  &  & \\
\hline
Curiosidade &  & X &  &  & \\
\hline
Esforço, persistência & X &  &  &  & \\
\hline
Expressão escrita &  & X &  &  & \\
\hline
Expressão oral &  & X &  &  & \\
\hline
Relacionamento com colegas & X &  &  &  & \\
\hline
\end{tabular}\\
\\
\textbf{Opinião sobre os antecedentes acadêmicos, profissionais e/ou técnicos do candidato:}
\\Marta es disciplinada, se desenvuelve muy en trabajo individual y en equipo. Ha sido una de mis mejores estudiantes en el curso de Teoria de Numeros Algebraica en el 2013. Ha tenido buen rendimiento en sus cursos de pregrado. Como orientadora de monitorias ha tenido exito y en seminarios y congresos ha mostrado interes por continuar estudiando.\\
\\
\textbf{Opinião sobre seu possível aproveitamento, se aceito no Programa:}
\\No dudo en recomendarla como excelente candidata a realizar estudios de maestria en matematicas, con seguridad tendra exito.\\ 
\\
\textbf{Outras informações relevantes:} \\
\\[0.3cm]
\textbf{Entre os estudantes que já conheceu, você diria que o candidato está entre os:}
\\
\begin{tabular}{|l|c|c|c|c|c|}
\hline
 & 5\% melhores & 10\% melhores & 25\% melhores & 50\% melhores & Não sabe \\
\hline
Como aluno, em aulas & X &  &  &  & \\
\hline
Como orientando & X &  &  &  & \\
\hline
\end{tabular}
\subsection*{Dados Recomendante} 
	Instituição (Institution): UNIVERSIDAD DEL CAUCA
\\ 
	Grau acadêmico mais alto obtido: doutor
	\ \ Área: MATEMATICAS
	\\
	Ano de obtenção deste grau: 1998
	\ \ 
	Instituição de obtenção deste grau : UNIVERSIDAD POLITECNICA DE MADRID
	\\ 
	Endereço institucional do recomendante: \\ DEPARTAMENTO DE MATEMATICAS, UNIVERSIDAD DEL CAUCA, POPAYAN, COLOMBIA, CALLE 5 No. 4 70\includepdf[pages={-},offset=35mm 0mm]{../../../upload/1257_2014-05-19_documentos.pdf}\includepdf[pages={-},offset=35mm 0mm]{../../../upload/1257_2014-05-19_historico.pdf} 
\begin{center}
Anexos.
\end{center}
\end{document}