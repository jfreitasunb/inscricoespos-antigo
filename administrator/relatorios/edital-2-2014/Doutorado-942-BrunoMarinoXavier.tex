\documentclass[11pt]{article}
\usepackage{graphicx,color}
\usepackage{pdfpages}
\usepackage[brazil]{babel}
\usepackage[utf8]{inputenc}
\addtolength{\hoffset}{-3cm} \addtolength{\textwidth}{6cm}
\addtolength{\voffset}{-.5cm} \addtolength{\textheight}{1cm}
%%%%%%%%%%%%%%%%%%%%%%%%%%%%%%%%%%%  To use Colors 
\title{\vspace*{-4cm} Ficha de Inscrição: \\Cod: 942\ \ Bruno Marino Xavier\ \ - \ \ Doutorado 
 }
\date{}

\begin{document}
\maketitle
\vspace*{-1.5cm}
\noindent Data de Nascimento:12/6/1989
\ \ \ Idade: 25   \ \ \ Sexo: Masculino
\\
Naturalidade: Braziliense  
\ \ \  Estado: DF
\ \ \  Nacionalidade: Brasileira
\ \ \ País: Brasil
\\        
Nome do pai : José Alípio de Souza Xavier
\ \ \ Nome da mãe: Norma Sueli Marino Alves          
\\[0.2cm]                     
\textbf{Endereço Pessoal} 
\\ 
\noindent Endereço residencial: SQN 108, bl H, apto 403
\\
        CEP: 70744-080 
\ \ \ Cidade: Brasília 
\ \ \ Estado: DF 
\ \ \ País: Brasil
\\		
		Telefone comercial : +55(61)32749119
\ \ \ Telefone residencial: +55(61)32749119
\ \ \ Telefone celular : +55(61)81239539
\\
E-mail principal: brunocqd@gmail.com
\ \ \ E-mail alternativo: 0 
\\[0.2cm] 
\textbf{Documentos Pessoais}
\\
\noindent Número de CPF : 73663042120
\ \ \ Número de Identidade (ou Passaporte para estrangeiros): 374890018
\\
Orgão emissor: SSP
\ \ \ Estado: DF
\ \ \ Data de emissão :1/8/2001
\\[0.3cm]
\textbf{Grau acadêmico mais alto obtido}
\\	
Curso:Matemática
\ \ \ Grau : mestre
\ \ \ Instituição : Universidade de Brasília
\\			
Ano de Conclusão ou Previsão: 2014
\\ 
Experiência Profissional mais recente. \ \  
Tem experiência: 0 Discente  
\ \ \ Instituição: Universidade de Brasília
\\  
Período - início: 2-2012
\ \ \ fim: 0-2014
\\[0.2cm] 
\textbf{Programa Pretendido:} Doutorado\ \ \ \textbf{Área:} GeometriaDiferencial\\
Interesse em bolsa: Sim
\\[0.3cm]		
\textbf{Dados dos Recomendantes} 
\\
1- Nome: Pedro Roitman
\ \ \ \  e-mail: roitman@mat.unb.br 
\\
2- Nome: Nilton Moura Barroso Neto
\ \ \ \ e-mail: n.m.b.neto@mat.unb.br
\\
3- Nome: Carlos Maber Carrión Riveros
\ \ \ \ e-mail: carlos@mat.unb.br
\\[0.2cm]
Motivação e expectativa do candidato em relação ao programa pretendido:
\\Ao longo do programa de mestrado desenvolvi gosto pela Geometria, cursando as disciplinas Geometria Diferencial 2, Geometria Riemanniana, Variedades Diferenciáveis e Tópicos em Geometria Diferencial. Nesse período, pesquisei temas relacionados à Geometria a fim de expor em algumas dessas disciplinas. Ainda, sob orientação do professor Pedro Roitman, pesquisamos assuntos relacionados à dissertação de Mestrado. Essas atividades me motivaram a continuar o estudo e a pesquisa. No programa de Doutorado, pretendo dar continuidade a essas práticas e atuar no meio acadêmico.\newpage\vspace*{-4cm}\subsection*{Carta de Recomendação - Pedro Roitman}Código Identificador: 1306\\Conhece-o candidato há quanto tempo (For how long have you known the applicant)? 
\ 2 anos
\\ Conhece-o sob as seguintes circunstâncias: aulas\ \ orientacao
	\ \ \ \  
\\ Conheçe o candidato sob outras circunstâncias: 
\\	Avaliações:\\
\begin{tabular}{|l|c|c|c|c|c|}
\hline
 & Excelente & Bom & Regular & Insuficiente & Não sabe \\
\hline
Desempenho acadêmico &  & X &  &  & \\
\hline
Capacidade de aprender novos conceitos &  & X &  &  & \\
\hline
Capacidade de trabalhar sozinho & X &  &  &  & \\
\hline
Criatividade &  & X &  &  & \\
\hline
Curiosidade & X &  &  &  & \\
\hline
Esforço, persistência & X &  &  &  & \\
\hline
Expressão escrita & X &  &  &  & \\
\hline
Expressão oral & X &  &  &  & \\
\hline
Relacionamento com colegas & X &  &  &  & \\
\hline
\end{tabular}\\
\\
\textbf{Opinião sobre os antecedentes acadêmicos, profissionais e/ou técnicos do candidato:}
\\Conheço o candidato como aluno de 2 cursos de pós graduação no MATUNB,
considero que ele tem uma boa formação e está bem preparado para iniciar um doutorado.
O candidato está finalizando a sua dissertação de mestrado sob minha orientação e pude constatar a sua motivação, capacidade e maturidade matemática ao longo da orientação.\\
\\
\textbf{Opinião sobre seu possível aproveitamento, se aceito no Programa:}
\\Creio que o candidato tem plenas condições de realizar um doutorado em matemática. O seu histórico como aluno do PET e mestrado indicam que ele deve ter um bom desempenho. \\ 
\\
\textbf{Outras informações relevantes:} \\Em sua dissertação o candidato conseguiu alguns resultados originais que podem servir de base para a sua tese de doutorado. O candidato se mostrou interessado em contituar as investigações sob a minha orientação. 
\\[0.3cm]
\textbf{Entre os estudantes que já conheceu, você diria que o candidato está entre os:}
\\
\begin{tabular}{|l|c|c|c|c|c|}
\hline
 & 5\% melhores & 10\% melhores & 25\% melhores & 50\% melhores & Não sabe \\
\hline
Como aluno, em aulas &  & X &  &  & \\
\hline
Como orientando &  & X &  &  & \\
\hline
\end{tabular}
\subsection*{Dados Recomendante} 
	Instituição (Institution): Universidade de Brasília
\\ 
	Grau acadêmico mais alto obtido: doutor
	\ \ Área: Matemática
	\\
	Ano de obtenção deste grau: 2002
	\ \ 
	Instituição de obtenção deste grau : Universidade Paris 7
	\\ 
	Endereço institucional do recomendante: \\ Departamento de Matemática UnB\newpage\vspace*{-4cm}\subsection*{Carta de Recomendação - Nilton Moura Barroso Neto}Código Identificador: 1307\\Conhece-o candidato há quanto tempo (For how long have you known the applicant)? 
\ 1 ano
\\ Conhece-o sob as seguintes circunstâncias: aulas\ \ 
	\ \ \ \  
\\ Conheçe o candidato sob outras circunstâncias: 
\\Avaliações: \\
\begin{tabular}{|l|c|c|c|c|c|}
\hline
 & Excelente & Bom & Regular & Insuficiente & Não sabe \\
\hline
Desempenho acadêmico &  & X &  &  & \\
\hline
Capacidade de aprender novos conceitos &  & X &  &  & \\
\hline
Capacidade de trabalhar sozinho & X &  &  &  & \\
\hline
Criatividade &  & X &  &  & \\
\hline
Curiosidade & X &  &  &  & \\
\hline
Esforço, persistência & X &  &  &  & \\
\hline
Expressão escrita & X &  &  &  & \\
\hline
Expressão oral & X &  &  &  & \\
\hline
Relacionamento com colegas & X &  &  &  & \\
\hline
\end{tabular}\\
\\
\textbf{Opinião sobre os antecedentes acadêmicos, profissionais e/ou técnicos do candidato:}
\\Durante a disciplina de variedades diferenciáveis, que ministrei como parte da programação das atividades no verão de 2014, o aluno Bruno Xavier demonstrou grande interesse e boa desenvoltura em relação aos tópicos abordados no curso. Além disso obteve um bom aproveitamento nas avaliações e mostrou bastante dedicação à disciplina.\\
\\
\textbf{Opinião sobre seu possível aproveitamento, se aceito no Programa:}
\\Considero que as características demonstradas pelo candidato têm grande valor para a formação de um bom matemático e acredito que com maior aprofundamento em seus estudos o aluno poderá desenvolver ainda mais esses atributos.\\ 
\\
\textbf{Outras informações relevantes:} \\
\\[0.3cm]
\textbf{Entre os estudantes que já conheceu, você diria que o candidato está entre os:}
\\
\begin{tabular}{|l|c|c|c|c|c|}
\hline
 & 5\% melhores & 10\% melhores & 25\% melhores & 50\% melhores & Não sabe \\
\hline
Como aluno, em aulas &  &  & X &  & \\
\hline
Como orientando &  &  &  &  & X\\
\hline
\end{tabular}
\subsection*{Dados Recomendante} 
	Instituição (Institution): Universidade de Brasília
\\ 
	Grau acadêmico mais alto obtido: doutor
	\ \ Área: Geometria
	\\
	Ano de obtenção deste grau: 2011
	\ \ 
	Instituição de obtenção deste grau : Universidade de Brasília
	\\ 
	Endereço institucional do recomendante: \\ Universidade de Brasília, Campus Darcy Ribeiro, Departamento de Matemática, 70910900, Brasília, DF.\newpage\vspace*{-4cm}\subsection*{Carta de Recomendação - Carlos Maber Carrión Riveros}Código Identificador: 1308\\Conhece-o candidato há quanto tempo (For how long have you known the applicant)? 
\ um ano
\\ Conhece-o sob as seguintes circunstâncias: aulas\ \ 
	\ \ \ \  
\\ Conheçe o candidato sob outras circunstâncias: 
\\Avaliações: \\
\begin{tabular}{|l|c|c|c|c|c|}
\hline
 & Excelente & Bom & Regular & Insuficiente & Não sabe \\
\hline
Desempenho acadêmico &  & X &  &  & \\
\hline
Capacidade de aprender novos conceitos &  & X &  &  & \\
\hline
Capacidade de trabalhar sozinho &  & X &  &  & \\
\hline
Criatividade &  & X &  &  & \\
\hline
Curiosidade &  & X &  &  & \\
\hline
Esforço, persistência & X &  &  &  & \\
\hline
Expressão escrita &  & X &  &  & \\
\hline
Expressão oral &  & X &  &  & \\
\hline
Relacionamento com colegas &  & X &  &  & \\
\hline
\end{tabular}\\
\\
\textbf{Opinião sobre os antecedentes acadêmicos, profissionais e/ou técnicos do candidato:}
\\O candidato foi meu aluno na disciplina Tópicos de Geometria, ele teve bom desempenho, apresentou muito interesse em apreender coisas novas. Participou muito durante as aulas e mostrou madurez nos seus conhecimentos.\\
\\
\textbf{Opinião sobre seu possível aproveitamento, se aceito no Programa:}
\\Acredito que o candidato tem condições para concluir satisfatoriamente o programa de doutorado, se for aceito no programa.\\ 
\\
\textbf{Outras informações relevantes:} \\
\\[0.3cm]
\textbf{Entre os estudantes que já conheceu, você diria que o candidato está entre os:}
\\
\begin{tabular}{|l|c|c|c|c|c|}
\hline
 & 5\% melhores & 10\% melhores & 25\% melhores & 50\% melhores & Não sabe \\
\hline
Como aluno, em aulas &  & X &  &  & \\
\hline
Como orientando &  &  &  &  & X\\
\hline
\end{tabular}
\subsection*{Dados Recomendante} 
	Instituição (Institution): Universidade de Brasília
\\ 
	Grau acadêmico mais alto obtido: doutor
	\ \ Área: Geometria Diferencial
	\\
	Ano de obtenção deste grau: 2001
	\ \ 
	Instituição de obtenção deste grau : Universidade de Brasília
	\\ 
	Endereço institucional do recomendante: \\ Universidade de Brasília
Departamento de Matemática\includepdf[pages={-},offset=35mm 0mm]{../../../upload/942_2014-05-19_historico.pdf}\includepdf[pages={-},offset=35mm 0mm]{../../../upload/942_2013-11-18_documentos.pdf}\includepdf[pages={-},offset=35mm 0mm]{../../../upload/942_2013-11-18_historico.pdf} 
\begin{center}
Anexos.
\end{center}
\end{document}