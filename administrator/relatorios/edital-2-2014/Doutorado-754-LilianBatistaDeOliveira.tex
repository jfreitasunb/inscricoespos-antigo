\documentclass[11pt]{article}
\usepackage{graphicx,color}
\usepackage{pdfpages}
\usepackage[brazil]{babel}
\usepackage[utf8]{inputenc}
\addtolength{\hoffset}{-3cm} \addtolength{\textwidth}{6cm}
\addtolength{\voffset}{-.5cm} \addtolength{\textheight}{1cm}
%%%%%%%%%%%%%%%%%%%%%%%%%%%%%%%%%%%  To use Colors 
\title{\vspace*{-4cm} Ficha de Inscrição: \\Cod: 754\ \ LILIAN BATISTA DE OLIVEIRA\ \ - \ \ Doutorado 
 }
\date{}

\begin{document}
\maketitle
\vspace*{-1.5cm}
\noindent Data de Nascimento:20/11/1987
\ \ \ Idade: 26   \ \ \ Sexo: Feminino
\\
Naturalidade: BELO HORIZONTE  
\ \ \  Estado: MG
\ \ \  Nacionalidade: BRASILEIRA
\ \ \ País: BRASIL
\\        
Nome do pai : JOSE LUCIO DE OLIVEIRA
\ \ \ Nome da mãe: BERNARDETE BATISTA DA SILVA          
\\[0.2cm]                     
\textbf{Endereço Pessoal} 
\\ 
\noindent Endereço residencial: rua bucareste
\\
        CEP: 31620-590 
\ \ \ Cidade: Belo horizonte 
\ \ \ Estado: MG 
\ \ \ País: BRASIL
\\		
		Telefone comercial : +0(31)34588765
\ \ \ Telefone residencial: +0(31)34588765
\ \ \ Telefone celular : +0(31)93481038
\\
E-mail principal: liubatista20@gmail.com
\ \ \ E-mail alternativo: liubatista\_20@hotmail.com 
\\[0.2cm] 
\textbf{Documentos Pessoais}
\\
\noindent Número de CPF : 07657075600
\ \ \ Número de Identidade (ou Passaporte para estrangeiros): 14504025
\\
Orgão emissor: ssp
\ \ \ Estado: MG
\ \ \ Data de emissão :4/12/2002
\\[0.3cm]
\textbf{Grau acadêmico mais alto obtido}
\\	
Curso:Matemática
\ \ \ Grau : mestre
\ \ \ Instituição : Universidade Federal de Minas Gerais
\\			
Ano de Conclusão ou Previsão: 2014
\\ 
Experiência Profissional mais recente. \ \  
Tem experiência: 0 Discente  
\ \ \ Instituição: UFMG
\\  
Período - início: 2-2012
\ \ \ fim: 2-2013
\\[0.2cm] 
\textbf{Programa Pretendido:} Doutorado\ \ \ \textbf{Área:} TeoriaDosNumeros\\
Interesse em bolsa: Sim
\\[0.3cm]		
\textbf{Dados dos Recomendantes} 
\\
1- Nome: Fábio Enrique Brochero Martinez
\ \ \ \  e-mail: fbrocher@mat.ufmg.br 
\\
2- Nome: Ana Cristina Vieira
\ \ \ \ e-mail: anacris@mat.ufmg.br
\\
3- Nome: Csaba Schneider
\ \ \ \ e-mail: csaba@mat.ufmg.br
\\[0.2cm]
Motivação e expectativa do candidato em relação ao programa pretendido:
\\Fiz meu mestrado na UFMG formalmente na área de álgebra, durante os estudos os resultados foram exigindo algum estudo de teoria dos números atrelado ao conhecimento de álgebra de grupos. O trabalho da dissertação de mestrado nos rendeu um artigo,Explicit factorization of xn1Fq, que pode ser visto em  httparxiv.orgabs1404.6281 e foi aceito em  Designs, Codes and Cryptography.
A partir dai o interesse nessa área de códigos, utilizando técnicas de teoria dos números se tornou cada vez mais interessante.
A UnB possui excelentes algebristas e professores que trabalham na área de teoria dos números. Acredito que estarei indo para o lugar certo afim de desenvolver melhor meu trabalho, com um grupo forte e sólido para me apoiar.
Atualmente estou matriculada no doutorado da UFMG, mas as bolsas estão escassas por aqui e gostaria de me dedicar completamente ao doutorado. 
Meu currículo lattes pode ser encontrado em  httplattes.cnpq.br4154198839676799 .\newpage\vspace*{-4cm}\subsection*{Carta de Recomendação - Fábio Enrique Brochero Martinez}Código Identificador: 829\\Conhece-o candidato há quanto tempo (For how long have you known the applicant)? 
\ 
\\ Conhece-o sob as seguintes circunstâncias: \ \ 
	\ \ \ \  
\\ Conheçe o candidato sob outras circunstâncias: 
\\	Avaliações:\\
\begin{tabular}{|l|c|c|c|c|c|}
\hline
 & Excelente & Bom & Regular & Insuficiente & Não sabe \\
\hline
Desempenho acadêmico &  &  &  &  & \\
\hline
Capacidade de aprender novos conceitos &  &  &  &  & \\
\hline
Capacidade de trabalhar sozinho &  &  &  &  & \\
\hline
Criatividade &  &  &  &  & \\
\hline
Curiosidade &  &  &  &  & \\
\hline
Esforço, persistência &  &  &  &  & \\
\hline
Expressão escrita &  &  &  &  & \\
\hline
Expressão oral &  &  &  &  & \\
\hline
Relacionamento com colegas &  &  &  &  & \\
\hline
\end{tabular}\\
\\
\textbf{Opinião sobre os antecedentes acadêmicos, profissionais e/ou técnicos do candidato:}
\\\\
\\
\textbf{Opinião sobre seu possível aproveitamento, se aceito no Programa:}
\\\\ 
\\
\textbf{Outras informações relevantes:} \\
\\[0.3cm]
\textbf{Entre os estudantes que já conheceu, você diria que o candidato está entre os:}
\\
\begin{tabular}{|l|c|c|c|c|c|}
\hline
 & 5\% melhores & 10\% melhores & 25\% melhores & 50\% melhores & Não sabe \\
\hline
Como aluno, em aulas &  &  &  &  & \\
\hline
Como orientando &  &  &  &  & \\
\hline
\end{tabular}
\subsection*{Dados Recomendante} 
	Instituição (Institution): Universidade Federal de Minas Gerais
\\ 
	Grau acadêmico mais alto obtido: doutor
	\ \ Área: Dinâmica Complexa
	\\
	Ano de obtenção deste grau: 2002
	\ \ 
	Instituição de obtenção deste grau : IMPA
	\\ 
	Endereço institucional do recomendante: \\ Departamento de Matemática
Universidade Federal de Minas Gerais
Av Antonio Carlos 6627
Belo Horizonte, MG
CEP CEP 30161-970 \newpage\vspace*{-4cm}\subsection*{Carta de Recomendação - Ana Cristina Vieira}Código Identificador: 830\\Conhece-o candidato há quanto tempo (For how long have you known the applicant)? 
\ Desde 2012, quando foi minha aluna em uma discipl.
\\ Conhece-o sob as seguintes circunstâncias: aulas\ \ 
	\ \ seminarios\ \  
\\ Conheçe o candidato sob outras circunstâncias: 
\\Avaliações: \\
\begin{tabular}{|l|c|c|c|c|c|}
\hline
 & Excelente & Bom & Regular & Insuficiente & Não sabe \\
\hline
Desempenho acadêmico & X &  &  &  & \\
\hline
Capacidade de aprender novos conceitos & X &  &  &  & \\
\hline
Capacidade de trabalhar sozinho & X &  &  &  & \\
\hline
Criatividade & X &  &  &  & \\
\hline
Curiosidade & X &  &  &  & \\
\hline
Esforço, persistência & X &  &  &  & \\
\hline
Expressão escrita &  & X &  &  & \\
\hline
Expressão oral &  & X &  &  & \\
\hline
Relacionamento com colegas & X &  &  &  & \\
\hline
\end{tabular}\\
\\
\textbf{Opinião sobre os antecedentes acadêmicos, profissionais e/ou técnicos do candidato:}
\\Lilian fez sua graduação e seu Mestrado na UFMG e tem uma ótima formação acadêmica. Sempre demonstrou muito entusiasmo e determinação em seus estudos. Quando foi minha aluna no curso de Álgebra Avançada no mestrado, se saiu muito bem e se destacou entre os melhores alunos.\\
\\
\textbf{Opinião sobre seu possível aproveitamento, se aceito no Programa:}
\\Acredito que Lilian está preparada para ingressar no doutorado e ter sucesso em sua conclusão. Ela é muito séria e gosta de Matemática, além disso tem instinto investigador. Estas são características importante para um estudante de doutorado.\\ 
\\
\textbf{Outras informações relevantes:} \\Lilian tem qualidades suficientes para o desenvolvimento de uma boa tese doutorado.
\\[0.3cm]
\textbf{Entre os estudantes que já conheceu, você diria que o candidato está entre os:}
\\
\begin{tabular}{|l|c|c|c|c|c|}
\hline
 & 5\% melhores & 10\% melhores & 25\% melhores & 50\% melhores & Não sabe \\
\hline
Como aluno, em aulas &  & X &  &  & \\
\hline
Como orientando &  &  &  &  & X\\
\hline
\end{tabular}
\subsection*{Dados Recomendante} 
	Instituição (Institution): UFMG
\\ 
	Grau acadêmico mais alto obtido: doutor
	\ \ Área: Matemática
	\\
	Ano de obtenção deste grau: 1997
	\ \ 
	Instituição de obtenção deste grau : UnB
	\\ 
	Endereço institucional do recomendante: \\ Departamento de Matemática
ICEx - UFMG
Av. Antonio Carlos 6627
Belo Horizonte - MG\newpage\vspace*{-4cm}\subsection*{Carta de Recomendação - Csaba Schneider}Código Identificador: 896\\Conhece-o candidato há quanto tempo (For how long have you known the applicant)? 
\ 
\\ Conhece-o sob as seguintes circunstâncias: \ \ 
	\ \ \ \  
\\ Conheçe o candidato sob outras circunstâncias: 
\\Avaliações: \\
\begin{tabular}{|l|c|c|c|c|c|}
\hline
 & Excelente & Bom & Regular & Insuficiente & Não sabe \\
\hline
Desempenho acadêmico &  &  &  &  & \\
\hline
Capacidade de aprender novos conceitos &  &  &  &  & \\
\hline
Capacidade de trabalhar sozinho &  &  &  &  & \\
\hline
Criatividade &  &  &  &  & \\
\hline
Curiosidade &  &  &  &  & \\
\hline
Esforço, persistência &  &  &  &  & \\
\hline
Expressão escrita &  &  &  &  & \\
\hline
Expressão oral &  &  &  &  & \\
\hline
Relacionamento com colegas &  &  &  &  & \\
\hline
\end{tabular}\\
\\
\textbf{Opinião sobre os antecedentes acadêmicos, profissionais e/ou técnicos do candidato:}
\\\\
\\
\textbf{Opinião sobre seu possível aproveitamento, se aceito no Programa:}
\\\\ 
\\
\textbf{Outras informações relevantes:} \\
\\[0.3cm]
\textbf{Entre os estudantes que já conheceu, você diria que o candidato está entre os:}
\\
\begin{tabular}{|l|c|c|c|c|c|}
\hline
 & 5\% melhores & 10\% melhores & 25\% melhores & 50\% melhores & Não sabe \\
\hline
Como aluno, em aulas &  &  &  &  & \\
\hline
Como orientando &  &  &  &  & \\
\hline
\end{tabular}
\subsection*{Dados Recomendante} 
	Instituição (Institution): UFMG
\\ 
	Grau acadêmico mais alto obtido: doutor
	\ \ Área: álgebra
	\\
	Ano de obtenção deste grau: 2000
	\ \ 
	Instituição de obtenção deste grau : The Australian National University
	\\ 
	Endereço institucional do recomendante: \\ Departamento de Matemática
Instituto de Ciências Exatas
Universidade Federal de Minas Gerais
Belo Horizonte MG\includepdf[pages={-},offset=35mm 0mm]{../../../upload/754_2014-05-30_historico.pdf}\includepdf[pages={-},offset=35mm 0mm]{../../../upload/754_2013-10-16_documentos.pdf}\includepdf[pages={-},offset=35mm 0mm]{../../../upload/754_2013-10-16_historico.pdf} 
\begin{center}
Anexos.
\end{center}
\end{document}