\documentclass[11pt]{article}
\usepackage{graphicx,color}
\usepackage{pdfpages}
\usepackage[brazil]{babel}
\usepackage[utf8]{inputenc}
\addtolength{\hoffset}{-3cm} \addtolength{\textwidth}{6cm}
\addtolength{\voffset}{-.5cm} \addtolength{\textheight}{1cm}
%%%%%%%%%%%%%%%%%%%%%%%%%%%%%%%%%%%  To use Colors 
\title{\vspace*{-4cm} Ficha de Inscrição: \\Cod: 1176\ \ Filipe Balduino Pires Fernandes\ \ - \ \ Mestrado 
 }
\date{}

\begin{document}
\maketitle
\vspace*{-1.5cm}
\noindent Data de Nascimento:23/1/1987
\ \ \ Idade: 27   \ \ \ Sexo: Masculino
\\
Naturalidade: Brasiliense  
\ \ \  Estado: DF
\ \ \  Nacionalidade: Brasileiro
\ \ \ País: Brasil
\\        
Nome do pai : Juvenal Fernandes dos Santos
\ \ \ Nome da mãe: Elza Balduino Pires          
\\[0.2cm]                     
\textbf{Endereço Pessoal} 
\\ 
\noindent Endereço residencial: SQS 411 BLOCO D APTO 208
\\
        CEP: 70277-040 
\ \ \ Cidade: BRASILIA 
\ \ \ Estado: DF 
\ \ \ País: BRASIL
\\		
		Telefone comercial : +0(61)33496501
\ \ \ Telefone residencial: +55(61)3349501
\ \ \ Telefone celular : +55(61)81041112
\\
E-mail principal: filipematunb@hotmail.com.br
\ \ \ E-mail alternativo: skyywayy@hotmail.com 
\\[0.2cm] 
\textbf{Documentos Pessoais}
\\
\noindent Número de CPF : 69458898153
\ \ \ Número de Identidade (ou Passaporte para estrangeiros): 2486801
\\
Orgão emissor: SSP
\ \ \ Estado: DF
\ \ \ Data de emissão :15/4/2004
\\[0.3cm]
\textbf{Grau acadêmico mais alto obtido}
\\	
Curso:Matemática
\ \ \ Grau : bacharel
\ \ \ Instituição : Universidade de Brasilia
\\			
Ano de Conclusão ou Previsão: 2013
\\ 
Experiência Profissional mais recente. \ \  
Tem experiência: Docente Discente  
\ \ \ Instituição: Universidade de Brasilia
\\  
Período - início: 1-2009
\ \ \ fim: 2-2013
\\[0.2cm] 
\textbf{Programa Pretendido:} Mestrado\\
Interesse em bolsa: Sim
\\[0.3cm]		
\textbf{Dados dos Recomendantes} 
\\
1- Nome: Aline Gomes da Silva Pinto
\ \ \ \  e-mail: pinto.aline@gmail.com 
\\
2- Nome: Hemar Teixeira Godinho
\ \ \ \ e-mail: h.t.godinho@mat.unb.br
\\
3- Nome: Mauro Luiz Rabelo
\ \ \ \ e-mail: rabelo@unb.br
\\[0.2cm]
Motivação e expectativa do candidato em relação ao programa pretendido:
\\Tenho como objetivo pessoal seguir carreira acadêmica na área de matemática. Com isso, creio que o programa de mestrado ira me propiciar oportunidade de seguir com os estudos.
Espero que o programa me ajude a aprofundar nas áreas básicas da matemática e também exponha o lado da pesquisa atual.\newpage\vspace*{-4cm}\subsection*{Carta de Recomendação - Aline Gomes da Silva Pinto}Código Identificador: 716\\Conhece-o candidato há quanto tempo (For how long have you known the applicant)? 
\ 5 anos
\\ Conhece-o sob as seguintes circunstâncias: aulas\ \ 
	\ \ \ \  
\\ Conheçe o candidato sob outras circunstâncias: 
\\	Avaliações:\\
\begin{tabular}{|l|c|c|c|c|c|}
\hline
 & Excelente & Bom & Regular & Insuficiente & Não sabe \\
\hline
Desempenho acadêmico &  & X &  &  & \\
\hline
Capacidade de aprender novos conceitos &  & X &  &  & \\
\hline
Capacidade de trabalhar sozinho &  & X &  &  & \\
\hline
Criatividade &  & X &  &  & \\
\hline
Curiosidade &  & X &  &  & \\
\hline
Esforço, persistência &  & X &  &  & \\
\hline
Expressão escrita & X &  &  &  & \\
\hline
Expressão oral &  & X &  &  & \\
\hline
Relacionamento com colegas &  &  &  &  & X\\
\hline
\end{tabular}\\
\\
\textbf{Opinião sobre os antecedentes acadêmicos, profissionais e/ou técnicos do candidato:}
\\O Felipe foi meu aluno no 2o semestre de 2009 na disciplina IAL. Foi um dos melhores alunos da turma naquela altura, seu desempenho foi excelente. No semestre seguinte atuou como meu monitor voluntario, desempenhando bom papel. No 1o semestre de 2011 foi meu aluno de Álgebra 1, onde tirou boas notas nas provas, ficando novamente com SS. Meu último contato com o candidato foi na disciplina Álgebra 2, disciplina já do bacharelado, no 1o semestre de 2012. Neste último contato seu desempenho não foi muito bom, sua média final foi 5,45, mostrando uma queda significativa no comparecimento das aulas e nas notas das provas.\\
\\
\textbf{Opinião sobre seu possível aproveitamento, se aceito no Programa:}
\\Pelo contato que teve comigo como aluno, o Felipe se mostrou com muito potencial quando iniciou sua graduação mas teve queda no desempenho ao longo da graduação. Ele apresenta bom potencial para seguir com os estudos caso tenha recuperado a motivação.\\ 
\\
\textbf{Outras informações relevantes:} \\Para que o aluno seja aceito no programa, recomendo sua aceitação, mas ressalto que seria bom analisar também seu desempenho nas outras disciplinas.
\\[0.3cm]
\textbf{Entre os estudantes que já conheceu, você diria que o candidato está entre os:}
\\
\begin{tabular}{|l|c|c|c|c|c|}
\hline
 & 5\% melhores & 10\% melhores & 25\% melhores & 50\% melhores & Não sabe \\
\hline
Como aluno, em aulas &  &  &  & X & \\
\hline
Como orientando &  &  &  & X & \\
\hline
\end{tabular}
\subsection*{Dados Recomendante} 
	Instituição (Institution): UnB
\\ 
	Grau acadêmico mais alto obtido: doutor
	\ \ Área: Álgebra
	\\
	Ano de obtenção deste grau: 2005
	\ \ 
	Instituição de obtenção deste grau : Unicamp
	\\ 
	Endereço institucional do recomendante: \\ UnB, Departamento de Matemática\newpage\vspace*{-4cm}\subsection*{Carta de Recomendação - Hemar Teixeira Godinho}Código Identificador: 794\\Conhece-o candidato há quanto tempo (For how long have you known the applicant)? 
\ 
\\ Conhece-o sob as seguintes circunstâncias: \ \ 
	\ \ \ \  
\\ Conheçe o candidato sob outras circunstâncias: 
\\Avaliações: \\
\begin{tabular}{|l|c|c|c|c|c|}
\hline
 & Excelente & Bom & Regular & Insuficiente & Não sabe \\
\hline
Desempenho acadêmico &  &  &  &  & \\
\hline
Capacidade de aprender novos conceitos &  &  &  &  & \\
\hline
Capacidade de trabalhar sozinho &  &  &  &  & \\
\hline
Criatividade &  &  &  &  & \\
\hline
Curiosidade &  &  &  &  & \\
\hline
Esforço, persistência &  &  &  &  & \\
\hline
Expressão escrita &  &  &  &  & \\
\hline
Expressão oral &  &  &  &  & \\
\hline
Relacionamento com colegas &  &  &  &  & \\
\hline
\end{tabular}\\
\\
\textbf{Opinião sobre os antecedentes acadêmicos, profissionais e/ou técnicos do candidato:}
\\\\
\\
\textbf{Opinião sobre seu possível aproveitamento, se aceito no Programa:}
\\\\ 
\\
\textbf{Outras informações relevantes:} \\
\\[0.3cm]
\textbf{Entre os estudantes que já conheceu, você diria que o candidato está entre os:}
\\
\begin{tabular}{|l|c|c|c|c|c|}
\hline
 & 5\% melhores & 10\% melhores & 25\% melhores & 50\% melhores & Não sabe \\
\hline
Como aluno, em aulas &  &  &  &  & \\
\hline
Como orientando &  &  &  &  & \\
\hline
\end{tabular}
\subsection*{Dados Recomendante} 
	Instituição (Institution): UnB
\\ 
	Grau acadêmico mais alto obtido: doutor
	\ \ Área: Teoria dos Números
	\\
	Ano de obtenção deste grau: 1992
	\ \ 
	Instituição de obtenção deste grau : University of Michigan
	\\ 
	Endereço institucional do recomendante: \\ Departamento de Matemática - UnB\newpage\vspace*{-4cm}\subsection*{Carta de Recomendação - Mauro Luiz Rabelo}Código Identificador: 194\\Conhece-o candidato há quanto tempo (For how long have you known the applicant)? 
\ 4 anos
\\ Conhece-o sob as seguintes circunstâncias: \ \ orientacao
	\ \ \ \  
\\ Conheçe o candidato sob outras circunstâncias: 
\\Avaliações: \\
\begin{tabular}{|l|c|c|c|c|c|}
\hline
 & Excelente & Bom & Regular & Insuficiente & Não sabe \\
\hline
Desempenho acadêmico &  &  & X &  & \\
\hline
Capacidade de aprender novos conceitos &  & X &  &  & \\
\hline
Capacidade de trabalhar sozinho & X &  &  &  & \\
\hline
Criatividade & X &  &  &  & \\
\hline
Curiosidade & X &  &  &  & \\
\hline
Esforço, persistência &  & X &  &  & \\
\hline
Expressão escrita &  & X &  &  & \\
\hline
Expressão oral &  & X &  &  & \\
\hline
Relacionamento com colegas &  & X &  &  & \\
\hline
\end{tabular}\\
\\
\textbf{Opinião sobre os antecedentes acadêmicos, profissionais e/ou técnicos do candidato:}
\\Filipe participou do grupo PET durante parte do período em que fui tutor. Sempre foi dedicado ao programa e praticamente respondia a todas as demandas que lhe eram feitas no grupo. Ele gosta de desafios matemáticos e trabalha com muita persistência. Algumas vezes se descuidava de algumas disciplinas, não por falta de capacidade, mas porque estava interessado em estudar temas de teoria dos números e álgebra, e acabava mergulhado nesses estudos. Isso acabou prejudicando seu histórico. No entanto, amadureceu muito ao longo do curso.\\
\\
\textbf{Opinião sobre seu possível aproveitamento, se aceito no Programa:}
\\Ele tem boas chances de conseguir cursar a pósgraduação com êxito.\\ 
\\
\textbf{Outras informações relevantes:} \\
\\[0.3cm]
\textbf{Entre os estudantes que já conheceu, você diria que o candidato está entre os:}
\\
\begin{tabular}{|l|c|c|c|c|c|}
\hline
 & 5\% melhores & 10\% melhores & 25\% melhores & 50\% melhores & Não sabe \\
\hline
Como aluno, em aulas &  &  & X &  & \\
\hline
Como orientando &  &  & X &  & \\
\hline
\end{tabular}
\subsection*{Dados Recomendante} 
	Instituição (Institution): UnB
\\ 
	Grau acadêmico mais alto obtido: doutor
	\ \ Área: Geometria Diferencial
	\\
	Ano de obtenção deste grau: 1987
	\ \ 
	Instituição de obtenção deste grau : UnB
	\\ 
	Endereço institucional do recomendante: \\ UnB\includepdf[pages={-},offset=35mm 0mm]{../../../upload/1176_2014-05-13_historico.pdf}\includepdf[pages={-},offset=35mm 0mm]{../../../upload/1176_2014-05-12_documentos.pdf}\includepdf[pages={-},offset=35mm 0mm]{../../../upload/1176_2014-05-12_historico.pdf} 
\begin{center}
Anexos.
\end{center}
\end{document}