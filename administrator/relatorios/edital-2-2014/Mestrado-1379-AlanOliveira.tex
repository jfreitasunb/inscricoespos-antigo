\documentclass[11pt]{article}
\usepackage{graphicx,color}
\usepackage{pdfpages}
\usepackage[brazil]{babel}
\usepackage[utf8]{inputenc}
\addtolength{\hoffset}{-3cm} \addtolength{\textwidth}{6cm}
\addtolength{\voffset}{-.5cm} \addtolength{\textheight}{1cm}
%%%%%%%%%%%%%%%%%%%%%%%%%%%%%%%%%%%  To use Colors 
\title{\vspace*{-4cm} Ficha de Inscrição: \\Cod: 1379\ \ Alan Oliveira\ \ - \ \ Mestrado 
 }
\date{}

\begin{document}
\maketitle
\vspace*{-1.5cm}
\noindent Data de Nascimento:5/12/1990
\ \ \ Idade: 23   \ \ \ Sexo: Masculino
\\
Naturalidade: São Paulo  
\ \ \  Estado: SP
\ \ \  Nacionalidade: Brasileiro
\ \ \ País: Brasil
\\        
Nome do pai : Jose Laci de Oliveira
\ \ \ Nome da mãe: Maria Jorvino de Lima          
\\[0.2cm]                     
\textbf{Endereço Pessoal} 
\\ 
\noindent Endereço residencial: QNO 08 Conjunto A Lotes 12
\\
        CEP: 72251-801 
\ \ \ Cidade: Ceilândia 
\ \ \ Estado: DF 
\ \ \ País: Brasil
\\		
		Telefone comercial : +0(83)35314239
\ \ \ Telefone residencial: +0(61)35852325
\ \ \ Telefone celular : +0(61)82818683
\\
E-mail principal: alanchristiansi@gmail.com
\ \ \ E-mail alternativo: 0 
\\[0.2cm] 
\textbf{Documentos Pessoais}
\\
\noindent Número de CPF : 06977490454
\ \ \ Número de Identidade (ou Passaporte para estrangeiros): 3364829
\\
Orgão emissor: SSP
\ \ \ Estado: DF
\ \ \ Data de emissão :28/1/2009
\\[0.3cm]
\textbf{Grau acadêmico mais alto obtido}
\\	
Curso:Gerenciamento de Projeto
\ \ \ Grau : especialista
\ \ \ Instituição : UniCEUB
\\			
Ano de Conclusão ou Previsão: 2014
\\ 
Experiência Profissional mais recente. \ \  
Tem experiência: 0 0  
\ \ \ Instituição: 0
\\  
Período - início: 0-0
\ \ \ fim: 0-0
\\[0.2cm] 
\textbf{Programa Pretendido:} Mestrado\\
Interesse em bolsa: Sim
\\[0.3cm]		
\textbf{Dados dos Recomendantes} 
\\
1- Nome: Leonardo Sousa Silva
\ \ \ \  e-mail: professorleonardosousa@gmail.com 
\\
2- Nome: Roberto Michelan
\ \ \ \ e-mail: roberto.michelan@fapce.edu.br
\\
3- Nome: Paulo Rogério Foina
\ \ \ \ e-mail: paulo.foina@uniceub.br
\\[0.2cm]
Motivação e expectativa do candidato em relação ao programa pretendido:
\\A razão que me impele a esta candidatura está diretamente ligada ao desejo de expandir conhecimentos em ciências exatas, além dos obtidos no bacharelado em sistemas de informação e na especialização em Gerenciamento de projetos, que estou a concluir.
Um dos motivos mais cruciais, de foro pessoal e acadêmico, que me levaram a ter interesse por esse mestrado está ligado a minha afeição pela matemática, assim como pela docência.
Penso que um mestrado em matemática caberia perfeitamente no contexto profissional que me vejo atuando. Oportunidades de trabalho nessa área na cidade e estado de onde venho são inúmeras, sendo esse, outro dos motivos que me motivaram a submeter este pleito.
Estou ciente que esta é um desafio importante para mim mas é, ao mesmo tempo, uma oportunidade rara, a qual, teria orgulho de participar.

Atenciosamente, Alan Christian.\newpage\vspace*{-4cm}\subsection*{Carta de Recomendação - Leonardo Sousa Silva}Código Identificador: 1404\\Conhece-o candidato há quanto tempo (For how long have you known the applicant)? 
\ 4 anos
\\ Conhece-o sob as seguintes circunstâncias: aulas\ \ 
	\ \ \ \  
\\ Conheçe o candidato sob outras circunstâncias: 
\\	Avaliações:\\
\begin{tabular}{|l|c|c|c|c|c|}
\hline
 & Excelente & Bom & Regular & Insuficiente & Não sabe \\
\hline
Desempenho acadêmico & X &  &  &  & \\
\hline
Capacidade de aprender novos conceitos & X &  &  &  & \\
\hline
Capacidade de trabalhar sozinho & X &  &  &  & \\
\hline
Criatividade & X &  &  &  & \\
\hline
Curiosidade & X &  &  &  & \\
\hline
Esforço, persistência & X &  &  &  & \\
\hline
Expressão escrita & X &  &  &  & \\
\hline
Expressão oral &  & X &  &  & \\
\hline
Relacionamento com colegas &  & X &  &  & \\
\hline
\end{tabular}\\
\\
\textbf{Opinião sobre os antecedentes acadêmicos, profissionais e/ou técnicos do candidato:}
\\Excelente aluno, assíduo, responsável, participativo e com ótimo rendimento nas avaliações e trabalhos.\\
\\
\textbf{Opinião sobre seu possível aproveitamento, se aceito no Programa:}
\\Será um aluno responsável, estudioso e que conseguirá acompanhar todo o programa de estudos. Tem uma grande possibilidade de ser destaque na turma.\\ 
\\
\textbf{Outras informações relevantes:} \\
\\[0.3cm]
\textbf{Entre os estudantes que já conheceu, você diria que o candidato está entre os:}
\\
\begin{tabular}{|l|c|c|c|c|c|}
\hline
 & 5\% melhores & 10\% melhores & 25\% melhores & 50\% melhores & Não sabe \\
\hline
Como aluno, em aulas &  &  & X &  & \\
\hline
Como orientando &  &  & X &  & \\
\hline
\end{tabular}
\subsection*{Dados Recomendante} 
	Instituição (Institution): FACULDADE PARAÍSO DO CEARÁ
\\ 
	Grau acadêmico mais alto obtido: mestre
	\ \ Área: ADMINISTRAÇÃO E FINANÇAS
	\\
	Ano de obtenção deste grau: 2013
	\ \ 
	Instituição de obtenção deste grau : UTIC  UNIVERSIDADE TECNOLÓGICA INTERCONTINENTAL 
	\\ 
	Endereço institucional do recomendante: \\ Rua Santa Isabel, número 65.
Bairro São Miguel. Juazeiro do Norte Ceará.\newpage\vspace*{-4cm}\subsection*{Carta de Recomendação - Roberto Michelan}Código Identificador: 1405\\Conhece-o candidato há quanto tempo (For how long have you known the applicant)? 
\ 4 anos
\\ Conhece-o sob as seguintes circunstâncias: aulas\ \ 
	\ \ \ \ outra 
\\ Conheçe o candidato sob outras circunstâncias: Coordenação de Curso Sistemas de Informação / Banca de Trabalho de Conclusão de Curso
\\Avaliações: \\
\begin{tabular}{|l|c|c|c|c|c|}
\hline
 & Excelente & Bom & Regular & Insuficiente & Não sabe \\
\hline
Desempenho acadêmico &  & X &  &  & \\
\hline
Capacidade de aprender novos conceitos &  & X &  &  & \\
\hline
Capacidade de trabalhar sozinho &  & X &  &  & \\
\hline
Criatividade &  & X &  &  & \\
\hline
Curiosidade & X &  &  &  & \\
\hline
Esforço, persistência & X &  &  &  & \\
\hline
Expressão escrita & X &  &  &  & \\
\hline
Expressão oral & X &  &  &  & \\
\hline
Relacionamento com colegas & X &  &  &  & \\
\hline
\end{tabular}\\
\\
\textbf{Opinião sobre os antecedentes acadêmicos, profissionais e/ou técnicos do candidato:}
\\Um aluno aplicado e focado em resultados. Consegue transpor problemas de um universo para o outro e realizar a análise de maneira crítica.\\
\\
\textbf{Opinião sobre seu possível aproveitamento, se aceito no Programa:}
\\Considero que Alan será um aluno muito dedicado, pois não tem dependência financeira de trabalho o que faz com que ele possa estar 24h imerso na pesquisa. Além disso, Alan tem facilidade com a área da mátemática.\\ 
\\
\textbf{Outras informações relevantes:} \\Em seu trabalho de conclusão de curso Alan demonstrou capacidade de finalização e coerência características importantes para um pesquisador.
\\[0.3cm]
\textbf{Entre os estudantes que já conheceu, você diria que o candidato está entre os:}
\\
\begin{tabular}{|l|c|c|c|c|c|}
\hline
 & 5\% melhores & 10\% melhores & 25\% melhores & 50\% melhores & Não sabe \\
\hline
Como aluno, em aulas & X &  &  &  & \\
\hline
Como orientando &  &  &  &  & X\\
\hline
\end{tabular}
\subsection*{Dados Recomendante} 
	Instituição (Institution): Faculdade Paraíso do Ceará
\\ 
	Grau acadêmico mais alto obtido: mestre
	\ \ Área: Engenharia Elétrica  e Computação
	\\
	Ano de obtenção deste grau: 2003
	\ \ 
	Instituição de obtenção deste grau : FEEC UNICAMP 
	\\ 
	Endereço institucional do recomendante: \\ R. Conceição, 1288 Bairro São Miguel, Juazeiro do Norte, Ceará, CEP 63010465\newpage\vspace*{-4cm}\subsection*{Carta de Recomendação - Paulo Rogério Foina}Código Identificador: 1406\\Conhece-o candidato há quanto tempo (For how long have you known the applicant)? 
\ 2 anos
\\ Conhece-o sob as seguintes circunstâncias: aulas\ \ 
	\ \ \ \  
\\ Conheçe o candidato sob outras circunstâncias: 
\\Avaliações: \\
\begin{tabular}{|l|c|c|c|c|c|}
\hline
 & Excelente & Bom & Regular & Insuficiente & Não sabe \\
\hline
Desempenho acadêmico &  & X &  &  & \\
\hline
Capacidade de aprender novos conceitos &  & X &  &  & \\
\hline
Capacidade de trabalhar sozinho & X &  &  &  & \\
\hline
Criatividade &  & X &  &  & \\
\hline
Curiosidade & X &  &  &  & \\
\hline
Esforço, persistência & X &  &  &  & \\
\hline
Expressão escrita &  & X &  &  & \\
\hline
Expressão oral &  & X &  &  & \\
\hline
Relacionamento com colegas & X &  &  &  & \\
\hline
\end{tabular}\\
\\
\textbf{Opinião sobre os antecedentes acadêmicos, profissionais e/ou técnicos do candidato:}
\\Enquanto aluno de curso de especialização, o Alan demonstrou interesse e dedicação com estudos e com as pesquisas que lhe foram atribuídas.\\
\\
\textbf{Opinião sobre seu possível aproveitamento, se aceito no Programa:}
\\Acredito que Alan terá um bom desempenho no programa, caso aceito, pois demonstrou atenção concentrada e interesse pela pesquisa acadêmica e em pensamentos abstratos.\\ 
\\
\textbf{Outras informações relevantes:} \\Alan sempre mostrou interesse em estudos matemáticos e, quando possível, propunha soluções abstratas e análises matemáticas para os problemas que lhe foram apresentados no curso de Gestão de Tecnologia
\\[0.3cm]
\textbf{Entre os estudantes que já conheceu, você diria que o candidato está entre os:}
\\
\begin{tabular}{|l|c|c|c|c|c|}
\hline
 & 5\% melhores & 10\% melhores & 25\% melhores & 50\% melhores & Não sabe \\
\hline
Como aluno, em aulas &  & X &  &  & \\
\hline
Como orientando &  &  &  &  & X\\
\hline
\end{tabular}
\subsection*{Dados Recomendante} 
	Instituição (Institution): Centro Universitário de Brasília   UniCEUB
\\ 
	Grau acadêmico mais alto obtido: doutor
	\ \ Área: Informática
	\\
	Ano de obtenção deste grau: 1985
	\ \ 
	Instituição de obtenção deste grau : PUC RJ
	\\ 
	Endereço institucional do recomendante: \\ SCLN 205 Bloco A sala 18  CEP 70843510  Brasilia DF	
\begin{figure}[!htb]
\includegraphics{../upload/1379_2014-05-30_documentos.jpg}
\end{figure}\includepdf[pages={-},offset=35mm 0mm]{../../../upload/1379_2014-05-30_historico.pdf}	
\begin{figure}[!htb]
\includegraphics{../upload/1379_2014-05-29_documentos.jpg}
\end{figure}	
\begin{figure}[!htb]
\includegraphics{../upload/1379_2014-05-29_historico.jpg}
\end{figure}\includepdf[pages={-},offset=35mm 0mm]{../../../upload/1379_2014-05-29_historico.pdf} 
\begin{center}
Anexos.
\end{center}
\end{document}