\documentclass[11pt]{article}
\usepackage{graphicx,color}
\usepackage{pdfpages}
\usepackage[brazil]{babel}
\usepackage[utf8]{inputenc}
\addtolength{\hoffset}{-3cm} \addtolength{\textwidth}{6cm}
\addtolength{\voffset}{-.5cm} \addtolength{\textheight}{1cm}
%%%%%%%%%%%%%%%%%%%%%%%%%%%%%%%%%%%  To use Colors 
\title{\vspace*{-4cm} Ficha de Inscrição: \\Cod: 1305\ \ Bruno Silva de Jesus\ \ - \ \ Mestrado 
 }
\date{}

\begin{document}
\maketitle
\vspace*{-1.5cm}
\noindent Data de Nascimento:2/3/1991
\ \ \ Idade: 23   \ \ \ Sexo: Masculino
\\
Naturalidade: Minaçu  
\ \ \  Estado: GO
\ \ \  Nacionalidade: Brasileiro
\ \ \ País: Brasil
\\        
Nome do pai : Antonio Alves de Jesus
\ \ \ Nome da mãe: Maria Balduino Silva          
\\[0.2cm]                     
\textbf{Endereço Pessoal} 
\\ 
\noindent Endereço residencial: Rua Nossa Senhora Auxiliadora Qd 73 Lt 13
\\
        CEP: 74948-120 
\ \ \ Cidade: Aparecida de Goiânia 
\ \ \ Estado: GO 
\ \ \ País: Brasil
\\		
		Telefone comercial : +55(62)35795830
\ \ \ Telefone residencial: +55(62)35795830
\ \ \ Telefone celular : +55(62)91192211
\\
E-mail principal: master-tomy@hotmail.com
\ \ \ E-mail alternativo: bc629526@gmail.com 
\\[0.2cm] 
\textbf{Documentos Pessoais}
\\
\noindent Número de CPF : 02902867190
\ \ \ Número de Identidade (ou Passaporte para estrangeiros): 6099322593
\\
Orgão emissor: SJSP
\ \ \ Estado: GO
\ \ \ Data de emissão :11/3/2004
\\[0.3cm]
\textbf{Grau acadêmico mais alto obtido}
\\	
Curso:Matemática
\ \ \ Grau : licenciado
\ \ \ Instituição : União das Faculdades Alfredo Nasser UNIFAN
\\			
Ano de Conclusão ou Previsão: 2013
\\ 
Experiência Profissional mais recente. \ \  
Tem experiência: Docente Discente  
\ \ \ Instituição: Unifan
\\  
Período - início: 1-2013
\ \ \ fim: 1-2014
\\[0.2cm] 
\textbf{Programa Pretendido:} Mestrado\\
Interesse em bolsa: Sim
\\[0.3cm]		
\textbf{Dados dos Recomendantes} 
\\
1- Nome: Ana Paula Machado Faria
\ \ \ \  e-mail: anapfmat@unifan.edu.br 
\\
2- Nome: Renata Goncalves Lacerda
\ \ \ \ e-mail: renatalg@hotmail.com
\\
3- Nome: Kelen Michela
\ \ \ \ e-mail: kelen@linha.tv
\\[0.2cm]
Motivação e expectativa do candidato em relação ao programa pretendido:
\\O programa de mestrado na conceituada universidade de Brasília ampliará meu horizonte de conhecimentos, com a grande gama de professores e livros que a universidade possui. Tenho em mente que a instituição oferece para seus alunos grandes oportunidades de diversos campos de pesquisa. Meu foco de ingressar no instituto de matemática me auxiliará no meu objetivo de me tornar um grande matemático, onde tenho ambições de absorver dos professores todo o conhecimento que puder. Não medirei esforços e compromisso para adquirir este conhecimento. 
Em suma, meu maior objetivo neste programa de mestrado é evoluir como matemático e um ser ainda mais capaz de produzir conhecimento cientifico.\newpage\vspace*{-4cm}\subsection*{Carta de Recomendação - Ana Paula Machado Faria}Código Identificador: 1400\\Conhece-o candidato há quanto tempo (For how long have you known the applicant)? 
\ 4 anos
\\ Conhece-o sob as seguintes circunstâncias: aulas\ \ 
	\ \ \ \  
\\ Conheçe o candidato sob outras circunstâncias: 
\\	Avaliações:\\
\begin{tabular}{|l|c|c|c|c|c|}
\hline
 & Excelente & Bom & Regular & Insuficiente & Não sabe \\
\hline
Desempenho acadêmico &  & X &  &  & \\
\hline
Capacidade de aprender novos conceitos &  & X &  &  & \\
\hline
Capacidade de trabalhar sozinho & X &  &  &  & \\
\hline
Criatividade &  & X &  &  & \\
\hline
Curiosidade &  & X &  &  & \\
\hline
Esforço, persistência &  & X &  &  & \\
\hline
Expressão escrita &  & X &  &  & \\
\hline
Expressão oral &  &  & X &  & \\
\hline
Relacionamento com colegas &  & X &  &  & \\
\hline
\end{tabular}\\
\\
\textbf{Opinião sobre os antecedentes acadêmicos, profissionais e/ou técnicos do candidato:}
\\\\
\\
\textbf{Opinião sobre seu possível aproveitamento, se aceito no Programa:}
\\\\ 
\\
\textbf{Outras informações relevantes:} \\
\\[0.3cm]
\textbf{Entre os estudantes que já conheceu, você diria que o candidato está entre os:}
\\
\begin{tabular}{|l|c|c|c|c|c|}
\hline
 & 5\% melhores & 10\% melhores & 25\% melhores & 50\% melhores & Não sabe \\
\hline
Como aluno, em aulas &  &  &  &  & \\
\hline
Como orientando &  &  &  &  & \\
\hline
\end{tabular}
\subsection*{Dados Recomendante} 
	Instituição (Institution): 
\\ 
	Grau acadêmico mais alto obtido: 
	\ \ Área: 
	\\
	Ano de obtenção deste grau: 
	\ \ 
	Instituição de obtenção deste grau : 
	\\ 
	Endereço institucional do recomendante: \\ \newpage\vspace*{-4cm}\subsection*{Carta de Recomendação - Renata Goncalves Lacerda}Código Identificador: 1401\\Conhece-o candidato há quanto tempo (For how long have you known the applicant)? 
\ a 2 anos
\\ Conhece-o sob as seguintes circunstâncias: aulas\ \ orientacao
	\ \ \ \  
\\ Conheçe o candidato sob outras circunstâncias: 
\\Avaliações: \\
\begin{tabular}{|l|c|c|c|c|c|}
\hline
 & Excelente & Bom & Regular & Insuficiente & Não sabe \\
\hline
Desempenho acadêmico & X &  &  &  & \\
\hline
Capacidade de aprender novos conceitos & X &  &  &  & \\
\hline
Capacidade de trabalhar sozinho & X &  &  &  & \\
\hline
Criatividade & X &  &  &  & \\
\hline
Curiosidade & X &  &  &  & \\
\hline
Esforço, persistência & X &  &  &  & \\
\hline
Expressão escrita &  & X &  &  & \\
\hline
Expressão oral & X &  &  &  & \\
\hline
Relacionamento com colegas & X &  &  &  & \\
\hline
\end{tabular}\\
\\
\textbf{Opinião sobre os antecedentes acadêmicos, profissionais e/ou técnicos do candidato:}
\\Conheço Bruno desde 2012, ano que comecei a ministrar aulas do curso de matemática ao mesmo, o qual sempre apresentou interesse e dedicação pelo conhecimento da matemática pura.
	O aluno também obteve excelentes resultados em todas as avaliações do curso, além de sempre expressar o seu desejo de aprofundar nos conteúdos estudados ao longo do curso.
	No ano de 2013, tive a oportunidade de orientalo em seu trabalho de final de curso, neste trabalho foi afirmado o potencial do aluno em investigar  e criar aplicações em matemática pura na física.\\
\\
\textbf{Opinião sobre seu possível aproveitamento, se aceito no Programa:}
\\Asseguro que o aluno referido será um ótimo PósGraduando em Matemática, já que possui determinação para trabalhar e aprender.\\ 
\\
\textbf{Outras informações relevantes:} \\
\\[0.3cm]
\textbf{Entre os estudantes que já conheceu, você diria que o candidato está entre os:}
\\
\begin{tabular}{|l|c|c|c|c|c|}
\hline
 & 5\% melhores & 10\% melhores & 25\% melhores & 50\% melhores & Não sabe \\
\hline
Como aluno, em aulas &  &  &  & X & \\
\hline
Como orientando &  &  &  & X & \\
\hline
\end{tabular}
\subsection*{Dados Recomendante} 
	Instituição (Institution): Universidade Federal de Goiás
\\ 
	Grau acadêmico mais alto obtido: mestre
	\ \ Área: Matemática Pura
	\\
	Ano de obtenção deste grau: 2003
	\ \ 
	Instituição de obtenção deste grau : Universidade Federal de Goiás
	\\ 
	Endereço institucional do recomendante: \\ Rua Harpia, Qd. 182, Lote 03, PQ. Amazônia, Goiânia, Goiás\newpage\vspace*{-4cm}\subsection*{Carta de Recomendação - Kelen Michela}Código Identificador: 1402\\Conhece-o candidato há quanto tempo (For how long have you known the applicant)? 
\ 3 anos
\\ Conhece-o sob as seguintes circunstâncias: aulas\ \ 
	\ \ seminarios\ \ outra 
\\ Conheçe o candidato sob outras circunstâncias: Organização e aplicação de projetos em escolas-campo de estágio.
\\Avaliações: \\
\begin{tabular}{|l|c|c|c|c|c|}
\hline
 & Excelente & Bom & Regular & Insuficiente & Não sabe \\
\hline
Desempenho acadêmico &  & X &  &  & \\
\hline
Capacidade de aprender novos conceitos &  & X &  &  & \\
\hline
Capacidade de trabalhar sozinho &  & X &  &  & \\
\hline
Criatividade &  & X &  &  & \\
\hline
Curiosidade & X &  &  &  & \\
\hline
Esforço, persistência & X &  &  &  & \\
\hline
Expressão escrita &  &  & X &  & \\
\hline
Expressão oral &  & X &  &  & \\
\hline
Relacionamento com colegas & X &  &  &  & \\
\hline
\end{tabular}\\
\\
\textbf{Opinião sobre os antecedentes acadêmicos, profissionais e/ou técnicos do candidato:}
\\Destaco que conheço o candidato desde o quarto período de seu curso de graduação em Licenciatura, como professora de Fundamentos da Matemática e nos semestres seguintes Matemática Financeira e Estágio Supervisionado. Foram precisamente nestas aulas, principalmente as de Estágio, que observei interesse e dedicação tanto para adquirir conhecimentos específicos da área da matemática, quanto pela busca de recursos que facilitariam o processo de ensino aprendizagem.
	Pude perceber também que o aluno não possuía uma excelente destreza para produção textual, mas era responsável, dedicado, motivado a aprender e possuía uma boa habilidade de se expressar oralmente, assim, seus projetos direcionados a escola campo de estágio alcançaram bons resultados.\\
\\
\textbf{Opinião sobre seu possível aproveitamento, se aceito no Programa:}
\\Acredito que tendo a oportunidade de ser aceito no programa o mesmo buscará um bom desempenho, pois sempre se apresentou motivado e interessado a adquirir conhecimento.\\ 
\\
\textbf{Outras informações relevantes:} \\
\\[0.3cm]
\textbf{Entre os estudantes que já conheceu, você diria que o candidato está entre os:}
\\
\begin{tabular}{|l|c|c|c|c|c|}
\hline
 & 5\% melhores & 10\% melhores & 25\% melhores & 50\% melhores & Não sabe \\
\hline
Como aluno, em aulas &  &  &  & X & \\
\hline
Como orientando &  &  &  &  & X\\
\hline
\end{tabular}
\subsection*{Dados Recomendante} 
	Instituição (Institution): Faculdade Alfredo Nasser
\\ 
	Grau acadêmico mais alto obtido: especialista
	\ \ Área: Formação de Professores
	\\
	Ano de obtenção deste grau: 2004
	\ \ 
	Instituição de obtenção deste grau : PUC GOIÁS  Pontifícia Universidade Católica de Goiás
	\\ 
	Endereço institucional do recomendante: \\ Av. Bela Vista N. 26, Jardim das Esmeraldas
Aparecida de Goiânia  Goiás
CEP. 74.905020	
\begin{figure}[!htb]
\includegraphics{../upload/1305_2014-05-29_documentos.jpg}
\end{figure}	
\begin{figure}[!htb]
\includegraphics{../upload/1305_2014-05-29_historico.jpg}
\end{figure}\includepdf[pages={-},offset=35mm 0mm]{../../../upload/1305_2014-05-29_historico.pdf} 
\begin{center}
Anexos.
\end{center}
\end{document}