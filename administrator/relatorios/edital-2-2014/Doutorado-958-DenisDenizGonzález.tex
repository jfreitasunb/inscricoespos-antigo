\documentclass[11pt]{article}
\usepackage{graphicx,color}
\usepackage{pdfpages}
\usepackage[brazil]{babel}
\usepackage[utf8]{inputenc}
\addtolength{\hoffset}{-3cm} \addtolength{\textwidth}{6cm}
\addtolength{\voffset}{-.5cm} \addtolength{\textheight}{1cm}
%%%%%%%%%%%%%%%%%%%%%%%%%%%%%%%%%%%  To use Colors 
\title{\vspace*{-4cm} Ficha de Inscrição: \\Cod: 958\ \ Denis Deniz González\ \ - \ \ Doutorado 
 }
\date{}

\begin{document}
\maketitle
\vspace*{-1.5cm}
\noindent Data de Nascimento:19/12/1988
\ \ \ Idade: 25   \ \ \ Sexo: Masculino
\\
Naturalidade: Cuba  
\ \ \  Estado: Outro
\ \ \  Nacionalidade: cubano
\ \ \ País: Cuba
\\        
Nome do pai : Bernardino Deniz Pérez
\ \ \ Nome da mãe: Clara Marisol González Jacomino          
\\[0.2cm]                     
\textbf{Endereço Pessoal} 
\\ 
\noindent Endereço residencial: Prolongación de Colón, Nro 18 entre Capitán Velazco y Barcelona. Santa Clara. Villa Clara
\\
        CEP: 50000-100 
\ \ \ Cidade: Santa Clara 
\ \ \ Estado: Outro 
\ \ \ País: Cuba
\\		
		Telefone comercial : +53(42)224746
\ \ \ Telefone residencial: +53(42)218675
\ \ \ Telefone celular : +53(58)252744
\\
E-mail principal: ddeniz@uclv.cu
\ \ \ E-mail alternativo: denisdenizglez@gmail.com 
\\[0.2cm] 
\textbf{Documentos Pessoais}
\\
\noindent Número de CPF : 70450736156
\ \ \ Número de Identidade (ou Passaporte para estrangeiros): O915929
\\
Orgão emissor: UCLV
\ \ \ Estado: Outro
\ \ \ Data de emissão :1/11/2012
\\[0.3cm]
\textbf{Grau acadêmico mais alto obtido}
\\	
Curso:Ciencia de la Computación
\ \ \ Grau : licenciado
\ \ \ Instituição : Universidad Central Marta Abreu de Las Villas
\\			
Ano de Conclusão ou Previsão: 2012
\\ 
Experiência Profissional mais recente. \ \  
Tem experiência: Docente 0  
\ \ \ Instituição: Universidad Central Marta Abreu de Las Villas
\\  
Período - início: 1-2013
\ \ \ fim: 2-2013
\\[0.2cm] 
\textbf{Programa Pretendido:} Doutorado\ \ \ \textbf{Área:} MatematicaAplicada\\
Interesse em bolsa: Sim
\\[0.3cm]		
\textbf{Dados dos Recomendantes} 
\\
1- Nome: Andre Brasil Cavalcante
\ \ \ \  e-mail: albrasilc@gmail.com 
\\
2- Nome: Carlos Alexander Recarey Morfa
\ \ \ \ e-mail: recarey@uclv.edu.cu
\\
3- Nome: Lucia Arguelles Cortes
\ \ \ \ e-mail: largue@uclv.edu.cu
\\[0.2cm]
Motivação e expectativa do candidato em relação ao programa pretendido:
\\ Aunque me gradué como licenciado en Ciencia de la Computación, siempre he mantenido una estrecha relación con la carrera de Matemática, cursando asignaturas propias de la misma como Topología, Medida e Integración, Análisis Funcional y Geometría Diferencial. Por lo que, con el programa de doctorado en Matemática de la UnB, pretendo desarrollar mis conocimientos y habilidades propios de esta madre ciencia. De manera general, pienso aprovechar todo el tiempo y espacio ofrecido por el programa para hacer míos los conocimientos compartidos por los excelentes profesores del mismo, y convertirme continuamente así en un mejor profesional.\newpage\vspace*{-4cm}\subsection*{Carta de Recomendação - Andre Brasil Cavalcante}Código Identificador: 972\\Conhece-o candidato há quanto tempo (For how long have you known the applicant)? 
\ 2 anos
\\ Conhece-o sob as seguintes circunstâncias: \ \ 
	\ \ \ \ outra 
\\ Conheçe o candidato sob outras circunstâncias: Recomendação
\\	Avaliações:\\
\begin{tabular}{|l|c|c|c|c|c|}
\hline
 & Excelente & Bom & Regular & Insuficiente & Não sabe \\
\hline
Desempenho acadêmico & X &  &  &  & \\
\hline
Capacidade de aprender novos conceitos & X &  &  &  & \\
\hline
Capacidade de trabalhar sozinho & X &  &  &  & \\
\hline
Criatividade & X &  &  &  & \\
\hline
Curiosidade & X &  &  &  & \\
\hline
Esforço, persistência &  & X &  &  & \\
\hline
Expressão escrita &  & X &  &  & \\
\hline
Expressão oral &  & X &  &  & \\
\hline
Relacionamento com colegas & X &  &  &  & \\
\hline
\end{tabular}\\
\\
\textbf{Opinião sobre os antecedentes acadêmicos, profissionais e/ou técnicos do candidato:}
\\O candidato demonstrou dedicação acadêmica e, em particular, um excelente desempenho na condução e elaboração de diferentes trabalhos de cooperação científica vinculados ao Programa de Pós graduação em Geotecnia. \\
\\
\textbf{Opinião sobre seu possível aproveitamento, se aceito no Programa:}
\\O candidato demonstra conhecimento e senso crítico para conduzir pesquisa de forma relevante em diferentes áreas da matemática.\\ 
\\
\textbf{Outras informações relevantes:} \\O candidato é medalhista em diferentes olimpíadas. Além de seu desempenho acadêmico, ele também interage bem com os colegas, é articulado, jovem, educado, informado e motivado. Eu acredito que ele será bem sucedido se aceito por este programa.
\\[0.3cm]
\textbf{Entre os estudantes que já conheceu, você diria que o candidato está entre os:}
\\
\begin{tabular}{|l|c|c|c|c|c|}
\hline
 & 5\% melhores & 10\% melhores & 25\% melhores & 50\% melhores & Não sabe \\
\hline
Como aluno, em aulas & X &  &  &  & \\
\hline
Como orientando & X &  &  &  & \\
\hline
\end{tabular}
\subsection*{Dados Recomendante} 
	Instituição (Institution): Universidade de Brasília
\\ 
	Grau acadêmico mais alto obtido: doutor
	\ \ Área: Geotecnia
	\\
	Ano de obtenção deste grau: 2004
	\ \ 
	Instituição de obtenção deste grau : Universidade de Brasília
	\\ 
	Endereço institucional do recomendante: \\ Campus Darcy Ribeiro
SG12 Programa de Pós graduação em Geotecnia
2a andar\newpage\vspace*{-4cm}\subsection*{Carta de Recomendação - Carlos Alexander Recarey Morfa}Código Identificador: 973\\Conhece-o candidato há quanto tempo (For how long have you known the applicant)? 
\ Más de 5 anos
\\ Conhece-o sob as seguintes circunstâncias: \ \ orientacao
	\ \ seminarios\ \ outra 
\\ Conheçe o candidato sob outras circunstâncias: Centro de Métodos Computacionales y Numéricos en la Ingeniería. Aula CIMNE Universidad Central de Las Villas, Cuba.
\\Avaliações: \\
\begin{tabular}{|l|c|c|c|c|c|}
\hline
 & Excelente & Bom & Regular & Insuficiente & Não sabe \\
\hline
Desempenho acadêmico & X &  &  &  & \\
\hline
Capacidade de aprender novos conceitos & X &  &  &  & \\
\hline
Capacidade de trabalhar sozinho & X &  &  &  & \\
\hline
Criatividade & X &  &  &  & \\
\hline
Curiosidade & X &  &  &  & \\
\hline
Esforço, persistência & X &  &  &  & \\
\hline
Expressão escrita & X &  &  &  & \\
\hline
Expressão oral & X &  &  &  & \\
\hline
Relacionamento com colegas & X &  &  &  & \\
\hline
\end{tabular}\\
\\
\textbf{Opinião sobre os antecedentes acadêmicos, profissionais e/ou técnicos do candidato:}
\\El Lic. Denis Deniz González ha mantenido una excelente trayectoria académica incluso antes de ingresar a la Universidad Central Marta Abreu de Las Villas UCLV. Siendo estudiante del Instituto Preuniversitario Vocacional de Ciencias Exactas IPVCE Ernesto Che Guevara de la ciudad de Santa Clara Villa Clara obtiene resultados sobresalientes en los concursos cubanos preuniversitarios de Matemática  e integra la Preselección Nacional de Matemática de Cuba.
Desde que ingresa a la UCLV para cursar la carrera de Ciencia de la Computación continúa con su interés por las matemáticas. Cada curso participa en diferentes concursos u olimpiadas cubanas e internacionales, alcanzando lugares destacados. Dentro de los más relevantes, cursando su segundo ao, gana la MEDALLA de ORO en la Octava Olimpiada Nacional de Matemática Raimundo Reguera. Internacionalmente, logra obtener la MEDALLA de BRONCE en las Olimpiadas Iberoamericanas de Matemática XII y XIV, respectivamente. Por otra parte, junto a dos compaeros, obtiene resultados nacionales cimeros y participa en la Final Caribea, región México y Centro América del prestigioso concurso de programación ACMICPC, consiguiendo ubicarse dentro de los diez primeros equipos.
Desde el primer curso se une al Centro de Métodos Computacionales y Numéricos en la Ingeniería Aula CIMNE en el que continua alcanzando resultados sobresalientes, muestra de ello lo constituye su Trabajo de Diploma titulado Segmentación de la aorta a partir de Imágenes de Resonancia Magnética cardiovasculares 4D con el que alcanza la máxima calificación, y posteriormente, participa en el Simposio Internacional de Estructuras, Geotecnia y Materiales de la Construcción celebrado en Cuba en 2013.
Debido a su interés por la matemática decide tomar como asignaturas opcionales Topología, Geometría Diferencial, Análisis Funcional y Medida e Integración, impartidas en la carrera de Matemática Pura. Como parte de la asignatura Geometría Diferencial implementa junto a otro compaero un paquete de funciones relacionadas a la asignatura en el software Wolfram Mathematica.
Se reconoce su trayectoria universitaria al otorgarle el Título de Oro, concluyendo sus estudios de pregrado con un índice académico final de 6.02, superior a 5 debido a bonificaciones por investigación científica.
Como parte de su trabajo dentro del Aula CIMNE participa en el Proyecto de Segmentación de la aorta a partir de Imágenes de Resonancia Magnética cardiovasculares 4D extracción de la línea central, el Proyecto de desarrollo de plugins que incorporen nuevas funcionalidades al software Autodesk Inventor y en marzo de 2014, termina de cursar el Diplomado Discrete Element Method impartido por dos profesores belgas en la UCLV.\\
\\
\textbf{Opinião sobre seu possível aproveitamento, se aceito no Programa:}
\\De ser aceptado dentro del programa de doctorado de la UnB estoy seguro de que tendrá un excelente desempeno, similar al mantenido hasta la fecha en todos los ámbitos.\\ 
\\
\textbf{Outras informações relevantes:} \\Siendo estudiante universitario se mantiene apoyando el movimiento de concursos preuniversitarios impartiendo en su tiempo libre temas de matemática elemental en el IPVCE Ernesto Che Guevara.
Luego de graduarse, imparte la asignatura de Ecuaciones Diferenciales Ordinarias a la carrera de Licenciatura en Matemática en la facultad MFC de la UCLV, y por otra parte se desempea como entrenador de los equipos universitarios a los diferentes niveles de los concursos regionales de la  ACMICPC.
\\[0.3cm]
\textbf{Entre os estudantes que já conheceu, você diria que o candidato está entre os:}
\\
\begin{tabular}{|l|c|c|c|c|c|}
\hline
 & 5\% melhores & 10\% melhores & 25\% melhores & 50\% melhores & Não sabe \\
\hline
Como aluno, em aulas & X &  &  &  & \\
\hline
Como orientando & X &  &  &  & \\
\hline
\end{tabular}
\subsection*{Dados Recomendante} 
	Instituição (Institution): Centro de Métodos Computacionales y Numéricos en la Ingeniería Aula CIMNE. Universidad Central Marta Abreu de Las Villas UCLV
\\ 
	Grau acadêmico mais alto obtido: doutor
	\ \ Área: Ciencias Técnicas
	\\
	Ano de obtenção deste grau: 1999
	\ \ 
	Instituição de obtenção deste grau : Universidad Central Marta Abreu de Las Villas UCLV.
	\\ 
	Endereço institucional do recomendante: \\ Universidad Central Marta Abreu de Las Villas. Carretera a Camajuaní Km cinco y medio, Santa Clara, Villa Clara, Cuba. C.P. 54830\newpage\vspace*{-4cm}\subsection*{Carta de Recomendação - Lucia Arguelles Cortes}Código Identificador: 974\\Conhece-o candidato há quanto tempo (For how long have you known the applicant)? 
\ 2008
\\ Conhece-o sob as seguintes circunstâncias: aulas\ \ 
	\ \ seminarios\ \  
\\ Conheçe o candidato sob outras circunstâncias: 
\\Avaliações: \\
\begin{tabular}{|l|c|c|c|c|c|}
\hline
 & Excelente & Bom & Regular & Insuficiente & Não sabe \\
\hline
Desempenho acadêmico & X &  &  &  & \\
\hline
Capacidade de aprender novos conceitos & X &  &  &  & \\
\hline
Capacidade de trabalhar sozinho & X &  &  &  & \\
\hline
Criatividade & X &  &  &  & \\
\hline
Curiosidade & X &  &  &  & \\
\hline
Esforço, persistência & X &  &  &  & \\
\hline
Expressão escrita & X &  &  &  & \\
\hline
Expressão oral &  & X &  &  & \\
\hline
Relacionamento com colegas & X &  &  &  & \\
\hline
\end{tabular}\\
\\
\textbf{Opinião sobre os antecedentes acadêmicos, profissionais e/ou técnicos do candidato:}
\\Desde que comienza en la universidad cursando la Carrera de Licenciatura en Ciencia de la Computación, se mantiene apoyando el movimiento de concursos preuniversitarios, impartiendo temas de matemática elemental en el IPVCE y participando además sistemáticamente, ya que cada curso participó en las Olimpiadas Nacionales Universitarias de Matemática Raimundo Reguera ONUM obteniendo resultados como bronce primer anno y oro segundo anno, Mención de Honor en la 10ma Olimpiada nacional universitaria de Matemática 2011, también en las Olimpiadas Iberoamericanas de Matemática Universitaria OIMU teniendo como resultados más destacados Mención de Honor segundo anno y bronce tercer anno y Primer lugar en el nivel universitario , Olimpiada por el día de la matemática 2010. En este anno participó junto a dos companneros en la competencia regional latinoamericana de la prestigiosa International Collegiate Programming Contest ACM ICPC alcanzando un cuarto lugar nacional. Este entrenamiento le permitió adquirir muchas habilidades en el desarrollo del pensamiento abstracto y analítico. 
Debido a su interés por la matemática decide tomar como asignaturas opcionales Topología, Geometría Diferencial, Análisis Funcional y Medida e Integración, impartidas en la carrera de Licenciatura en Matemática. Como parte de la asignatura Geometría Diferencial implementó junto a otro compannero un paquete de funciones relacionadas con la asignatura en el software Wolfram Mathematica. Fue alumno ayudante de las asignaturas de Geometría Analítica y Análisis Matemático durante dos cursos. También resultó destacada su participación en eventos, ya que obtuvo Mención en la Competencia de habilidades en el Taller Científico Estudiantil del evento internacional COMPUMAT 2011 y Destacado en Fórum Científico Estudiantil a nivel de facultad en quinto anno. Convalidó asignaturas como Algebra I, Algebra II y Ecuaciones Diferenciales, participó en exámenes de premio de asignaturas como Análisis Matemático, Algebra y  Matemática Discreta lo cual hizo posible que se graduara con el reconocimiento de Título de Oro, concluyendo sus estudios de pregrado con un índice académico final de 6.02 sobre 5.\\
\\
\textbf{Opinião sobre seu possível aproveitamento, se aceito no Programa:}
\\En caso de ser aceptado en el programa, su poder analítico,  sagacidad, creatividad y dedicación al estudio garantizarían el éxito de su desempenno\\ 
\\
\textbf{Outras informações relevantes:} \\Posee una sobresaliente trayectoria de participación en concursos y olimpíadas desde la enseanza media, etapa en la que obtuvo el primer lugar provincial en el concurso de Física 2004 y medalla de plata en el Concurso Nacional Preuniversitario de Matemática. A partir de este momento integra la Preselección Nacional de Matemática y es significativo sealar que concluyó sus estudios preuniversitarios con un promedio general de 100 puntos.
Desde el primer curso se une al Centro de Investigación de Métodos Computacionales y Numéricos en la Ingeniería CIMCNI, grupo de investigación de la universidad. Posteriormente, comienza a trabajar en la UCLV específicamente en este centro Adicionalmente, impartió la asignatura de Ecuaciones Diferenciales Ordinarias a la carrera de Matemática en la facultad MFC, y por otra parte se desempenna como entrenador de los equipos universitarios a los diferentes niveles de los concursos regionales de la  ACMICPC. Como parte de su trabajo dentro del CIMCNI ha participado en dos Proyectos asociados al procesamiento de imágenes.
\\[0.3cm]
\textbf{Entre os estudantes que já conheceu, você diria que o candidato está entre os:}
\\
\begin{tabular}{|l|c|c|c|c|c|}
\hline
 & 5\% melhores & 10\% melhores & 25\% melhores & 50\% melhores & Não sabe \\
\hline
Como aluno, em aulas &  & X &  &  & \\
\hline
Como orientando &  & X &  &  & \\
\hline
\end{tabular}
\subsection*{Dados Recomendante} 
	Instituição (Institution): Universidad Central de Las Villas
\\ 
	Grau acadêmico mais alto obtido: doutor
	\ \ Área: Ciencias Técnicas
	\\
	Ano de obtenção deste grau: 1998
	\ \ 
	Instituição de obtenção deste grau : Universidad Central de Las Villas
	\\ 
	Endereço institucional do recomendante: \\ Carretera a Camajuaní Km 5 1/2, C.P. 54830 Santa Clara, Villa Clara, Cuba\includepdf[pages={-},offset=35mm 0mm]{../../../upload/958_2014-05-13_documentos.pdf}\includepdf[pages={-},offset=35mm 0mm]{../../../upload/958_2014-05-13_historico.pdf}\includepdf[pages={-},offset=35mm 0mm]{../../../upload/958_2013-11-04_documentos.pdf}\includepdf[pages={-},offset=35mm 0mm]{../../../upload/958_2013-11-04_historico.pdf} 
\begin{center}
Anexos.
\end{center}
\end{document}