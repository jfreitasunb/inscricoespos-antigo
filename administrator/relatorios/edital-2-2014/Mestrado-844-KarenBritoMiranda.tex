\documentclass[11pt]{article}
\usepackage{graphicx,color}
\usepackage{pdfpages}
\usepackage[brazil]{babel}
\usepackage[utf8]{inputenc}
\addtolength{\hoffset}{-3cm} \addtolength{\textwidth}{6cm}
\addtolength{\voffset}{-.5cm} \addtolength{\textheight}{1cm}
%%%%%%%%%%%%%%%%%%%%%%%%%%%%%%%%%%%  To use Colors 
\title{\vspace*{-4cm} Ficha de Inscrição: \\Cod: 844\ \ Karen Brito Miranda\ \ - \ \ Mestrado 
 }
\date{}

\begin{document}
\maketitle
\vspace*{-1.5cm}
\noindent Data de Nascimento:16/1/1993
\ \ \ Idade: 21   \ \ \ Sexo: Feminino
\\
Naturalidade: Guaraí  
\ \ \  Estado: TO
\ \ \  Nacionalidade: Brasileira
\ \ \ País: Brasil
\\        
Nome do pai : Raimundo Tavares de Miranda
\ \ \ Nome da mãe: Lucilene Souza Brito          
\\[0.2cm]                     
\textbf{Endereço Pessoal} 
\\ 
\noindent Endereço residencial: Rua Souza Porto
\\
        CEP: 77805-100 
\ \ \ Cidade: Araguaína 
\ \ \ Estado: TO 
\ \ \ País: Brasil
\\		
		Telefone comercial : +0(63)92520714
\ \ \ Telefone residencial: +0(63)34142759
\ \ \ Telefone celular : +0(63)92914403
\\
E-mail principal: karen\_@uft.edu.br
\ \ \ E-mail alternativo: karen.m4t@gmail.com 
\\[0.2cm] 
\textbf{Documentos Pessoais}
\\
\noindent Número de CPF : 04200819171
\ \ \ Número de Identidade (ou Passaporte para estrangeiros): 1119953
\\
Orgão emissor: SSP
\ \ \ Estado: TO
\ \ \ Data de emissão :24/6/2008
\\[0.3cm]
\textbf{Grau acadêmico mais alto obtido}
\\	
Curso:Matemática
\ \ \ Grau : licenciado
\ \ \ Instituição : Universidade Federal do Toccantins
\\			
Ano de Conclusão ou Previsão: 2013
\\ 
Experiência Profissional mais recente. \ \  
Tem experiência: 0 Discente  
\ \ \ Instituição: Universidade Federal do Toccantins
\\  
Período - início: 1-2010
\ \ \ fim: 1-2013
\\[0.2cm] 
\textbf{Programa Pretendido:} Mestrado\\
Interesse em bolsa: Sim
\\[0.3cm]		
\textbf{Dados dos Recomendantes} 
\\
1- Nome: Odair Vieira dos Santos
\ \ \ \  e-mail: odairuft@yahoo.com.br 
\\
2- Nome: Fernanda Vital de Paula
\ \ \ \ e-mail: fernandavital@uft.edu.br
\\
3- Nome: Sinval de Oliveira
\ \ \ \ e-mail: sinval@uft.edu.br
\\[0.2cm]
Motivação e expectativa do candidato em relação ao programa pretendido:
\\No decorrer da graduação fui percebendo a importância da qualificação profissional e incentivada pelos professores do meu curso a me preparar para o ingresso em um mestrado. Desde então, busquei participar de projetos com os professores que visassem uma preparação maior para o mestrado. Posso dizer com veemência que para mim será a realização de um sonho ser selecionada para o programa. Particularmente, gosto ação da pesquisa, de descobrir coisas novas, de manter o conhecimento em constante construção.
Tenho que confessar que me sinto mais atraída pela Matemática Pura e Aplicada, além disso, percebo que há uma necessidade de profissionais nessa área no meu curso. Aliado a esses motivos, cresce em mim o desejo de prosseguir estudando, pesquisando e ensinando, contribuindo de alguma forma, para o progresso do conhecimento. Uma das possibilidades que encontrei para que isso aconteça foi o meu ingresso em um curso de mestrado.
Portanto, espero ser aceita no programa e aproveitar o máximo possível para o crescimento do meu conhecimento, superando as dificuldades e os desafios do percurso. Creio ser importante mencionar que não pretendo parar no mestrado, quero prosseguir estudando e me aperfeiçoando sempre. Também anseio por não deixar o conhecimento morrer em mim, mas permitir seu prolongamento em outras pessoas através do ensino.\newpage\vspace*{-4cm}\subsection*{Carta de Recomendação - Odair Vieira dos Santos}Código Identificador: 458\\Conhece-o candidato há quanto tempo (For how long have you known the applicant)? 
\ 3 anos
\\ Conhece-o sob as seguintes circunstâncias: aulas\ \ 
	\ \ \ \  
\\ Conheçe o candidato sob outras circunstâncias: 
\\	Avaliações:\\
\begin{tabular}{|l|c|c|c|c|c|}
\hline
 & Excelente & Bom & Regular & Insuficiente & Não sabe \\
\hline
Desempenho acadêmico & X &  &  &  & \\
\hline
Capacidade de aprender novos conceitos & X &  &  &  & \\
\hline
Capacidade de trabalhar sozinho &  & X &  &  & \\
\hline
Criatividade &  & X &  &  & \\
\hline
Curiosidade & X &  &  &  & \\
\hline
Esforço, persistência & X &  &  &  & \\
\hline
Expressão escrita &  & X &  &  & \\
\hline
Expressão oral &  & X &  &  & \\
\hline
Relacionamento com colegas &  & X &  &  & \\
\hline
\end{tabular}\\
\\
\textbf{Opinião sobre os antecedentes acadêmicos, profissionais e/ou técnicos do candidato:}
\\Conheço esta aluna há 3 anos e sempre pude ver de perto a sua dedicação durante o curso e se envolvendo em vários projetos que o curso oferecia. Aluna muito dedicada e com uma capacidade cognitiva excelente para as mais variadas tarefas que o curso apresentava.\\
\\
\textbf{Opinião sobre seu possível aproveitamento, se aceito no Programa:}
\\Acredito que a candidata sairá muito bem no mestrado, já que a mesma sempre se mostrou estimulada, dedicada e bastante responsável nas suas atividades acadêmicas durante toda a sua graduação.\\ 
\\
\textbf{Outras informações relevantes:} \\A candidata sempre foi envolvida nos eventos que o curso oferecia e também participou de vários congressos e eventos durante a sua graduação.
\\[0.3cm]
\textbf{Entre os estudantes que já conheceu, você diria que o candidato está entre os:}
\\
\begin{tabular}{|l|c|c|c|c|c|}
\hline
 & 5\% melhores & 10\% melhores & 25\% melhores & 50\% melhores & Não sabe \\
\hline
Como aluno, em aulas &  & X &  &  & \\
\hline
Como orientando &  & X &  &  & \\
\hline
\end{tabular}
\subsection*{Dados Recomendante} 
	Instituição (Institution): 
\\ 
	Grau acadêmico mais alto obtido: 
	\ \ Área: 
	\\
	Ano de obtenção deste grau: 
	\ \ 
	Instituição de obtenção deste grau : 
	\\ 
	Endereço institucional do recomendante: \\ \newpage\vspace*{-4cm}\subsection*{Carta de Recomendação - Fernanda Vital de Paula}Código Identificador: 1231\\Conhece-o candidato há quanto tempo (For how long have you known the applicant)? 
\ 3 anos
\\ Conhece-o sob as seguintes circunstâncias: aulas\ \ 
	\ \ seminarios\ \  
\\ Conheçe o candidato sob outras circunstâncias: 
\\Avaliações: \\
\begin{tabular}{|l|c|c|c|c|c|}
\hline
 & Excelente & Bom & Regular & Insuficiente & Não sabe \\
\hline
Desempenho acadêmico &  & X &  &  & \\
\hline
Capacidade de aprender novos conceitos &  & X &  &  & \\
\hline
Capacidade de trabalhar sozinho &  & X &  &  & \\
\hline
Criatividade &  & X &  &  & \\
\hline
Curiosidade & X &  &  &  & \\
\hline
Esforço, persistência &  & X &  &  & \\
\hline
Expressão escrita & X &  &  &  & \\
\hline
Expressão oral & X &  &  &  & \\
\hline
Relacionamento com colegas & X &  &  &  & \\
\hline
\end{tabular}\\
\\
\textbf{Opinião sobre os antecedentes acadêmicos, profissionais e/ou técnicos do candidato:}
\\A aluna foi bolsista do Programa Pibid, o que lhe permitiu um domínio no desenvolvimento de artigos e trabalhos científicos. Durante o curso a aluna participou ativamente de todos os eventos,regionais, nacionais e locais que teve oportunidade de participar. A aluna visa sempre sua qualificação participando de programas e projetos vigentes no Curso, na área de Matemática Pura ou na área de Educação. A aluna tem grande 
potencial para dar continuidade em seus estudos ingressando neste mestrado.Vale destacar que a Monografia da aluna é intitulada O Polinômio Característico aplicado à análise da estabilidade de sistemas autônomos caso linear, o qual permitiu que a mesma se aprofundasse em certos temas importantes para seu desempenho no mestrado para a realização do trabalho.\\
\\
\textbf{Opinião sobre seu possível aproveitamento, se aceito no Programa:}
\\A aluna é muito dedicada e esforçada buscando sempre ótimas notas como pode ser observado em seu histórico e sempre busca o esclarecimento de 
dúvidas buscando os colegas e os professores. A aluna é tranquila, sendo fácil conviver com a mesma e trabalha muito bem 
individualmente ou em equipe.\\ 
\\
\textbf{Outras informações relevantes:} \\
\\[0.3cm]
\textbf{Entre os estudantes que já conheceu, você diria que o candidato está entre os:}
\\
\begin{tabular}{|l|c|c|c|c|c|}
\hline
 & 5\% melhores & 10\% melhores & 25\% melhores & 50\% melhores & Não sabe \\
\hline
Como aluno, em aulas & X &  &  &  & \\
\hline
Como orientando & X &  &  &  & \\
\hline
\end{tabular}
\subsection*{Dados Recomendante} 
	Instituição (Institution): universidade federal do tocantins
\\ 
	Grau acadêmico mais alto obtido: mestre
	\ \ Área: estatística aplicada
	\\
	Ano de obtenção deste grau: 2011
	\ \ 
	Instituição de obtenção deste grau : universidade federal de viçosa
	\\ 
	Endereço institucional do recomendante: \\ Centro de Ciências Integradas

Av. Paraguai, sn  esquina com Rua Uxiramas

Setor Cimba

CEP 77.814970
\newpage\vspace*{-4cm}\subsection*{Carta de Recomendação - Sinval de Oliveira}Código Identificador: 1232\\Conhece-o candidato há quanto tempo (For how long have you known the applicant)? 
\ 24 meses
\\ Conhece-o sob as seguintes circunstâncias: aulas\ \ 
	\ \ \ \ outra 
\\ Conheçe o candidato sob outras circunstâncias: Bolsista Pibid
\\Avaliações: \\
\begin{tabular}{|l|c|c|c|c|c|}
\hline
 & Excelente & Bom & Regular & Insuficiente & Não sabe \\
\hline
Desempenho acadêmico & X &  &  &  & \\
\hline
Capacidade de aprender novos conceitos & X &  &  &  & \\
\hline
Capacidade de trabalhar sozinho & X &  &  &  & \\
\hline
Criatividade & X &  &  &  & \\
\hline
Curiosidade & X &  &  &  & \\
\hline
Esforço, persistência & X &  &  &  & \\
\hline
Expressão escrita &  & X &  &  & \\
\hline
Expressão oral & X &  &  &  & \\
\hline
Relacionamento com colegas & X &  &  &  & \\
\hline
\end{tabular}\\
\\
\textbf{Opinião sobre os antecedentes acadêmicos, profissionais e/ou técnicos do candidato:}
\\Karen Brito Miranda foi uma das alunas exemplares do Curso de Licenciatura em Matemática da UFT, Campus de Araguaína. Sempre dedicada, com hábitos de estudos regulares. Nos dois últimos semestre de seu curso participou de ações voltadas para formação matemática avançada.
A candidata foi minha aluna de Estágio no ensino médio, onde desenvolveu com responsabilidade todas as ações previstas.  \\
\\
\textbf{Opinião sobre seu possível aproveitamento, se aceito no Programa:}
\\Creio que a candidata reuni qualidades importantes para conduzir estudos de caráter avançado no âmbito da matemática, entre elas, persistência, hábitos de estudos. \\ 
\\
\textbf{Outras informações relevantes:} \\Durante 18 meses a candidata foi minha orientada no Programa PIBID, onde conduziu com responsabilidade as tarefas que lhe foram confiadas. O mérito de seu trabalho no projeto foi reconhecido externamente através de parecer por duplo cego com a publicação de um artigo na forma de relato de experiência no XI Encontro Nacional de Educação Matemática. 
\\[0.3cm]
\textbf{Entre os estudantes que já conheceu, você diria que o candidato está entre os:}
\\
\begin{tabular}{|l|c|c|c|c|c|}
\hline
 & 5\% melhores & 10\% melhores & 25\% melhores & 50\% melhores & Não sabe \\
\hline
Como aluno, em aulas & X &  &  &  & \\
\hline
Como orientando & X &  &  &  & \\
\hline
\end{tabular}
\subsection*{Dados Recomendante} 
	Instituição (Institution): Universidade Federal do Tocantins  UFT Campus de Araguaína
\\ 
	Grau acadêmico mais alto obtido: doutor
	\ \ Área: Educação Matemática
	\\
	Ano de obtenção deste grau: 2013
	\ \ 
	Instituição de obtenção deste grau : Univerisdade Estadual Paulista Júlio de Mesquita Filho UNESP Campus de Rio Claro SP
	\\ 
	Endereço institucional do recomendante: \\ Rua Fortaleza, 133 Casa 03 Setor Brasil
77824390 Araguaína TO\includepdf[pages={-},offset=35mm 0mm]{../../../upload/844_2014-05-07_historico.pdf}\includepdf[pages={-},offset=35mm 0mm]{../../../upload/844_2013-10-31_documentos.pdf}\includepdf[pages={-},offset=35mm 0mm]{../../../upload/844_2013-10-31_historico.pdf}\includepdf[pages={-},offset=35mm 0mm]{../../../upload/844_2013-10-24_documentos.pdf}\includepdf[pages={-},offset=35mm 0mm]{../../../upload/844_2013-10-24_historico.pdf} 
\begin{center}
Anexos.
\end{center}
\end{document}