\documentclass[11pt]{article}
\usepackage{graphicx,color}
\usepackage{pdfpages}
\usepackage[brazil]{babel}
\usepackage[utf8]{inputenc}
\addtolength{\hoffset}{-3cm} \addtolength{\textwidth}{6cm}
\addtolength{\voffset}{-.5cm} \addtolength{\textheight}{1cm}
%%%%%%%%%%%%%%%%%%%%%%%%%%%%%%%%%%%  To use Colors 
\title{\vspace*{-4cm} Ficha de Inscrição: \\Cod: 635\ \ Matheus Giacomel Viero\ \ - \ \ Mestrado 
 }
\date{}

\begin{document}
\maketitle
\vspace*{-1.5cm}
\noindent Data de Nascimento:22/1/1987
\ \ \ Idade: 27   \ \ \ Sexo: Masculino
\\
Naturalidade: Farroupilha  
\ \ \  Estado: RS
\ \ \  Nacionalidade: Brasileiro
\ \ \ País: Brasil
\\        
Nome do pai : Larry Juares Viero
\ \ \ Nome da mãe: Rosa Maria Giacomel          
\\[0.2cm]                     
\textbf{Endereço Pessoal} 
\\ 
\noindent Endereço residencial: Rua Ari Zanonato n 29
\\
        CEP: 95180-000 
\ \ \ Cidade: Farroupilha 
\ \ \ Estado: RS 
\ \ \ País: Brasil
\\		
		Telefone comercial : +51(54)32612903
\ \ \ Telefone residencial: +51(54)32689043
\ \ \ Telefone celular : +51(54)81330016
\\
E-mail principal: mgviero@gmail.com
\ \ \ E-mail alternativo: shape\_mu@hotmail.com 
\\[0.2cm] 
\textbf{Documentos Pessoais}
\\
\noindent Número de CPF : 01795158026
\ \ \ Número de Identidade (ou Passaporte para estrangeiros): 7092350987
\\
Orgão emissor: SSP
\ \ \ Estado: RS
\ \ \ Data de emissão :22/1/1987
\\[0.3cm]
\textbf{Grau acadêmico mais alto obtido}
\\	
Curso:Matemática
\ \ \ Grau : bacharel
\ \ \ Instituição : Universidade Federal do Rio Grande
\\			
Ano de Conclusão ou Previsão: 2013
\\ 
Experiência Profissional mais recente. \ \  
Tem experiência: Docente Discente  
\ \ \ Instituição: 0
\\  
Período - início: 0-0
\ \ \ fim: 0-0
\\[0.2cm] 
\textbf{Programa Pretendido:} Mestrado\\
Interesse em bolsa: Sim
\\[0.3cm]		
\textbf{Dados dos Recomendantes} 
\\
1- Nome: Adriano De Cezaro
\ \ \ \  e-mail: decezaromtm@gmail.com 
\\
2- Nome: Matheus Jatkoske Lazo
\ \ \ \ e-mail: matheusjlazo@gmail.com
\\
3- Nome: Leandro Sebben Bellicanta
\ \ \ \ e-mail: lscanta@gmail.com
\\[0.2cm]
Motivação e expectativa do candidato em relação ao programa pretendido:
\\Me formei em matemática aplicada e cursei todas as cadeiras do curso de ciências econômicas, falta apenas apresentar o tcc para me formar em economia. Meu interesse é estudar na área de inferência em processos estocásticos área que já estudei para meu trabalho de conclusão de curso, no qual o tema principal foi a obtenção e solução da equação de Black Scholes para opções de compra do tipo europeia. Minha expectativa é consolidar meu conhecimento em matemática e me dedicar o máximo possível ao programa.\newpage\vspace*{-4cm}\subsection*{Carta de Recomendação - Adriano De Cezaro}Código Identificador: 642\\Conhece-o candidato há quanto tempo (For how long have you known the applicant)? 
\ 4,5 anos
\\ Conhece-o sob as seguintes circunstâncias: aulas\ \ orientacao
	\ \ \ \  
\\ Conheçe o candidato sob outras circunstâncias: 
\\	Avaliações:\\
\begin{tabular}{|l|c|c|c|c|c|}
\hline
 & Excelente & Bom & Regular & Insuficiente & Não sabe \\
\hline
Desempenho acadêmico &  & X &  &  & \\
\hline
Capacidade de aprender novos conceitos &  & X &  &  & \\
\hline
Capacidade de trabalhar sozinho & X &  &  &  & \\
\hline
Criatividade &  & X &  &  & \\
\hline
Curiosidade & X &  &  &  & \\
\hline
Esforço, persistência & X &  &  &  & \\
\hline
Expressão escrita &  & X &  &  & \\
\hline
Expressão oral &  & X &  &  & \\
\hline
Relacionamento com colegas & X &  &  &  & \\
\hline
\end{tabular}\\
\\
\textbf{Opinião sobre os antecedentes acadêmicos, profissionais e/ou técnicos do candidato:}
\\O Matheus concluiu o curso de Bacharelado em Matemática Aplicada na FURG em 2013, concumitantemente, com o curso de Administração pela mesma instituição. Durante boa parte de sua graduação o Matheus manteve atividades de pesquisa em Iniciação científica sob minha orientação, relacionando conceitos de matemática e suas aplicação a economia e finanças quantitativas. Deste modo, além das disciplinas correntos do curso de graduação, o Matheus também estudou a parte conceitos necessários para o desenvolvimento de sua pesquia, a qual culminou com o sue trabalho de conclusão de curso. Dos cursos dos quais ele foi meu aluno, sempre teve destacada posição entre os alunos, mostrando maturidade, criatividade e desenvoltura para resolver problemas e entender conceitos matemáticos.
\\
\\
\textbf{Opinião sobre seu possível aproveitamento, se aceito no Programa:}
\\Dada a formação acadêmica do Matheus, eu acredito que seu desempenho seja muito bom, de forma que eu não exitaria em aceitálo em meu programa. 

Dos cursos dos quais ele foi meu aluno, sempre teve destacada posição entre os alunos, mostrando maturidade, criatividade e desenvoltura para resolver problemas e entender conceitos.\\ 
\\
\textbf{Outras informações relevantes:} \\O Matheus esteve envolvido com atividades de iniciação científica durante toda a sua formação, mostrando independência científica desde o início. 

Assim, eu o recomendo para o curso de mestrado em matemática da UNB
\\[0.3cm]
\textbf{Entre os estudantes que já conheceu, você diria que o candidato está entre os:}
\\
\begin{tabular}{|l|c|c|c|c|c|}
\hline
 & 5\% melhores & 10\% melhores & 25\% melhores & 50\% melhores & Não sabe \\
\hline
Como aluno, em aulas &  & X &  &  & \\
\hline
Como orientando & X &  &  &  & \\
\hline
\end{tabular}
\subsection*{Dados Recomendante} 
	Instituição (Institution): Universidade Federal do Rio Grande
\\ 
	Grau acadêmico mais alto obtido: doutor
	\ \ Área: Matemática Aplicada - Análise Numérica
	\\
	Ano de obtenção deste grau: 2010
	\ \ 
	Instituição de obtenção deste grau : IMPA
	\\ 
	Endereço institucional do recomendante: \\ Universidade Federal do Rio Grande FURG
Instituto de Matemática Estatística e Física IMEF
Av. Itália km 8 Campus Carreiros  96203-900 Rio Grande RS\newpage\vspace*{-4cm}\subsection*{Carta de Recomendação - Matheus Jatkoske Lazo}Código Identificador: 643\\Conhece-o candidato há quanto tempo (For how long have you known the applicant)? 
\ 
\\ Conhece-o sob as seguintes circunstâncias: \ \ 
	\ \ \ \  
\\ Conheçe o candidato sob outras circunstâncias: 
\\Avaliações: \\
\begin{tabular}{|l|c|c|c|c|c|}
\hline
 & Excelente & Bom & Regular & Insuficiente & Não sabe \\
\hline
Desempenho acadêmico &  &  &  &  & \\
\hline
Capacidade de aprender novos conceitos &  &  &  &  & \\
\hline
Capacidade de trabalhar sozinho &  &  &  &  & \\
\hline
Criatividade &  &  &  &  & \\
\hline
Curiosidade &  &  &  &  & \\
\hline
Esforço, persistência &  &  &  &  & \\
\hline
Expressão escrita &  &  &  &  & \\
\hline
Expressão oral &  &  &  &  & \\
\hline
Relacionamento com colegas &  &  &  &  & \\
\hline
\end{tabular}\\
\\
\textbf{Opinião sobre os antecedentes acadêmicos, profissionais e/ou técnicos do candidato:}
\\\\
\\
\textbf{Opinião sobre seu possível aproveitamento, se aceito no Programa:}
\\\\ 
\\
\textbf{Outras informações relevantes:} \\
\\[0.3cm]
\textbf{Entre os estudantes que já conheceu, você diria que o candidato está entre os:}
\\
\begin{tabular}{|l|c|c|c|c|c|}
\hline
 & 5\% melhores & 10\% melhores & 25\% melhores & 50\% melhores & Não sabe \\
\hline
Como aluno, em aulas &  &  &  &  & \\
\hline
Como orientando &  &  &  &  & \\
\hline
\end{tabular}
\subsection*{Dados Recomendante} 
	Instituição (Institution): Instituto de Matemática, Estatística e Física - Universidade Federal do Rio Grande
\\ 
	Grau acadêmico mais alto obtido: doutor
	\ \ Área: Física Matemática
	\\
	Ano de obtenção deste grau: 2006
	\ \ 
	Instituição de obtenção deste grau : USP/São Carlos
	\\ 
	Endereço institucional do recomendante: \\ Instituto de Matemática, Estatística e Física
Universidade Federal do Rio Grande - FURG
Campus Carreiros
Av. Itália, Km 8
Rio Grande - RS
Brasil
96.201-900\newpage\vspace*{-4cm}\subsection*{Carta de Recomendação - Leandro Sebben Bellicanta}Código Identificador: 644\\Conhece-o candidato há quanto tempo (For how long have you known the applicant)? 
\ Três anos.
\\ Conhece-o sob as seguintes circunstâncias: aulas\ \ 
	\ \ \ \  
\\ Conheçe o candidato sob outras circunstâncias: 
\\Avaliações: \\
\begin{tabular}{|l|c|c|c|c|c|}
\hline
 & Excelente & Bom & Regular & Insuficiente & Não sabe \\
\hline
Desempenho acadêmico &  & X &  &  & \\
\hline
Capacidade de aprender novos conceitos &  & X &  &  & \\
\hline
Capacidade de trabalhar sozinho & X &  &  &  & \\
\hline
Criatividade &  &  &  &  & X\\
\hline
Curiosidade & X &  &  &  & \\
\hline
Esforço, persistência & X &  &  &  & \\
\hline
Expressão escrita &  & X &  &  & \\
\hline
Expressão oral &  & X &  &  & \\
\hline
Relacionamento com colegas & X &  &  &  & \\
\hline
\end{tabular}\\
\\
\textbf{Opinião sobre os antecedentes acadêmicos, profissionais e/ou técnicos do candidato:}
\\O candidato realizou dois cursos de graduação  Matemática Aplicada e Economia  simultaneamente, o que, de certa forma, demonstra uma boa capacidade de trabalho.\\
\\
\textbf{Opinião sobre seu possível aproveitamento, se aceito no Programa:}
\\Acredito que o candidato possui condições intelectuais e emocionais para concluir um curso de Mestrado em Matemática.\\ 
\\
\textbf{Outras informações relevantes:} \\Acredito ainda que o candidato esteja decidido a fazer um curso de Pós graduação, na área de matemática, em uma grande instituição.
\\[0.3cm]
\textbf{Entre os estudantes que já conheceu, você diria que o candidato está entre os:}
\\
\begin{tabular}{|l|c|c|c|c|c|}
\hline
 & 5\% melhores & 10\% melhores & 25\% melhores & 50\% melhores & Não sabe \\
\hline
Como aluno, em aulas &  & X &  &  & \\
\hline
Como orientando &  &  &  &  & X\\
\hline
\end{tabular}
\subsection*{Dados Recomendante} 
	Instituição (Institution): Universidade Federal do Rio Grande - FURG.
\\ 
	Grau acadêmico mais alto obtido: doutor
	\ \ Área: Mat. Aplicada
	\\
	Ano de obtenção deste grau: 2002
	\ \ 
	Instituição de obtenção deste grau : IME - USP - São Paulo.
	\\ 
	Endereço institucional do recomendante: \\ Universidade Federal do Rio Grande - FURG
Sala L18 - IMEF
Av Itália Km 8, Campus Carreiros,
Rio Grande RS
96203-900\includepdf[pages={-},offset=35mm 0mm]{../../../upload/635_2014-05-29_documentos.pdf}\includepdf[pages={-},offset=35mm 0mm]{../../../upload/635_2014-05-29_historico.pdf}\includepdf[pages={-},offset=35mm 0mm]{../../../upload/635_2013-06-04_documentos.pdf}\includepdf[pages={-},offset=35mm 0mm]{../../../upload/635_2013-06-04_historico.pdf} 
\begin{center}
Anexos.
\end{center}
\end{document}