\documentclass[11pt]{article}
\usepackage{graphicx,color}
\usepackage{pdfpages}
\usepackage[brazil]{babel}
\usepackage[utf8]{inputenc}
\addtolength{\hoffset}{-3cm} \addtolength{\textwidth}{6cm}
\addtolength{\voffset}{-.5cm} \addtolength{\textheight}{1cm}
%%%%%%%%%%%%%%%%%%%%%%%%%%%%%%%%%%%  To use Colors 
\title{\vspace*{-4cm} Ficha de Inscrição: \\Cod: 1344\ \ Talita  Santos\ \ - \ \ Mestrado 
 }
\date{}

\begin{document}
\maketitle
\vspace*{-1.5cm}
\noindent Data de Nascimento:17/7/1991
\ \ \ Idade: 23   \ \ \ Sexo: Feminino
\\
Naturalidade: Nanuquense   
\ \ \  Estado: MG
\ \ \  Nacionalidade: Brasileira
\ \ \ País: Brasil
\\        
Nome do pai : Antonio Luiz dos santos
\ \ \ Nome da mãe: Sonia Copo dos Santos          
\\[0.2cm]                     
\textbf{Endereço Pessoal} 
\\ 
\noindent Endereço residencial: Rua Pouso Alegre, 122
\\
        CEP: 39860-000 
\ \ \ Cidade: Nanuque 
\ \ \ Estado: MG 
\ \ \ País: Brasil
\\		
		Telefone comercial : +55(33)36211055
\ \ \ Telefone residencial: +55(33)36211055
\ \ \ Telefone celular : +55(27)996399881
\\
E-mail principal: talita-c.s@hotmail.com
\ \ \ E-mail alternativo: talitagirl155@hotmail.com 
\\[0.2cm] 
\textbf{Documentos Pessoais}
\\
\noindent Número de CPF : 10586122630
\ \ \ Número de Identidade (ou Passaporte para estrangeiros): 16011395
\\
Orgão emissor: SSP
\ \ \ Estado: MG
\ \ \ Data de emissão :23/9/2005
\\[0.3cm]
\textbf{Grau acadêmico mais alto obtido}
\\	
Curso:Matematica
\ \ \ Grau : bacharel
\ \ \ Instituição : Universidade Federal do Espirito Santo UFES
\\			
Ano de Conclusão ou Previsão: 2014
\\ 
Experiência Profissional mais recente. \ \  
Tem experiência: Docente Discente  
\ \ \ Instituição: 0
\\  
Período - início: 0-0
\ \ \ fim: 0-0
\\[0.2cm] 
\textbf{Programa Pretendido:} Mestrado\\
Interesse em bolsa: Sim
\\[0.3cm]		
\textbf{Dados dos Recomendantes} 
\\
1- Nome: Michel Guimaraes Coswosck
\ \ \ \  e-mail: michelguimaraes96@gmail.com 
\\
2- Nome: Aldo Vignatti
\ \ \ \ e-mail: aldovignatti5@gmail.com
\\
3- Nome: Isaac Pinheiro dos Santos
\ \ \ \ e-mail: isaacps@gmail.com
\\[0.2cm]
Motivação e expectativa do candidato em relação ao programa pretendido:
\\Durante o curso de matemática bacharelado, escolhi seguir a área de matemática pura, após ter cursado algumas disciplinas como, Álgebra Linear I, Álgebra Linear II, Estruturas Algébricas, Analise I e Analise II. Logo depois, me encantei ainda mais pela matemática pura quando realizei as disciplinas Álgebra I, Álgebra II, Tópicos em Álgebra e Calculo Avançado. Foi através dessas disciplinas que resolvi seguir a área de pesquisa em matemática pura.
Meu interesse no ingresso na Universidade de Brasilia está inteiramente ligado às áreas de pesquisas realizadas em álgebra. Outro grande fator que influenciou a minha escolha, foi o fato da universidade ser bem conceituada no mercado, ter qualidade de ensino e um quadro docente muito bom e diversificado.
\newpage\vspace*{-4cm}\subsection*{Carta de Recomendação - Michel Guimaraes Coswosck}Código Identificador: 1348\\Conhece-o candidato há quanto tempo (For how long have you known the applicant)? 
\ 3 anos
\\ Conhece-o sob as seguintes circunstâncias: aulas\ \ 
	\ \ \ \  
\\ Conheçe o candidato sob outras circunstâncias: 
\\	Avaliações:\\
\begin{tabular}{|l|c|c|c|c|c|}
\hline
 & Excelente & Bom & Regular & Insuficiente & Não sabe \\
\hline
Desempenho acadêmico &  & X &  &  & \\
\hline
Capacidade de aprender novos conceitos &  & X &  &  & \\
\hline
Capacidade de trabalhar sozinho &  & X &  &  & \\
\hline
Criatividade &  & X &  &  & \\
\hline
Curiosidade &  & X &  &  & \\
\hline
Esforço, persistência &  & X &  &  & \\
\hline
Expressão escrita &  & X &  &  & \\
\hline
Expressão oral &  & X &  &  & \\
\hline
Relacionamento com colegas &  & X &  &  & \\
\hline
\end{tabular}\\
\\
\textbf{Opinião sobre os antecedentes acadêmicos, profissionais e/ou técnicos do candidato:}
\\A candidata teve um bom desempenho nas disciplinas nas quais ministrei. Se destacou pela participação nas aulas,perserverância e na capacidade de resolver problemas. Exerce a atividade de monitora. \\
\\
\textbf{Opinião sobre seu possível aproveitamento, se aceito no Programa:}
\\Acredito que a candidata, caso seja aceita no programa de mestrado da instituição, terá um bom desempenho. A aluna possue interesse pelo conhecimento matemático e motivação para estudos avançados. \\ 
\\
\textbf{Outras informações relevantes:} \\
\\[0.3cm]
\textbf{Entre os estudantes que já conheceu, você diria que o candidato está entre os:}
\\
\begin{tabular}{|l|c|c|c|c|c|}
\hline
 & 5\% melhores & 10\% melhores & 25\% melhores & 50\% melhores & Não sabe \\
\hline
Como aluno, em aulas &  & X &  &  & \\
\hline
Como orientando &  &  & X &  & \\
\hline
\end{tabular}
\subsection*{Dados Recomendante} 
	Instituição (Institution): Universidade Federal do Espírito Santo 
\\ 
	Grau acadêmico mais alto obtido: mestre
	\ \ Área: Teoria dos Números
	\\
	Ano de obtenção deste grau: 2002
	\ \ 
	Instituição de obtenção deste grau : Universidade Federal Fluminense
	\\ 
	Endereço institucional do recomendante: \\ michelcoswosckceunes.ufes.br\newpage\vspace*{-4cm}\subsection*{Carta de Recomendação - Aldo Vignatti}Código Identificador: 1349\\Conhece-o candidato há quanto tempo (For how long have you known the applicant)? 
\ 3 anos
\\ Conhece-o sob as seguintes circunstâncias: aulas\ \ 
	\ \ \ \ outra 
\\ Conheçe o candidato sob outras circunstâncias: Além das aulas também como aluna dos colegas de trabalho
\\Avaliações: \\
\begin{tabular}{|l|c|c|c|c|c|}
\hline
 & Excelente & Bom & Regular & Insuficiente & Não sabe \\
\hline
Desempenho acadêmico &  & X &  &  & \\
\hline
Capacidade de aprender novos conceitos & X &  &  &  & \\
\hline
Capacidade de trabalhar sozinho & X &  &  &  & \\
\hline
Criatividade &  & X &  &  & \\
\hline
Curiosidade &  & X &  &  & \\
\hline
Esforço, persistência & X &  &  &  & \\
\hline
Expressão escrita &  & X &  &  & \\
\hline
Expressão oral & X &  &  &  & \\
\hline
Relacionamento com colegas & X &  &  &  & \\
\hline
\end{tabular}\\
\\
\textbf{Opinião sobre os antecedentes acadêmicos, profissionais e/ou técnicos do candidato:}
\\Como aluna, sempre recebeu elogios dos colegas de trabalho. A candidata possui uma boa formação em matemática. Possui conhecimento suficiente para ingressar em um mestrado que exija matemática aplicada ou matemática pura.\\
\\
\textbf{Opinião sobre seu possível aproveitamento, se aceito no Programa:}
\\Acredito que se possuir condições financeiras, por exemplo, uma bolsa, para que possa se dedicar integralmente, a candidata cumprirá todos os requisitos exigidos para a conclusão do mestrado e no período esperado, ou seja, dois anos.\\ 
\\
\textbf{Outras informações relevantes:} \\
\\[0.3cm]
\textbf{Entre os estudantes que já conheceu, você diria que o candidato está entre os:}
\\
\begin{tabular}{|l|c|c|c|c|c|}
\hline
 & 5\% melhores & 10\% melhores & 25\% melhores & 50\% melhores & Não sabe \\
\hline
Como aluno, em aulas & X &  &  &  & \\
\hline
Como orientando & X &  &  &  & \\
\hline
\end{tabular}
\subsection*{Dados Recomendante} 
	Instituição (Institution): UFES
\\ 
	Grau acadêmico mais alto obtido: doutor
	\ \ Área: Análise. Matemática
	\\
	Ano de obtenção deste grau: 2005
	\ \ 
	Instituição de obtenção deste grau : UFRJ
	\\ 
	Endereço institucional do recomendante: \\ Centro Universitário Norte do Espírito Santo
Rodovia BR 101 Norte, Km 60, Bairro Litorâneo, CEP. 29.932540, Tel. 55 27 3312.1511, Fax. 55 27 3312.1510
São Mateus  ES  Sítio Eletrônico  httpwww.ceunes.ufes.br
\newpage\vspace*{-4cm}\subsection*{Carta de Recomendação - Isaac Pinheiro dos Santos}Código Identificador: 1350\\Conhece-o candidato há quanto tempo (For how long have you known the applicant)? 
\ 04 anos
\\ Conhece-o sob as seguintes circunstâncias: aulas\ \ 
	\ \ \ \ outra 
\\ Conheçe o candidato sob outras circunstâncias: Como coordenador de curso
\\Avaliações: \\
\begin{tabular}{|l|c|c|c|c|c|}
\hline
 & Excelente & Bom & Regular & Insuficiente & Não sabe \\
\hline
Desempenho acadêmico &  & X &  &  & \\
\hline
Capacidade de aprender novos conceitos &  & X &  &  & \\
\hline
Capacidade de trabalhar sozinho & X &  &  &  & \\
\hline
Criatividade &  & X &  &  & \\
\hline
Curiosidade & X &  &  &  & \\
\hline
Esforço, persistência &  & X &  &  & \\
\hline
Expressão escrita &  &  & X &  & \\
\hline
Expressão oral &  & X &  &  & \\
\hline
Relacionamento com colegas & X &  &  &  & \\
\hline
\end{tabular}\\
\\
\textbf{Opinião sobre os antecedentes acadêmicos, profissionais e/ou técnicos do candidato:}
\\A Talita é estudante do curso de bacharelado em Matemática da UFES, Campus de São Mateus. Ela tem se mostrado uma aluna muito interessada em estudar tópicos avançados de matemática. Seu desempenho no curso é bom e eu acredito que ela tem condições de dar prosseguimento nos estudos a nível de mestrado, e posteriormente, doutorado.\\
\\
\textbf{Opinião sobre seu possível aproveitamento, se aceito no Programa:}
\\Observando o seu desempenho e esforço na graduação, com certeza ela não terá muita dificuldade em concluir um curso de mestrado.\\ 
\\
\textbf{Outras informações relevantes:} \\Além de ser uma aluna dedicada, é uma pessoa dócil que tem um bom relacionamento com colegas, técnicos e professores.
\\[0.3cm]
\textbf{Entre os estudantes que já conheceu, você diria que o candidato está entre os:}
\\
\begin{tabular}{|l|c|c|c|c|c|}
\hline
 & 5\% melhores & 10\% melhores & 25\% melhores & 50\% melhores & Não sabe \\
\hline
Como aluno, em aulas &  & X &  &  & \\
\hline
Como orientando &  &  &  &  & X\\
\hline
\end{tabular}
\subsection*{Dados Recomendante} 
	Instituição (Institution): Universidade Federal do Espírito Santo
\\ 
	Grau acadêmico mais alto obtido: doutor
	\ \ Área: Modelagem Computacional
	\\
	Ano de obtenção deste grau: 2007
	\ \ 
	Instituição de obtenção deste grau : Laboratório Nacional de Computação Científica, LNCC
	\\ 
	Endereço institucional do recomendante: \\ Departamento de Matemática Aplicada  DMA
Centro Universitário Norte do Espírito Santo  CEUNES
Universidade Federal do Espírito Santo  UFES
Rodovia BR 101 Norte, km. 60, Bairro Litorâneo
CEP. 29932540 São Mateus  ES \includepdf[pages={-},offset=35mm 0mm]{../../../upload/1344_2014-05-26_documentos.pdf}\includepdf[pages={-},offset=35mm 0mm]{../../../upload/1344_2014-05-26_historico.pdf} 
\begin{center}
Anexos.
\end{center}
\end{document}