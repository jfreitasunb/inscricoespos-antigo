\documentclass[11pt]{article}
\usepackage{graphicx,color}
\usepackage{pdfpages}
\usepackage[brazil]{babel}
\usepackage[utf8]{inputenc}
\addtolength{\hoffset}{-3cm} \addtolength{\textwidth}{6cm}
\addtolength{\voffset}{-.5cm} \addtolength{\textheight}{1cm}
%%%%%%%%%%%%%%%%%%%%%%%%%%%%%%%%%%%  To use Colors 
\title{\vspace*{-4cm} Ficha de Inscrição: \\Cod: 1261\ \ TONY KLEVERSON NOGUEIRA\ \ - \ \ Doutorado 
 }
\date{}

\begin{document}
\maketitle
\vspace*{-1.5cm}
\noindent Data de Nascimento:8/6/1989
\ \ \ Idade: 25   \ \ \ Sexo: Masculino
\\
Naturalidade: LIMOEIRO DO NORTE  
\ \ \  Estado: CE
\ \ \  Nacionalidade: BRASILEIRO
\ \ \ País: BRASIL
\\        
Nome do pai : CLEONICE LUCAS NOGUEIRA
\ \ \ Nome da mãe: CLEONICE LUCAS NOGUEIRA          
\\[0.2cm]                     
\textbf{Endereço Pessoal} 
\\ 
\noindent Endereço residencial: RUA EDUVAL XAVIER PINTO, 2754
\\
        CEP: 62900-000 
\ \ \ Cidade: RUSSAS 
\ \ \ Estado: PB 
\ \ \ País: BRASIL
\\		
		Telefone comercial : +55(83)87278738
\ \ \ Telefone residencial: +55(83)98011937
\ \ \ Telefone celular : +55(88)96883628
\\
E-mail principal: tonykleverson@hotmail.com
\ \ \ E-mail alternativo: tonykleverson@gmail.com 
\\[0.2cm] 
\textbf{Documentos Pessoais}
\\
\noindent Número de CPF : 03069437354
\ \ \ Número de Identidade (ou Passaporte para estrangeiros): 2005030044687
\\
Orgão emissor: SSP
\ \ \ Estado: PB
\ \ \ Data de emissão :18/11/2009
\\[0.3cm]
\textbf{Grau acadêmico mais alto obtido}
\\	
Curso:MATEMÁTICA
\ \ \ Grau : mestre
\ \ \ Instituição : UFPB
\\			
Ano de Conclusão ou Previsão: 2014
\\ 
Experiência Profissional mais recente. \ \  
Tem experiência: Docente Discente  
\ \ \ Instituição: UFPB
\\  
Período - início: 1-2012
\ \ \ fim: 1-2014
\\[0.2cm] 
\textbf{Programa Pretendido:} Doutorado\ \ \ \textbf{Área:} Analise\\
Interesse em bolsa: Sim
\\[0.3cm]		
\textbf{Dados dos Recomendantes} 
\\
1- Nome: Daniel Marinho Pellegrino
\ \ \ \  e-mail: dmpellegrino@gmail.com 
\\
2- Nome: Flank David Morais Bezerra
\ \ \ \ e-mail: flank@mat.ufpb.br
\\
3- Nome: Joedson Silva dos Santos
\ \ \ \ e-mail: joedsonmat@gmail.com
\\[0.2cm]
Motivação e expectativa do candidato em relação ao programa pretendido:
\\O fato de ser um centro de excelência, de conceito máximo perante a Capes, já é uma motivação e tanto para eu querer cursar o doutorado na UnB. Os meus professores falam bem e indicam esse centro. Além disso, tenho amigos cursando o doutorado ai e se mostram bem satisfeitos com o programa, a instituição e os professores. Minha expectativa é, primeiro, de ser selecionado para o programa, e depois disso me esforçar o máximo nas disciplinas, a fim de ficar pronto para a pesquisa e depois produzir o máximo que eu puder. \newpage\vspace*{-4cm}\subsection*{Carta de Recomendação - Daniel Marinho Pellegrino}Código Identificador: 1385\\Conhece-o candidato há quanto tempo (For how long have you known the applicant)? 
\ 3 anos
\\ Conhece-o sob as seguintes circunstâncias: aulas\ \ orientacao
	\ \ seminarios\ \  
\\ Conheçe o candidato sob outras circunstâncias: 
\\	Avaliações:\\
\begin{tabular}{|l|c|c|c|c|c|}
\hline
 & Excelente & Bom & Regular & Insuficiente & Não sabe \\
\hline
Desempenho acadêmico &  & X &  &  & \\
\hline
Capacidade de aprender novos conceitos &  & X &  &  & \\
\hline
Capacidade de trabalhar sozinho & X &  &  &  & \\
\hline
Criatividade &  & X &  &  & \\
\hline
Curiosidade & X &  &  &  & \\
\hline
Esforço, persistência & X &  &  &  & \\
\hline
Expressão escrita &  & X &  &  & \\
\hline
Expressão oral &  & X &  &  & \\
\hline
Relacionamento com colegas & X &  &  &  & \\
\hline
\end{tabular}\\
\\
\textbf{Opinião sobre os antecedentes acadêmicos, profissionais e/ou técnicos do candidato:}
\\O candidato tem boa formação acadêmica e conhecimentos suficientes para ingressar num programa de doutorado em matemática.\\
\\
\textbf{Opinião sobre seu possível aproveitamento, se aceito no Programa:}
\\O candidato tem excelente capacidade de trabalho, perseverança e conhecimento matemático. Acredito que tem condições de ter um bom desempenho no programa.\\ 
\\
\textbf{Outras informações relevantes:} \\
\\[0.3cm]
\textbf{Entre os estudantes que já conheceu, você diria que o candidato está entre os:}
\\
\begin{tabular}{|l|c|c|c|c|c|}
\hline
 & 5\% melhores & 10\% melhores & 25\% melhores & 50\% melhores & Não sabe \\
\hline
Como aluno, em aulas &  & X &  &  & \\
\hline
Como orientando & X &  &  &  & \\
\hline
\end{tabular}
\subsection*{Dados Recomendante} 
	Instituição (Institution): UFPB
\\ 
	Grau acadêmico mais alto obtido: doutor
	\ \ Área: Matemática
	\\
	Ano de obtenção deste grau: 2002
	\ \ 
	Instituição de obtenção deste grau : Unicamp
	\\ 
	Endereço institucional do recomendante: \\  Universidade Federal da Paraíba  Campus I
Departamento de Matemática
Coordenação do Programa de Pósgraduação em Matemática
CEP 58.051900, João Pessoa  PB\newpage\vspace*{-4cm}\subsection*{Carta de Recomendação - Flank David Morais Bezerra}Código Identificador: 743\\Conhece-o candidato há quanto tempo (For how long have you known the applicant)? 
\ Dois anos e seis meses
\\ Conhece-o sob as seguintes circunstâncias: aulas\ \ 
	\ \ \ \  
\\ Conheçe o candidato sob outras circunstâncias: 
\\Avaliações: \\
\begin{tabular}{|l|c|c|c|c|c|}
\hline
 & Excelente & Bom & Regular & Insuficiente & Não sabe \\
\hline
Desempenho acadêmico & X &  &  &  & \\
\hline
Capacidade de aprender novos conceitos & X &  &  &  & \\
\hline
Capacidade de trabalhar sozinho &  & X &  &  & \\
\hline
Criatividade &  & X &  &  & \\
\hline
Curiosidade & X &  &  &  & \\
\hline
Esforço, persistência & X &  &  &  & \\
\hline
Expressão escrita &  &  & X &  & \\
\hline
Expressão oral &  &  & X &  & \\
\hline
Relacionamento com colegas & X &  &  &  & \\
\hline
\end{tabular}\\
\\
\textbf{Opinião sobre os antecedentes acadêmicos, profissionais e/ou técnicos do candidato:}
\\O candidato Tony K. Nogueira foi meu aluno na disciplina Análise no RN no primeiro semestre de 2012 e obteve um excelente desempenho, o mesmo foi aprovado com uma das  duas melhores médias da turma. O aluno Tony, naquela ocasião, havia recentemente aceito no nosso Programa de Pós Graduação de Mestrado Acadêmico em Matemática com uma das melhores notas e desempenho no curso de Verão. Certamente foi um dos melhores alunos que tive contato nestes três anos como membro do programa de pós graduação.\\
\\
\textbf{Opinião sobre seu possível aproveitamento, se aceito no Programa:}
\\O candidato durante o seu mestrado acadêmico em Matemática obteve um ótimo desempenho, além de um bom desempenho nas disciplinas cursadas ao longo do mesmo, tem apresentado um grande interesse em dar continuidade aos estudos de pós graduação, eu recomendo fortemente o candidato para o programa de pós graduação de doutorado em Matemática. \\ 
\\
\textbf{Outras informações relevantes:} \\O candidato Tony K. Nogueira foi orientado no seu mestrado pelo Prof. Dr. Daniel M. Pellegrino, membro permanente do nosso Programa de Pós Graduação de Mestrado Acadêmica em Matemática, e renomado pesquisador em Análise Funcional e Teoria dos Números, e atualmente segue trabalhando com o Prof. Pellegrino dando continuidade ao seu trabalho de pesquisa em Análise matemática.
\\[0.3cm]
\textbf{Entre os estudantes que já conheceu, você diria que o candidato está entre os:}
\\
\begin{tabular}{|l|c|c|c|c|c|}
\hline
 & 5\% melhores & 10\% melhores & 25\% melhores & 50\% melhores & Não sabe \\
\hline
Como aluno, em aulas & X &  &  &  & \\
\hline
Como orientando &  &  &  &  & X\\
\hline
\end{tabular}
\subsection*{Dados Recomendante} 
	Instituição (Institution): Universidade Federal da Paraíba
\\ 
	Grau acadêmico mais alto obtido: doutor
	\ \ Área: Análise, Sistemas Dinâmicos Não Lineares
	\\
	Ano de obtenção deste grau: 2010
	\ \ 
	Instituição de obtenção deste grau : Universidade de São Paulo
	\\ 
	Endereço institucional do recomendante: \\ Universidade Federal da Paraína, CCEN, Departamento de Matemática, João Pessoa, Cidade Universitária Campua I, CEP 58051-900.\newpage\vspace*{-4cm}\subsection*{Carta de Recomendação - Joedson Silva dos Santos}Código Identificador: 1386\\Conhece-o candidato há quanto tempo (For how long have you known the applicant)? 
\ Quase um ano
\\ Conhece-o sob as seguintes circunstâncias: \ \ 
	\ \ \ \ outra 
\\ Conheçe o candidato sob outras circunstâncias: participei da banca de defesa de dissertação do candidato
\\Avaliações: \\
\begin{tabular}{|l|c|c|c|c|c|}
\hline
 & Excelente & Bom & Regular & Insuficiente & Não sabe \\
\hline
Desempenho acadêmico & X &  &  &  & \\
\hline
Capacidade de aprender novos conceitos & X &  &  &  & \\
\hline
Capacidade de trabalhar sozinho & X &  &  &  & \\
\hline
Criatividade &  & X &  &  & \\
\hline
Curiosidade & X &  &  &  & \\
\hline
Esforço, persistência & X &  &  &  & \\
\hline
Expressão escrita &  & X &  &  & \\
\hline
Expressão oral &  & X &  &  & \\
\hline
Relacionamento com colegas & X &  &  &  & \\
\hline
\end{tabular}\\
\\
\textbf{Opinião sobre os antecedentes acadêmicos, profissionais e/ou técnicos do candidato:}
\\Acompanhei o candidato apenas no último período do mestrado. É um aluno dedicado, que toma iniciativas próprias e que tem uma ótima capacidade de aprendizado.\\
\\
\textbf{Opinião sobre seu possível aproveitamento, se aceito no Programa:}
\\Pelo nível de sua dissertação podemos ver que o candidato possui um bom conhecimento teórico e uma grande habilidade em sua área, Análise, e assim eu julgo que o mesmo estar apto a cursar o doutorado em qualquer IES.\\ 
\\
\textbf{Outras informações relevantes:} \\
\\[0.3cm]
\textbf{Entre os estudantes que já conheceu, você diria que o candidato está entre os:}
\\
\begin{tabular}{|l|c|c|c|c|c|}
\hline
 & 5\% melhores & 10\% melhores & 25\% melhores & 50\% melhores & Não sabe \\
\hline
Como aluno, em aulas & X &  &  &  & \\
\hline
Como orientando & X &  &  &  & \\
\hline
\end{tabular}
\subsection*{Dados Recomendante} 
	Instituição (Institution): Universidade Federal da Paraíba
\\ 
	Grau acadêmico mais alto obtido: doutor
	\ \ Área: Análise
	\\
	Ano de obtenção deste grau: 2011
	\ \ 
	Instituição de obtenção deste grau : Universidade Federal de Pernambuco
	\\ 
	Endereço institucional do recomendante: \\ Universidade Federal da Paraíba, Campus I
Castelo Branco
58051900  João Pessoa, PB, Brasil
\includepdf[pages={-},offset=35mm 0mm]{../../../upload/1261_2014-05-28_documentos.pdf}\includepdf[pages={-},offset=35mm 0mm]{../../../upload/1261_2014-05-28_historico.pdf}	
\begin{figure}[!htb]
\includegraphics{../upload/1261_2014-05-27_documentos.jpg}
\end{figure}\includepdf[pages={-},offset=35mm 0mm]{../../../upload/1261_2014-05-27_historico.pdf} 
\begin{center}
Anexos.
\end{center}
\end{document}