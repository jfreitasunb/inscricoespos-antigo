\documentclass[11pt]{article}
\usepackage{graphicx,color}
\usepackage{pdfpages}
\usepackage[brazil]{babel}
\usepackage[utf8]{inputenc}
\addtolength{\hoffset}{-3cm} \addtolength{\textwidth}{6cm}
\addtolength{\voffset}{-.5cm} \addtolength{\textheight}{1cm}
%%%%%%%%%%%%%%%%%%%%%%%%%%%%%%%%%%%  To use Colors 
\title{\vspace*{-4cm} Ficha de Inscrição: \\Cod: 193\ \ Romelia Santos de Araújo\ \ - \ \ Mestrado 
 }
\date{}

\begin{document}
\maketitle
\vspace*{-1.5cm}
\noindent Data de Nascimento:12/6/1989
\ \ \ Idade: 25   \ \ \ Sexo: Feminino
\\
Naturalidade: Cruz das Almas  
\ \ \  Estado: BA
\ \ \  Nacionalidade: Brasileira
\ \ \ País: Brasil
\\        
Nome do pai : José Reis de Araújo
\ \ \ Nome da mãe: Maria José Santos de Araújo          
\\[0.2cm]                     
\textbf{Endereço Pessoal} 
\\ 
\noindent Endereço residencial: Rua Alexandre Ferreira de Souza
\\
        CEP: 44380-000 
\ \ \ Cidade: Cruz das Almas 
\ \ \ Estado: BA 
\ \ \ País: Brasil
\\		
		Telefone comercial : +55(75)81215222
\ \ \ Telefone residencial: +55(75)91828981
\ \ \ Telefone celular : +55(75)81253674
\\
E-mail principal: romelia.mat@gmail.com
\ \ \ E-mail alternativo: romelia-araujo@hotmail.com 
\\[0.2cm] 
\textbf{Documentos Pessoais}
\\
\noindent Número de CPF : 04047173576
\ \ \ Número de Identidade (ou Passaporte para estrangeiros): 1446109810
\\
Orgão emissor: SSP
\ \ \ Estado: BA
\ \ \ Data de emissão :23/2/2005
\\[0.3cm]
\textbf{Grau acadêmico mais alto obtido}
\\	
Curso:Especialização em Matemática
\ \ \ Grau : especialista
\ \ \ Instituição : Universidade Estadual de Feira de Santana
\\			
Ano de Conclusão ou Previsão: 2014
\\ 
Experiência Profissional mais recente. \ \  
Tem experiência: Docente Discente  
\ \ \ Instituição: Universidade Federal do Recôncavo da Bahia
\\  
Período - início: 1-2012
\ \ \ fim: 1-2014
\\[0.2cm] 
\textbf{Programa Pretendido:} Mestrado\\
Interesse em bolsa: Sim
\\[0.3cm]		
\textbf{Dados dos Recomendantes} 
\\
1- Nome: Claudiano Goulart
\ \ \ \  e-mail: goulart.fsa@gmail.com 
\\
2- Nome: Erikson Alexandre Fonseca dos Santos
\ \ \ \ e-mail: eriksonmat@gmail.com
\\
3- Nome: Kisnney Emiliano de Almeida
\ \ \ \ e-mail: kisnney@gmail.com
\\[0.2cm]
Motivação e expectativa do candidato em relação ao programa pretendido:
\\Minha expectativa em relação ao Mestrado em Matemática da UNB reside em complementar minha formação acadêmica na área da Matemática pura. Além de, desenvolver estudos que qualifique para à pesquisa e para docência superior.\newpage\vspace*{-4cm}\subsection*{Carta de Recomendação - Claudiano Goulart}Código Identificador: 1334\\Conhece-o candidato há quanto tempo (For how long have you known the applicant)? 
\ A aproximadamente 1 ano
\\ Conhece-o sob as seguintes circunstâncias: aulas\ \ orientacao
	\ \ \ \  
\\ Conheçe o candidato sob outras circunstâncias: 
\\	Avaliações:\\
\begin{tabular}{|l|c|c|c|c|c|}
\hline
 & Excelente & Bom & Regular & Insuficiente & Não sabe \\
\hline
Desempenho acadêmico &  & X &  &  & \\
\hline
Capacidade de aprender novos conceitos & X &  &  &  & \\
\hline
Capacidade de trabalhar sozinho & X &  &  &  & \\
\hline
Criatividade &  & X &  &  & \\
\hline
Curiosidade & X &  &  &  & \\
\hline
Esforço, persistência & X &  &  &  & \\
\hline
Expressão escrita &  & X &  &  & \\
\hline
Expressão oral & X &  &  &  & \\
\hline
Relacionamento com colegas & X &  &  &  & \\
\hline
\end{tabular}\\
\\
\textbf{Opinião sobre os antecedentes acadêmicos, profissionais e/ou técnicos do candidato:}
\\A candidata foi minha aluna na disciplina Geometria diferencial durante o curso Especialização em Matemática da Universidade Estadual de Feira de Santana. Atualmente oriento o seu trabalho de conclusão do referido curso. durante este contato ela sempre demonstrou interesse e conhecimento dos conteúdos discutidos.  Tem boa expressão oral e escrita e motivação pra estudo de novos conteúdos.  \\
\\
\textbf{Opinião sobre seu possível aproveitamento, se aceito no Programa:}
\\A candidata demonstra ter grande interesse pela matemática pura em especial  geometria diferencial. Acredito que a especialização em matemática, onde teve o aprofundamento em disciplinas como Análise, Álgebra linear e Geometria diferencial tenha capacitado a candidata para o ingresso e bom desenvolvimento no programa de mestrado desejado.\\ 
\\
\textbf{Outras informações relevantes:} \\
\\[0.3cm]
\textbf{Entre os estudantes que já conheceu, você diria que o candidato está entre os:}
\\
\begin{tabular}{|l|c|c|c|c|c|}
\hline
 & 5\% melhores & 10\% melhores & 25\% melhores & 50\% melhores & Não sabe \\
\hline
Como aluno, em aulas &  & X &  &  & \\
\hline
Como orientando & X &  &  &  & \\
\hline
\end{tabular}
\subsection*{Dados Recomendante} 
	Instituição (Institution): Universidade Estadual de Feira de Santana   UEFS
\\ 
	Grau acadêmico mais alto obtido: doutor
	\ \ Área: Matemática      Geometria Diferencial
	\\
	Ano de obtenção deste grau: 2013
	\ \ 
	Instituição de obtenção deste grau : Universidade de Brasilia    UnB
	\\ 
	Endereço institucional do recomendante: \\ Universidade Estadual de Feira de Santana
Departamento de Ciências Exatas
Avenida Transnordestina s.n. 
Novo Horizonte    
Feira de Santana BA     
CEP 44.036.900
Telefone   75.3161.8086\newpage\vspace*{-4cm}\subsection*{Carta de Recomendação - Erikson Alexandre Fonseca dos Santos}Código Identificador: 1335\\Conhece-o candidato há quanto tempo (For how long have you known the applicant)? 
\ Há 4 anos
\\ Conhece-o sob as seguintes circunstâncias: aulas\ \ orientacao
	\ \ seminarios\ \  
\\ Conheçe o candidato sob outras circunstâncias: 
\\Avaliações: \\
\begin{tabular}{|l|c|c|c|c|c|}
\hline
 & Excelente & Bom & Regular & Insuficiente & Não sabe \\
\hline
Desempenho acadêmico &  & X &  &  & \\
\hline
Capacidade de aprender novos conceitos &  & X &  &  & \\
\hline
Capacidade de trabalhar sozinho & X &  &  &  & \\
\hline
Criatividade &  & X &  &  & \\
\hline
Curiosidade &  & X &  &  & \\
\hline
Esforço, persistência &  & X &  &  & \\
\hline
Expressão escrita &  & X &  &  & \\
\hline
Expressão oral &  &  & X &  & \\
\hline
Relacionamento com colegas &  & X &  &  & \\
\hline
\end{tabular}\\
\\
\textbf{Opinião sobre os antecedentes acadêmicos, profissionais e/ou técnicos do candidato:}
\\Lecionei 3 disciplinas para a estudante e durante os cursos a mesma teve um bom desempenho em todos os cursos. É uma aluna que tem um bom relacionamento com outros estudantes, inclusive os auxiliando sempre que possível. É razoavelmente participativa em aula e dinâmica. Também a orientei no Trabalho de Conclusão de Curso, no qual ele se desenvolveu muito bem.\\
\\
\textbf{Opinião sobre seu possível aproveitamento, se aceito no Programa:}
\\Acredito que a discente possa ter um bom desempenho no referido programa. Ela consegue assimilar muito bem os novos conceitos, tem curiosidade a cerca dos temas abordados, é bastante esforçada e persistente. Cumpre muito bem horários e prazos e é bem disciplinada. Ela possui uma boa escrita matemática e consegue redigir com certa desenvoltura.\\ 
\\
\textbf{Outras informações relevantes:} \\
\\[0.3cm]
\textbf{Entre os estudantes que já conheceu, você diria que o candidato está entre os:}
\\
\begin{tabular}{|l|c|c|c|c|c|}
\hline
 & 5\% melhores & 10\% melhores & 25\% melhores & 50\% melhores & Não sabe \\
\hline
Como aluno, em aulas &  &  & X &  & \\
\hline
Como orientando &  & X &  &  & \\
\hline
\end{tabular}
\subsection*{Dados Recomendante} 
	Instituição (Institution): UNIVERSIDADE FEDERAL DO RECONCAVO DA BAHIA 
\\ 
	Grau acadêmico mais alto obtido: mestre
	\ \ Área: Geometria
	\\
	Ano de obtenção deste grau: 2009
	\ \ 
	Instituição de obtenção deste grau : UNIVERSIDADE FEDERAL DE ALAGOAS
	\\ 
	Endereço institucional do recomendante: \\ Centro de Ciências Exatas e Tecnológicas CETEC
Campus Universitário de Cruz das Almas
Rua Rui Barbosa, n 710  Centro  Cruz das Almas, BA
CEP 44380000\newpage\vspace*{-4cm}\subsection*{Carta de Recomendação - Kisnney Emiliano de Almeida}Código Identificador: 1336\\Conhece-o candidato há quanto tempo (For how long have you known the applicant)? 
\ 1 ano e meio
\\ Conhece-o sob as seguintes circunstâncias: aulas\ \ 
	\ \ \ \  
\\ Conheçe o candidato sob outras circunstâncias: 
\\Avaliações: \\
\begin{tabular}{|l|c|c|c|c|c|}
\hline
 & Excelente & Bom & Regular & Insuficiente & Não sabe \\
\hline
Desempenho acadêmico &  & X &  &  & \\
\hline
Capacidade de aprender novos conceitos &  & X &  &  & \\
\hline
Capacidade de trabalhar sozinho &  & X &  &  & \\
\hline
Criatividade &  & X &  &  & \\
\hline
Curiosidade &  & X &  &  & \\
\hline
Esforço, persistência & X &  &  &  & \\
\hline
Expressão escrita & X &  &  &  & \\
\hline
Expressão oral &  &  &  &  & X\\
\hline
Relacionamento com colegas &  & X &  &  & \\
\hline
\end{tabular}\\
\\
\textbf{Opinião sobre os antecedentes acadêmicos, profissionais e/ou técnicos do candidato:}
\\A aluna fez dois cursos nível especialização comigo, Algebra Linear e Estruturas Algebricas. No curso de Álgebra Linear ela teve um desempenho muito bom. No de Estruturas Algébricas, que consistia em um curso básico de anéis e grupos, ela foi aprovada, mas com dificuldade. No entanto, me parece que boa parte da dificuldade vinha do pouco tempo disponível para estudo, porque ela estava trabalhando e morando em outra cidade. Além disso, álgebra aparentemente não é sua área principal de interesse e ela dizia que minha matéria era a que mais lhe trazia dificuldade. Suas demonstrações eram em geral bem escritas e ela tinha boas ideias. Creio que a aluna possua um potencial além de seu desempenho em meus cursos.\\
\\
\textbf{Opinião sobre seu possível aproveitamento, se aceito no Programa:}
\\Acredito que a aluna possa ter um bom desempenho em um mestrado em matemática. Apesar de ter uma formação com algumas deficiências, ela já teve contato com várias áreas da matemática através da especialização, é uma aluna muito interessada e com uma boa intuição. Creio que, se dedicando exclusivamente ao mestrado, ela possa explorar melhor seu potencial e ter uma carreira promissora como pesquisadora.\\ 
\\
\textbf{Outras informações relevantes:} \\
\\[0.3cm]
\textbf{Entre os estudantes que já conheceu, você diria que o candidato está entre os:}
\\
\begin{tabular}{|l|c|c|c|c|c|}
\hline
 & 5\% melhores & 10\% melhores & 25\% melhores & 50\% melhores & Não sabe \\
\hline
Como aluno, em aulas &  & X &  &  & \\
\hline
Como orientando &  &  &  &  & X\\
\hline
\end{tabular}
\subsection*{Dados Recomendante} 
	Instituição (Institution): Universidade Estadual de Feira de Santana
\\ 
	Grau acadêmico mais alto obtido: doutor
	\ \ Área: Teoria de Grupos
	\\
	Ano de obtenção deste grau: 2012
	\ \ 
	Instituição de obtenção deste grau : Unicamp
	\\ 
	Endereço institucional do recomendante: \\ Av. Transnordestina, SN 
Bairro Novo Horizonte
CEP 44.036.900
Módulo V 
Departamento de Ciências Exatas \includepdf[pages={-},offset=35mm 0mm]{../../../upload/193_2014-05-23_documentos.pdf}\includepdf[pages={-},offset=35mm 0mm]{../../../upload/193_2014-05-23_historico.pdf}\includepdf[pages={-},offset=35mm 0mm]{../../../upload/193_2014-05-21_documentos.pdf}\includepdf[pages={-},offset=35mm 0mm]{../../../upload/193_2014-05-21_historico.pdf}\includepdf[pages={-},offset=35mm 0mm]{../../../upload/193_2012-10-25_documentos.pdf}\includepdf[pages={-},offset=35mm 0mm]{../../../upload/193_2012-10-25_historico.pdf} 
\begin{center}
Anexos.
\end{center}
\end{document}