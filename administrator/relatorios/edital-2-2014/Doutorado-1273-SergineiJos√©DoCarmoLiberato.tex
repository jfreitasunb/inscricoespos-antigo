\documentclass[11pt]{article}
\usepackage{graphicx,color}
\usepackage{pdfpages}
\usepackage[brazil]{babel}
\usepackage[utf8]{inputenc}
\addtolength{\hoffset}{-3cm} \addtolength{\textwidth}{6cm}
\addtolength{\voffset}{-.5cm} \addtolength{\textheight}{1cm}
%%%%%%%%%%%%%%%%%%%%%%%%%%%%%%%%%%%  To use Colors 
\title{\vspace*{-4cm} Ficha de Inscrição: \\Cod: 1273\ \ Serginei José  do Carmo Liberato\ \ - \ \ Doutorado 
 }
\date{}

\begin{document}
\maketitle
\vspace*{-1.5cm}
\noindent Data de Nascimento:11/7/1989
\ \ \ Idade: 25   \ \ \ Sexo: Masculino
\\
Naturalidade: Ponte Nova  
\ \ \  Estado: MG
\ \ \  Nacionalidade: Brasileira
\ \ \ País: Brasil
\\        
Nome do pai : Francisco Liberato
\ \ \ Nome da mãe: Ana Maria do Carmo Liberato          
\\[0.2cm]                     
\textbf{Endereço Pessoal} 
\\ 
\noindent Endereço residencial: Fazenda Fragoso, SN
\\
        CEP: 35444-000 
\ \ \ Cidade: Amparo da Serra 
\ \ \ Estado: MG 
\ \ \ País: Brasil
\\		
		Telefone comercial : +0(31)83368500
\ \ \ Telefone residencial: +0(31)83368500
\ \ \ Telefone celular : +0(31)83368500
\\
E-mail principal: sergineiliberato@gmail.com
\ \ \ E-mail alternativo: serginei.liberato@ufv.br 
\\[0.2cm] 
\textbf{Documentos Pessoais}
\\
\noindent Número de CPF : 09749709640
\ \ \ Número de Identidade (ou Passaporte para estrangeiros): 15502328
\\
Orgão emissor: SSP
\ \ \ Estado: MG
\ \ \ Data de emissão :29/9/2004
\\[0.3cm]
\textbf{Grau acadêmico mais alto obtido}
\\	
Curso:Matemática
\ \ \ Grau : mestre
\ \ \ Instituição : Universidade Federal de Viçosa
\\			
Ano de Conclusão ou Previsão: 2014
\\ 
Experiência Profissional mais recente. \ \  
Tem experiência: Docente 0  
\ \ \ Instituição: Universidade Federal de Viçosa Campus Florstal
\\  
Período - início: 1-2014
\ \ \ fim: nselecionado-nselecionado
\\[0.2cm] 
\textbf{Programa Pretendido:} Doutorado\ \ \ \textbf{Área:} SistemasDinamicos\\
Interesse em bolsa: Sim
\\[0.3cm]		
\textbf{Dados dos Recomendantes} 
\\
1- Nome: Alexandre Miranda Alves
\ \ \ \  e-mail: amalves@ufv.br 
\\
2- Nome: Catarina Mendes de Jesus
\ \ \ \ e-mail: cmendes@ufv.br
\\
3- Nome: Marinês Guerreiro
\ \ \ \ e-mail: marinesguerreiro0208@gmail.com
\\[0.2cm]
Motivação e expectativa do candidato em relação ao programa pretendido:
\\Durante o mestrado trabalhei com Sistemas Dinâmicos, a área me motivou muito pela riqueza das questões estudadas, além disso as relações com a geometria que procurei desenvolver no meu trabalho me deixou profundamente extasiado no estudo da dinâmica. Diante do exposto e  para continuar os estudos na área acredito que a UNB é a melhor opção para que eu possa aprimorar meus conhecimentos, espero que no curso de doutorado eu possa colaborar para o desenvolvimento da boa matemática e para que a mesma tenha sua importância reconhecida.\newpage\vspace*{-4cm}\subsection*{Carta de Recomendação - Alexandre Miranda Alves}Código Identificador: 1274\\Conhece-o candidato há quanto tempo (For how long have you known the applicant)? 
\ 4 anos
\\ Conhece-o sob as seguintes circunstâncias: aulas\ \ orientacao
	\ \ \ \  
\\ Conheçe o candidato sob outras circunstâncias: 
\\	Avaliações:\\
\begin{tabular}{|l|c|c|c|c|c|}
\hline
 & Excelente & Bom & Regular & Insuficiente & Não sabe \\
\hline
Desempenho acadêmico &  & X &  &  & \\
\hline
Capacidade de aprender novos conceitos &  &  & X &  & \\
\hline
Capacidade de trabalhar sozinho &  & X &  &  & \\
\hline
Criatividade &  &  & X &  & \\
\hline
Curiosidade &  & X &  &  & \\
\hline
Esforço, persistência &  & X &  &  & \\
\hline
Expressão escrita &  &  & X &  & \\
\hline
Expressão oral &  &  & X &  & \\
\hline
Relacionamento com colegas & X &  &  &  & \\
\hline
\end{tabular}\\
\\
\textbf{Opinião sobre os antecedentes acadêmicos, profissionais e/ou técnicos do candidato:}
\\Acredito que ele ainda tenha algumas lacunas a serem preenchidas, devidas a sua formação, mas evoluiu bem durante o período do mestrado.\\
\\
\textbf{Opinião sobre seu possível aproveitamento, se aceito no Programa:}
\\Pelo que vem demonstrando no mestrado, acredito que ele terá condições de desenvolver uma tese em Matemática. De fato, ele apresenta iniciativa procurando artigos complementares ao seu trabalho, certa autonomia precisa de orientação, mas consegue desenvolver vários tópicos por conta própria e sempre está atento a críticas recebendo bem as mesmas.\\ 
\\
\textbf{Outras informações relevantes:} \\Desde que o conheci, tem evoluído constantemente. Tornouse mais independente, apresenta interesse ao participar de seminários e palestras e tem desenvolvido razoavelmente um espírito crítico.  Observo que o mesmo tinha muitas lacunas em sua formação básica, mas vem evoluindo constantemente.
O candidato é responsável, dedicado e apresenta entusiasmo. Tem iniciativa em buscar textos alternativos nas disciplinas cursadas  e apresenta certa autonomia em seu trabalho de mestardo.
\\[0.3cm]
\textbf{Entre os estudantes que já conheceu, você diria que o candidato está entre os:}
\\
\begin{tabular}{|l|c|c|c|c|c|}
\hline
 & 5\% melhores & 10\% melhores & 25\% melhores & 50\% melhores & Não sabe \\
\hline
Como aluno, em aulas &  &  & X &  & \\
\hline
Como orientando &  &  & X &  & \\
\hline
\end{tabular}
\subsection*{Dados Recomendante} 
	Instituição (Institution): Universidade Federal de Viçosa
\\ 
	Grau acadêmico mais alto obtido: doutor
	\ \ Área: Sistemas Dinâmicos
	\\
	Ano de obtenção deste grau: 2009
	\ \ 
	Instituição de obtenção deste grau : UFMG
	\\ 
	Endereço institucional do recomendante: \\ Departamento de Matemática UFV 
Avenida Peter Henry Rolfs, sn
Campus Universitário. Viçosa  MG
CEP 36570900
Tel. 31 38992393



\newpage\vspace*{-4cm}\subsection*{Carta de Recomendação - Catarina Mendes de Jesus}Código Identificador: 238\\Conhece-o candidato há quanto tempo (For how long have you known the applicant)? 
\ 4 anos
\\ Conhece-o sob as seguintes circunstâncias: aulas\ \ orientacao
	\ \ \ \  
\\ Conheçe o candidato sob outras circunstâncias: 
\\Avaliações: \\
\begin{tabular}{|l|c|c|c|c|c|}
\hline
 & Excelente & Bom & Regular & Insuficiente & Não sabe \\
\hline
Desempenho acadêmico &  & X &  &  & \\
\hline
Capacidade de aprender novos conceitos &  & X &  &  & \\
\hline
Capacidade de trabalhar sozinho &  & X &  &  & \\
\hline
Criatividade &  & X &  &  & \\
\hline
Curiosidade &  & X &  &  & \\
\hline
Esforço, persistência &  & X &  &  & \\
\hline
Expressão escrita &  & X &  &  & \\
\hline
Expressão oral &  & X &  &  & \\
\hline
Relacionamento com colegas & X &  &  &  & \\
\hline
\end{tabular}\\
\\
\textbf{Opinião sobre os antecedentes acadêmicos, profissionais e/ou técnicos do candidato:}
\\O gosta muito de matemática e é muito esforçado, teve dificuldades em umas disciplinas e fez iniciação científica em introdução a singularidades sob minha orientação, não teve bom rendimento por não identificar muito com o tema.  No mestrado não posso dizer, mas parece que saiu bem, e estava afastada e somente pude ver que fez uma ótima apresentação na defesa.\\
\\
\textbf{Opinião sobre seu possível aproveitamento, se aceito no Programa:}
\\Imagino que o candidato pode encontrar algumas dificuldades, por não ter uma base muito sólida, mas com o esforço de sua natureza e com boa orientação acredito que ele possa superar. \\ 
\\
\textbf{Outras informações relevantes:} \\
\\[0.3cm]
\textbf{Entre os estudantes que já conheceu, você diria que o candidato está entre os:}
\\
\begin{tabular}{|l|c|c|c|c|c|}
\hline
 & 5\% melhores & 10\% melhores & 25\% melhores & 50\% melhores & Não sabe \\
\hline
Como aluno, em aulas &  &  & X &  & \\
\hline
Como orientando &  &  &  & X & \\
\hline
\end{tabular}
\subsection*{Dados Recomendante} 
	Instituição (Institution): UFV
\\ 
	Grau acadêmico mais alto obtido: doutor
	\ \ Área: Geometria e Topologia
	\\
	Ano de obtenção deste grau: 2001
	\ \ 
	Instituição de obtenção deste grau : PUC-Rio
	\\ 
	Endereço institucional do recomendante: \\ Departamento de Matemática, Campus UFV, Viçosa - MG
CEP 36570-000\newpage\vspace*{-4cm}\subsection*{Carta de Recomendação - Marinês Guerreiro}Código Identificador: 1329\\Conhece-o candidato há quanto tempo (For how long have you known the applicant)? 
\ 
\\ Conhece-o sob as seguintes circunstâncias: \ \ 
	\ \ \ \  
\\ Conheçe o candidato sob outras circunstâncias: 
\\Avaliações: \\
\begin{tabular}{|l|c|c|c|c|c|}
\hline
 & Excelente & Bom & Regular & Insuficiente & Não sabe \\
\hline
Desempenho acadêmico &  &  &  &  & \\
\hline
Capacidade de aprender novos conceitos &  &  &  &  & \\
\hline
Capacidade de trabalhar sozinho &  &  &  &  & \\
\hline
Criatividade &  &  &  &  & \\
\hline
Curiosidade &  &  &  &  & \\
\hline
Esforço, persistência &  &  &  &  & \\
\hline
Expressão escrita &  &  &  &  & \\
\hline
Expressão oral &  &  &  &  & \\
\hline
Relacionamento com colegas &  &  &  &  & \\
\hline
\end{tabular}\\
\\
\textbf{Opinião sobre os antecedentes acadêmicos, profissionais e/ou técnicos do candidato:}
\\\\
\\
\textbf{Opinião sobre seu possível aproveitamento, se aceito no Programa:}
\\\\ 
\\
\textbf{Outras informações relevantes:} \\
\\[0.3cm]
\textbf{Entre os estudantes que já conheceu, você diria que o candidato está entre os:}
\\
\begin{tabular}{|l|c|c|c|c|c|}
\hline
 & 5\% melhores & 10\% melhores & 25\% melhores & 50\% melhores & Não sabe \\
\hline
Como aluno, em aulas &  &  &  &  & \\
\hline
Como orientando &  &  &  &  & \\
\hline
\end{tabular}
\subsection*{Dados Recomendante} 
	Instituição (Institution): UNIVERSIDADE FEDERAL DE VIÇOSA
\\ 
	Grau acadêmico mais alto obtido: doutor
	\ \ Área: MATEMÁTICA
	\\
	Ano de obtenção deste grau: 1997
	\ \ 
	Instituição de obtenção deste grau : UNIVERSITY OF MANCHESTER
	\\ 
	Endereço institucional do recomendante: \\ UFV  Departamento de Matemática
36570900  Viçosa  MG\includepdf[pages={-},offset=35mm 0mm]{../../../upload/1273_2014-05-13_documentos.pdf}\includepdf[pages={-},offset=35mm 0mm]{../../../upload/1273_2014-05-13_historico.pdf} 
\begin{center}
Anexos.
\end{center}
\end{document}