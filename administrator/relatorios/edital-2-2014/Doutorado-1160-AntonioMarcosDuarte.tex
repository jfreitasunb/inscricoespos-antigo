\documentclass[11pt]{article}
\usepackage{graphicx,color}
\usepackage{pdfpages}
\usepackage[brazil]{babel}
\usepackage[utf8]{inputenc}
\addtolength{\hoffset}{-3cm} \addtolength{\textwidth}{6cm}
\addtolength{\voffset}{-.5cm} \addtolength{\textheight}{1cm}
%%%%%%%%%%%%%%%%%%%%%%%%%%%%%%%%%%%  To use Colors 
\title{\vspace*{-4cm} Ficha de Inscrição: \\Cod: 1160\ \ Antonio Marcos Duarte\ \ - \ \ Doutorado 
 }
\date{}

\begin{document}
\maketitle
\vspace*{-1.5cm}
\noindent Data de Nascimento:13/4/1990
\ \ \ Idade: 24   \ \ \ Sexo: Masculino
\\
Naturalidade: Palmeira dos Indios  
\ \ \  Estado: AL
\ \ \  Nacionalidade: Brasileiro
\ \ \ País: Brasil
\\        
Nome do pai : Antonio Duarte Lima
\ \ \ Nome da mãe: Maria do Socorro Peixoto de Franca          
\\[0.2cm]                     
\textbf{Endereço Pessoal} 
\\ 
\noindent Endereço residencial: Av Alagoas 200 bairro Palmeira de Fora
\\
        CEP: 57608-180 
\ \ \ Cidade: Palmeira dos Indios 
\ \ \ Estado: AL 
\ \ \ País: Brasil
\\		
		Telefone comercial : +0(82)96048735
\ \ \ Telefone residencial: +0(83)81116192
\ \ \ Telefone celular : +55(83)96615287
\\
E-mail principal: amdf@dme.ufcg.edu.br
\ \ \ E-mail alternativo: antonio\_marcos200@hotmail.com 
\\[0.2cm] 
\textbf{Documentos Pessoais}
\\
\noindent Número de CPF : 08564776480
\ \ \ Número de Identidade (ou Passaporte para estrangeiros): 3197872-0
\\
Orgão emissor: SCJDS
\ \ \ Estado: AL
\ \ \ Data de emissão :11/5/2006
\\[0.3cm]
\textbf{Grau acadêmico mais alto obtido}
\\	
Curso:Matemática
\ \ \ Grau : mestre
\ \ \ Instituição : UNIVERSIDADE FEDERAL DE CAMPINA GRANDE
\\			
Ano de Conclusão ou Previsão: 2014
\\ 
Experiência Profissional mais recente. \ \  
Tem experiência: Docente Discente  
\ \ \ Instituição: UNIVERSIDADE FEDERAL DE CAMPINA GRANDE
\\  
Período - início: 2-2012
\ \ \ fim: 0-2014
\\[0.2cm] 
\textbf{Programa Pretendido:} Doutorado\ \ \ \textbf{Área:} Algebra\\
Interesse em bolsa: Sim
\\[0.3cm]		
\textbf{Dados dos Recomendantes} 
\\
1- Nome: Antonio Pereira Brandão Júnior
\ \ \ \  e-mail: brandao@dme.ufcg.edu.br 
\\
2- Nome: Diogo Diniz Pereira da Silva e Silva
\ \ \ \ e-mail: diogo@dme.ufcg.edu.br
\\
3- Nome: Jefferson Abrantes dos Santos
\ \ \ \ e-mail: jefferson@dme.ufcg.edu.br
\\[0.2cm]
Motivação e expectativa do candidato em relação ao programa pretendido:
\\Tendo em vista que o PPGMATUnB constitui um dos maiores centros de excelência em pesquisa Matemática, em nível nacional e internacional, assim como um dos mais tradicionais e conceituados centros de pesquisa do pais, é de se esperar que desperte o interesse de diversos estudantes e pesquisadores desta área, bem como de áreas afins. Comigo não foi diferente.  

É bem conhecido que o PPGMATUnB possui um admirável corpo docente que contribui cientificamente para as diversas áreas da matemática, sobretudo para a álgebra, que é a minha área de interesse no programa. Interesse esse motivado pela minha afinidade com a área e por poder prosseguir na mesma linha de pesquisa a qual estou inserido no momento, como mestrando, PITeoria. Sendo assim, o PPGMATUnB fornece um ambiente favorável ao desenvolvimento de minha formação como pesquisador, assim como, a possibilidade de continuar com minhas pesquisas. 

Em relação a minha formação acadêmica, ingressei em 2008 no curso de Graduação em Licenciatura Plena em Matemática da Universidade Estadual de Alagoas, UNEAL. Durante o período da graduação, desenvolvi diversas atividades de extensão, tais como monitor de matemática do curso prévestibular oferecido pela própria instituição. Além disso, participei como aluno bolsista de Projeto de Iniciação Cientifica, PIBIC, cujo titulo foi A Solução de Schwarzschild Revisitada, financiado pela Fundação de Amparo à Pesquisa do Estado de Alagoas, FAPEAL. Conclui a graduação em 2012.

Atualmente, estou concluindo o curso de Mestrado Acadêmico do Programa de PósGraduação em Matemática da Universidade Federal de Campina Grande, UFCG, com bolsa CAPES, trabalhando com ênfase na área de Álgebra, sendo orientado pelo Prof. Dr. Antônio Pereira Brandão Junior, o qual me direcionou na realização de diversas disciplinas na área em questão, bem como para minha formação como matemático. Durante o curso de mestrado, mais precisamente durante as férias, venho participando de cursos extracurriculares que tem ajudado a complementar minha formação, como é o caso do curso Equações Algébricas ofertado durante o 29 Coloquio do IMPA, e do curso  Variedades Diferenciáveis ofertado durante a XLIII Escola de Verão do MATUnB.

Ademais, minha área de estudo está concentrada no estudo das Identidades Polinomiais, mais especificamente, minha dissertação tem como tema Crescimento Polinomial das Codimensões e Uma Caracterização de Álgebras com Crescimento Polinomial das Codimensões. Sendo assim, trabalhar em um departamento onde essa linha de pesquisa PITeoria vem sendo desenvolvida  e onde poderei encontrar pesquisadores de renome internacional, seria uma oportunidade única. Além disso, a possibilidade de intercâmbio com grandes centros internacionais em pesquisa matemática, para fins de pesquisa, agradame muito.

Acredito ter uma boa base matemática para ingressar no doutorado do PPGMATUnB, prosseguindo ou não na minha linha pesquisa atual, almejando ter a melhor formação possível, de maneira a vir contribuir para o desenvolvimento da matemática na instituição de ensino a qual futuramente irei trabalhar, pesquisando e lecionando. Sou capaz de trabalhar em grupo ou sozinho, de maneira organizada. Nas atividades onde tenho ocupado o meu tempo, dei provas de flexibilidade e de iniciativa, procurando dar o melhor de mim com empenho e dedicação. Portanto, venho por este meio apresentar a minha inscrição.

Att. Antonio Marcos Duarte de França\newpage\vspace*{-4cm}\subsection*{Carta de Recomendação - Antonio Pereira Brandão Júnior}Código Identificador: 330\\Conhece-o candidato há quanto tempo (For how long have you known the applicant)? 
\ Há 3 anos
\\ Conhece-o sob as seguintes circunstâncias: aulas\ \ orientacao
	\ \ \ \  
\\ Conheçe o candidato sob outras circunstâncias: 
\\	Avaliações:\\
\begin{tabular}{|l|c|c|c|c|c|}
\hline
 & Excelente & Bom & Regular & Insuficiente & Não sabe \\
\hline
Desempenho acadêmico & X &  &  &  & \\
\hline
Capacidade de aprender novos conceitos & X &  &  &  & \\
\hline
Capacidade de trabalhar sozinho & X &  &  &  & \\
\hline
Criatividade &  & X &  &  & \\
\hline
Curiosidade & X &  &  &  & \\
\hline
Esforço, persistência & X &  &  &  & \\
\hline
Expressão escrita &  & X &  &  & \\
\hline
Expressão oral &  & X &  &  & \\
\hline
Relacionamento com colegas & X &  &  &  & \\
\hline
\end{tabular}\\
\\
\textbf{Opinião sobre os antecedentes acadêmicos, profissionais e/ou técnicos do candidato:}
\\O candidato Antônio Marcos cursou a graduação no interior do Estado de Alagoas. Em agosto de 2012, ingressou no programa de mestrado em matemática da UFCG, sob minha orientação. Apesar de ter vindo da graduação com algumas deficiências, o candidato teve um excelente desempenho no mestrado, demonstrando grande disposição para os estudos e capacidade intelectual. Atualmente, está escrevendo sua dissertação de mestrado, com previsão para defesa dentro do prazo estabelecido. Considero o Antonio Marcos um candidato com bons antecedentes acadêmicos.\\
\\
\textbf{Opinião sobre seu possível aproveitamento, se aceito no Programa:}
\\O candidato tem bom potencial para estudar matemática avançada e para pesquisa. Caso seja aceito no programa de doutorado em matemática da UNB, acredito que o candidato terá um desempenho muito bom e que concluirá o curso dentro do prazo estabelecido.\\ 
\\
\textbf{Outras informações relevantes:} \\
\\[0.3cm]
\textbf{Entre os estudantes que já conheceu, você diria que o candidato está entre os:}
\\
\begin{tabular}{|l|c|c|c|c|c|}
\hline
 & 5\% melhores & 10\% melhores & 25\% melhores & 50\% melhores & Não sabe \\
\hline
Como aluno, em aulas &  & X &  &  & \\
\hline
Como orientando &  &  & X &  & \\
\hline
\end{tabular}
\subsection*{Dados Recomendante} 
	Instituição (Institution): Universidade Federal de Campina Grande  -  UFCG
\\ 
	Grau acadêmico mais alto obtido: doutor
	\ \ Área: Álgebra
	\\
	Ano de obtenção deste grau: 2006
	\ \ 
	Instituição de obtenção deste grau : UNICAMP
	\\ 
	Endereço institucional do recomendante: \\ Unidade Acadêmica de Matemática  CCT  UFCG
Av. Aprígio Veloso, 882  Bodocongó
Campina Grande  PB
CEP 58.429970      Caixa Postal 10044\newpage\vspace*{-4cm}\subsection*{Carta de Recomendação - Diogo Diniz Pereira da Silva e Silva}Código Identificador: 782\\Conhece-o candidato há quanto tempo (For how long have you known the applicant)? 
\ 1 ano e meio
\\ Conhece-o sob as seguintes circunstâncias: aulas\ \ 
	\ \ \ \  
\\ Conheçe o candidato sob outras circunstâncias: 
\\Avaliações: \\
\begin{tabular}{|l|c|c|c|c|c|}
\hline
 & Excelente & Bom & Regular & Insuficiente & Não sabe \\
\hline
Desempenho acadêmico &  & X &  &  & \\
\hline
Capacidade de aprender novos conceitos &  & X &  &  & \\
\hline
Capacidade de trabalhar sozinho &  & X &  &  & \\
\hline
Criatividade &  & X &  &  & \\
\hline
Curiosidade &  & X &  &  & \\
\hline
Esforço, persistência & X &  &  &  & \\
\hline
Expressão escrita &  & X &  &  & \\
\hline
Expressão oral &  & X &  &  & \\
\hline
Relacionamento com colegas & X &  &  &  & \\
\hline
\end{tabular}\\
\\
\textbf{Opinião sobre os antecedentes acadêmicos, profissionais e/ou técnicos do candidato:}
\\O candidato demonstrou boa capacidade de aprendizado nas disciplinas cursadas comigo.\\
\\
\textbf{Opinião sobre seu possível aproveitamento, se aceito no Programa:}
\\O candidato tem plenas condições técnicas e demonstra grande interesse em fazer doutorado acredito que concluirá seu doutorado com bom desempenho.\\ 
\\
\textbf{Outras informações relevantes:} \\
\\[0.3cm]
\textbf{Entre os estudantes que já conheceu, você diria que o candidato está entre os:}
\\
\begin{tabular}{|l|c|c|c|c|c|}
\hline
 & 5\% melhores & 10\% melhores & 25\% melhores & 50\% melhores & Não sabe \\
\hline
Como aluno, em aulas &  & X &  &  & \\
\hline
Como orientando &  &  &  &  & X\\
\hline
\end{tabular}
\subsection*{Dados Recomendante} 
	Instituição (Institution): Universidade Federal de Campina Grande
\\ 
	Grau acadêmico mais alto obtido: doutor
	\ \ Área: Álgebra
	\\
	Ano de obtenção deste grau: 2010
	\ \ 
	Instituição de obtenção deste grau : Unicamp
	\\ 
	Endereço institucional do recomendante: \\ Rua Aprígio Veloso, 887 - Bairro Universitário. Campina Grande  PB.\newpage\vspace*{-4cm}\subsection*{Carta de Recomendação - Jefferson Abrantes dos Santos}Código Identificador: 1309\\Conhece-o candidato há quanto tempo (For how long have you known the applicant)? 
\ 2 anos
\\ Conhece-o sob as seguintes circunstâncias: aulas\ \ 
	\ \ \ \  
\\ Conheçe o candidato sob outras circunstâncias: 
\\Avaliações: \\
\begin{tabular}{|l|c|c|c|c|c|}
\hline
 & Excelente & Bom & Regular & Insuficiente & Não sabe \\
\hline
Desempenho acadêmico &  & X &  &  & \\
\hline
Capacidade de aprender novos conceitos &  & X &  &  & \\
\hline
Capacidade de trabalhar sozinho & X &  &  &  & \\
\hline
Criatividade &  & X &  &  & \\
\hline
Curiosidade & X &  &  &  & \\
\hline
Esforço, persistência & X &  &  &  & \\
\hline
Expressão escrita &  & X &  &  & \\
\hline
Expressão oral &  & X &  &  & \\
\hline
Relacionamento com colegas & X &  &  &  & \\
\hline
\end{tabular}\\
\\
\textbf{Opinião sobre os antecedentes acadêmicos, profissionais e/ou técnicos do candidato:}
\\O aluno se mostra muito motivado com o curso. O mesmo está sempre presente no departamento interagindo com os colegas. \\
\\
\textbf{Opinião sobre seu possível aproveitamento, se aceito no Programa:}
\\Acredito plenamente, que o candidato tem condição de realizar um ótimo doutorado.  \\ 
\\
\textbf{Outras informações relevantes:} \\Fui professor do candidato na disciplina Análise Real do PPGMat, e percebi que o mesmo possui facilidade em compreender novos conteúdos. 
\\[0.3cm]
\textbf{Entre os estudantes que já conheceu, você diria que o candidato está entre os:}
\\
\begin{tabular}{|l|c|c|c|c|c|}
\hline
 & 5\% melhores & 10\% melhores & 25\% melhores & 50\% melhores & Não sabe \\
\hline
Como aluno, em aulas &  & X &  &  & \\
\hline
Como orientando &  &  &  &  & X\\
\hline
\end{tabular}
\subsection*{Dados Recomendante} 
	Instituição (Institution): Universidade Federal de Campina Grande
\\ 
	Grau acadêmico mais alto obtido: doutor
	\ \ Área: Análise
	\\
	Ano de obtenção deste grau: 2011
	\ \ 
	Instituição de obtenção deste grau : UNB
	\\ 
	Endereço institucional do recomendante: \\ Av. Aprígio Veloso, 785, Bodocongó, Caixa Postal 10.044,
Cep 58429970  Campina Grande  PB  Brasil
FoneFax 55832101.1030 , email chefiadme.ufcg.edu.br \includepdf[pages={-},offset=35mm 0mm]{../../../upload/1160_2014-05-19_documentos.pdf}\includepdf[pages={-},offset=35mm 0mm]{../../../upload/1160_2014-05-19_historico.pdf}\includepdf[pages={-},offset=35mm 0mm]{../../../upload/1160_2013-12-13_documentos.pdf}\includepdf[pages={-},offset=35mm 0mm]{../../../upload/1160_2013-12-13_historico.pdf} 
\begin{center}
Anexos.
\end{center}
\end{document}