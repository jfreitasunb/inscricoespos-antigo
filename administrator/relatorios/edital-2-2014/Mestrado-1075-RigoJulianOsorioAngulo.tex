\documentclass[11pt]{article}
\usepackage{graphicx,color}
\usepackage{pdfpages}
\usepackage[brazil]{babel}
\usepackage[utf8]{inputenc}
\addtolength{\hoffset}{-3cm} \addtolength{\textwidth}{6cm}
\addtolength{\voffset}{-.5cm} \addtolength{\textheight}{1cm}
%%%%%%%%%%%%%%%%%%%%%%%%%%%%%%%%%%%  To use Colors 
\title{\vspace*{-4cm} Ficha de Inscrição: \\Cod: 1075\ \ Rigo Julian  Osorio Angulo\ \ - \ \ Mestrado 
 }
\date{}

\begin{document}
\maketitle
\vspace*{-1.5cm}
\noindent Data de Nascimento:8/2/1985
\ \ \ Idade: 29   \ \ \ Sexo: Masculino
\\
Naturalidade: Rosas  
\ \ \  Estado: Outro
\ \ \  Nacionalidade: Colombiana
\ \ \ País: Colombia
\\        
Nome do pai : Rigo Osorio
\ \ \ Nome da mãe: Otilia Angulo          
\\[0.2cm]                     
\textbf{Endereço Pessoal} 
\\ 
\noindent Endereço residencial: Carrera 4 Calle 9 Número 30
\\
        CEP: 71608-000 
\ \ \ Cidade: Popayán 
\ \ \ Estado: Outro 
\ \ \ País: Colombia
\\		
		Telefone comercial : +0(57)392125
\ \ \ Telefone residencial: +0(57)392125
\ \ \ Telefone celular : +0(57)314701693
\\
E-mail principal: julian.math666@gmail.com
\ \ \ E-mail alternativo: osorio08-85@hotmail.com 
\\[0.2cm] 
\textbf{Documentos Pessoais}
\\
\noindent Número de CPF : 0
\ \ \ Número de Identidade (ou Passaporte para estrangeiros): 76150539
\\
Orgão emissor: Colombia
\ \ \ Estado: Outro
\ \ \ Data de emissão :12/3/2003
\\[0.3cm]
\textbf{Grau acadêmico mais alto obtido}
\\	
Curso:Matemáticas
\ \ \ Grau : outro
\ \ \ Instituição : Universidad del Cauca
\\			
Ano de Conclusão ou Previsão: 2013
\\ 
Experiência Profissional mais recente. \ \  
Tem experiência: Docente Discente  
\ \ \ Instituição: Institución Educativa Promoción Social Guanacas
\\  
Período - início: 1-2014
\ \ \ fim: 1-2014
\\[0.2cm] 
\textbf{Programa Pretendido:} Mestrado\\
Interesse em bolsa: Sim
\\[0.3cm]		
\textbf{Dados dos Recomendantes} 
\\
1- Nome: Carlos Alberto Trujillo Solarte
\ \ \ \  e-mail: trujillo@unicauca.edu.co 
\\
2- Nome: Willy Sierra Arrollo
\ \ \ \ e-mail: wsierra@unicauca.edu.co
\\
3- Nome: Maribel Diaz Noguera
\ \ \ \ e-mail: mddiaz@unicauca.edu.co
\\[0.2cm]
Motivação e expectativa do candidato em relação ao programa pretendido:
\\En la culminación de nuestros estudios superiores debemos plantearnos diferentes horizontes y analizar las oportunidades que tenemos, enfocarnos y desarrollarnos tanto personal como profesionalmente. La curiosidad y el deseo de aprender son unas de las principales cualidades del ser humano y que muy meticulosamente me han inculcado mis maestros, busco como matemático poder obtener en un futuro un título de Maestría, y porque no, si es posible posteriormente un título de Doctorado, desarrollándome como un buen profesional para así brindar un apoyo y conocimiento de calidad a los futuros miembros de esta comunidad en mi pais. Para llegar a tal punto debo seguir formando y aprendiendo de todas las experiencias que pueda, por estas razones deseo cursar la maestría que la UnB ofrece en dicho programa y dado que además cuento con excelentes referencias a cerca de la calidad académica de esta importante universidad. 
Gracias al comité de Posgrados de la Universidad de Brasilia, espero tener una respuesta satisfactoria y así poder demostrar mi potencial, el deseo y la disposición para seguir mi carrera.\newpage\vspace*{-4cm}\subsection*{Carta de Recomendação - Carlos Alberto Trujillo Solarte}Código Identificador: 1296\\Conhece-o candidato há quanto tempo (For how long have you known the applicant)? 
\ 36 meses
\\ Conhece-o sob as seguintes circunstâncias: aulas\ \ orientacao
	\ \ seminarios\ \  
\\ Conheçe o candidato sob outras circunstâncias: 
\\	Avaliações:\\
\begin{tabular}{|l|c|c|c|c|c|}
\hline
 & Excelente & Bom & Regular & Insuficiente & Não sabe \\
\hline
Desempenho acadêmico &  & X &  &  & \\
\hline
Capacidade de aprender novos conceitos &  & X &  &  & \\
\hline
Capacidade de trabalhar sozinho & X &  &  &  & \\
\hline
Criatividade &  & X &  &  & \\
\hline
Curiosidade &  & X &  &  & \\
\hline
Esforço, persistência & X &  &  &  & \\
\hline
Expressão escrita &  & X &  &  & \\
\hline
Expressão oral &  & X &  &  & \\
\hline
Relacionamento com colegas & X &  &  &  & \\
\hline
\end{tabular}\\
\\
\textbf{Opinião sobre os antecedentes acadêmicos, profissionais e/ou técnicos do candidato:}
\\Julian trabaja bien en actividades individuales y en equipo. Es disciplinado y tiene mucho interes en continuar estudiando. Es un buen candidato para estudios de maestria en matematica, area algebra y teoria de numeros.\\
\\
\textbf{Opinião sobre seu possível aproveitamento, se aceito no Programa:}
\\Su interes y espiritu de superacion lo hacen merecedor de una admision al programa. Seguro tendra exitos en sus estudios.\\ 
\\
\textbf{Outras informações relevantes:} \\No dudo en recomendarlo. Trabajo conmigo en seminarios y pude identificar su potencial,
\\[0.3cm]
\textbf{Entre os estudantes que já conheceu, você diria que o candidato está entre os:}
\\
\begin{tabular}{|l|c|c|c|c|c|}
\hline
 & 5\% melhores & 10\% melhores & 25\% melhores & 50\% melhores & Não sabe \\
\hline
Como aluno, em aulas &  & X &  &  & \\
\hline
Como orientando & X &  &  &  & \\
\hline
\end{tabular}
\subsection*{Dados Recomendante} 
	Instituição (Institution): UNIVERSIDAD DEL CAUCA
\\ 
	Grau acadêmico mais alto obtido: doutor
	\ \ Área: MATEMATICAS
	\\
	Ano de obtenção deste grau: 1998
	\ \ 
	Instituição de obtenção deste grau : UNIVERSIDAD POLITECNICA DE MADRID
	\\ 
	Endereço institucional do recomendante: \\ DEPARTAMENTO DE MATEMATICAS, UNIVERSIDAD DEL CAUCA, POPAYAN, COLOMBIA, CALLE 5 No. 4 70\newpage\vspace*{-4cm}\subsection*{Carta de Recomendação - Willy Sierra Arrollo}Código Identificador: 1411\\Conhece-o candidato há quanto tempo (For how long have you known the applicant)? 
\ 24 meses
\\ Conhece-o sob as seguintes circunstâncias: aulas\ \ 
	\ \ \ \  
\\ Conheçe o candidato sob outras circunstâncias: 
\\Avaliações: \\
\begin{tabular}{|l|c|c|c|c|c|}
\hline
 & Excelente & Bom & Regular & Insuficiente & Não sabe \\
\hline
Desempenho acadêmico &  & X &  &  & \\
\hline
Capacidade de aprender novos conceitos &  & X &  &  & \\
\hline
Capacidade de trabalhar sozinho &  & X &  &  & \\
\hline
Criatividade &  & X &  &  & \\
\hline
Curiosidade & X &  &  &  & \\
\hline
Esforço, persistência &  & X &  &  & \\
\hline
Expressão escrita &  & X &  &  & \\
\hline
Expressão oral &  & X &  &  & \\
\hline
Relacionamento com colegas &  &  &  &  & X\\
\hline
\end{tabular}\\
\\
\textbf{Opinião sobre os antecedentes acadêmicos, profissionais e/ou técnicos do candidato:}
\\He orientado a Julian dos cursos, uno de Análisis Complejo y otro de Análisis III, el cual es un curso introductorio de teoría de la medida, en ambos cursos Julian obtuvo una buena nota. Por su rendimiento en los cursos que le orienté, considero que Julian cuenta con buena formación en el área de Análisis. De otro lado, entiendo que Julian ha realizado su trabajo de grado en Teoría de Números, esto seguro contribuyó a que Julian adquiriera cierto grado de independencia y mejorara su capacidad de escribir, esto fue la única debilidad que noté en él en los cursos que le orienté. Su trabajo de grado además lo obligó a profundizar en temas de Álgebra y Teoría de Números que no se tratan en nuestros cursos habituales, seguro fortaleció esta área.

No tengo conocimiento de los antecedentes de Julian en otras áreas fundamentales de nuestro pregrado como geometría y topología, pero seguro tomó cursos de ambas asignaturas.\\
\\
\textbf{Opinião sobre seu possível aproveitamento, se aceito no Programa:}
\\Por lo expresado anteriormente, considero que Julian cuenta con una sólida formación en áreas básicas como Análisis y Álgebra, además de tomar cursos en áreas importantes como Geometría y Topología. Esto sin duda le permitirá afrontar sin dificultades sus estudios de postgrado, en caso de ser aceptado.\\ 
\\
\textbf{Outras informações relevantes:} \\
\\[0.3cm]
\textbf{Entre os estudantes que já conheceu, você diria que o candidato está entre os:}
\\
\begin{tabular}{|l|c|c|c|c|c|}
\hline
 & 5\% melhores & 10\% melhores & 25\% melhores & 50\% melhores & Não sabe \\
\hline
Como aluno, em aulas &  & X &  &  & \\
\hline
Como orientando &  &  &  &  & X\\
\hline
\end{tabular}
\subsection*{Dados Recomendante} 
	Instituição (Institution): Universidad del Cauca
\\ 
	Grau acadêmico mais alto obtido: doutor
	\ \ Área: Matemáticas, Teoría Geométrica de Funciones.
	\\
	Ano de obtenção deste grau: 2010
	\ \ 
	Instituição de obtenção deste grau : Pontificia Universidad Católica de Chile
	\\ 
	Endereço institucional do recomendante: \\ Calle 5 número 4 70, Departamento de Matemáticas, Universidad del Cauca, Popayán, Colombia\newpage\vspace*{-4cm}\subsection*{Carta de Recomendação - Maribel Diaz Noguera}Código Identificador: 1412\\Conhece-o candidato há quanto tempo (For how long have you known the applicant)? 
\ 6 anos
\\ Conhece-o sob as seguintes circunstâncias: aulas\ \ 
	\ \ \ \  
\\ Conheçe o candidato sob outras circunstâncias: 
\\Avaliações: \\
\begin{tabular}{|l|c|c|c|c|c|}
\hline
 & Excelente & Bom & Regular & Insuficiente & Não sabe \\
\hline
Desempenho acadêmico &  & X &  &  & \\
\hline
Capacidade de aprender novos conceitos &  & X &  &  & \\
\hline
Capacidade de trabalhar sozinho &  & X &  &  & \\
\hline
Criatividade &  & X &  &  & \\
\hline
Curiosidade & X &  &  &  & \\
\hline
Esforço, persistência & X &  &  &  & \\
\hline
Expressão escrita &  & X &  &  & \\
\hline
Expressão oral & X &  &  &  & \\
\hline
Relacionamento com colegas & X &  &  &  & \\
\hline
\end{tabular}\\
\\
\textbf{Opinião sobre os antecedentes acadêmicos, profissionais e/ou técnicos do candidato:}
\\Durante el tiempo que Julian fué mi alumno, se destaco por su responsabilidad, dedicación y cumplimiento, eso le permitió lograr un buen desempeno en las asignaturas que cursó conmigo.\\
\\
\textbf{Opinião sobre seu possível aproveitamento, se aceito no Programa:}
\\Considero que Julian tendrá un buen desempeno en el programa de Maestría, pues tiene capacidades académicas y considero que es una persona supremamente responsable.\\ 
\\
\textbf{Outras informações relevantes:} \\Con relación a lo personal, se destaca por ser una persona respetuosa de las personas y entidades con las que tiene relación, considero que tiene buena capacidad de relacionarse en cualquier ambiente.
\\[0.3cm]
\textbf{Entre os estudantes que já conheceu, você diria que o candidato está entre os:}
\\
\begin{tabular}{|l|c|c|c|c|c|}
\hline
 & 5\% melhores & 10\% melhores & 25\% melhores & 50\% melhores & Não sabe \\
\hline
Como aluno, em aulas &  &  &  & X & \\
\hline
Como orientando &  &  &  &  & X\\
\hline
\end{tabular}
\subsection*{Dados Recomendante} 
	Instituição (Institution): Universidad del Cauca
\\ 
	Grau acadêmico mais alto obtido: mestre
	\ \ Área: Algebra
	\\
	Ano de obtenção deste grau: 2008
	\ \ 
	Instituição de obtenção deste grau : Universidade Estadual de Campinas, Brasil
	\\ 
	Endereço institucional do recomendante: \\ Dirección carrera 2 A N 3N111 Sector Tulcán.
Teléfono 8209800 ext. 2371\includepdf[pages={-},offset=35mm 0mm]{../../../upload/1075_2014-05-30_documentos.pdf}\includepdf[pages={-},offset=35mm 0mm]{../../../upload/1075_2014-05-30_historico.pdf} 
\begin{center}
Anexos.
\end{center}
\end{document}