\documentclass[11pt]{article}
\usepackage{graphicx,color}
\usepackage{pdfpages}
\usepackage[brazil]{babel}
\usepackage[utf8]{inputenc}
\addtolength{\hoffset}{-3cm} \addtolength{\textwidth}{6cm}
\addtolength{\voffset}{-.5cm} \addtolength{\textheight}{1cm}
%%%%%%%%%%%%%%%%%%%%%%%%%%%%%%%%%%%  To use Colors 
\title{\vspace*{-4cm} Ficha de Inscrição: \\Cod: 1148\ \ Christe Hélida Moreira Montijo\ \ - \ \ Mestrado 
 }
\date{}

\begin{document}
\maketitle
\vspace*{-1.5cm}
\noindent Data de Nascimento:3/1/1990
\ \ \ Idade: 24   \ \ \ Sexo: Feminino
\\
Naturalidade: Campos Belos  
\ \ \  Estado: GO
\ \ \  Nacionalidade: brasileira
\ \ \ País: Brasil
\\        
Nome do pai : Almir Moreira do Nascimento
\ \ \ Nome da mãe: Lazará Maria Montijo          
\\[0.2cm]                     
\textbf{Endereço Pessoal} 
\\ 
\noindent Endereço residencial: UnB Colina BL K ap 305 Asa Norte Brasília DF
\\
        CEP: 70910-900 
\ \ \ Cidade: Brasília 
\ \ \ Estado: DF 
\ \ \ País: Brasil
\\		
		Telefone comercial : +0(63)92456160
\ \ \ Telefone residencial: +0(63)92456160
\ \ \ Telefone celular : +0(63)92456160
\\
E-mail principal: christe.montijo@hotmail.com
\ \ \ E-mail alternativo: 0 
\\[0.2cm] 
\textbf{Documentos Pessoais}
\\
\noindent Número de CPF : 03633262199
\ \ \ Número de Identidade (ou Passaporte para estrangeiros): 900296
\\
Orgão emissor: SSP
\ \ \ Estado: DF
\ \ \ Data de emissão :19/5/2004
\\[0.3cm]
\textbf{Grau acadêmico mais alto obtido}
\\	
Curso:Matemática
\ \ \ Grau : licenciado
\ \ \ Instituição : Universidade Federal do Tocantins
\\			
Ano de Conclusão ou Previsão: 2012
\\ 
Experiência Profissional mais recente. \ \  
Tem experiência: 0 0  
\ \ \ Instituição: 0
\\  
Período - início: 0-0
\ \ \ fim: 0-0
\\[0.2cm] 
\textbf{Programa Pretendido:} Mestrado\\
Interesse em bolsa: Sim
\\[0.3cm]		
\textbf{Dados dos Recomendantes} 
\\
1- Nome: Noraí Romeu Rocco
\ \ \ \  e-mail: norai.rocco@gmail.com 
\\
2- Nome: Romes Antônio Borges
\ \ \ \ e-mail: romes@ufg.br
\\
3- Nome: Gilberto Fernandes Vieira
\ \ \ \ e-mail: gilbertovieira05@gmail.com
\\[0.2cm]
Motivação e expectativa do candidato em relação ao programa pretendido:
\\Em 2012 me inscrevi e fui aceita no programa de mestrado aqui na UnB. Sempre sonhei em fazer mestrado, no quanto poderia aprender com essa oportunidade, uma vez que sempre fui encantada com a matemática, entretanto, as coisas não ocorreram como planejei. Quando entrei, sabia que enfrentaria muitas dificuldades, devido as minhas deficiências acadêmicas, com relação ao nível da instituição que é a UnB, mas acreditava que no decorrer do curso, essas deficiências seriam naturalmente sanadas.
Cheguei a fazer quatro disciplinas, Geometria Diferencial, Probabilidade, Análise no Rn e Álgebra. Consegui ser aprovada em três dentre essas, a única que reprovei foi probabilidade, porém, logo em seguida fiz e passei no exame de qualificação referente à mesma. Mas, apesar das aprovações, não estava satisfeita com meu rendimento, minhas notas nas disciplinas foram baixas e não me sentia preparada o suficiente para continuar.
Então decidi pedir desligamento, cursar algumas disciplinas que considero indispensáveis para a minha formação, como aluna especial nos cursos de Matemática da UnB e depois tentar retornar ao mestrado.
Agora, um ano depois, estou cursando como aluna especial as seguintes disciplinas Geometria Diferencial, Variáveis complexas e Álgebra 2. E semestre passado fiz Álgebra 3, também como aluna especial, enquanto revisava os conteúdos de Análise no Rn e Álgebra Linear. Além disso, fiz o curso de verão em Topologia Geral no início do ano, aliás, é uma disciplina surpreendente, e enriqueceu em cem por cento minha vida acadêmica.
Todas essas mudanças foram essenciais para minha formação.
Sinto me preparada para voltar e fazer um bom mestrado, pois acredito que essa decisão de parar e recomeçar foi muito importante para o meu crescimento acadêmico, tanto em relação às disciplinas que fiz como aluna especial, que serviu como uma forma de nivelamento, quanto em relação ao que já havia estudado no mestrado, pois esse ultimo ano foi essencial para o amadurecimento de todo o conhecimento que adquiri durante o mestrado.
Quero destacar que, de forma alguma, olho para o tempo que estive no mestrado como um tempo perdido, pelo contrário, apesar de não ter continuado, aprendi muito, apenas a falta dos conhecimentos que deveria ter adquirido na minha graduação, pois minha formação é licenciatura em Matemática, fez com que me recusasse a continuar, diante da possibilidade de poder fazer melhor. 
Estou preparada e com muitas expectativas com a possibilidade de voltar para o mestrado, e apesar de gostar muito com todas as áreas, estou encantada com álgebra, então é a área que quero continuar estudando e fazer minha dissertação.
\newpage\vspace*{-4cm}\subsection*{Carta de Recomendação - Noraí Romeu Rocco}Código Identificador: 1201\\Conhece-o candidato há quanto tempo (For how long have you known the applicant)? 
\ 
\\ Conhece-o sob as seguintes circunstâncias: \ \ 
	\ \ \ \  
\\ Conheçe o candidato sob outras circunstâncias: 
\\	Avaliações:\\
\begin{tabular}{|l|c|c|c|c|c|}
\hline
 & Excelente & Bom & Regular & Insuficiente & Não sabe \\
\hline
Desempenho acadêmico &  &  &  &  & \\
\hline
Capacidade de aprender novos conceitos &  &  &  &  & \\
\hline
Capacidade de trabalhar sozinho &  &  &  &  & \\
\hline
Criatividade &  &  &  &  & \\
\hline
Curiosidade &  &  &  &  & \\
\hline
Esforço, persistência &  &  &  &  & \\
\hline
Expressão escrita &  &  &  &  & \\
\hline
Expressão oral &  &  &  &  & \\
\hline
Relacionamento com colegas &  &  &  &  & \\
\hline
\end{tabular}\\
\\
\textbf{Opinião sobre os antecedentes acadêmicos, profissionais e/ou técnicos do candidato:}
\\\\
\\
\textbf{Opinião sobre seu possível aproveitamento, se aceito no Programa:}
\\\\ 
\\
\textbf{Outras informações relevantes:} \\
\\[0.3cm]
\textbf{Entre os estudantes que já conheceu, você diria que o candidato está entre os:}
\\
\begin{tabular}{|l|c|c|c|c|c|}
\hline
 & 5\% melhores & 10\% melhores & 25\% melhores & 50\% melhores & Não sabe \\
\hline
Como aluno, em aulas &  &  &  &  & \\
\hline
Como orientando &  &  &  &  & \\
\hline
\end{tabular}
\subsection*{Dados Recomendante} 
	Instituição (Institution): UnB
\\ 
	Grau acadêmico mais alto obtido: doutor
	\ \ Área: Álgebra
	\\
	Ano de obtenção deste grau: 1980
	\ \ 
	Instituição de obtenção deste grau : UnB
	\\ 
	Endereço institucional do recomendante: \\ UnB ICC Centro BT 403 
Tel 31077347\newpage\vspace*{-4cm}\subsection*{Carta de Recomendação - Romes Antônio Borges}Código Identificador: 1398\\Conhece-o candidato há quanto tempo (For how long have you known the applicant)? 
\ 
\\ Conhece-o sob as seguintes circunstâncias: \ \ 
	\ \ \ \  
\\ Conheçe o candidato sob outras circunstâncias: 
\\Avaliações: \\
\begin{tabular}{|l|c|c|c|c|c|}
\hline
 & Excelente & Bom & Regular & Insuficiente & Não sabe \\
\hline
Desempenho acadêmico &  &  &  &  & \\
\hline
Capacidade de aprender novos conceitos &  &  &  &  & \\
\hline
Capacidade de trabalhar sozinho &  &  &  &  & \\
\hline
Criatividade &  &  &  &  & \\
\hline
Curiosidade &  &  &  &  & \\
\hline
Esforço, persistência &  &  &  &  & \\
\hline
Expressão escrita &  &  &  &  & \\
\hline
Expressão oral &  &  &  &  & \\
\hline
Relacionamento com colegas &  &  &  &  & \\
\hline
\end{tabular}\\
\\
\textbf{Opinião sobre os antecedentes acadêmicos, profissionais e/ou técnicos do candidato:}
\\\\
\\
\textbf{Opinião sobre seu possível aproveitamento, se aceito no Programa:}
\\\\ 
\\
\textbf{Outras informações relevantes:} \\
\\[0.3cm]
\textbf{Entre os estudantes que já conheceu, você diria que o candidato está entre os:}
\\
\begin{tabular}{|l|c|c|c|c|c|}
\hline
 & 5\% melhores & 10\% melhores & 25\% melhores & 50\% melhores & Não sabe \\
\hline
Como aluno, em aulas &  &  &  &  & \\
\hline
Como orientando &  &  &  &  & \\
\hline
\end{tabular}
\subsection*{Dados Recomendante} 
	Instituição (Institution): 
\\ 
	Grau acadêmico mais alto obtido: 
	\ \ Área: 
	\\
	Ano de obtenção deste grau: 
	\ \ 
	Instituição de obtenção deste grau : 
	\\ 
	Endereço institucional do recomendante: \\ \newpage\vspace*{-4cm}\subsection*{Carta de Recomendação - Gilberto Fernandes Vieira}Código Identificador: 1399\\Conhece-o candidato há quanto tempo (For how long have you known the applicant)? 
\ 2 anos
\\ Conhece-o sob as seguintes circunstâncias: aulas\ \ 
	\ \ \ \  
\\ Conheçe o candidato sob outras circunstâncias: No curso de verão
\\Avaliações: \\
\begin{tabular}{|l|c|c|c|c|c|}
\hline
 & Excelente & Bom & Regular & Insuficiente & Não sabe \\
\hline
Desempenho acadêmico &  & X &  &  & \\
\hline
Capacidade de aprender novos conceitos &  &  & X &  & \\
\hline
Capacidade de trabalhar sozinho &  &  & X &  & \\
\hline
Criatividade &  &  &  &  & X\\
\hline
Curiosidade & X &  &  &  & \\
\hline
Esforço, persistência & X &  &  &  & \\
\hline
Expressão escrita &  &  & X &  & \\
\hline
Expressão oral &  &  &  &  & X\\
\hline
Relacionamento com colegas & X &  &  &  & \\
\hline
\end{tabular}\\
\\
\textbf{Opinião sobre os antecedentes acadêmicos, profissionais e/ou técnicos do candidato:}
\\Não a conheço fora do curso Tópicos de Análise Real da 41 Escola de Verão MatUnB.\\
\\
\textbf{Opinião sobre seu possível aproveitamento, se aceito no Programa:}
\\Acredito que ela deve se sair bem, tendo em vista que ela já cursou várias disciplinas como aluna especial e, além disso, tem muito interesse em fazer  esse mestrado e aprender matemática.\\ 
\\
\textbf{Outras informações relevantes:} \\Nada a declarar
\\[0.3cm]
\textbf{Entre os estudantes que já conheceu, você diria que o candidato está entre os:}
\\
\begin{tabular}{|l|c|c|c|c|c|}
\hline
 & 5\% melhores & 10\% melhores & 25\% melhores & 50\% melhores & Não sabe \\
\hline
Como aluno, em aulas &  &  & X &  & \\
\hline
Como orientando &  &  &  &  & X\\
\hline
\end{tabular}
\subsection*{Dados Recomendante} 
	Instituição (Institution): Universidade Federal de Campina Grande
\\ 
	Grau acadêmico mais alto obtido: doutor
	\ \ Área: Equações Diferenciais Elípticas
	\\
	Ano de obtenção deste grau: 2010
	\ \ 
	Instituição de obtenção deste grau : Universidade de Brasília
	\\ 
	Endereço institucional do recomendante: \\ Rua Sergio Moreira de Figueiredo, SN, Bairro Casas Populares, Cajazeiras, Paraíba, CEP 58900 000\includepdf[pages={-},offset=35mm 0mm]{../../../upload/1148_2014-05-29_documentos.pdf}\includepdf[pages={-},offset=35mm 0mm]{../../../upload/1148_2014-05-29_historico.pdf}\includepdf[pages={-},offset=35mm 0mm]{../../../upload/1148_2013-12-05_documentos.pdf}\includepdf[pages={-},offset=35mm 0mm]{../../../upload/1148_2013-12-05_historico.pdf} 
\begin{center}
Anexos.
\end{center}
\end{document}