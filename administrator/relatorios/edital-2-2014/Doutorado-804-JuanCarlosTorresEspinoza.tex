\documentclass[11pt]{article}
\usepackage{graphicx,color}
\usepackage{pdfpages}
\usepackage[brazil]{babel}
\usepackage[utf8]{inputenc}
\addtolength{\hoffset}{-3cm} \addtolength{\textwidth}{6cm}
\addtolength{\voffset}{-.5cm} \addtolength{\textheight}{1cm}
%%%%%%%%%%%%%%%%%%%%%%%%%%%%%%%%%%%  To use Colors 
\title{\vspace*{-4cm} Ficha de Inscrição: \\Cod: 804\ \ Juan Carlos Torres Espinoza\ \ - \ \ Doutorado 
 }
\date{}

\begin{document}
\maketitle
\vspace*{-1.5cm}
\noindent Data de Nascimento:15/7/1983
\ \ \ Idade: 31   \ \ \ Sexo: Masculino
\\
Naturalidade: Trujillo  
\ \ \  Estado: Outro
\ \ \  Nacionalidade: Peruano
\ \ \ País: Peru
\\        
Nome do pai : Washington Pascual Torres Ruiz
\ \ \ Nome da mãe: María del Pilar Espinoza Llanos          
\\[0.2cm]                     
\textbf{Endereço Pessoal} 
\\ 
\noindent Endereço residencial: Rua José de Rezende Bastos 120
\\
        CEP: 36037-205 
\ \ \ Cidade: Juiz de Fora 
\ \ \ Estado: MG 
\ \ \ País: Brasil
\\		
		Telefone comercial : +0(32)32369192
\ \ \ Telefone residencial: +0(32)32369192
\ \ \ Telefone celular : +0(32)91547007
\\
E-mail principal: juantorres1507@gmail.com
\ \ \ E-mail alternativo: 0 
\\[0.2cm] 
\textbf{Documentos Pessoais}
\\
\noindent Número de CPF : 01914894685
\ \ \ Número de Identidade (ou Passaporte para estrangeiros): 5502339
\\
Orgão emissor: DIGEMIN
\ \ \ Estado: MG
\ \ \ Data de emissão :27/10/2011
\\[0.3cm]
\textbf{Grau acadêmico mais alto obtido}
\\	
Curso:Matemática
\ \ \ Grau : mestre
\ \ \ Instituição : Universidade Federal de Juiz de Fora
\\			
Ano de Conclusão ou Previsão: 2014
\\ 
Experiência Profissional mais recente. \ \  
Tem experiência: Docente Discente  
\ \ \ Instituição: Universidade Federal de Juiz de Fora
\\  
Período - início: 1-2012
\ \ \ fim: 1-2014
\\[0.2cm] 
\textbf{Programa Pretendido:} Doutorado\ \ \ \textbf{Área:} Analise\\
Interesse em bolsa: Sim
\\[0.3cm]		
\textbf{Dados dos Recomendantes} 
\\
1- Nome: Olimpio Hiroshi Miyagaki
\ \ \ \  e-mail: ohmiyagaki@gmail.com 
\\
2- Nome: Flaviana Andrea Ribeiro
\ \ \ \ e-mail: flaviana.ribeiro@ufjf.edu.br
\\
3- Nome: Fábio Rodrigues Pereira
\ \ \ \ e-mail: fabio.pereira@ufjf.edu.br
\\[0.2cm]
Motivação e expectativa do candidato em relação ao programa pretendido:
\\Motivação
Quero aprofundar meus conhecimentos na área das Edps, bem como adquirir experiência na pesquisa e divulgação científica nesta área.
Expectativas
Desenvolver pesquisa na área, me permitir a posibilidade de trabalhar com o ensino e pesquisa em matemática em instituções de ensino superior.\newpage\vspace*{-4cm}\subsection*{Carta de Recomendação - Olimpio Hiroshi Miyagaki}Código Identificador: 682\\Conhece-o candidato há quanto tempo (For how long have you known the applicant)? 
\ 3 anos
\\ Conhece-o sob as seguintes circunstâncias: aulas\ \ orientacao
	\ \ \ \  
\\ Conheçe o candidato sob outras circunstâncias: 
\\	Avaliações:\\
\begin{tabular}{|l|c|c|c|c|c|}
\hline
 & Excelente & Bom & Regular & Insuficiente & Não sabe \\
\hline
Desempenho acadêmico & X &  &  &  & \\
\hline
Capacidade de aprender novos conceitos & X &  &  &  & \\
\hline
Capacidade de trabalhar sozinho & X &  &  &  & \\
\hline
Criatividade & X &  &  &  & \\
\hline
Curiosidade & X &  &  &  & \\
\hline
Esforço, persistência & X &  &  &  & \\
\hline
Expressão escrita &  &  & X &  & \\
\hline
Expressão oral &  &  & X &  & \\
\hline
Relacionamento com colegas &  & X &  &  & \\
\hline
\end{tabular}\\
\\
\textbf{Opinião sobre os antecedentes acadêmicos, profissionais e/ou técnicos do candidato:}
\\Aluno tem um excelente potencial, bem preparado, responsavel e dedicado.\\
\\
\textbf{Opinião sobre seu possível aproveitamento, se aceito no Programa:}
\\Nao tenho duvidas do seu potencial tecnico, fara um belo trabalho\\ 
\\
\textbf{Outras informações relevantes:} \\O ponto fraco dele eh que ele fica muito nervoso na prova, isso, tem dificultado bastante. Mas se der a ele um artigo, ele estudara bem e sem usar o professor, pois o mesmo gosta muito de estudar e dirimir suas duvidas so.
\\[0.3cm]
\textbf{Entre os estudantes que já conheceu, você diria que o candidato está entre os:}
\\
\begin{tabular}{|l|c|c|c|c|c|}
\hline
 & 5\% melhores & 10\% melhores & 25\% melhores & 50\% melhores & Não sabe \\
\hline
Como aluno, em aulas &  & X &  &  & \\
\hline
Como orientando &  & X &  &  & \\
\hline
\end{tabular}
\subsection*{Dados Recomendante} 
	Instituição (Institution): universidade federal de juiz de fora
\\ 
	Grau acadêmico mais alto obtido: doutor
	\ \ Área: analise
	\\
	Ano de obtenção deste grau: 1987
	\ \ 
	Instituição de obtenção deste grau : unb
	\\ 
	Endereço institucional do recomendante: \\ departamento de matematica, UFJF, campus, bairro martelos 33.036.000 juiz de fora MG\newpage\vspace*{-4cm}\subsection*{Carta de Recomendação - Flaviana Andrea Ribeiro}Código Identificador: 925\\Conhece-o candidato há quanto tempo (For how long have you known the applicant)? 
\ 18 meses
\\ Conhece-o sob as seguintes circunstâncias: aulas\ \ 
	\ \ \ \  
\\ Conheçe o candidato sob outras circunstâncias: 
\\Avaliações: \\
\begin{tabular}{|l|c|c|c|c|c|}
\hline
 & Excelente & Bom & Regular & Insuficiente & Não sabe \\
\hline
Desempenho acadêmico &  & X &  &  & \\
\hline
Capacidade de aprender novos conceitos &  & X &  &  & \\
\hline
Capacidade de trabalhar sozinho & X &  &  &  & \\
\hline
Criatividade &  & X &  &  & \\
\hline
Curiosidade & X &  &  &  & \\
\hline
Esforço, persistência & X &  &  &  & \\
\hline
Expressão escrita & X &  &  &  & \\
\hline
Expressão oral &  & X &  &  & \\
\hline
Relacionamento com colegas & X &  &  &  & \\
\hline
\end{tabular}\\
\\
\textbf{Opinião sobre os antecedentes acadêmicos, profissionais e/ou técnicos do candidato:}
\\Juan fez Matemática no Peru e como todo aluno que temos recebido de lá ele chegou com uma base muito boa de matemática em nível de bacharelado e algumas disciplinas já em nível de pós graduação. No  mestrado na UFJF, ele cursou disciplinas de formação básica como Análise no Rn, Álgebra Avançada, Geometria Diferencial e EDO. Acredito que ele tenha uma boa base para iniciar o doutorado.\\
\\
\textbf{Opinião sobre seu possível aproveitamento, se aceito no Programa:}
\\Juan é uma aluno melhor do que o seu histórico demonstra. Em provas só escreve o que tem certeza e se for capaz de concluir a questão, o que ás vezes é ruim pois  às vezes tem boas ideias e está no caminho certo. Durante o curso que fez comigo, percebi que ele tem aptidão para a Matemática, além de motivação e persistência  para estudo mais avançados. Aposto no potencial dele para concluir com sucesso um doutorado em Matemática.\\ 
\\
\textbf{Outras informações relevantes:} \\
\\[0.3cm]
\textbf{Entre os estudantes que já conheceu, você diria que o candidato está entre os:}
\\
\begin{tabular}{|l|c|c|c|c|c|}
\hline
 & 5\% melhores & 10\% melhores & 25\% melhores & 50\% melhores & Não sabe \\
\hline
Como aluno, em aulas &  &  & X &  & \\
\hline
Como orientando &  &  &  &  & X\\
\hline
\end{tabular}
\subsection*{Dados Recomendante} 
	Instituição (Institution): Universidade Federal de Juiz de Fora
\\ 
	Grau acadêmico mais alto obtido: doutor
	\ \ Área: Geometria Algébrica
	\\
	Ano de obtenção deste grau: 2007
	\ \ 
	Instituição de obtenção deste grau : Universidade Fedral de Minas Gerais
	\\ 
	Endereço institucional do recomendante: \\ Universidade Federal de Juiz de Fora
Departamento de Matemática 
Instituto de Ciências Exatas
Campus Universitário
Bairro São Pedro, Juiz de Fora 
Minas Gerais
CEP 36036330\newpage\vspace*{-4cm}\subsection*{Carta de Recomendação - Fábio Rodrigues Pereira}Código Identificador: 1278\\Conhece-o candidato há quanto tempo (For how long have you known the applicant)? 
\ 2 anos
\\ Conhece-o sob as seguintes circunstâncias: \ \ 
	\ \ \ \ outra 
\\ Conheçe o candidato sob outras circunstâncias: Coordenei o mestrado no primeiro ano de curso do candidato e participei de sua banca de conclusão.
\\Avaliações: \\
\begin{tabular}{|l|c|c|c|c|c|}
\hline
 & Excelente & Bom & Regular & Insuficiente & Não sabe \\
\hline
Desempenho acadêmico &  & X &  &  & \\
\hline
Capacidade de aprender novos conceitos &  & X &  &  & \\
\hline
Capacidade de trabalhar sozinho & X &  &  &  & \\
\hline
Criatividade &  & X &  &  & \\
\hline
Curiosidade & X &  &  &  & \\
\hline
Esforço, persistência & X &  &  &  & \\
\hline
Expressão escrita &  & X &  &  & \\
\hline
Expressão oral &  & X &  &  & \\
\hline
Relacionamento com colegas & X &  &  &  & \\
\hline
\end{tabular}\\
\\
\textbf{Opinião sobre os antecedentes acadêmicos, profissionais e/ou técnicos do candidato:}
\\Conversando com os docentes que fizeram a seleção para o mestrado em matemática da UFJF, ele obteve um bom desempenho.\\
\\
\textbf{Opinião sobre seu possível aproveitamento, se aceito no Programa:}
\\Acredito no potencial do aluno para estudos avançados, pois possui boa formação em Matemática. Além disso, é um aluno responsável e independente no que diz respeito à iniciativa ao trabalho e a pesquisa.
Com certeza não faltará esforços ao candidato para a conclusão do curso pretendido.\\ 
\\
\textbf{Outras informações relevantes:} \\Acredito que no Doutorado, o candidato terá condições de fazer um bom trabalho original, visto que é um aluno dedicado e adquiriu os conceitos necessários para atingir esse objetivo.
O candidato evoluiu bastante no período que esteve cursando o mestrado em Matemática e acredito que terá um bom aproveitamento no curso.

\\[0.3cm]
\textbf{Entre os estudantes que já conheceu, você diria que o candidato está entre os:}
\\
\begin{tabular}{|l|c|c|c|c|c|}
\hline
 & 5\% melhores & 10\% melhores & 25\% melhores & 50\% melhores & Não sabe \\
\hline
Como aluno, em aulas &  & X &  &  & \\
\hline
Como orientando &  &  &  &  & X\\
\hline
\end{tabular}
\subsection*{Dados Recomendante} 
	Instituição (Institution): Universidade Federal de Juiz de Fora
\\ 
	Grau acadêmico mais alto obtido: doutor
	\ \ Área: Equações Diferenciais Parciais
	\\
	Ano de obtenção deste grau: 2005
	\ \ 
	Instituição de obtenção deste grau : Unicamp
	\\ 
	Endereço institucional do recomendante: \\ Universidade Federal de Juiz de Fora
Rua José Lourenço Kelmer, sn  Campus Universitário 
Bairro São Pedro  CEP 36036900  Juiz de Fora  MG
\includepdf[pages={-},offset=35mm 0mm]{../../../upload/804_2014-05-13_documentos.pdf}\includepdf[pages={-},offset=35mm 0mm]{../../../upload/804_2014-05-13_historico.pdf}\includepdf[pages={-},offset=35mm 0mm]{../../../upload/804_2013-10-30_documentos.pdf}\includepdf[pages={-},offset=35mm 0mm]{../../../upload/804_2013-10-30_historico.pdf} 
\begin{center}
Anexos.
\end{center}
\end{document}