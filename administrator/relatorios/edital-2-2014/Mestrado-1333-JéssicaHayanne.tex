\documentclass[11pt]{article}
\usepackage{graphicx,color}
\usepackage{pdfpages}
\usepackage[brazil]{babel}
\usepackage[utf8]{inputenc}
\addtolength{\hoffset}{-3cm} \addtolength{\textwidth}{6cm}
\addtolength{\voffset}{-.5cm} \addtolength{\textheight}{1cm}
%%%%%%%%%%%%%%%%%%%%%%%%%%%%%%%%%%%  To use Colors 
\title{\vspace*{-4cm} Ficha de Inscrição: \\Cod: 1333\ \ Jéssica  Hayanne\ \ - \ \ Mestrado 
 }
\date{}

\begin{document}
\maketitle
\vspace*{-1.5cm}
\noindent Data de Nascimento:10/5/1993
\ \ \ Idade: 21   \ \ \ Sexo: Feminino
\\
Naturalidade: Nova Olimpia  
\ \ \  Estado: MT
\ \ \  Nacionalidade: Brasileira
\ \ \ País: Brasil
\\        
Nome do pai : Jesse Coelho
\ \ \ Nome da mãe: Raquel da Silva Vieira Coelho          
\\[0.2cm]                     
\textbf{Endereço Pessoal} 
\\ 
\noindent Endereço residencial: Rua Tibiriça
\\
        CEP: 78390-000 
\ \ \ Cidade: Barra do Bugres 
\ \ \ Estado: MT 
\ \ \ País: Brasil
\\		
		Telefone comercial : +0(65)99197035
\ \ \ Telefone residencial: +0(65)33613099
\ \ \ Telefone celular : +0(65)99983774
\\
E-mail principal: jessika\_coelho@hotmail.com
\ \ \ E-mail alternativo: kekakoelho@gmail.com 
\\[0.2cm] 
\textbf{Documentos Pessoais}
\\
\noindent Número de CPF : 04469646113
\ \ \ Número de Identidade (ou Passaporte para estrangeiros): 22858598
\\
Orgão emissor: SSP
\ \ \ Estado: MT
\ \ \ Data de emissão :12/5/2008
\\[0.3cm]
\textbf{Grau acadêmico mais alto obtido}
\\	
Curso:Matemática
\ \ \ Grau : licenciado
\ \ \ Instituição : Universidade do Estado de Mato Grosso
\\			
Ano de Conclusão ou Previsão: 2013
\\ 
Experiência Profissional mais recente. \ \  
Tem experiência: Docente 0  
\ \ \ Instituição: Universidade do Estado de Mato Grosso
\\  
Período - início: 1-2014
\ \ \ fim: 1-2014
\\[0.2cm] 
\textbf{Programa Pretendido:} Mestrado\\
Interesse em bolsa: Sim
\\[0.3cm]		
\textbf{Dados dos Recomendantes} 
\\
1- Nome: Minéia Cappellari Fagundes
\ \ \ \  e-mail: mineiacf@gmail.com 
\\
2- Nome: Edinéia Aparecida dos Santos Galvanin
\ \ \ \ e-mail: galvanin@gmail.com
\\
3- Nome: Maria Elizabete Rambo Kochhann
\ \ \ \ e-mail: beterambo@gmail.com
\\[0.2cm]
Motivação e expectativa do candidato em relação ao programa pretendido:
\\Minha motivação para o programa pretendido inicialmente surgiu pelo conceito e prestigio que a universidade tem na sociedade, posteriormente pelo fato de ter escolhido a profissão da docência e ter como pensamento que a profissão exige um aprimoramento constante. A disciplina me fascina e seria muito gratificante ingressar nesse programa para poder vivenciar novos desafios e me realizar pessoalmente e profissionalmente. Por estar exercendo a profissão na Universidade do Estado do Mato Grosso foi possível vivenciar momentos e situações incríveis, onde me vi na necessidade de estar sempre melhorando e aperfeiçoando meus conhecimentos. Com o mestrado, professores altamente capacitados e com uma universidade com um conceito ótimo como a Universidade de Brasilia, visualizei a oportunidade que estava a procura. Com muita dedicação pretendo iniciar e concluir o programa pretendido.
Atenciosamente,
Jéssica Hayanne\newpage\vspace*{-4cm}\subsection*{Carta de Recomendação - Minéia Cappellari Fagundes}Código Identificador: 1393\\Conhece-o candidato há quanto tempo (For how long have you known the applicant)? 
\ 6 anos
\\ Conhece-o sob as seguintes circunstâncias: aulas\ \ 
	\ \ \ \  
\\ Conheçe o candidato sob outras circunstâncias: 
\\	Avaliações:\\
\begin{tabular}{|l|c|c|c|c|c|}
\hline
 & Excelente & Bom & Regular & Insuficiente & Não sabe \\
\hline
Desempenho acadêmico &  & X &  &  & \\
\hline
Capacidade de aprender novos conceitos &  & X &  &  & \\
\hline
Capacidade de trabalhar sozinho & X &  &  &  & \\
\hline
Criatividade &  & X &  &  & \\
\hline
Curiosidade & X &  &  &  & \\
\hline
Esforço, persistência & X &  &  &  & \\
\hline
Expressão escrita &  & X &  &  & \\
\hline
Expressão oral & X &  &  &  & \\
\hline
Relacionamento com colegas & X &  &  &  & \\
\hline
\end{tabular}\\
\\
\textbf{Opinião sobre os antecedentes acadêmicos, profissionais e/ou técnicos do candidato:}
\\Enquanto acadêmica Jéssica Hayanne sempre apresentou bom desempenho. Ministrei as disciplinas de Geometria Analítica e Vetorial e Cálculo Diferencial e Integral III, onde nesses disciplinas a aluna era interessada  e buscava sempre informações extras a sala de aula, mostrando interesse na área.  \\
\\
\textbf{Opinião sobre seu possível aproveitamento, se aceito no Programa:}
\\Acredito que Jéssica deve apresentar um desempenho esperado pelo programa pois sempre estava envolvida nas atividades propostas de maneira muito ativa, mesmo em trabalhos em grupos como individuais.\\ 
\\
\textbf{Outras informações relevantes:} \\O perfil de Jéssica corresponde com o Mestrado que está pleiteando a vaga, acredito que será de grande crescimento acadêmico para a mesma e para o estado de Mato Grosso que está carente de profissionais nessa área. 
\\[0.3cm]
\textbf{Entre os estudantes que já conheceu, você diria que o candidato está entre os:}
\\
\begin{tabular}{|l|c|c|c|c|c|}
\hline
 & 5\% melhores & 10\% melhores & 25\% melhores & 50\% melhores & Não sabe \\
\hline
Como aluno, em aulas &  &  &  & X & \\
\hline
Como orientando &  &  &  &  & X\\
\hline
\end{tabular}
\subsection*{Dados Recomendante} 
	Instituição (Institution): Universidade do Estado de Mato Grosso UNEMAT
\\ 
	Grau acadêmico mais alto obtido: doutor
	\ \ Área: Engenharia Elátrica
	\\
	Ano de obtenção deste grau: 2014
	\ \ 
	Instituição de obtenção deste grau : Universidade Estadual Paulista Julio de Mesquita UNESP
	\\ 
	Endereço institucional do recomendante: \\ www.novoportal.unemat.brindex.php\newpage\vspace*{-4cm}\subsection*{Carta de Recomendação - Edinéia Aparecida dos Santos Galvanin}Código Identificador: 1394\\Conhece-o candidato há quanto tempo (For how long have you known the applicant)? 
\ 4 anos
\\ Conhece-o sob as seguintes circunstâncias: aulas\ \ 
	\ \ \ \  
\\ Conheçe o candidato sob outras circunstâncias: 
\\Avaliações: \\
\begin{tabular}{|l|c|c|c|c|c|}
\hline
 & Excelente & Bom & Regular & Insuficiente & Não sabe \\
\hline
Desempenho acadêmico &  & X &  &  & \\
\hline
Capacidade de aprender novos conceitos &  & X &  &  & \\
\hline
Capacidade de trabalhar sozinho &  & X &  &  & \\
\hline
Criatividade & X &  &  &  & \\
\hline
Curiosidade & X &  &  &  & \\
\hline
Esforço, persistência &  & X &  &  & \\
\hline
Expressão escrita &  & X &  &  & \\
\hline
Expressão oral &  & X &  &  & \\
\hline
Relacionamento com colegas & X &  &  &  & \\
\hline
\end{tabular}\\
\\
\textbf{Opinião sobre os antecedentes acadêmicos, profissionais e/ou técnicos do candidato:}
\\A candidata possui plenas condições para o estudo em nível de mestrado, considerando que a mesma possui bons antecedentes acadêmicos como bom aproveitamento de estudos no curso, participação em eventos etc.\\
\\
\textbf{Opinião sobre seu possível aproveitamento, se aceito no Programa:}
\\Acredito que irá continuar com o mesmo pensamento agregador e da busca por novos conhecimentos\\ 
\\
\textbf{Outras informações relevantes:} \\
\\[0.3cm]
\textbf{Entre os estudantes que já conheceu, você diria que o candidato está entre os:}
\\
\begin{tabular}{|l|c|c|c|c|c|}
\hline
 & 5\% melhores & 10\% melhores & 25\% melhores & 50\% melhores & Não sabe \\
\hline
Como aluno, em aulas &  & X &  &  & \\
\hline
Como orientando &  & X &  &  & \\
\hline
\end{tabular}
\subsection*{Dados Recomendante} 
	Instituição (Institution): Universidade do Estado de Mato Grosso
\\ 
	Grau acadêmico mais alto obtido: doutor
	\ \ Área: Geociências
	\\
	Ano de obtenção deste grau: 2007
	\ \ 
	Instituição de obtenção deste grau : Universidade Estadual Paulista UNESP
	\\ 
	Endereço institucional do recomendante: \\ Rua A, sn, bairro Cohab São Raimundo Barra do Bugres, MT  CEP 78390000\newpage\vspace*{-4cm}\subsection*{Carta de Recomendação - Maria Elizabete Rambo Kochhann}Código Identificador: 1395\\Conhece-o candidato há quanto tempo (For how long have you known the applicant)? 
\ 5 anos
\\ Conhece-o sob as seguintes circunstâncias: aulas\ \ 
	\ \ seminarios\ \ outra 
\\ Conheçe o candidato sob outras circunstâncias: Escritos
\\Avaliações: \\
\begin{tabular}{|l|c|c|c|c|c|}
\hline
 & Excelente & Bom & Regular & Insuficiente & Não sabe \\
\hline
Desempenho acadêmico & X &  &  &  & \\
\hline
Capacidade de aprender novos conceitos & X &  &  &  & \\
\hline
Capacidade de trabalhar sozinho & X &  &  &  & \\
\hline
Criatividade &  & X &  &  & \\
\hline
Curiosidade & X &  &  &  & \\
\hline
Esforço, persistência & X &  &  &  & \\
\hline
Expressão escrita &  & X &  &  & \\
\hline
Expressão oral & X &  &  &  & \\
\hline
Relacionamento com colegas & X &  &  &  & \\
\hline
\end{tabular}\\
\\
\textbf{Opinião sobre os antecedentes acadêmicos, profissionais e/ou técnicos do candidato:}
\\É uma aluna de excelente desempenho acadêmico. Luta para aprender conceitos que possam ser novos em uma aula e na seguinte já os domina com propriedade.\\
\\
\textbf{Opinião sobre seu possível aproveitamento, se aceito no Programa:}
\\Será a aluna que sempre foi, esforçada, que tem metas e luta pelo que quer.\\ 
\\
\textbf{Outras informações relevantes:} \\Possui bom relacionamento com os colegas e professores. É determinada e sabe valorizar os conhecimentos que constrói.
\\[0.3cm]
\textbf{Entre os estudantes que já conheceu, você diria que o candidato está entre os:}
\\
\begin{tabular}{|l|c|c|c|c|c|}
\hline
 & 5\% melhores & 10\% melhores & 25\% melhores & 50\% melhores & Não sabe \\
\hline
Como aluno, em aulas & X &  &  &  & \\
\hline
Como orientando & X &  &  &  & \\
\hline
\end{tabular}
\subsection*{Dados Recomendante} 
	Instituição (Institution): Universidade do Estado de Mato Grosso   UNEMAT
\\ 
	Grau acadêmico mais alto obtido: doutor
	\ \ Área: Educação para a Ciência
	\\
	Ano de obtenção deste grau: 2007
	\ \ 
	Instituição de obtenção deste grau : Unesp
	\\ 
	Endereço institucional do recomendante: \\ Rua A sn Bairro Cohab São Raimundo
78390000 Barra do Bugres  MT\includepdf[pages={-},offset=35mm 0mm]{../../../upload/1333_2014-05-29_documentos.pdf}\includepdf[pages={-},offset=35mm 0mm]{../../../upload/1333_2014-05-29_historico.pdf} 
\begin{center}
Anexos.
\end{center}
\end{document}