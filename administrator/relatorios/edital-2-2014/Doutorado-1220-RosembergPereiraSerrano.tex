\documentclass[11pt]{article}
\usepackage{graphicx,color}
\usepackage{pdfpages}
\usepackage[brazil]{babel}
\usepackage[utf8]{inputenc}
\addtolength{\hoffset}{-3cm} \addtolength{\textwidth}{6cm}
\addtolength{\voffset}{-.5cm} \addtolength{\textheight}{1cm}
%%%%%%%%%%%%%%%%%%%%%%%%%%%%%%%%%%%  To use Colors 
\title{\vspace*{-4cm} Ficha de Inscrição: \\Cod: 1220\ \ Rosemberg Pereira Serrano\ \ - \ \ Doutorado 
 }
\date{}

\begin{document}
\maketitle
\vspace*{-1.5cm}
\noindent Data de Nascimento:17/11/1974
\ \ \ Idade: 39   \ \ \ Sexo: Masculino
\\
Naturalidade: Goiânia  
\ \ \  Estado: GO
\ \ \  Nacionalidade: Brasileira
\ \ \ País: Brasil
\\        
Nome do pai : Renato Cardozo Serrano
\ \ \ Nome da mãe: Canuta Pereira Serrano          
\\[0.2cm]                     
\textbf{Endereço Pessoal} 
\\ 
\noindent Endereço residencial: Rua das Margaridas, Q. 15 L. 05, Casa  02
\\
        CEP: 74913-370 
\ \ \ Cidade: Aparecida de Goiânia 
\ \ \ Estado: GO 
\ \ \ País: Brasil
\\		
		Telefone comercial : +0(62)39362160
\ \ \ Telefone residencial: +0(62)35496199
\ \ \ Telefone celular : +0(62)91364510
\\
E-mail principal: rpsmath@yahoo.com.br
\ \ \ E-mail alternativo: rosemberg.serrano@ueg.br 
\\[0.2cm] 
\textbf{Documentos Pessoais}
\\
\noindent Número de CPF : 76106438153
\ \ \ Número de Identidade (ou Passaporte para estrangeiros): 3155819-1662848
\\
Orgão emissor: SSP
\ \ \ Estado: GO
\ \ \ Data de emissão :28/11/1990
\\[0.3cm]
\textbf{Grau acadêmico mais alto obtido}
\\	
Curso:Matemática
\ \ \ Grau : bacharel
\ \ \ Instituição : Universidade Federal de Goiás
\\			
Ano de Conclusão ou Previsão: 1999
\\ 
Experiência Profissional mais recente. \ \  
Tem experiência: Docente Discente  
\ \ \ Instituição: Universidade Estadual de Goiás
\\  
Período - início: 2-2006
\ \ \ fim: 2-2014
\\[0.2cm] 
\textbf{Programa Pretendido:} Doutorado\ \ \ \textbf{Área:} Algebra\\
Interesse em bolsa: Nao
\\[0.3cm]		
\textbf{Dados dos Recomendantes} 
\\
1- Nome: Antonio Paques
\ \ \ \  e-mail: paques@mat.ufrgs.br 
\\
2- Nome: Plamen Emilov Kochloukov
\ \ \ \ e-mail: plamen@ime.unicamp.br
\\
3- Nome: Rosane Maria de Castilho
\ \ \ \ e-mail: rosanecastilho@ueg.br
\\[0.2cm]
Motivação e expectativa do candidato em relação ao programa pretendido:
\\Estou fazendo inscrição ao doutorado em Matemática, por que é de suma importância para minha vida acadêmica profissional e atualmente me encontro em uma posição financeira e familiar mais propicia para minha dedicação ao programa do curso. 

Acredito que possuo potencial para desenvolver o programa do curso, por ter boa capacidade intelectual, alta motivação para estudos avançados. Ao longo dos anos de academia e trabalho profissional desenvolvi capacidade de concentração aguçada para trabalhos individuais, iniciativa e desempenho para atividades em grupo, trato com respeito  colegas e superiores e possuo facilidade de expressão escrita e oral. 

Em relação ao histórico acadêmico, sempre me mostrei assíduo e perseverante.

Pretendo durante o programa do curso adquirir conhecimentos avançados de Matemática e continuar pesquisando nas áreas de álgebra e ou teoria dos números. Disponho de condições atualmente de solicitar afastamento integral do trabalho para dedicar me integramente ao programa de doutoramento. 

O programa de pós graduação fornecido pelo Instituto de Matemática da UnB tornará meu objetivo alcançável, pois além de sentir prazer em fazer estudos na área de álgebra e ou teoria dos números, o programa possui professores que eu gostaria de trabalhar, pesquisar, aprender e compartilhar experiências. 
\newpage\vspace*{-4cm}\subsection*{Carta de Recomendação - Antonio Paques}Código Identificador: 1298\\Conhece-o candidato há quanto tempo (For how long have you known the applicant)? 
\ desde 2002
\\ Conhece-o sob as seguintes circunstâncias: \ \ orientacao
	\ \ \ \  
\\ Conheçe o candidato sob outras circunstâncias: 
\\	Avaliações:\\
\begin{tabular}{|l|c|c|c|c|c|}
\hline
 & Excelente & Bom & Regular & Insuficiente & Não sabe \\
\hline
Desempenho acadêmico &  &  & X &  & \\
\hline
Capacidade de aprender novos conceitos &  &  & X &  & \\
\hline
Capacidade de trabalhar sozinho &  &  & X &  & \\
\hline
Criatividade &  &  & X &  & \\
\hline
Curiosidade &  &  & X &  & \\
\hline
Esforço, persistência &  &  & X &  & \\
\hline
Expressão escrita &  &  & X &  & \\
\hline
Expressão oral &  &  & X &  & \\
\hline
Relacionamento com colegas &  &  & X &  & \\
\hline
\end{tabular}\\
\\
\textbf{Opinião sobre os antecedentes acadêmicos, profissionais e/ou técnicos do candidato:}
\\Minha opinião sobre o candidato diz respeito à época em que esteve sob minha orientação como aluno de mestrado. Foi um aluno mediano. Defendeu sua dissertação em 2002. De lá para cá não tivemos mais nenhum contato, exceto nos ocasiões em que solicitou carta de recomendação minha para ingresso no programa de doutorado da UFG. Creio que esta é a segunda vez que isto ocorre. Não tenho, portanto, como fazer considerações sobre seu desempenho acadêmico nos dias de hoje. O candidato tem conhecimento disso e escrevo esta carta por pura insistência dele.  Contudo, levando em consideração o relato que ele me fez, via email, sobre suas atividades acadêmicas como professor na UEG, fico com a impressão de que ele amadureceu e tem se esforçado em ampliar seus conhecimentos de matemática e, em particular, da álgebra. Como professor da UEG tem ministrado disciplinas na graduação, tanto na licenciatura em matemática quanto nos cursos superiores de tecnologia. Na licenciatura é coordenador dos trabalhos de conclusão de curso, assim como tem orientado monografias de graduação e projetos de iniciação científica.
Por outro lado ele tem manifestado grande interesse e vontade de ingressar em um curso de doutorado em matemática. Não sei dizer se hoje ele tem potencial para tanto, mas acho que ele merece uma chance.\\
\\
\textbf{Opinião sobre seu possível aproveitamento, se aceito no Programa:}
\\Pelas considerações feitas acima, não saberia dizer.\\ 
\\
\textbf{Outras informações relevantes:} \\Nada a acrescentar.
\\[0.3cm]
\textbf{Entre os estudantes que já conheceu, você diria que o candidato está entre os:}
\\
\begin{tabular}{|l|c|c|c|c|c|}
\hline
 & 5\% melhores & 10\% melhores & 25\% melhores & 50\% melhores & Não sabe \\
\hline
Como aluno, em aulas &  &  &  & X & \\
\hline
Como orientando &  &  &  & X & \\
\hline
\end{tabular}
\subsection*{Dados Recomendante} 
	Instituição (Institution): Universidade Federal do Rio Grande do Sul
\\ 
	Grau acadêmico mais alto obtido: doutor
	\ \ Área: Matemática
	\\
	Ano de obtenção deste grau: 1977
	\ \ 
	Instituição de obtenção deste grau : Universidade Estadual de Campinas
	\\ 
	Endereço institucional do recomendante: \\ Instituto de Matemática, Universidade Federal do Rio Grande do Sul, Avenida Bento Gonçalves 9500, Cep 91509900, Porto Alegre, RS\newpage\vspace*{-4cm}\subsection*{Carta de Recomendação - Plamen Emilov Kochloukov}Código Identificador: 1299\\Conhece-o candidato há quanto tempo (For how long have you known the applicant)? 
\ Há aproximadamente 10 anos.
\\ Conhece-o sob as seguintes circunstâncias: aulas\ \ 
	\ \ seminarios\ \  
\\ Conheçe o candidato sob outras circunstâncias: 
\\Avaliações: \\
\begin{tabular}{|l|c|c|c|c|c|}
\hline
 & Excelente & Bom & Regular & Insuficiente & Não sabe \\
\hline
Desempenho acadêmico &  &  & X &  & \\
\hline
Capacidade de aprender novos conceitos &  & X &  &  & \\
\hline
Capacidade de trabalhar sozinho & X &  &  &  & \\
\hline
Criatividade &  & X &  &  & \\
\hline
Curiosidade &  & X &  &  & \\
\hline
Esforço, persistência & X &  &  &  & \\
\hline
Expressão escrita &  & X &  &  & \\
\hline
Expressão oral &  & X &  &  & \\
\hline
Relacionamento com colegas &  & X &  &  & \\
\hline
\end{tabular}\\
\\
\textbf{Opinião sobre os antecedentes acadêmicos, profissionais e/ou técnicos do candidato:}
\\Conheço o candidato desde o tempo que ele foi aluno de mestrado no IMECC. Ele não se destacava com seu desempenho nas aulas, até chegou a reprovar em algumas matérias no mestrado. Mas parece que foi durante o período de adaptação, mais para o final do seu mestrado Rosemberg já começou a mostrar que é capaz de estudar e aprender conceitos, teorias e métodos  profundos em álgebra. Portanto acredito que o histórico escolar do candidato NÃO reflete adequadamente as suas capacidades. Estive na banca na defesa de mestrado do candidato, e fiquei impressionado. Ele decidiu escrever na lousa. Sem anotações em mãos sequer, ele fez uma exposição excelente, durante uma hora escrevendo na lousa e explicando os principais resultados da dissertação, com desenvoltura e conhecimento.\\
\\
\textbf{Opinião sobre seu possível aproveitamento, se aceito no Programa:}
\\Acredito que Rosemberg tem bom potencial de trabalho sozinho, e que poderá ser uma boa escolha. Aqui quero ressaltar que ele defendeu o mestrado há um bom tempo, e eu não tive mais contatos com ele. Portanto as informações que eu estou providenciando são de tempos remotos.\\ 
\\
\textbf{Outras informações relevantes:} \\Rosemberg foi aluno de mestrado do prof. Antonio Paques. Antes da defesa, o orientador do Rosemberg viajou para o exterior, e me pediu para eu atender o aluno caso haja necessidade. Rosemberg veio me procurar algumas vezes. Eu tinha conhecimento do desempenho não muito convincente nas matérias cursadas por ele. Mas tenho de admitir que fiquei impressionado com ele. Rosemberg tinha perguntas sobre os artigos indicados pelo orientador, e como os tópicos foram um pouco longe da minha área, tive de ler com muito cuidado e detalhes. Então percebi que em alguns trechos, aqueles que Rosemberg tinha apontado, as coisas não estavam boas no respectivo artigo. Até catei alguns errinhos. E o Rosemberg tinha chegado aos pontos fracos daquele artigo. Não são muitos os alunos de mestrado que conseguem fazer tais coisas. Ainda mais aqueles etiquetados como medianos. 
Na minha opinião Rosemberg merece uma chance. Não posso garantir que ele consiga fazer um bom doutorado, afinal ele defendeu seu mestrado há muitos anos. Mas as minhas impressões do mestrado dele foram muito posiitvas.
\\[0.3cm]
\textbf{Entre os estudantes que já conheceu, você diria que o candidato está entre os:}
\\
\begin{tabular}{|l|c|c|c|c|c|}
\hline
 & 5\% melhores & 10\% melhores & 25\% melhores & 50\% melhores & Não sabe \\
\hline
Como aluno, em aulas &  &  & X &  & \\
\hline
Como orientando &  & X &  &  & \\
\hline
\end{tabular}
\subsection*{Dados Recomendante} 
	Instituição (Institution): Departamento de Matemática, IMECC, UNICAMP
\\ 
	Grau acadêmico mais alto obtido: doutor
	\ \ Área: álgebra
	\\
	Ano de obtenção deste grau: 1987
	\ \ 
	Instituição de obtenção deste grau : Sofia University, Sofia, Bulgaria
	\\ 
	Endereço institucional do recomendante: \\ IMECC, UNICAMP
Rua Sergio Buarque de Holanda, 651
Cidade Universitaria Zeferino Vaz
Campinas, SP, CEP 13083859\newpage\vspace*{-4cm}\subsection*{Carta de Recomendação - Rosane Maria de Castilho}Código Identificador: 1300\\Conhece-o candidato há quanto tempo (For how long have you known the applicant)? 
\ cinco anos
\\ Conhece-o sob as seguintes circunstâncias: aulas\ \ 
	\ \ seminarios\ \ outra 
\\ Conheçe o candidato sob outras circunstâncias: coordenação de TC da UEG
\\Avaliações: \\
\begin{tabular}{|l|c|c|c|c|c|}
\hline
 & Excelente & Bom & Regular & Insuficiente & Não sabe \\
\hline
Desempenho acadêmico & X &  &  &  & \\
\hline
Capacidade de aprender novos conceitos & X &  &  &  & \\
\hline
Capacidade de trabalhar sozinho & X &  &  &  & \\
\hline
Criatividade &  & X &  &  & \\
\hline
Curiosidade & X &  &  &  & \\
\hline
Esforço, persistência & X &  &  &  & \\
\hline
Expressão escrita & X &  &  &  & \\
\hline
Expressão oral & X &  &  &  & \\
\hline
Relacionamento com colegas & X &  &  &  & \\
\hline
\end{tabular}\\
\\
\textbf{Opinião sobre os antecedentes acadêmicos, profissionais e/ou técnicos do candidato:}
\\Conheci o prof. Rosemberg Serrano no Colegiado do curdo de Matemática da UEG em 2009, quando ambos trabalhávamos como docentes no Campus da Cidade de Goiás. Posteriormente o prof. Rosemberg assumiu a coordenação de Trabalho de Curso do colegiado, vindo a apresentar inovações nas normativas relativas à metodologia e avaliação de desempenho dos alunos, proposta que, por seu grau de excelência e objetividade, foi adotada por dois outros colegiados do Campus.\\
\\
\textbf{Opinião sobre seu possível aproveitamento, se aceito no Programa:}
\\Pensando um programa de Pós Graduação como um espaço acadêmico no qual as potencialidades individuais são evidenciadas a partir de demandas específicas, penso que as características individuais do candidato o credenciam a cursar o programa desta instituição.\\ 
\\
\textbf{Outras informações relevantes:} \\O prof. Rosemberg é colaborador, há três anos, no processo de autoavaliação institucional da UEG, mostrando se sempre solícito quanto à tabulação dos dados e ou quanto à escolha do melhor método estatístico a ser aplicado na análise dos dados. Sua participação ativa nos eventos acadêmicos realizados pelo colegiado também é uma característica a ser destacada.
\\[0.3cm]
\textbf{Entre os estudantes que já conheceu, você diria que o candidato está entre os:}
\\
\begin{tabular}{|l|c|c|c|c|c|}
\hline
 & 5\% melhores & 10\% melhores & 25\% melhores & 50\% melhores & Não sabe \\
\hline
Como aluno, em aulas &  & X &  &  & \\
\hline
Como orientando &  & X &  &  & \\
\hline
\end{tabular}
\subsection*{Dados Recomendante} 
	Instituição (Institution): Universidade Estadual de Goiás
\\ 
	Grau acadêmico mais alto obtido: doutor
	\ \ Área: Psicologia das Organizações
	\\
	Ano de obtenção deste grau: 2010
	\ \ 
	Instituição de obtenção deste grau : Universidade Catolica Argentina UCSF
	\\ 
	Endereço institucional do recomendante: \\ Universidade Estadual de Goiás. Campus Aparecida de Goiânia. Rua Mucuri, sn. Setor Conde dos Arcos. Aparecida de Goiânia. Goiás\includepdf[pages={-},offset=35mm 0mm]{../../../upload/1220_2014-05-19_documentos.pdf}\includepdf[pages={-},offset=35mm 0mm]{../../../upload/1220_2014-05-19_historico.pdf} 
\begin{center}
Anexos.
\end{center}
\end{document}