\documentclass[11pt]{article}
\usepackage{graphicx,color}
\usepackage{pdfpages}
\usepackage[brazil]{babel}
\usepackage[utf8]{inputenc}
\addtolength{\hoffset}{-3cm} \addtolength{\textwidth}{6cm}
\addtolength{\voffset}{-.5cm} \addtolength{\textheight}{1cm}
%%%%%%%%%%%%%%%%%%%%%%%%%%%%%%%%%%%  To use Colors 
\title{\vspace*{-4cm} Ficha de Inscrição: \\Cod: 467\ \ Valter Borges Sampaio Juniior\ \ - \ \ Doutorado 
 }
\date{}

\begin{document}
\maketitle
\vspace*{-1.5cm}
\noindent Data de Nascimento:24/8/1987
\ \ \ Idade: 27   \ \ \ Sexo: Masculino
\\
Naturalidade: Cruz das Almas  
\ \ \  Estado: BA
\ \ \  Nacionalidade: Brasileiro
\ \ \ País: Brasio
\\        
Nome do pai : Valter Borges Sampaio
\ \ \ Nome da mãe: Ana Cristina Conceição de Aquino          
\\[0.2cm]                     
\textbf{Endereço Pessoal} 
\\ 
\noindent Endereço residencial: SCRLN 706
\\
        CEP: 70740-517 
\ \ \ Cidade: Brasília 
\ \ \ Estado: DF 
\ \ \ País: Brasil
\\		
		Telefone comercial : +0(61)92252469
\ \ \ Telefone residencial: +0(61)81660569
\ \ \ Telefone celular : +0(61)92252469
\\
E-mail principal: nablavalter@gmail.com
\ \ \ E-mail alternativo: 0 
\\[0.2cm] 
\textbf{Documentos Pessoais}
\\
\noindent Número de CPF : 03147641511
\ \ \ Número de Identidade (ou Passaporte para estrangeiros): 0945137834
\\
Orgão emissor: ssp
\ \ \ Estado: DF
\ \ \ Data de emissão :27/5/2008
\\[0.3cm]
\textbf{Grau acadêmico mais alto obtido}
\\	
Curso:Matemática
\ \ \ Grau : mestre
\ \ \ Instituição : Universidade de Brasília
\\			
Ano de Conclusão ou Previsão: 2014
\\ 
Experiência Profissional mais recente. \ \  
Tem experiência: Docente Discente  
\ \ \ Instituição: 0
\\  
Período - início: 0-0
\ \ \ fim: 0-0
\\[0.2cm] 
\textbf{Programa Pretendido:} Doutorado\ \ \ \textbf{Área:} GeometriaDiferencial\\
Interesse em bolsa: Sim
\\[0.3cm]		
\textbf{Dados dos Recomendantes} 
\\
1- Nome: João Paulo dos Santos
\ \ \ \  e-mail: j.p.santos@mat.unb.br 
\\
2- Nome: Noraí Romeu Rocco
\ \ \ \ e-mail: norai.rocco@gmail.com
\\
3- Nome: Xia Changyu
\ \ \ \ e-mail: x.changyu@mat.unb.br
\\[0.2cm]
Motivação e expectativa do candidato em relação ao programa pretendido:
\\A minha expectativa com relação ao programa de doutorado em matemática da Universidade de Brasília, consiste em aperfeiçoar meus conhecimentos em matemática nas áreas de Geometria Diferencial, topologia, Análise e Álgebra. O objetivo é pesquisar na área de geometria diferencial, e sei que para contribuir com a matemática no estágio em que ela se encontra preciso saber, pelo menos, o máximo destas áreas que citei.\newpage\vspace*{-4cm}\subsection*{Carta de Recomendação - João Paulo dos Santos}Código Identificador: 1267\\Conhece-o candidato há quanto tempo (For how long have you known the applicant)? 
\ 9 meses
\\ Conhece-o sob as seguintes circunstâncias: \ \ orientacao
	\ \ \ \  
\\ Conheçe o candidato sob outras circunstâncias: 
\\	Avaliações:\\
\begin{tabular}{|l|c|c|c|c|c|}
\hline
 & Excelente & Bom & Regular & Insuficiente & Não sabe \\
\hline
Desempenho acadêmico & X &  &  &  & \\
\hline
Capacidade de aprender novos conceitos & X &  &  &  & \\
\hline
Capacidade de trabalhar sozinho & X &  &  &  & \\
\hline
Criatividade &  & X &  &  & \\
\hline
Curiosidade & X &  &  &  & \\
\hline
Esforço, persistência & X &  &  &  & \\
\hline
Expressão escrita &  & X &  &  & \\
\hline
Expressão oral & X &  &  &  & \\
\hline
Relacionamento com colegas & X &  &  &  & \\
\hline
\end{tabular}\\
\\
\textbf{Opinião sobre os antecedentes acadêmicos, profissionais e/ou técnicos do candidato:}
\\O contato com o candidato ocorreu durante o periodo de orientacao no mestrado. Ao consultar o seu historico academico, nota se que o candidato possui um desempenho muito bom, com mencoes muito boas nas disciplinas cursadas. Com relacao ao estagio de pesquisa, o candidato tambem teve um desempenho muito bom, demonstrando bastante iniciativa e empenho no desenvolvimento das atividades e na busca por conteudo que ajudasse e complementasse na sua dissertacao. Realizou uma apresentacao oral na Semana da Matematica do departamento e ira realizar uma apresentacao no seminario de Geometria Diferencial.\\
\\
\textbf{Opinião sobre seu possível aproveitamento, se aceito no Programa:}
\\Acredito que o candidato ira manter a postura descrita no campo anterior, o que ira permitir que mantenha a otima qualidade do desempenho apresentado. Alem disso, o candidato possui uma boa independencia para trabalhar, tanto na resolucoes de problemas quanto no estudo da literatura. Como ja cursou quase todas as disciplinas necessarias para a qualificacao e possui um bom background no tema que estudou para a dissertacao, podera se qualificar em pouco tempo e ja comecar a trabalhar com a pesquisa.\\ 
\\
\textbf{Outras informações relevantes:} \\O candidato nao foi meu aluno em uma disciplina regular, mas acompanha as minhas aulas na posgraduaçao este semestre. Minha classificaçao relativa aos demais alunos, em aulas, refere se a sua participacao em aula, com comentarios e sugestoes pertinentes.
\\[0.3cm]
\textbf{Entre os estudantes que já conheceu, você diria que o candidato está entre os:}
\\
\begin{tabular}{|l|c|c|c|c|c|}
\hline
 & 5\% melhores & 10\% melhores & 25\% melhores & 50\% melhores & Não sabe \\
\hline
Como aluno, em aulas &  & X &  &  & \\
\hline
Como orientando &  & X &  &  & \\
\hline
\end{tabular}
\subsection*{Dados Recomendante} 
	Instituição (Institution): Universidade de Brasilia
\\ 
	Grau acadêmico mais alto obtido: doutor
	\ \ Área: Geometria Diferencial
	\\
	Ano de obtenção deste grau: 2012
	\ \ 
	Instituição de obtenção deste grau : Universidade de Brasilia
	\\ 
	Endereço institucional do recomendante: \\ Departamento de Matemática  Universidade de Brasília 
Campus Universitário Darcy Ribeiro 
ICC Centro  Bloco A  Asa Norte 
70910900 Brasília  DF\newpage\vspace*{-4cm}\subsection*{Carta de Recomendação - Noraí Romeu Rocco}Código Identificador: 1201\\Conhece-o candidato há quanto tempo (For how long have you known the applicant)? 
\ há uns dois anos ou mais.
\\ Conhece-o sob as seguintes circunstâncias: aulas\ \ 
	\ \ \ \  
\\ Conheçe o candidato sob outras circunstâncias: 
\\Avaliações: \\
\begin{tabular}{|l|c|c|c|c|c|}
\hline
 & Excelente & Bom & Regular & Insuficiente & Não sabe \\
\hline
Desempenho acadêmico &  & X &  &  & \\
\hline
Capacidade de aprender novos conceitos & X &  &  &  & \\
\hline
Capacidade de trabalhar sozinho &  & X &  &  & \\
\hline
Criatividade &  & X &  &  & \\
\hline
Curiosidade & X &  &  &  & \\
\hline
Esforço, persistência & X &  &  &  & \\
\hline
Expressão escrita &  & X &  &  & \\
\hline
Expressão oral & X &  &  &  & \\
\hline
Relacionamento com colegas & X &  &  &  & \\
\hline
\end{tabular}\\
\\
\textbf{Opinião sobre os antecedentes acadêmicos, profissionais e/ou técnicos do candidato:}
\\O candidato é oriundo do nosso mestrado e seu histórico escolar exibe um bom desempenho. De fato, é um candidato que, vindo de um curso de graduação concluído em Instituto Tecnológico, superou as dificuldades e se mostrou apto a dar prosseguimento nos estudos em níveis mais avançados.  Ele cursou comigo a disciplina de Introdução à Álgebra no primeiro período de 2013 e foi muito bem. Ficou com a menção MS, o que achei muito significativa tendo em vista a sua formação na graduação.  \\
\\
\textbf{Opinião sobre seu possível aproveitamento, se aceito no Programa:}
\\O candidato é muito dedicado, leva os seus estudos com muita responsabilidade e convive diariamente com outros colegas no Departamento. Pelo que pude observar durante o seu primeiro semestre no mestrado, pelo desempenho nos exames do mestrado e pelas conversas subsequentes, acredito que ele está apto a prosseguir com sucesso no doutorado.\\ 
\\
\textbf{Outras informações relevantes:} \\O Walter é uma pessoa do bem, amigável e dedicado. Ele é persistente, tem muita garra e boa vontade para superar as dificuldades. Gosta de desafios e de aprender coisas novas. 
\\[0.3cm]
\textbf{Entre os estudantes que já conheceu, você diria que o candidato está entre os:}
\\
\begin{tabular}{|l|c|c|c|c|c|}
\hline
 & 5\% melhores & 10\% melhores & 25\% melhores & 50\% melhores & Não sabe \\
\hline
Como aluno, em aulas &  &  & X &  & \\
\hline
Como orientando &  &  &  &  & X\\
\hline
\end{tabular}
\subsection*{Dados Recomendante} 
	Instituição (Institution): UnB
\\ 
	Grau acadêmico mais alto obtido: doutor
	\ \ Área: Álgebra
	\\
	Ano de obtenção deste grau: 1980
	\ \ 
	Instituição de obtenção deste grau : UnB
	\\ 
	Endereço institucional do recomendante: \\ UnB ICC Centro BT 403 
Tel 31077347\newpage\vspace*{-4cm}\subsection*{Carta de Recomendação - Xia Changyu}Código Identificador: 1269\\Conhece-o candidato há quanto tempo (For how long have you known the applicant)? 
\ um ano
\\ Conhece-o sob as seguintes circunstâncias: aulas\ \ 
	\ \ \ \  
\\ Conheçe o candidato sob outras circunstâncias: 
\\Avaliações: \\
\begin{tabular}{|l|c|c|c|c|c|}
\hline
 & Excelente & Bom & Regular & Insuficiente & Não sabe \\
\hline
Desempenho acadêmico & X &  &  &  & \\
\hline
Capacidade de aprender novos conceitos &  & X &  &  & \\
\hline
Capacidade de trabalhar sozinho &  & X &  &  & \\
\hline
Criatividade &  & X &  &  & \\
\hline
Curiosidade & X &  &  &  & \\
\hline
Esforço, persistência &  & X &  &  & \\
\hline
Expressão escrita &  & X &  &  & \\
\hline
Expressão oral &  & X &  &  & \\
\hline
Relacionamento com colegas &  & X &  &  & \\
\hline
\end{tabular}\\
\\
\textbf{Opinião sobre os antecedentes acadêmicos, profissionais e/ou técnicos do candidato:}
\\Ele já estudou muitas coisas em Geometria Riemanniana.\\
\\
\textbf{Opinião sobre seu possível aproveitamento, se aceito no Programa:}
\\Ele vai ter progressos grandes.\\ 
\\
\textbf{Outras informações relevantes:} \\
\\[0.3cm]
\textbf{Entre os estudantes que já conheceu, você diria que o candidato está entre os:}
\\
\begin{tabular}{|l|c|c|c|c|c|}
\hline
 & 5\% melhores & 10\% melhores & 25\% melhores & 50\% melhores & Não sabe \\
\hline
Como aluno, em aulas &  &  & X &  & \\
\hline
Como orientando &  &  &  &  & X\\
\hline
\end{tabular}
\subsection*{Dados Recomendante} 
	Instituição (Institution): UnB
\\ 
	Grau acadêmico mais alto obtido: doutor
	\ \ Área: Geometria Diferencial
	\\
	Ano de obtenção deste grau: 1989
	\ \ 
	Instituição de obtenção deste grau : Fudan University
	\\ 
	Endereço institucional do recomendante: \\ Departamento de Matemática, UnB.\includepdf[pages={-},offset=35mm 0mm]{../../../upload/467_2014-05-12_historico.pdf}\includepdf[pages={-},offset=35mm 0mm]{../../../upload/467_2013-11-19_documentos.pdf}\includepdf[pages={-},offset=35mm 0mm]{../../../upload/467_2013-11-19_historico.pdf} 
\begin{center}
Anexos.
\end{center}
\end{document}