\documentclass[11pt]{article}
\usepackage{graphicx,color}
\usepackage{pdfpages}
\usepackage[brazil]{babel}
\usepackage[utf8]{inputenc}
\addtolength{\hoffset}{-3cm} \addtolength{\textwidth}{6cm}
\addtolength{\voffset}{-.5cm} \addtolength{\textheight}{1cm}
%%%%%%%%%%%%%%%%%%%%%%%%%%%%%%%%%%%  To use Colors 
\title{\vspace*{-4cm} Ficha de Inscrição: \\Cod: 1170\ \ Thiago Augusto Silva Dourado\ \ - \ \ Mestrado 
 }
\date{}

\begin{document}
\maketitle
\vspace*{-1.5cm}
\noindent Data de Nascimento:26/10/1984
\ \ \ Idade: 29   \ \ \ Sexo: Masculino
\\
Naturalidade: Murutinga do Sul  
\ \ \  Estado: SP
\ \ \  Nacionalidade: Brasileiro
\ \ \ País: Brasil
\\        
Nome do pai : João Celso Dourado
\ \ \ Nome da mãe: Elda Maria Silva          
\\[0.2cm]                     
\textbf{Endereço Pessoal} 
\\ 
\noindent Endereço residencial: Rua Treze de Maio 1084
\\
        CEP: 16901-018 
\ \ \ Cidade: Andradina 
\ \ \ Estado: SP 
\ \ \ País: Brasil
\\		
		Telefone comercial : +55(11)44440455
\ \ \ Telefone residencial: +55(11)44440455
\ \ \ Telefone celular : +55(11)948609257
\\
E-mail principal: doranykov@gmail.com
\ \ \ E-mail alternativo: doranykov@yahoo.com.br 
\\[0.2cm] 
\textbf{Documentos Pessoais}
\\
\noindent Número de CPF : 30947717854
\ \ \ Número de Identidade (ou Passaporte para estrangeiros): 440778992
\\
Orgão emissor: SSP
\ \ \ Estado: SP
\ \ \ Data de emissão :20/1/1999
\\[0.3cm]
\textbf{Grau acadêmico mais alto obtido}
\\	
Curso:Matemática
\ \ \ Grau : bacharel
\ \ \ Instituição : Universidade Federal de Mato Grosso do Sul
\\			
Ano de Conclusão ou Previsão: 2010
\\ 
Experiência Profissional mais recente. \ \  
Tem experiência: 0 0  
\ \ \ Instituição: 0
\\  
Período - início: 0-0
\ \ \ fim: 0-0
\\[0.2cm] 
\textbf{Programa Pretendido:} Mestrado\\
Interesse em bolsa: Sim
\\[0.3cm]		
\textbf{Dados dos Recomendantes} 
\\
1- Nome: Paulo Ribenboim
\ \ \ \  e-mail: wilson@impa.br 
\\
2- Nome: Carlos Gustavo Tamm de Araújo Moreira
\ \ \ \ e-mail: gugu@impa.br
\\
3- Nome: Antonio Carlos Tamrozzi
\ \ \ \ e-mail: tamarozzi@yandex.com
\\[0.2cm]
Motivação e expectativa do candidato em relação ao programa pretendido:
\\O quadro docente do programa de pósgraduação em matemática da UnB foi à motivação preponderante de minha inscrição. Por meio de conselhos e orientações, tanto de pessoas que foram alunos deste programa como de profissionais de outros departamentos, inferese que aí está concentrado o grupo mais proeminente de álgebra e teoria dos números de todo país. Assim sendo, minha expectativa quanto a este programa é a de solidificar minha formação básica em matemática, sobretudo na área de análise, para que, num segundo momento, possa se dar seguimento a uma pesquisa, aí sim num nível mais elevado, na área de teoria dos números, focando principal, mas não exclusivamente, os problemas diofantinos sobre corpos. \newpage\vspace*{-4cm}\subsection*{Carta de Recomendação - Paulo Ribenboim}Código Identificador: 1208\\Conhece-o candidato há quanto tempo (For how long have you known the applicant)? 
\ 3 anos
\\ Conhece-o sob as seguintes circunstâncias: aulas\ \ 
	\ \ \ \  
\\ Conheçe o candidato sob outras circunstâncias: Aulas de orientação
\\	Avaliações:\\
\begin{tabular}{|l|c|c|c|c|c|}
\hline
 & Excelente & Bom & Regular & Insuficiente & Não sabe \\
\hline
Desempenho acadêmico &  &  &  &  & X\\
\hline
Capacidade de aprender novos conceitos & X &  &  &  & \\
\hline
Capacidade de trabalhar sozinho & X &  &  &  & \\
\hline
Criatividade &  &  &  &  & X\\
\hline
Curiosidade & X &  &  &  & \\
\hline
Esforço, persistência & X &  &  &  & \\
\hline
Expressão escrita & X &  &  &  & \\
\hline
Expressão oral & X &  &  &  & \\
\hline
Relacionamento com colegas &  &  &  &  & X\\
\hline
\end{tabular}\\
\\
\textbf{Opinião sobre os antecedentes acadêmicos, profissionais e/ou técnicos do candidato:}
\\Sei que foi aluno da PUCRio\\
\\
\textbf{Opinião sobre seu possível aproveitamento, se aceito no Programa:}
\\Será excelente\\ 
\\
\textbf{Outras informações relevantes:} \\Aluno com maturidade e qualidade
\\[0.3cm]
\textbf{Entre os estudantes que já conheceu, você diria que o candidato está entre os:}
\\
\begin{tabular}{|l|c|c|c|c|c|}
\hline
 & 5\% melhores & 10\% melhores & 25\% melhores & 50\% melhores & Não sabe \\
\hline
Como aluno, em aulas &  & X &  &  & \\
\hline
Como orientando &  & X &  &  & \\
\hline
\end{tabular}
\subsection*{Dados Recomendante} 
	Instituição (Institution): Queens University, Kingston, Canada
\\ 
	Grau acadêmico mais alto obtido: doutor
	\ \ Área: Matemática  Álgebra e Teoria dos Números
	\\
	Ano de obtenção deste grau: 1957
	\ \ 
	Instituição de obtenção deste grau : Universidade de São Paulo  USP
	\\ 
	Endereço institucional do recomendante: \\ Queens University, Kingston, Canada\newpage\vspace*{-4cm}\subsection*{Carta de Recomendação - Carlos Gustavo Tamm de Araújo Moreira}Código Identificador: 720\\Conhece-o candidato há quanto tempo (For how long have you known the applicant)? 
\ Há 2 anos
\\ Conhece-o sob as seguintes circunstâncias: \ \ 
	\ \ seminarios\ \  
\\ Conheçe o candidato sob outras circunstâncias: 
\\Avaliações: \\
\begin{tabular}{|l|c|c|c|c|c|}
\hline
 & Excelente & Bom & Regular & Insuficiente & Não sabe \\
\hline
Desempenho acadêmico &  & X &  &  & \\
\hline
Capacidade de aprender novos conceitos &  & X &  &  & \\
\hline
Capacidade de trabalhar sozinho &  & X &  &  & \\
\hline
Criatividade &  & X &  &  & \\
\hline
Curiosidade & X &  &  &  & \\
\hline
Esforço, persistência &  & X &  &  & \\
\hline
Expressão escrita &  & X &  &  & \\
\hline
Expressão oral &  & X &  &  & \\
\hline
Relacionamento com colegas &  & X &  &  & \\
\hline
\end{tabular}\\
\\
\textbf{Opinião sobre os antecedentes acadêmicos, profissionais e/ou técnicos do candidato:}
\\Thiago Dourado é paulista e fez sua graduação na UFMS, em Mato Grosso do Sul. Passou cerca de um ano e meio no Rio de Janeiro. Fez um seminário informal de Teoria dos Números sob minha orientação no IMPA, em que estudou parte do livro de Teoria dos Números do Projeto Euclides do qual sou coautor. Também participou ativamente de um seminário organizado pelo Prof. Paulo Ribenboim, que visitou o IMPA na época. Em ambas as atividades mostrou bastante entusiasmo pela Matemática e por Teoria dos Números em particular, também mostrando interesse por Álgebra e Geometria Algébrica. Chegou a começar o mestrado na PUC do Rio, mas teve dificuldades de adaptação ao Rio de Janeiro e acabou voltando para São Paulo. Thiago tem muito entusiasmo pela Matemática e bastante cultura e curiosidade. Creio que pode prosseguir com bastante sucesso seus estudos de PósGraduação.\\
\\
\textbf{Opinião sobre seu possível aproveitamento, se aceito no Programa:}
\\Thiago Dourado tem muito entusiasmo pela matemática e tem feito estudos avançados principalmente nas áreas de Teoria dos Números e Álgebra, sob a orientação informal de Paulo Ribenboim, que tem visitado o Brasil com frequência nos últimos anos. Creio que pode fazer um Mestrado muito bom na UnB.\\ 
\\
\textbf{Outras informações relevantes:} \\
\\[0.3cm]
\textbf{Entre os estudantes que já conheceu, você diria que o candidato está entre os:}
\\
\begin{tabular}{|l|c|c|c|c|c|}
\hline
 & 5\% melhores & 10\% melhores & 25\% melhores & 50\% melhores & Não sabe \\
\hline
Como aluno, em aulas &  &  & X &  & \\
\hline
Como orientando &  & X &  &  & \\
\hline
\end{tabular}
\subsection*{Dados Recomendante} 
	Instituição (Institution): IMPA
\\ 
	Grau acadêmico mais alto obtido: doutor
	\ \ Área: Matemática
	\\
	Ano de obtenção deste grau: 1993
	\ \ 
	Instituição de obtenção deste grau : IMPA
	\\ 
	Endereço institucional do recomendante: \\ Estrada D. Castorina, 110 - Jardim Botânico
CEP 22460-320 - Rio de Janeiro - RJ\newpage\vspace*{-4cm}\subsection*{Carta de Recomendação - Antonio Carlos Tamrozzi}Código Identificador: 1209\\Conhece-o candidato há quanto tempo (For how long have you known the applicant)? 
\ Oito anos
\\ Conhece-o sob as seguintes circunstâncias: aulas\ \ orientacao
	\ \ seminarios\ \  
\\ Conheçe o candidato sob outras circunstâncias: 
\\Avaliações: \\
\begin{tabular}{|l|c|c|c|c|c|}
\hline
 & Excelente & Bom & Regular & Insuficiente & Não sabe \\
\hline
Desempenho acadêmico &  & X &  &  & \\
\hline
Capacidade de aprender novos conceitos &  & X &  &  & \\
\hline
Capacidade de trabalhar sozinho &  & X &  &  & \\
\hline
Criatividade &  & X &  &  & \\
\hline
Curiosidade &  &  & X &  & \\
\hline
Esforço, persistência &  &  & X &  & \\
\hline
Expressão escrita & X &  &  &  & \\
\hline
Expressão oral & X &  &  &  & \\
\hline
Relacionamento com colegas &  &  & X &  & \\
\hline
\end{tabular}\\
\\
\textbf{Opinião sobre os antecedentes acadêmicos, profissionais e/ou técnicos do candidato:}
\\Os antecedentes do candidato, na área acadêmica são muito bons. Nas disciplinas que ministrei ele se saiu bem, tinha boas questões e observações, e sempre se mostrou bastante interessado em assuntos variados. Entretanto, o grande êxito do candidato, durante sua graduação, foi sua monografia, ele fez, sob minha orientação, um trabalho bastante rico e organizativo sobre teoria dos grupos, que culminou em um ótimo texto sobre o tema, chegando até a ser utilizado por alunos de iniciação cientifica até os dias de hoje. \\
\\
\textbf{Opinião sobre seu possível aproveitamento, se aceito no Programa:}
\\Em minha opinião o candidato tem todas as ferramentas para ter um bom desempenho neste programa. Ele é atencioso e se dá bem com assuntos novos, além de ter uma excelente escrita e oratória. \\ 
\\
\textbf{Outras informações relevantes:} \\Recentemente o candidato passou um período no Rio de Janeiro, onde este em companhia de grandes profissionais, como foi o caso do Professor Paulo Ribenboim, da Queens University, e, como tem mantido contato regular comigo, pode perceber um grande avanço nos estudos e conhecimentos matemáticos do candidato. Periodicamente ele retorna a UFMS, para pequenos ciclos de apresentações. 

\\[0.3cm]
\textbf{Entre os estudantes que já conheceu, você diria que o candidato está entre os:}
\\
\begin{tabular}{|l|c|c|c|c|c|}
\hline
 & 5\% melhores & 10\% melhores & 25\% melhores & 50\% melhores & Não sabe \\
\hline
Como aluno, em aulas &  &  & X &  & \\
\hline
Como orientando & X &  &  &  & \\
\hline
\end{tabular}
\subsection*{Dados Recomendante} 
	Instituição (Institution): Universidade Federal de Mato Grosso do Sul
\\ 
	Grau acadêmico mais alto obtido: doutor
	\ \ Área: Álgebra
	\\
	Ano de obtenção deste grau: 2006
	\ \ 
	Instituição de obtenção deste grau : Universidade de Brasília
	\\ 
	Endereço institucional do recomendante: \\ Universidade Federal de Mato Grosso do Sul
Unidade II
Av. Ranulpho Marques Leal, 3484
Bairro Distr. Industrial
CEP 79620080
Três Lagoas MS \includepdf[pages={-},offset=35mm 0mm]{../../../upload/1170_2014-05-04_documentos.pdf}\includepdf[pages={-},offset=35mm 0mm]{../../../upload/1170_2014-05-04_historico.pdf}\includepdf[pages={-},offset=35mm 0mm]{../../../upload/1170_2014-02-21_documentos.pdf}\includepdf[pages={-},offset=35mm 0mm]{../../../upload/1170_2014-02-21_historico.pdf} 
\begin{center}
Anexos.
\end{center}
\end{document}